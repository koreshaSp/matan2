% 2022.11.11 Lecture 9

\begin{remrk}
 Если $x$ --- точка Лебега функции $f \in L^{1}(X,\mu)$, то выполнено
 \begin{align}
  \label{equation:remark:lebesgue_point_weaker}
  \lim_{r \to 0} \frac{1}{\mu(B_r)} \int\limits_{B_r} f(y) \, d\mu(y) = f(x)
 .\end{align} 
\end{remrk}
\begin{proof}
 По основной оценке интеграла из \eqref{equation:definition:lebesgue_point} следует
 \begin{align*}
  \lim_{r \to 0} \frac{1}{\mu(B_r)} \int\limits_{B_R} (f(y) - f(x)) d\mu(y) = 0
 .\end{align*} Перенося $f(x)$ в правую часть, получаем \eqref{equation:remark:lebesgue_point_weaker}.
\end{proof}

\begin{exmpl}
 Пусть $X = \Z$, $\mu = \sum_{k \in \Z} \delta_{\left\{ k \right\}}$ --- считающая мера. Тогда для точки $x \in \Z$:
 \begin{align*}
  \frac{1}{\mu(B_r)} \int\limits_{B_r} f(y) \, d\mu  (y) = \frac{1}{2N + 1} \sum_{\left| k - x \right| \leqslant N} f(k), \quad N = \left\lfloor r \right\rfloor 
 .\end{align*} В действительности, при $r < 1$ остается только одно слагаемое $f(x)$ --- внутри шара будет одна целая точка. Поэтому, все точки $x \in \Z$ являются точками Лебега любой функции $f$ на $\Z$.
\end{exmpl}

\begin{thm}[%
 ]
 \label{theorem:almost_all_points_are_lebesgue_points}
 Пусть $(X, \rho)$ --- сепарабельное метрическое пространство, $\mu$ --- борелевская регулярная мера на $X$, обладающая свойством удвоения.

 Пусть $f \in L^{1}(X,\mu)$. Тогда $\mu$-почти все точки $x \in X$ являются точками Лебега функции $f$.
\end{thm}

\begin{proof}[\normalfont\textsc{Доказательство теоремы \ref{theorem:almost_all_points_are_lebesgue_points}}]\

 Рассмотрим сначала простой случай: пусть функция $f \in C(X)$ непрерывна. Возьмём любую точку $x \in X$. Тогда при малых $r$ для любого $y \in B_r$
 \begin{align*}
  \left| f(x) - f(y) \right| \leqslant \delta(r)
 ,\end{align*}
 где $\delta(r) \to 0$ при $r \to 0$. Поэтому,
 \begin{align*}
  \frac{1}{\mu(B_r)} \int\limits_{B_r} \left| f(x) - f(y) \right| \, d\mu(y) \leqslant \frac{1}{\mu(B_r)} \delta(r) \cdot \int\limits_{B_r} \, d\mu   = \delta(r) \to 0
 .\end{align*} Для непрерывных функций теорема доказана. Более того, в случае непрерывной функции абсолютно все точки $X$ являются точками Лебега (а не $\mu$-почти все).

 Пусть теперь  $f \in L^{1}(X, \mu)$ --- любая суммируемая функция. Будем пока считать, что пространство $X$ ограничено.

 Возьмём любой $\eps > 0$ и рассмотрим множество <<$\eps$-плохих>> точек \begin{align*}
  A(\eps) = \left\{ x \in X \Mid \limsup_{r \to 0} \frac{1}{\mu(B_r)} \int\limits_{B_r}  \left| f(y) - f(x) \right| \, d\mu(y) > \eps   \right\}
 .\end{align*} Наша задача --- показать, что $\mu(A(\eps)) = 0$. Возьмём любую непрерывную функцию $g \in C(X)$. Сделаем <<$\eps / 3$-приём>>: запишем неравенство
 \begin{align*}
  &\frac{1}{\mu(B_r)} \int\limits_{B_r} \left| f(x) - f(y) \right| \, d\mu(y) \leqslant \\
  \leqslant &\frac{1}{\mu(B_r)} \left( \int\limits_{B_r} \left| f(x) - g(x) \right| \, d\mu(y)  + \int\limits_{B_r} \left| g(x) - g(y) \right| \, d\mu(y)  + \int\limits_{B_r} \left| g(y) - f(y) \right| \, d\mu(y) \right) \\
  \leqslant &\left| f(x) - g(x) \right| + \frac{1}{\mu(B_r)} \int\limits_{B_r} \left| g(x) - g(y) \right| \, d\mu(y) + (M^{\ast}(g-f))(x)
 .\end{align*} 
 Значит, множество $A(\eps)$ укладывается в три множества: $A(\eps) \subset A_1(\eps) \cup A_2(\eps) \cup A_3(\eps)$, где \begin{align*}
  A_1(\eps) &= \left\{ x \in X \Mid \left| f(x) - g(x) \right| > \eps / 3 \right\}, \\
  A_2(\eps) &= \left\{ x \in X \Mid \limsup_{r \to 0} \frac{1}{\mu(B_r)} \int\limits_{B_r} \left| g(x) - g(y) \right| \, d\mu(y) > \eps / 3   \right\}, \\
  A_3(\eps) &= \left\{ x \in X \Mid (M^{\ast}(f-g))(x) > \eps / 3 \right\}
  .\end{align*} Поэтому есть оценка  \begin{align*}
  \mu(A(\eps)) \leqslant \mu(A_1(\eps)) + \mu(A_2(\eps)) + \mu(A_3(\eps))
 .\end{align*} Оценим сверху меры каждого из трёх множеств.
 \begin{itemize}
  \item Для $A_1(\eps)$ воспользуемся неравенством \ref{theorem:chebishev_inequality} Чебышева: \begin{align*}
    \mu(A_1(\eps)) \leqslant \frac{3}{\eps} \left\| f-g \right\|_{L^{1}(X,\mu)}
   .\end{align*}

  \item Заметим, что $A_2(\eps) = \varnothing$, потому что мы уже доказали, что все точки являются точками Лебега непрерывной функции. Значит, $\mu(A_2(\eps)) = 0$.

  \item Для $A_3(\eps)$ воспользуемся теоремой \ref{theorem:hardy_littlewood} Харди-Литтлвуда: \begin{align*}
    \mu(A_3(\eps)) \leqslant \frac{3C}{\eps} \left\| f-g \right\|_{L^{1}(X,\mu)}
   ,\end{align*} где $C$ --- константа, зависящая лишь от меры. Напомним, что мы пока считаем пространство $X$ ограниченным.
 \end{itemize}

 Соберём теперь всё вместе:
 \begin{align*}
  \mu(A(\eps)) \leqslant \frac{3 + 3C}{\eps} \left\| f - g \right\|_{L^{1}(X,\mu)} = C' \left\| f-g \right\|_{L^{1}(X,\mu)}
  .\end{align*} Отметим, что эта оценка выполнена для любой непрерывной функции $g \in C(X)$, причём константа $C'$ не зависит от $g$. Так как мера $\mu$ конечна (следует из ограниченности $X$ и свойства удвоения) и регулярна, то по теореме \ref{theorem:continous_functions_dense_set_in_lebesgue_space} $C(X)$ всюду плотно в  $L^{1}(X,\mu)$. Тогда $\left\| f-g \right\|_{L^{1}(X,\mu)}$  можно сделать сколь угодно малым. Отсюда получаем $\mu(A(\eps)) = 0$ для любого  $\eps > 0$. Значит, \begin{align*}
  \mu(A(0)) = \mu \left( \bigcup_{n=1}^{\infty} A\left(\frac{1}{n}\right) \right) = 0
 .\end{align*} Следовательно, при $\mu$-почти все точки $x \in X$ являются точками Лебега.

 Теперь рассмотрим общий случай, когда пространство $X$ не ограничено. Представим его в виде счётного объединения шаров:
 \begin{align*}
  X = \bigcup_{n=1}^{\infty} B(x_0, n), \quad x_0 \in X
 .\end{align*} На каждом шаре множество $T_n$ точек, не являющихся точками Лебега имеет нулевую меру. Тогда и множество всех не-точек-Лебега имеет меру нуль:
 \begin{align*}
  \mu \left( \bigcup_{n=1}^{\infty} T_n \right) = 0
 ,\end{align*}  поэтому $\mu$-почти все точки пространства --- точки Лебега.
\end{proof}

\begin{crly*}
 Пусть $E \subset \R$ --- измеримое множество. Тогда для $\lambda_1$-почти всех точек $x \in E$ существует предел \begin{align*}
  \lim_{\delta \to 0} \frac{\left| E \cap (x - \delta, x + \delta) \right|}{2\delta} = 1
 .\end{align*} Это ответ на мотивирующую задачу \ref{problem:motivatinal_problem_good_points_in_closed_subset} в начале параграфа.
\end{crly*}
\begin{proof}
 Рассмотрим характеристическую функцию $\chi_E$. Так как $\R$ --- сепарабельное метрическое пространство, а мера Лебега $\lao$ регулярна и обладает условием удвоения, то $\lao$-почти все точки $x \in \R$ являются точками Лебега функции $\chi_E$. Тогда, $\lao$-почти все точки $x \in E$ являются точками Лебега множества $E$. 

 Здесь остался один недоказанный момент: \textit{регулярность меры Лебега}. Она будет проверена позже, в теореме \ref{theorem:lebesgue_measure_is_regular}, но можно обойтись и без неё: можно локализовать задачу на сколь угодно большой куб $C$. По лемме \ref{lemma:finite_measure_on_compact_space_is_regular}  мера Лебега будет регулярна на компакте, и поэтому $\lao$-почти все точки $x \in E \cap C$ будут точками Лебега множества $E \cap C$. Увеличивая куб $C$, получим нужное.
\end{proof}

\section{Произведение мер. Теоремы Тонелли и Фубини.}

Есть очень важное геометрическое наблюдение, называемое \textit{принципом Кавальери}: объём тела равен интегралу $\int_{a}^{b} S(x) \, dx$, где $S(x)$ -- площадь сечения тела плоскостью, перпендикулярной определённой оси, и имеющую координату $x$ по этой оси. Прежде чем мы перейдём к конкретике --- мере Лебега в $\R^{n}$ --- мы изучим аналоги этого утверждения в теории меры. Общая идея состоит в том, что не важно, по какой переменной сначала интегрировать, результат будет один и тот же. Об этом как раз и будут теоремы Тонелли и Фубини.

Но прежде этого, мы вернемся на элементарный уровень. Наша текущая задача --- имея два пространства $X$, $Y$ с мерами $\mu$ и $\nu$ построить меру на $X \times Y$. Причём хотелось бы, чтобы на прямоугольнике $A \times B$ ($A \subset X$, $B \subset Y$) эта мера имела естественное значение $\mu(A) \cdot \nu(B)$.

\begin{lm}
 \label{lemma:product_of_semirings_is_semiring}
 Пусть  $\p$, $\mathcal{Q}$ --- полукольца. Тогда множество \begin{align*}
  \p \times \mathcal{Q} = \left\{ P \times Q \Mid P \in \p, Q \in \mathcal{Q} \right\}
 \end{align*}  --- полукольцо.
\end{lm}
\begin{proof}
 Проверим все три аксиомы полукольца.
 \begin{enumerate}
  \item $\varnothing \in \p \times \mathcal{Q}$, так как $\varnothing = \varnothing \times \varnothing$, и $\varnothing \in \p$, $\varnothing \in \mathcal{Q}$.
  \item Пусть $P_1 \times Q_1 \in \mathcal{P} \times \mathcal{Q}$ и  $P_2 \times Q_2 \in \p \times \mathcal{Q}$. Тогда \begin{align*}
    (P_1 \times Q_1) \cap (P_2 \times Q_2) = (P_1 \cap P_2) \times (Q_1 \cap Q_2) \in \p \times \mathcal{Q}
   ,\end{align*} так как $P_1 \cap P_2 \in \p$ и $Q_1 \cap Q_2 \in \mathcal{Q}$.
  \item Пусть $P_1 \times Q_1 \in \p \times \mathcal{Q}$ и $P_2 \times Q_2 \in \p \times \mathcal{Q}$. Удобно рассмотреть эти множества на плоскости в качестве сетки $3 \times 3$ (см. рис \ref{fig:semiring-product-set-difference}).
   \begin{figure}[ht]
    \centering
    \incfig{semiring-product-set-difference}
    \caption{Общее положение двух множеств из произведения полуколец.}
    \label{fig:semiring-product-set-difference}
   \end{figure}
   Рассмотрим множества 
   \begin{align*}
    A = P_1 \setminus P_2, & &B = P_1 \cap P_2, & &C = P_2 \setminus P_1
   \end{align*} и 
   \begin{align*}
    E = Q_1 \setminus Q_2, & &F = Q_1 \cap Q_2, & &G = Q_2 \setminus Q_1
   .\end{align*}
   Каждое из множеств $A, B, C, E, F, G$ является дизъюнктным конечным объединением элементов (своего) полукольца. Значит, каждое из девяти возможных произведений является конечным дизъюнктным объединением множеств из $\p \times \mathcal{Q}$. А разность можно выразить явно через три ячейки сетки: \begin{align*}
    P_1 \times Q_1 \setminus P_2 \times Q_2 = (A \times E) \sqcup (B \times E) \sqcup (A \times F)
   .\end{align*}
 \end{enumerate}
\end{proof}
\begin{lm}
 \label{lemma:semiring_to_algebra}
 Пусть $\p$ --- полукольцо над $X$, причём $X \in \p$. Тогда множество \begin{align*}
  \A = \left\{ \bigsqcup_{k=1}^{N} P_k \Mid \{P_{k}\}_{k=1}^{N} \subset \p  \right\}
 \end{align*} конечных дизъюнктных объединений множеств полукольца  --- алгебра.
\end{lm}
\begin{proof}\
 \begin{enumerate}
  \item $\varnothing \in \A$, так как $\varnothing \in \p$.
  \item Пересечение двух элементов из $\A$ попадает в $\A$: \begin{align*}
    \bigsqcup_{k=1}^{N} P_k \cap \bigsqcup_{j=1}^{M} Q_j = \bigsqcup_{k=1}^{N} \bigsqcup_{j=1}^{M} (P_k \cap Q_j)
   .\end{align*} 
  \item Равенство \begin{align*}
    X \setminus \bigsqcup_{k=1}^{N} P_k = \bigsqcup_{j=1}^{M} Q_j
   \end{align*} для дополнения --- это первый пункт леммы \ref{lemma:about_subordinate_partition} о подчинённом разбиении, доказанной в начале семестра.
 \end{enumerate}
\end{proof}
\begin{remrk}
 \label{remark:semiring_to_algebra}
 Алгебра $\A$ из леммы \ref{lemma:semiring_to_algebra} называется \textit{порождённой полукольцом $\p$}. Ясно, что $\A$ --- минимальная алгебра, содержащая $\p$. Также ясно, что если $\B$ --- минимальная $\sigma$-алгебра, содержащая $\p$, то $\A \subset \B$.

 Также заметим, что условие $X \in \p$ в лемме \ref{lemma:semiring_to_algebra} существенно: без него нельзя доказать аксиому с дополнением. Например, лемму нельзя применить для полукольца ячеек $\p_1$.
\end{remrk}

Чтобы построить меру на $X \times Y$ нам сначала нужно определится, на какой $\sigma$-алгебре эта мера будет построена. Это приводит нас к следующему определению.
\begin{df}
 Пусть $\A \subset 2^X$, $\B \subset 2^Y$ --- $\sigma$-алгебры. Введём обозначение для множества \begin{align*}
  \A \times \B = \left\{ A \times B \Mid A \in \A, B \in \B \right\}
 .\end{align*} Это множество является полукольцом над $X \times Y$, но в общем случае не является $\sigma$-алгеброй. Поэтому, \textit{тензорным произведением} $\sigma$-алгебр $\A$ и $\B$ называется множество $\A \otimes \B$  --- наименьшая $\sigma$-алгебра, содержащая полукольцо $\A \times \B$.

 Также введём удобные обозначения для \textit{сечений} множества $E \subset X \times Y$ по элементу $x \in X$:
 \begin{align*}
  E_x &= \left\{ y \in Y \Mid (x, y) \in E \right\} \subset Y
 \end{align*} и по элементу $y \in Y$:
 \begin{align*}
  E^{y} &= \left\{ x \in X \Mid (x, y) \in E \right\} \subset X
 .\end{align*} Для понимания см. рис. \ref{fig:projections}.

 \begin{figure}[ht]
  \centering
  \incfig{projections}
  \caption{Сечение $E_x$.}
  \label{fig:projections}
 \end{figure}

\end{df}

\begin{lm}
 \label{lemma:cross_sections_of_sigma_algebra_product_is_measurable}
 Если $E \in \A \otimes \B$, то для любых $x \in X$ и $y \in Y$ множество $E_x$ измеримо в $\B$, а множество $E^y$ измеримо в $\A$.
\end{lm}
\begin{proof}[\normalfont\textsc{Доказательство}]
 Пусть $\mathcal{C}$ --- множество всех таких $E \subset \A \otimes \B$, что $E_x \in \B$ и $E^{y} \in \A$ для любых $x \in X$, $y \in Y$. Поймём, что $\mathcal{C}$ --- $\sigma$-алгебра:
 \begin{enumerate}
  \item $\varnothing \in \mathcal{C}$, так как $\varnothing_x = \varnothing^{y} = \varnothing$.
  \item Если $\{E_{k}\}_{k=1}^{\infty} \subset \mathcal{C} $, то \begin{align*}
    \left( \bigcup_{k=1}^{\infty} E_k \right)_x = \bigcup_{k=1}^{\infty} \underbrace{(E_k)_x}_{\in \mathcal{B}} \in \mathcal{B}
   .\end{align*} Для сечения по $y$ проверяется аналогично.
  \item Если $E \in \mathcal{C}$, то
   \begin{align*}
    (E^{c})_x = Y \setminus E_x \in \B
   .\end{align*} Для сечения по $y$ проверяется аналогично.
 \end{enumerate}

 Значит, $\mathcal{C}$ --- $\sigma$-алгебра. Кроме того, $\mathcal{C} \supset \A \times \B$, так как если $A \in \A$ и $B \in \B$, то \begin{align*}
  (A \times B)_x = \begin{cases}
   B, \text{ если } x \in A \\
   \varnothing, \text{ иначе}
  \end{cases} \in \B
  ,\end{align*} и \begin{align*}
  (A \times B)^{y} = \begin{cases}
   A, \text{ если } y \in B  \\
   \varnothing, \text{ иначе}
  \end{cases} \in \A
 .\end{align*} Раз $\A \otimes \B$ наименьшая, то $\mathcal{C} = \A \otimes \B$, что завершает доказательство леммы.
\end{proof}

Теперь начнём построение меры на тензорном произведении $\A \otimes \B$. Естественным путём нужно сначала её задать на полукольце $\A \times \B$.

\begin{lm}
 \label{lemma:continuation_of_measure_on_sigma_algebra_product}
 Пусть $(X, \A, \mu)$ и $(Y, \B, \nu)$ --- пространства с мерами. Тогда функция \begin{align*}
  A \times B \mapsto \mu(A) \cdot \nu(B), \quad A \in \A, B \in \B
 \end{align*} является счётно-аддитивной на полукольце $\A \times \B$.
\end{lm}
\begin{proof}
 Пусть $A \times B = \bigsqcup_{i=1}^{\infty} A_i \times B_i$. Тогда для любых $x \in X$, $y \in Y$ выполняется тривиальное равенство
 \begin{align}
  \nonumber
  \chi_A(x) \cdot \chi_B(y) = \chi_{A \times B}(x, y) &= \sum_{i=1}^{\infty} \chi_{A_i \times B_i}(x, y)  = \sum_{i=1}^{\infty} \chi_{A_i}(x) \cdot \chi_{B_i}(y) \\
  \label{equation:lemma:continuation_of_measure_on_sigma_algebra_product:characteristic_functions_equality}
  \implies \chi_A(x) \cdot \chi_B(y) &= \sum_{i=1}^{\infty} \chi_{A_i}(x) \cdot \chi_{B_i}(y)
 .\end{align} Так как по предыдущей лемме \ref{lemma:cross_sections_of_sigma_algebra_product_is_measurable} функции вида $\chi_{\tilde A}$, $\chi_{\tilde B}$ для любых $\tilde A \in \A$, $\tilde B \in \B$ измеримы, то мы можем проинтегрировать равенство \eqref{equation:lemma:continuation_of_measure_on_sigma_algebra_product:characteristic_functions_equality} по переменной $x$, зафиксировав при этом $y$:
 \begin{align}
  \nonumber
  \chi_B(y) \int\limits_{X} \chi_A(x) \, d\mu(x)   &= \int\limits_{X} \left(\sum_{i=1}^{\infty} \chi_{A_i}(x) \cdot \chi_{B_i}(y) \right) \, d\mu(x) \implies \\
  \nonumber
  \implies \chi_B(y) \mu(A) &= \sum_{i=1}^{\infty} \int\limits_{X} \chi_{A_i}(x) \cdot \chi_{B_i}(y) \, d\mu(x) \implies \\
  \label{equation:lemma:continuation_of_measure_on_sigma_algebra_product:equality_integrated_by_x}
  \implies \mu(A) \chi_B(y) &= \sum_{i=1}^{\infty} \mu(A_i) \chi_{B_i}(y)
 .\end{align} Менять местами знак интеграла и знак бесконечной суммы можно по теореме \ref{theorem:levi} Леви: нужно рассмотреть неотрицательные функции
 \begin{align*}
  f_n(x) = \sum_{i=1}^{n} \chi_{A_i}(x) \cdot \chi_{B_i}(y)
  ,\end{align*} возрастающие к сумме ряда. Теперь проинтегрируем \eqref{equation:lemma:continuation_of_measure_on_sigma_algebra_product:equality_integrated_by_x} по $y$ и получим то, что нужно: \begin{align*}
  \mu(A) \nu(B) &= \int\limits_{Y} \mu(A) \chi_B(y) \, d\nu(y) = \int\limits_{Y} \left( \sum_{i=1}^{\infty} \mu(A_i) \chi_{B_i}(y) \right) d\nu(y) =  \\
  &= \sum_{i=1}^{\infty} \int\limits_{Y} \mu(A_i)  \chi_{B_i}(y) d\nu(y) = \sum_{i=1}^{\infty} \mu(A_i) \nu(B_i)
 .\end{align*} 
\end{proof}

Счётная-аддитивность позволяет продолжить меру-произведение с полукольца $\A \times \B$ на $\sigma$-алгебру $\A \otimes \B$. Чтобы быть абсолютно уверенными в единственности стандартного продолжению, далее у нас будет \textbf{соглашение}: в этом параграфе все меры $\sigma$-конечны.

\begin{df}
 \label{definition:measure_product}
 Пусть $\mu$, $\nu$ --- $\sigma$-конечные меры на $\sigma$-алгебрах $\A$, $\B$ соответственно. Тогда \textit{произведением мер} $\mu$ и $\nu$ называется мера $\mu \times \nu$ на $\sigma$-алгебре $\A \otimes \B$, являющаяся стандартным продолжением функции множеств \begin{align*}
  A \times B \mapsto \mu(A) \cdot \nu(B)
 \end{align*}  на полукольце $\A \times \B$.
\end{df}
\begin{remrk*}
 Произведение мер $\mu \times \nu$ существует и единственно по теореме \ref{theorem:caratheodory} Каратеодори, так как меры $\mu$ и $\nu$ $\sigma$-конечны (см. следствие \ref{corollary:sigma-finite-caratheodory-continuation-is-unique}).
\end{remrk*}

Интуитивно понятно, что не важно, в каком порядке перемножать меры (и $\sigma$-алгебры): меры $\mu \times \nu$ и $\nu \times \mu$ ($\sigma$-алгебры $\A \otimes \B$ и $\B \otimes \A$) должны быть эквивалентными. Это соображение будет удобно тем, что в некоторых случаях мы можем сослаться на симметричность и доказать только один случай. Поэтому, формализуем его.

\begin{df*}
 Пусть $E \subset X \times Y$. Обозначим
 \begin{align*}
  E^{\top} = \left\{ (y,x) \mid (x,y) \in E \right\} \subset Y \times X
 .\end{align*} 
\end{df*}
\begin{remrk*}\
 \begin{itemize}
  \item Для множества $A \times B \subset X \times Y$, где $A \subset X$, $B \subset Y$, выполнено
   \begin{align*}
    (A \times B)^{\top} = B \times A
   .\end{align*} 

  \item Для множеств $E, E_1, E_2, \ldots \subset X \times Y$ выполнены равенства
   \begin{align*}
    (E^{\top})^{c} &= (E^{c})^{\top}, \\
    E_1^{\top} \cup E_2^{\top} \cup \ldots &= (E_1 \cup E_2 \cup \ldots)^{\top}, \\
    E_1^{\top} \cap E_2^{\top} \cap \ldots &= (E_1 \cap E_2 \cap \ldots)^{\top}
   .\end{align*} 
 \end{itemize}
\end{remrk*}
\begin{lm}
 \label{lemma:commutativity_of_sigma_algebra_tensor_product}
 Пусть $\A \subset 2^{X}$, $\B \subset 2^{Y}$ --- $\sigma$-алгебры. Тогда для $E \subset X \times Y$ выполняется
 \begin{align*}
  E \in \A \otimes \B \iff E^{\top} \in \B \otimes \A
 .\end{align*} 
\end{lm}
\begin{proof}
 Понятно, что достаточно доказать только в сторону ($\Longrightarrow$), ведь обратную сторону можно получить, поменяв местами $\A$ и $\B$ и применив прямую сторону. Рассмотрим множество
 \begin{align*}
  J = \left\{ E \in \A \otimes \B \Mid E^{\top} \in \B \otimes \A \right\}
 .\end{align*} Тривиально проверяется, что $J$ --- $\sigma$-алгебра, и что $J$ содержит $\A \times \B$. Значит, $J = \A \otimes \B$.
\end{proof}
\begin{lm}
 \label{lemma:commutativity_of_product_measures}
 Пусть $(X,\A,\mu)$ и $(Y,\B,\nu)$ --- пространства с мерами. Тогда для любого $E \in \A \otimes \B$ верно
 \begin{align*}
  (\mu \times \nu)(E) = (\nu \times \mu)(E^{\top})
 .\end{align*} 
\end{lm}
\begin{proof}
 Применим явную формулу стандартного продолжения (замечание \ref{remark:measure_explicit_formula}):
 \begin{align*}
  (\mu \times \nu)(E) &= \inf \left\{ \sum_{k=1}^{\infty} \mu(A_k) \nu(B_k) \Mid A_k \in \A, B_k \in \B,\; E \subset \bigcup_{k=1}^{\infty} A_k \times B_k \right\} = \\
  &= \inf \left\{ \sum_{k=1}^{\infty} \nu(B_k) \mu(A_k) \Mid B_k \in \B, A_k \in \A,\; E^{\top} \subset \bigcup_{k=1}^{\infty} B_k \times A_k \right\} = \\
  &= (\nu \times \mu)(E^{\top})
 .\end{align*} 
\end{proof}

Теперь, имея понятие произведения мер, мы можем перейти к аналогу принципа Кавальери в теории меры --- основному способу вычислить меру измеримого множества $E \subset X \times Y$.

\begin{thm}[%
 Принцип Кавальери]
 \label{theorem:principle_cavalieri} 

 Пусть $\mu$, $\nu$ --- конечные меры на  $\sigma$-алгебрах $\A$, $\B$  в $X$, $Y$ соответственно. Тогда для любого множества  $E \in \A \otimes \B$  выполнено следующее.
 \begin{enumerate}
  \item Функция $x \mapsto \nu(E_x)$  измерима относительно $\A$.
   \label{enum1:theorem:principle_cavalieri}
  \item Функция $y \mapsto \mu(E^{y})$  измерима относительно $\B$.
   \label{enum2:theorem:principle_cavalieri}
  \item Имеет место равенство
   \begin{align}
    \label{equation1:theorem:principle_cavalieri}
    (\mu \times \nu)(E) &= \int\limits_{X} \nu(E_x) \, d\mu = \\
    \label{equation2:theorem:principle_cavalieri}
    &=\int\limits_{Y} \mu(E^{y}) \, d\nu
   .\end{align} 
   \label{enum3:theorem:principle_cavalieri}
 \end{enumerate}
\end{thm}

Для доказательства принципа Кавальери нам потребуется ещё одна структура над множествами.

\begin{df*}
 Пусть $X$  --- множество. Набор подмножеств $C \subset 2^{X}$ называется \textit{монотонным классом}, если для любых $\{E_{k}\}_{k=1}^{\infty} \subset C $ таких, что $ E_1 \subset E_2 \subset \ldots \subset E_k \subset \ldots $ верно $\bigcup_{k=1}^{\infty} E_k \in C$, и для любых $\{F_{k}\}_{k=1}^{\infty} \subset C $  таких, что $ F_1 \supset F_2 \supset \ldots \supset F_k \supset \ldots $ верно $\bigcap_{k=1}^{\infty} F_k \in C$.
\end{df*}
\begin{exmpl*}
 Любая $\sigma$-алгебра является монотонным классом (можно взять счётное объединение, пересечение).
\end{exmpl*}
\begin{remrk*}
 Как и для $\sigma$-алгебр, существует понятие \textit{наименьшего монотонного класса}, содержащего семейство $S \subset 2^{X}$: это просто пересечение всех монотонных классов, содержащих $S$. Хотя бы один такой монотонный класс есть ($2^{X}$), а также пересечение любого набора монотонных классов --- монотонный класс.
\end{remrk*}

\begin{lm}
 \label{lemma:algebra_and_monotonous_class_is_sigma_algebra}
 Если семейство $\A \subset 2^{X}$ является одновременно алгеброй и монотонным классом, то $\A$ --- $\sigma$-алгебра.
\end{lm}
\begin{proof}
 Достаточно проверить только аксиому со счётным объединением. Пусть $\{A_{n}\}_{k=1}^{\infty} \subset \A$. Тогда
 \begin{align*}
  \bigcup_{k=1}^{\infty} A_k = \bigcup_{n=1}^{\infty} \left( \bigcup_{k=1}^{n} A_k \right)
 .\end{align*} Так как множества $\bigcup_{k=1}^{n} $  возрастают, а $\A$  --- монотонный класс, то их объединение $\bigcup_{k=1}^{\infty} A_k$  принадлежит $\A$.
\end{proof}

\begin{lm}[%
 о монотонном классе]
 \label{lemma:about_monotonous_class}

 Пусть $\A$ --- алгебра, $\B$ --- наименьшая $\sigma$-алгебра, содержащая $\A$, а $C$ --- наименьший монотонный класс, содержащий $\A$. Тогда $\B = C$.
\end{lm}
\begin{proof}
 Так как $\B$  --- монотонный класс, содержащий $\A$, а $C$ --- наименьший монотонный класс, содержащий $\A$, то $C \subset \B$. Теорема автоматически будет доказана, если мы покажем, что $C$ является $\sigma$-алгеброй --- тогда мы аналогичным образом получим $\B \subset C$ и заключим $\B = C$. Но по лемме \ref{lemma:algebra_and_monotonous_class_is_sigma_algebra} достаточно показать, что $C$ --- алгебра. Проверим тогда для $C$ все три аксиомы алгебры. Проверки будут иметь <<структурный>> вид, как обычно бывает в теории меры.
 \begin{enumerate}
  \item $\varnothing \in \A \subset C$.
  \item Докажем, что монотонный класс $C$ замкнут относительно операции пересечения двух множеств. Для каждого множества $E \in C$ определим множество
   \begin{align*}
    I_E = \left\{ F \in C \mid E \cap F \in C \right\}
   .\end{align*} Ясно, что $I_E \subset C$. Нам достаточно показать, что $I_E = C$ для всякого  $E \in C$. Для этого  достаточно проверить следующие два пункта:
   \begin{itemize}
    \item $I_E$  --- монотонный класс. Действительно, пусть $\{F_{n}\}_{n=1}^{\infty} \subset I_E$, и
     \begin{align*}
      F_1 \subset F_2 \subset \ldots
     \end{align*} Тогда
     \begin{align*}
      E \cap F_1 \subset E \cap F_2 \subset \ldots
     \end{align*} Но так как все $E \cap F_n \in C$, а $C$ --- монотонный класс, то
     \begin{align*}
      E \cap \bigcup_{n=1}^{\infty} F_n = \bigcup_{n=1}^{\infty} (E \cap F_n) \in C
     .\end{align*} По определению $I_E$:
     \begin{align*}
      \bigcup_{n=1}^{\infty} F_n \in C
     .\end{align*} Мы показали замкнутость $I_E$ относительно взятия объединения возрастающей последовательности множеств, замкнутость относительно взятия пересечения убывающей последовательности множеств проверяется аналогично. 
    \item $I_E \supset \A$. В случае $E \in \A$ включение $I_E \supset \A$ верно, так как алгебра $\A$ замкнута относительно пересечения. Но тогда включение верно для любого $E \in C$: ведь по симметричности определения $I_E$ имеем $F \in I_E \iff E \in I_F$, а последнее уже проверено для $E \in \A$.
   \end{itemize} 
  \item Докажем, что $C$ замкнут относительно взятия дополнения. Рассмотрим множество
   \begin{align*}
    J = \left\{ E \in C \mid E^{c} \in C \right\}
   .\end{align*} Ясно, что $J \subset C$. Нам достаточно показать $J = C$. Для этого достаточно проверить два пункта:
   \begin{itemize}
    \item $J$ --- монотонный класс. Действительно, пусть $\{E_{n}\}_{n=1}^{\infty} \subset J$, и
     \begin{align*}
      E_1 \subset E_2 \subset \ldots
     \end{align*} Тогда
     \begin{align*}
      E_1^{c} \supset E_2^{c} \supset \ldots
     \end{align*} Так как все $E_n^{c} \in C$, а $C$ --- монотонный класс, то и
     \begin{align*}
      \left(\bigcup_{n=1}^{\infty} E_n\right)^{c} = \bigcap_{n=1}^{\infty} E_n^{c} \in C
     .\end{align*} По определению $J$  получаем
     \begin{align*}
      \bigcup_{n=1}^{\infty} E_n \in J
     .\end{align*} Для убывающих множеств совершенно аналогично.
    \item $J \supset \A$ очевидно, так как алгебра $\A$ замкнута относительно взятия дополнения.
   \end{itemize}
 \end{enumerate}

\end{proof}

\begin{proof}[\normalfont\textsc{Доказательство принципа Кавальери (теорема \ref{theorem:principle_cavalieri})}]\

 Сначала покажем, что достаточно доказывать пункт \ref{enum1:theorem:principle_cavalieri} и равенство \eqref{equation1:theorem:principle_cavalieri}: пункт \ref{enum2:theorem:principle_cavalieri} и равенство \eqref{equation2:theorem:principle_cavalieri} можно из них вывести, учитывая симметричность из лемм \ref{lemma:commutativity_of_sigma_algebra_tensor_product} и \ref{lemma:commutativity_of_product_measures}. Действительно, рассмотрим множество $E^{\top} \in \B \otimes \A$. Тогда функция $y \mapsto \mu(E^{\top}_y) = \mu(E_y)$ измерима, что сразу доказывает пункт \ref{enum2:theorem:principle_cavalieri}. Также по лемме \ref{lemma:commutativity_of_product_measures} и равенству \eqref{equation1:theorem:principle_cavalieri} имеем
 \begin{align*}
  (\mu \times \nu)(E) = (\nu \times \mu)(E^{\top}) = \int\limits_{Y} \mu(E^{\top}_y) d\nu = \int\limits_{Y} \mu(E^{y}) \, d\nu
 ,\end{align*} что доказывает \eqref{equation2:theorem:principle_cavalieri}.

 Как обычно, доказательство будет структурным. Пусть $C$ --- семейство всех множеств $E \in \A \otimes \B$, удовлетворяющих пункту \ref{enum1:theorem:principle_cavalieri} и равенству \eqref{equation1:theorem:principle_cavalieri}. Докажем, что $C$  --- монотонный класс, содержащий алгебру, порожденную полукольцом $\A \times \B$ (см. замечание \ref{remark:semiring_to_algebra}, $X \in Y \in \A \times \B$). Тогда по лемме \ref{lemma:about_monotonous_class} о монотонном классе будет $C \supset \A \otimes \B \implies C = \A \otimes \B$, что завершит доказательство.

 \begin{itemize}
  \item Шаг 0. Докажем, что квадрат $E = A \times B \in C$, где $A \in \A$, $B \in \B$. Проверим пункт \ref{enum1:theorem:principle_cavalieri}: функция \begin{align*}
    \nu(E_x) = \begin{cases}
     \nu(B), \text{ если } x \in A  \\
     0, \text{ иначе }
    \end{cases} = \nu(B) \cdot \chi_A(x)
    \end{align*} измерима относительно $\A$, так как $A$ --- измеримое множество. Проверим равенство \eqref{equation1:theorem:principle_cavalieri}. По определению меры $\mu \times \nu$ имеем \begin{align*}
    (\mu \times \nu)(E) = \mu(A) \nu(B)
    .\end{align*} Вычислим интеграл из \eqref{equation1:theorem:principle_cavalieri}: \begin{align*}
    \int\limits_{X} \nu(E_x) \, d\mu &= \nu(B) \int\limits_{X} \chi_A(x) \, d\mu   = \nu(B) \mu(A) = (\mu \times \nu)(E).
   \end{align*} 

  \item Шаг 1. Докажем, что $E = \bigsqcup_{i=1}^{N} A_i \times B_i \in C $  для любого $N < \infty$, $A_i \in \A$, $B_i \in \B$. Для любого $x \in X$: \begin{align*}
    \nu(E_x) = \sum_{i=1}^{N} \nu((E_{i})_x)
   ,\end{align*} где $E_i = A_i \times B_i$. Каждая функция измерима, значит и сумма измерима. Пункт \ref{enum1:theorem:principle_cavalieri} проверен. Равенство \eqref{equation1:theorem:principle_cavalieri} верно по линейности интеграла:
   \begin{align*}
    \int\limits_{X} \nu(E_x) \, d\mu  &= \int\limits_{X} \sum_{i=1}^{N} \nu((E_i)_x) \, d\mu  = \sum_{i=1}^{N} \int\limits_{X} \nu((E_i)_x) \, d\mu  = \\
    &= \sum_{i=1}^{N} \nu(B_i) \int\limits_{X} \chi_{A_i}(x) \, d\mu  = \sum_{i=1}^{N} \nu(B_i) \mu(A_i) = (\mu \times \nu)(E)
   .\end{align*}

   В результате этого шага мы показали, что $C$ содержит алгебру из леммы \ref{lemma:semiring_to_algebra}, порождённую полукольцом $\A \times \B$.

  \item Шаг 2. Осталось доказать, что $C$  --- монотонный класс. Проверим по определению: пусть есть последовательности множеств $\{E_{n}\}_{n=1}^{\infty}\subset C $, $\{F_{n}\}_{n=1}^{\infty} \subset C $ такие, что $E_1 \subset E_2 \subset \ldots$ и $F_1 \supset F_2 \supset \ldots$.  Пусть $E = \bigcup_{n=1}^{\infty} E_n$ и $F = \bigcap_{n=1}^{\infty} F_n $. Проверим пункт \ref{enum1:theorem:principle_cavalieri} для $E$: \begin{align*}
    \nu (E_x) = \nu \left( \left( \bigcup_{n=1}^{\infty} E_n \right)_x \right) = \nu \left( \bigcup_{n=1}^{\infty} (E_n)_x \right) = \lim_{n \to \infty} \nu((E_{n})_x)
   \end{align*} по непрерывности меры сверху. Следовательно, функция $\nu(E_x)$ измерима как предел измеримых функций $\nu((E_n)_x)$. Аналогично проверим пункт \ref{enum1:theorem:principle_cavalieri} для $F$:
   \begin{align*}
    \nu(F_x) = \nu \left( \left( \bigcap_{n=1}^{\infty} F_n \right)_x \right) = \nu \left( \bigcap_{n=1}^{\infty} (F_n)_x \right) = \lim_{n \to \infty} \nu((F_{n})_x) 
   \end{align*} по непрерывности меры снизу для конечной меры $\nu$.

   Осталось проверить равенство \eqref{equation1:theorem:principle_cavalieri}. По непрерывности меры $\mu \times \nu$ сверху:
   \begin{align*}
    (\mu \times \nu)(E) &= \lim_{n \to \infty} (\mu \times \nu)(E_n) = \lim_{n \to \infty} \int\limits_{X} \nu((E_n)_x)  \, d\mu
   .\end{align*} Далее, так как неотрицательные измеримые функции $\nu((E_n)_x)$ возрастают, то по теореме \ref{theorem:levi} Леви
   \begin{align*}
    (\mu \times \nu)(E) &= \lim_{n \to \infty} \int\limits_{X} \nu((E_{n_x}))   \, d\mu = \int\limits_{X} \left( \lim_{n \to \infty} \nu((E_n)_x)  \right) \, d\mu = \int\limits_{X} \nu(E_x) \, d\mu
   .\end{align*} 

   Для $F$ всё то же самое, но нужно воспользоваться непрерывностью конечной меры $\mu \times \nu$ снизу, и вместо теоремы \ref{theorem:levi} Леви воспользоваться теоремой \ref{theorem:lebesgue-majoring-convergence} Лебега о мажорируемой сходимости:
   \begin{align*}
    (\mu \times \nu)(F) &= \lim_{n \to \infty} (\mu \times \nu)(F_n) = \lim_{n \to \infty} \int\limits_{X} \nu((F_n)_x) \, d\mu = \\
    &= \int\limits_{X} \left( \lim_{n \to \infty} \nu((F_n)_x)  \right) \, d\mu = \int\limits_{X} \nu(F_x) \, d\mu
   .\end{align*} Здесь константная мажоранта $g = \nu(X)$ с учётом конечности меры $\mu$ (см. замечание \ref{remrk:finite_measure:theorem_lebesgue_majoring_convergence}).
 \end{itemize}

 Таким образом, $C$ --- монотонный класс, содержащий алгебру, порождённую полукольцом $\A \times \B$. Значит, $C = \A \otimes \B$, что завершает доказательство принципа Кавальери.
\end{proof}

\begin{remrk*}
 {\color{red} Принцип Кавальери верен и для $\sigma$-конечных мер, но я уже не успеваю дописать это.}
\end{remrk*}

