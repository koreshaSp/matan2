% 2022.11.11 Lecture 9

\begin{remrk}
 Если $x$ --- точка Лебега функции $f \in L^{1}(X,\mu)$, то выполнено \begin{align*}
  \lim_{r \to 0} \frac{1}{\mu(B_r)} \int\limits_{B_r} f(y) \, d\mu(y) = f(x)
 .\end{align*} 
\end{remrk}
\begin{proof}
 Действительно, \begin{align*}
  &\lim_{r \to 0}  \frac{1}{\mu(B_r)} \int\limits_{B_r} f(y) \, d\mu (y) = f(x) \iff \\
  \iff &\lim_{r \to 0} \frac{1}{\mu(B_r)} \int\limits_{B_r} (f(y) - f(x)) \, d\mu  (y) = 0 \impliedby \\
  \impliedby &\lim_{r \to 0}  \frac{1}{\mu(B_r)} \int\limits_{B_r} \left| f(y) - f(x) \right| \, d\mu(y) = 0
, \end{align*} а последнее верно, так как $x$ --- точка Лебега.
\end{proof}

\begin{thm}[%
]
\label{theorem:almost_all_points_are_lebesgue_points}
 Пусть $(X, \rho)$ --- сепарабельное метрическое пространство, $\mu$ --- борелевская регулярная мера на $X$, обладающая условием удвоения: для любого шара $B$ и некоторого $c > 0$ выполнено $\mu(2B) \leqslant c \mu(B)$, причем $\mu(B) < \infty$.

 Пусть $f \in L^{1}(X,\mu)$. Тогда $\mu$-почти все точки $x \in X$ являются точками Лебега функции $f$.
\end{thm}
\begin{remrk}
 Пусть $X = \Z$, $\mu = \sum_{k \in \Z} \delta_{\left\{ k \right\}}$ --- считающая мера. Тогда \begin{align*}
  \frac{1}{\mu(B_r)} \int\limits_{B_r} f(y) \, d\mu  (y) = \frac{1}{N + 1} \sum_{\left| k - x \right| \leqslant \frac{N}{2}} f(k), \quad N = [r]
 .\end{align*} В действительности, при малых $r$ остается только одно слагаемое: $f(x)$. Так как внутри шара будет только одна целая точка.
\end{remrk}

\begin{proof}[\normalfont\textsc{Доказательство теоремы \ref{theorem:almost_all_points_are_lebesgue_points}}] 
 Простой случай: пусть $f \in C(X)$  --- непрерывна. Тогда при малых $r$ для любого $y \in B_r = B_r(x)$ \begin{align*}
  \left| f(x) - f(y) \right| \leqslant \delta(r)
 ,\end{align*} где $\delta(r) \to 0$. Поэтому, \begin{align*}
  \frac{1}{\mu(B_r)} \int\limits_{B_r} \left| f(x) - f(y) \right| \, d\mu(y) \leqslant \frac{1}{\mu(B_r)} \delta(r) \cdot \int\limits_{B_r} \, d\mu   = \delta(r) \to 0
 .\end{align*} Для непрерывных функций теорема доказана, более того, в случае непрерывной функции абсолютно все точки $X$ являются точками Лебега (а не $\mu$-почти все).

 Пусть теперь  $f \in L^{1}(X, \mu)$ --- любая. Можно считать, что $X$ ограниченное (например, $X = B_1$), так как утверждение теоремы локально. Возьмём любое $\eps > 0$ и рассмотрим множество <<плохих>> точек \begin{align*}
  A(\eps) = \left\{ x \in X \Mid \limsup_{r \to 0} \frac{1}{\mu(B_r)} \int\limits_{B_r}  \left| f(y) - f(x) \right| \, d\mu(y) > \eps   \right\}
 .\end{align*} Возьмём любую непрерывную функцию $g \in C(X)$. Сделаем <<$\frac{\eps}{3}$-приём>>: запишем неравенство \begin{align*}
 &\frac{1}{\mu(B_r)} \int\limits_{B_r} \left| f(x) - f(y) \right| \, d\mu(y) \leqslant \\
 \leqslant &\frac{1}{\mu(B_r)} \left( \int\limits_{B_r} \left| f(x) - g(x) \right| \, d\mu(y)  + \int\limits_{B_r} \left| g(x) - g(y) \right| \, d\mu(y)  + \int\limits_{B_r} \left| g(y) - f(y) \right| \, d\mu(y) \right) \\
 \leqslant &\left| f(x) - g(x) \right| + \frac{1}{\mu(B_r)} \int\limits_{B_r} \left| g(x) - g(y) \right| \, d\mu(y) + (M^{\ast}(g-f))(x)
 .\end{align*} Значит, множество $A(\eps)$ укладывается в три множества $A(\eps) \subset A_1(\eps) \cup A_2(\eps) \cup A_3(\eps)$, где \begin{align*}
 A_1(\eps) &= \left\{ x \in X \Mid \left| f(x) - g(x) \right| > \frac{\eps}{3} \right\}, \\
 A_2(\eps) &= \left\{ x \in X \Mid \limsup_{r \to 0} \frac{1}{\mu(B_r)} \int\limits_{B_r} \left| g(x) - g(y) \right| \, d\mu(y) > \frac{\eps}{3}   \right\}, \\
 A_3(\eps) &= \left\{ x \in X \Mid (M^{\ast}(f-g))(x) > \frac{\eps}{3} \right\}
 .\end{align*} Хотим доказать $\mu(A(\eps)) = 0$. Имеем:  \begin{align*}
  \mu(A(\eps)) \leqslant \mu(A_1(\eps)) + \mu(A_2(\eps)) + \mu(A_3(\eps))
 .\end{align*} Оценим сверху меры каждого из трёх множеств.

 Для $A_1(\eps)$ воспользуемся неравенством \ref{theorem:chebishev_inequality} Чебышева: \begin{align*}
  \mu(A_1(\eps)) \leqslant \frac{3}{\eps} \left\| f-g \right\|_{L^{1}(X,\mu)}
 .\end{align*}

 Для $A_2(\eps)$ заметим, что $A_2(\eps) = \varnothing$, потому что для непрерывной функции $g$, как мы уже доказали, все точки являются точками Лебега. Значит, $\mu(A_2(\eps)) = 0$.

  Для $A_3(\eps)$ воспользуемся теоремой \ref{theorem:hardy_littlewood} Харди-Литтлвуда: \begin{align*}
   \mu(A_3(\eps)) \leqslant \frac{3C}{\eps} \left\| f-g \right\|_{L^{1}(X,\mu)}
  ,\end{align*} где $C$ --- константа, зависящая лишь от меры.

  Соберём всё вместе:
  \begin{align*}
   \mu(A(\eps)) \leqslant \frac{3 + 3C}{\eps} \left\| f - g \right\|_{L^{1}(X,\mu)} = C' \left\| f-g \right\|_{L^{1}(X,\mu)}
   .\end{align*} Отметим, что это выполнено для любой непрерывной функции $g \in C(X)$, причём константа $C'$ не зависит от $g$. Так как по теореме \ref{theorem:continous_functions_dense_set_in_lebesgue_space} $C(X)$ плотно в  $L^{1}(X,\mu)$, то $\left\| f-g \right\|_{L^{1}(X,\mu)}$  можно сделать сколь угодно малым. Отсюда получаем $\mu(A(\eps)) = 0$ для любого  $\eps > 0$. Значит, по непрерывности конечной меры снизу имеем \begin{align*}
   \mu(A(0)) = \mu \left( \bigcap_{n=1}^{\infty} A\left(\frac{1}{n}\right) \right) = 0
  .\end{align*} Следовательно, при $\mu$-почти всех $x \in X$  имеем \begin{align*}
  \lim_{r \to 0} \frac{1}{\mu(B_r)} \int\limits_{B_r} \left| f(x) - f(y) \right| \, d\mu  (y) = 0
  .\end{align*} Теорема доказана.

\end{proof}
\begin{crly}
 Пусть $E \subset \R$ --- замкнутое подмножество. Тогда для $\lambda_1$-почти всех точек $x \in E$ существует предел \begin{align*}
  \lim_{\delta \to 0} \frac{\left| E \cap (x - \delta, x + \delta) \right|}{2\delta} = 1
 .\end{align*} Это ответ на мотивационную задачу \ref{problem:motivatinal_problem_good_points_in_closed_subset} в начале параграфа.
\end{crly}
\begin{proof}
 Можно считать $E \subset [-R, R]$, потому что всё локально. Мера $\lao \rvert_{\B([-R, R])}$  --- регулярная мера, удовлетворяющая условию удвоения на сепарабельном метрическом пространстве $[-R,R]$. По теореме \ref{theorem:almost_all_points_are_lebesgue_points} для измеримой функции $\chi_E$ и для $\lao$-почти всех $x \in [-R, R]$  имеем \begin{align*}
  \lim_{\delta \to 0} \frac{1}{\lao(B_{\delta})} \int\limits_{B_{\delta}} \chi_E \, d\mu  = \chi_E(x)
 .\end{align*} Если $x \in E$, то правая часть равна $1$, а левая часть равна \begin{align*}
  \frac{\left| E \cap B_{\delta} \right|}{\left| B_{\delta} \right|}
 .\end{align*}

 Здесь остался один недоказанный момент: \textbf{регулярность меры Лебега}. Пока это отложим.
\end{proof}

\section{Произведение мер. Теоремы Тонелли и Фубини.}

Есть очень важное геометрическое наблюдение, называемое \textit{принципом Кавальери}: площадь фигуры равна интегралу функции длин сечений по определённой оси. Мы будем изучать аналог этого утверждения в теории меры. Интересно, что дискретная переформулировка этого утверждения --- теорема Тонелли (задача 1 из первого листочка), утверждающая, что если есть ряд по двум параметрам, то не важно, в каком порядке переменных брать сумму. Более общая идея состоит в том, что не важно, по какой переменной сначала интегрировать, результат будет один и тот же.

Но прежде этого, мы вернемся на элементарный уровень, спустимся на землю с вершины теории меры.

\begin{lm}
 \label{lemma:product_of_semirings_is_semiring}
 Пусть  $\p$, $\mathcal{Q}$ --- полукольца. Тогда множество \begin{align*}
  \p \times \mathcal{Q} = \left\{ P \times Q \Mid P \in \p, Q \in \mathcal{Q} \right\}
 \end{align*}  --- полукольцо.
\end{lm}
\begin{proof}
 Проверим все три аксиомы полукольца.
 \begin{enumerate}
  \item $\varnothing \in \p \times \mathcal{Q}$, так как $\varnothing = \varnothing \times \varnothing$, и $\varnothing \in \p$, $\varnothing \in \mathcal{Q}$.
  \item Пусть $P_1 \times Q_1 \in \mathcal{P} \times \mathcal{Q}$ и  $P_2 \times Q_2 \in \p \times \mathcal{Q}$. Тогда \begin{align*}
    (P_1 \times Q_1) \cap (P_2 \times Q_2) = (P_1 \cap P_2) \times (Q_1 \cap Q_2) \in \p \times \mathcal{Q}
   ,\end{align*} так как $P_1 \cap P_2 \in \p$ и $Q_1 \cap Q_2 \in \mathcal{Q}$.
  \item Пусть $P_1 \times Q_1 \in \p \times \mathcal{Q}$ и $P_2 \times Q_2 \in \p \times \mathcal{Q}$. Удобно рассмотреть эти множества на плоскости в качестве сетки $3 \times 3$ (см. рис \ref{fig:semiring-product-set-difference}).
   \begin{figure}[ht]
    \centering
    \incfig{semiring-product-set-difference}
    \caption{Общее положение двух множеств из произведения полуколец.}
    \label{fig:semiring-product-set-difference}
   \end{figure}
   Рассмотрим множества $A = P_1 \setminus P_2, B = P_1 \cap P_2, C = P_2 \setminus P_1$ и $E = Q_1 \setminus Q_2, F = Q_1 \cap Q_2, G = Q_2 \setminus Q_1$. Каждое из множеств $A, B, C, E, F, G$ является дизъюнктным конечным объединением ячеек (в своём полукольце). Значит, каждая из девяти клеток является конечным объединением ячеек. А разность можно выразить явно через три ячейки сетки: \begin{align*}
    P_1 \times Q_1 \setminus P_2 \times Q_2 = (A \times E) \sqcup (B \times E) \sqcup (A \times F)
   .\end{align*}
 \end{enumerate}
\end{proof}
\begin{lm}
 \label{lemma:semiring_to_algebra}
 Пусть $\p$ --- полукольцо над $X$, причём $X \in \p$. Тогда множество \begin{align*}
  \A = \left\{ \bigsqcup_{k=1}^{N} P_k \Mid \{P_{k}\}_{k=1}^{N} \subset \p  \right\}
 \end{align*} конечных дизъюнктных объединений множеств полукольца  --- алгебра.
\end{lm}
\begin{proof}\
 \begin{enumerate}
  \item $\varnothing, X \in \A$, так как $\varnothing, X \in \p$.
  \item Пересечение двух элементов из $\A$ попадает в $\A$: \begin{align*}
   \bigsqcup_{k=1}^{N} P_k \cap \bigsqcup_{j=1}^{M} Q_j = \bigsqcup_{k=1}^{N} \bigsqcup_{j=1}^{M} (P_k \cap Q_j)
  .\end{align*} 
 \item Равенство \begin{align*}
  X \setminus \bigsqcup_{k=1}^{N} P_k = \bigsqcup_{j=1}^{M} Q_j
 \end{align*} для дополнения --- это первый пункт леммы \ref{lemma:about_subordinate_partition} о подчинённом разбиении, доказанной в начале семестра.
 \end{enumerate}
\end{proof}
\begin{df}
 Пусть $\A \subset 2^X$, $\B \subset 2^Y$ --- $\sigma$-алгебры. Введём обозначение для множества \begin{align*}
  \A \times \B = \left\{ A \times B \Mid A \in \A, B \in \B \right\}
 .\end{align*} Это множество является полукольцом над $X \times Y$, но в общем случае не является $\sigma$-алгеброй. Поэтому, \textit{тензорным произведением} $\sigma$-алгебр $\A$ и $\B$ называется множество $\A \otimes \B$  --- наименьшая $\sigma$-алгебра, содержащая полукольцо $\A \times \B$.

 Также введём удобные обозначения для \textit{сечений} множества $E \subset X \times Y$ по элементу $x \in X$:
\begin{align*}
 E_x &= \left\{ y \in Y \Mid (x, y) \in E \right\} \subset Y
\end{align*} и по элементу $y \in Y$:
\begin{align*}
 E^{y} &= \left\{ x \in X \Mid (x, y) \in E \right\} \subset X
.\end{align*} Для понимания см. рис. \ref{fig:projections}.

\begin{figure}[ht]
    \centering
    \incfig{projections}
    \caption{Сечение $E_x$.}
    \label{fig:projections}
\end{figure}

\end{df}

\begin{lm}
 \label{lemma:cross_sections_of_sigma_algebra_product_is_measurable}
 Если $E \in \A \otimes \B$, то для любых $x \in X$ и $y \in Y$ множество $E_x$ измеримо в $\B$, а множество $E^y$ измеримо в $\A$.
\end{lm}
\begin{proof}[\normalfont\textsc{Доказательство}]
 Заведём $\sigma$-алгебру 
\begin{align*}
\mathcal{C} = \left\{ E \subset \A \otimes \B \Mid E_x \in \B, E^{y} \in \A \right\}
.\end{align*} Достаточно понять, что $\mathcal{C}$ ---  $\sigma$ -алгебра:
\begin{enumerate}
 \item $\varnothing, X \times Y \in \mathcal{C}$.
 \item Если $\{E_{k}\}_{k=1}^{\infty} \subset \mathcal{C} $, то \begin{align*}
   \left( \bigcup_{k=1}^{\infty} E_k \right)_x = \bigcup_{k=1}^{\infty} \underbrace{(E_k)_x}_{\in \mathcal{B}} \implies \bigcup_{k=1}^{\infty} (E_k)_x \in \mathcal{B}
 .\end{align*} Для сечения по $y$ проверяется аналогично.
\item Если $E \in \mathcal{C}$, то
 \begin{align*}
  (E^{c})_x = Y \setminus E_x \in \B
 .\end{align*} Для сечения по $y$ проверяется аналогично.
\end{enumerate}

Значит, $\mathcal{C}$ --- $\sigma$-алгебра. Кроме того, $\mathcal{C} \supset \A \times \B$, так как если $A \in \A$ и $B \in \B$, то \begin{align*}
 (A \times B)_x = \begin{cases}
  B, \text{ если } x \in A \\
  \varnothing, \text{ иначе}
 \end{cases} \in \B
,\end{align*} и \begin{align*}
(A \times B)^{y} = \begin{cases}
 A, \text{ если } y \in B  \\
 \varnothing, \text{ иначе}
\end{cases} \in \A
,\end{align*} Раз $\A \otimes \B$ наименьшая, то $\mathcal{C} = \A \otimes \B$, что завершает доказательство леммы.
\end{proof}
\begin{lm}
 \label{lemma:continuation_of_measure_on_sigma_algebra_product}

 Пусть $\A$ и $\B$ --- $\sigma$-алгебры в $X$, $Y$ соответственно. Пусть $\mu, \nu$ --- меры на $\A$, $\B$ соответственно. Тогда функция \begin{align*}
  \varphi_0(A \times B) = \mu(A) \cdot \nu(B), \quad A \in \A, B \in \B
 \end{align*} является счётно-аддитивной на полукольце $\A \times \B$.
\end{lm}
\begin{proof}
 Пусть $A \times B = \bigsqcup_{i=1}^{\infty} A_i \times B_i$. Тогда для любых $x \in X$, $y \in Y$ выполняется тривиальное равенство: \begin{align*}
  \chi_A(x) \cdot \chi_B(y) = \chi_{A \times B}(x, y) &= \sum_{i=1}^{\infty} \chi_{A_i \times B_i}(x, y)  = \sum_{i=1}^{\infty} \chi_{A_i}(x) \cdot \chi_{B_i}(y) \\
  \implies \chi_A(x) \cdot \chi_B(y) &= \sum_{i=1}^{\infty} \chi_{A_i}(x) \cdot \chi_{B_i}(y)
  .\end{align*} Так как по предыдущей лемме \ref{lemma:cross_sections_of_sigma_algebra_product_is_measurable} функции вида $\chi_{\tilde A}$, $\chi_{\tilde B}$ для любых $\tilde A \in \A$, $\tilde B \in \B$ измеримы, то мы можем проинтегрировать равенство по переменной $x$ (и по мере $\mu$), зафиксировав при этом $y$:  \begin{align*}
  \chi_B(y) \int\limits_{X} \chi_A(x) \, d\mu(x)   &= \int\limits_{X} \left( \lim_{n \to \infty} f_n(x)  \right) \, d\mu(x) \implies \\
 \implies \chi_B(y) \mu(A) &= \int\limits_{X} \left( \lim_{n \to \infty} f_n(x)  \right) \, d\mu(x)
,\end{align*} где \begin{align*}
f_n(x) = \sum_{i=1}^{n} \chi_{A_i}(x) \cdot \chi_{B_i}(y)
.\end{align*} Заметим, что функции $f_n$ простые, неотрицательные, монотонно возрастают и ограничены в каждой точке $x \in X$. Поэтому, по теореме \ref{theorem:levi} Леви можно поменять интеграл и предел местами:
 \begin{align*}
  \chi_B(y) \mu(A) &= \lim_{n \to \infty} \int\limits_{X} f_n(x) \, d\mu(x) = \\
  &= \lim_{n \to \infty} \int\limits_{X} \left( \sum_{i=1}^{n} \chi_{A_i}(x) \cdot \chi_{B_i}(y) \right) \, d\mu(x) =    \\
  &= \lim_{n \to \infty} \sum_{i=1}^{n} \int\limits_{X} \chi_{A_i}(x) \cdot \chi_{B_i}(y) \, d\mu(x) =    \\
  &= \sum_{i=1}^{\infty} \chi_{B_i}(y) \mu(A_i) \implies \\
  \implies \chi_B(y) \mu(A) &= \sum_{i=1}^{\infty} \mu(A_i) \chi_{B_i}(y)
 .\end{align*} Теперь проинтегрируем по $y$ (по мере $\nu$), и получим то, что нужно: \begin{align*}
 \nu(B) \mu(A) = \sum_{i=1}^{\infty} \mu(A_i) \nu(B_i)
.\end{align*} 
\end{proof}

Далее у нас будет \textbf{соглашение}: в этом параграфе все меры $\sigma$-конечны.

\begin{df}
 \label{definition:measure_product}
 Пусть $\mu$, $\nu$ --- $\sigma$-конечные меры на $\sigma$-алгебрах $\A$, $\B$ соответственно. Тогда, \textit{произведением мер} $\mu$ и $\nu$ называется мера $\mu \times \nu$ на $\sigma$-алгебре $\A \otimes \B$, являющаяся стандартным продолжением функции множеств $\varphi_0$ на полукольце $\A \times \B$.
\end{df}
\begin{remrk*}
 Произведение мер $\mu \times \nu$ существует и единственно по теореме Каратеодори, так как меры $\mu$, $\nu$ $\sigma$-конечны.
\end{remrk*}

\begin{thm}[%
Принцип Кавальери]
\label{theorem:principle_cavalieri} 

Пусть $\mu$, $\nu$ --- конечные меры на  $\sigma$-алгебрах $\A$, $\B$  в $X$, $Y$ соответственно. Тогда для любого множества  $E \in \A \otimes \B$  выполнено следующее.
\begin{enumerate}
 \item Функция $x \mapsto \nu(E_x)$  измерима относительно $\A$.
 \label{enum1:theorem:principle_cavalieri}
 \item Функция $y \mapsto \mu(E^{y})$  измерима относительно $\B$.
 \label{enum2:theorem:principle_cavalieri}
\item Имеет место равенство \begin{align*}
   (\mu \times \nu)(E) = \int\limits_{X} \nu(E_x) \, d\mu = \int\limits_{Y} \mu(E^{y}) \, d\nu
 .\end{align*} 
 \label{enum3:theorem:principle_cavalieri}
\end{enumerate}
\end{thm}

\begin{df}
 Пусть $X$  --- множество. Набор подмножеств $C \subset 2^{X}$ называется \textit{монотонным классом}, если для любых $\{E_{k}\}_{k=1}^{\infty} \subset C $ таких, что $ E_1 \subset E_2 \subset \ldots \subset E_k \subset \ldots $ верно $\bigcup_{k=1}^{\infty} E_k \in C$, и для любых $\{F_{k}\}_{k=1}^{\infty} \subset C $  таких, что $ F_1 \supset F_2 \supset \ldots \supset F_k \supset \ldots $ верно $\bigcap_{k=1}^{\infty} F_k \in C$.
\end{df}
\begin{exmpl*}
 Любая $\sigma$-алгебра является монотонным классом (можно взять счётное объединение, пересечение).
\end{exmpl*}
\begin{remrk}
  Как и для $\sigma$-алгебр, существует понятие \textit{наименьшего монотонного класса}, содержащего семейство $S \subset 2^{X}$: это просто пересечение всех монотонных классов, содержащих $S$. Хотя бы один такой монотонный класс есть ($2^{X}$), а также пересечение любого набора монотонных классов --- монотонный класс.
\end{remrk}
\begin{lm}[%
о монотонном классе]
\label{lemma:about_monotonous_class}
 
Пусть $\A$ --- алгебра, $\B$ --- наименьшая $\sigma$-алгебра, содержащая $\A$, и $C$ --- наименьший монотонный класс, содержащий $\A$. Тогда $\B = C$.
\end{lm}
\begin{proof}
 Так как $\B$  --- монотонный класс, то $C \subset \B$ . Надо показать, что $C$  --- $\sigma$-алгебра, и тогда получим $\B \subset C \implies \B = C$.

 Для всякого множества $E \in C$ определим множество
\begin{align*}
C(E) = \left\{ F \in C \Mid F \setminus E, F \cap E, E \setminus F \in C \right\}
.\end{align*} Заметим следующие факты.
 \begin{enumerate}
  \item По построению $C(E) \subset C$.
   \label{С_E_cc_C:lemma:about_monotonous_class}
  \item $C(E)$ --- монотонный класс. Действительно, если $F_1 \subset F_2 \subset \ldots$ или $F_1 \supset F_2 \supset \ldots$, то каждые из множеств $F \setminus E$, $F \cap E$, $E \setminus F$ вложены, и их <<предел>> принадлежит $C$, из чего следует, что и <<предел>> $F_k$ принадлежит $C$.
  \item Если $E, F \in C$, то $E \in C(F) \iff F \in C(E)$, так как построение симметрично.
   \label{С_E_symmetry:lemma:about_monotonous_class}
  \item Если $E \in \A$, то $C = C(E)$. Докажем: пусть $F \in \A$. Тогда $F \in C(E)$, ведь $F \setminus E, E \cap F, E \setminus F \in \A \subset C$. Значит, $\A \subset C(E)$, и, поскольку $C$ --- наименьший монотонный класс, содержащий $\A$, то $C \subset C(E)$. Отсюда $C = C(E)$ по пункту \ref{С_E_cc_C:lemma:about_monotonous_class}.
   \label{С_E_eq_C_for_E_in_A:lemma:about_monotonous_class}
  \item Если $E \in C$, то $C = C(E)$. Докажем: возьмём любое множество $F \in \A$. Тогда по пункту \ref{С_E_symmetry:lemma:about_monotonous_class} $F \in C(E) \iff E \in C(F)$. Так как $F \in \A$, то по пункту \ref{С_E_eq_C_for_E_in_A:lemma:about_monotonous_class} $E \in C(F)$ выполнено для любого $E \in C$. Значит, для любых $E \in C$ и $F \in \A$ выполнено $F \in C(E)$. Это означает, что $\A \subset C(E) \implies C \subset C(E) \implies C = C(E)$.
  \item $C(X)$ --- это алгебра:
   \begin{enumerate}
    \item $\varnothing \in C(X)$, так как $\varnothing \in \A \subset C$, $\varnothing \setminus X = \varnothing \cap X = \varnothing \in C$, $X \setminus \varnothing = X \in \A \subset C$.
    \item Пусть $F_1, F_2 \in C(X)$. Тогда $F_1 \cap F_2 \in C$. Но по пункту \ref{С_E_eq_C_for_E_in_A:lemma:about_monotonous_class} $C = C(X)$, значит, $F_1 \cap F_2 \in C(X)$.
    \item Пусть $F \in C(X)$. Тогда $X \setminus F \in C = C(X)$.
   \end{enumerate}
   \label{С_E_is_algebra:lemma:about_monotonous_class}
 \end{enumerate}

 Мы поняли, что $C = C(E)$ для любого  $E \in C$, в частности для $E = X$. Но тогда, $C$  --- это алгебра, так как $C(X)$ --- это алгебра по пункту \ref{С_E_is_algebra:lemma:about_monotonous_class}. Осталось понять, что $C$ ---  $\sigma$-алгебра.

 Это верно, так как, если $\{A_{k}\}_{k=1}^{\infty} \subset C $, то \begin{align*}
  \bigcup_{k=1}^{\infty} A_k = \bigcup_{k=1}^{\infty} \left( \bigcup_{n=1}^{k} A_k \right) = \bigcup_{k=1}^{\infty} B_k \in C
 ,\end{align*} $B_k \in C$, так как $C$ --- алгебра и $B_k$ вложено возрастают. Значит, $C$ --- $\sigma$-алгебра, и лемма доказана.
\end{proof}

\begin{proof}[\normalfont\textsc{Доказательство принципа Кавальери \ref{theorem:principle_cavalieri}}]
 Пусть $C$ --- семейство всех множеств $E \in \A \otimes \B$, удовлетворяющих всем трём  свойствам принципа Кавальери.

 Докажем, что $C$  --- монотонный класс, содержащий алгебру, порожденную полукольцом $\A \times \B$. Тогда по лемме \ref{lemma:about_monotonous_class} будет $C = \A \otimes \B$, что завершит доказательство.

 Шаг 0. Докажем, что квадрат $E = A \times B \in C$, где $A \in \A$, $B \in \B$. Проверим пункт \ref{enum1:theorem:principle_cavalieri}: функция \begin{align*}
  \nu(E_x) = \begin{cases}
   \nu(B), \text{ если } x \in A  \\
   0, \text{ иначе }
  \end{cases} = \nu(B) \cdot \chi_A(x)
 \end{align*} измерима относительно $\A$, так как $A$ измеримое множество. Пункт \ref{enum2:theorem:principle_cavalieri} проверяется аналогично. Проверим пункт \ref{enum3:theorem:principle_cavalieri}. По определению меры $\mu \times \nu$ имеем \begin{align*}
 (\mu \times \nu)(E) = \mu(A) \nu(B)
.\end{align*} Возьмём и вычислим оба интеграла из пункта \ref{enum3:theorem:principle_cavalieri}: \begin{align*}
\int\limits_{X} \nu(E_x) \, d\mu &= \nu(B) \int\limits_{X} \chi_A(x) \, d\mu   = \nu(B) \mu(A) = (\mu \times \nu)(E), \\
\int\limits_{Y} \mu(E^{y}) \, d\nu &= \mu(A) \int\limits_{Y} \chi_B(y) \, d\nu = \mu(A) \nu(B)  = (\mu \times \nu)(E).
\end{align*} 

 Шаг 1. Докажем, что $\bigsqcup_{i=1}^{N} A_i \times B_i \in C $  для любого $N < \infty$, $A_i \in \A$, $B_i \in \B$. Для любого $x \in X$: \begin{align*}
  \nu(E_x) = \sum_{i=1}^{N} \nu((E_{i})_x)
 ,\end{align*} где $E_i = A_i \times B_i$. Каждая функция измерима, значит и сумма измерима. Пункт \ref{enum1:theorem:principle_cavalieri} проверем, пункт \ref{enum2:theorem:principle_cavalieri} проверяется аналогично. Третий пункт верен по линейности интеграла: \begin{align*}
 \int\limits_{X} \nu(E_x) \, d\mu  &= \int\limits_{X} \sum_{i=1}^{N} \nu((E_i)_x) \, d\mu  = \\
 &= \sum_{i=1}^{N} \int\limits_{X} \nu((E_i)_x) \, d\mu  = \\
 &= \sum_{i=1}^{N} \nu(B_i) \int\limits_{X} \chi_{A_i}(x) \, d\mu  = \\
 &= \sum_{i=1}^{N} \nu(B_i) \mu(A_i)= \\
 &= (\mu \times \nu)(E)
 .\end{align*} Для $\int_{Y} \mu(E^{y}) \, d\nu $ аналогично.

 В результате шага 1 мы показали, что $C$ содержит алгебру из леммы \ref{lemma:semiring_to_algebra}, содержащую полукольцо $\A \times \B$.

 Шаг 2. Осталось доказать, что $C$  --- монотонный класс. Проверим по определению. Возьмём $E_1 \subset E_2 \subset \ldots$  или $F_1 \supset F_2 \supset \ldots$  Пусть $E = \bigcup_{k=1}^{\infty} E_k$ и $F = \bigcap_{i=1}^{\infty} F_i $. Проверим пункт \ref{enum1:theorem:principle_cavalieri} для $E$: \begin{align*}
  \nu (E_x) = \nu \left( \left( \bigcup_{k=1}^{\infty} E_k \right)_x \right) = \lim_{k \to \infty} \nu((E_{k})_x)
  .\end{align*} Следовательно, функция $\nu(E_x)$ измерима как предел измеримых функций $\nu((E_k)_x)$. Проверим пункт \ref{enum1:theorem:principle_cavalieri} для $F$: всё точно так же, но надо воспользоваться конечностью меры: \begin{align*}
 \nu(F_x) = \nu \left( \bigcap_{i=1}^{\infty} F_i \right) = \lim_{i \to \infty} \nu((F_{i})_x) 
.\end{align*} Пункт \ref{enum2:theorem:principle_cavalieri} проверяется аналогично.

Осталось проверить пункт \ref{enum3:theorem:principle_cavalieri}. По теореме \ref{theorem:lebesgue-majoring-convergence} Лебега о мажорирующей сходимости можно записать \begin{align*}
  (\mu \times \nu)(E) &= \lim_{k \to \infty} (\mu \times \nu)(E_k) = \lim_{k \to \infty} \int\limits_{X} \nu((E_{k})_x) \, d\mu   (x) =  \int\limits_{X} \nu(E_x)  \, d\mu  (x)
 \end{align*} так как функции $f_k(x) = \nu((E_k)_x)$ возрастают и ограничены числом $\nu(Y)$. Для $F$ аналогично:

 \begin{align*}
  (\mu \times \nu)(F)  = \lim_{k \to \infty} (\mu \times \nu)(F_i) = \lim_{k \to \infty} \int\limits_{X} \nu(F_{ix}) \, d\mu = \int\limits_{X} \nu(F_x) \, d\mu(x)
 .\end{align*} Для $\int_{Y} \mu(E^{y}) \, d\nu  $ аналогично.

 Итак, мы показали, что $C$ --- монотонный класс. Тогда $C = \A \otimes \B$, и все множества хорошие. Принцип Кавальери доказан.
\end{proof}

