% 2022.11.11 Lecture 9

\begin{thm}[%
]
\label{theorem:almost_all_points_are_lebesgue_points}
 Пусть $(X, \rho)$ --- сепарабельное метрическое пространство, $\mu$ --- борелевская регулярная мера на $X$, обладающая условием удвоения: $\mu(2B) \leqslant c \mu(B)$  для любого шара $B$  и некоторого $c > 0$, причем $\mu(B) < \infty$  для любого шара $B$.

 Пусть $f \in L^{1}(X,\mu)$. Тогда $\mu$-почти все точки $x \in X$ являются точками Лебега: \begin{align}
  \label{equation:theorem_almost_all_points_are_lebesgue_points_1}
  \lim_{r \to 0} \frac{1}{\mu(B_r)} \int\limits_{B_r} \left| f(x) - f(y) \right| \, d\mu(y) = 0
 ,\end{align} где $B_r = \left\{ y \in X \Mid \rho(y, x) < r \right\}$. В частности, при $\mu$-почти всех $x \in X$  \begin{align}
  \label{equation:theorem_almost_all_points_are_lebesgue_points_2}
 \lim\limits_{r \to 0} \frac{1}{\mu(B_r)} \int\limits_{B_r} f(y) \, d\mu  (y) = f(x)
 .\end{align} 
\end{thm}
\begin{remrk}
 Пусть $X = \Z$, $\mu = \sum_{k \in \Z} \delta_{\left\{ k \right\}}$. Тогда \begin{align*}
  \frac{1}{\mu(B_r)} \int\limits_{B_r} f(y) \, d\mu  (y) = \frac{1}{N + 1} \sum_{\left| k - x \right| \leqslant \frac{N}{2}} f(k), \quad N = [r]
 .\end{align*} В действительности, при малых $r$ остается только одно слагаемое: $f(x)$. Так как внутри шара будет только одна целая точка.
\end{remrk}

\begin{proof}
 Простой случай: пусть $f \in C(X, \rho)$  --- непрерывна. Тогда при малых $r$  \begin{align*}
  \left| f(x) - f(y) \right| \leqslant \delta(r)
 ,\end{align*} где $\delta(r) \to 0$  для любого $y \in B_r = B_r(x)$ . Поэтому, \begin{align*}
  \frac{1}{\mu(B_r)} \int\limits_{B_r} \left| f(x) - f(y) \right| \, d\mu(y) \leqslant \frac{1}{\mu(B_r)} \delta(r) \cdot \int\limits_{B_r} \, d\mu   = \delta(r) \to 0
 .\end{align*} Для непрерывных доказали.

 Пусть теперь  $f \in L^{1}(X, \mu)$ --- любая. Можно считать, что $X = B_1$, то есть $X$ ограниченное, так как утверждение теоремы локально. Возьмём $\eps > 0$, возьмём любую непрерывную $g \in C(X)$ и рассмотрим множество \begin{align*}
  A(\eps) = \left\{ x \in X \Mid \varlimsup_{r \to 0} \frac{1}{\mu(B_r)} \int\limits_{B_r}  \left| f(y) - f(x) \right| \, d\mu(y) > \eps   \right\}
 .\end{align*} Есть неравенство ($\frac{\eps}{3}$-приём) \begin{align*}
  \frac{1}{\mu(B_r)} \int\limits_{B_r} \left| f(x) - f(y) \right| \, d\mu(y) \leqslant \\
   \leqslant \frac{1}{\mu(B_r)} \int\limits_{B_r} \left| f(x) - g(x) \right| \, d\mu(y)  + \frac{1}{\mu(B_r)} \int\limits_{B_r} \left| g(x) - g(y) \right| \, d\mu(y)  + \\ + \frac{1}{\mu(B_r)} \int\limits_{B_r} \left| g(y) - f(y) \right| \, d\mu(y) \leqslant \\
   \leqslant \left| f(x) - g(x) \right| + \frac{1}{\mu(B_r)} \int\limits_{B_r} \left| g(x) - g(y) \right| \, d\mu(y) + (M^{\ast}(g-f))(x)
 .\end{align*} Значит, указанное множество укладывается в три множества: \begin{align*}
 \left\{ x \in X \Mid \left| f(x) - g(x) \right| > \frac{\eps}{3} \right\} \cup \left\{ x \in X \Mid \varlimsup_{r \to 0} \frac{1}{\mu(B_r)} \int\limits_{B_r} \left| g(x) - g(y) \right| \, d\mu(y) > \frac{\eps}{3}   \right\} \cup \\ \cup \left\{ x \in X \Mid (M^{\ast}(g - f))(x) > \frac{\eps}{3} \right\} = \\
 = A_{1}(\eps) \cup A_2(\eps) \cup A_3(\eps)
 .\end{align*} Хотим доказать $\mu(A(\eps)) = 0$. Имеем:  \begin{align*}
  \mu(A(\eps)) <+ \mu(A_1(\eps)) + \mu(A_2(\eps)) + \mu(A_3(\eps))
 .\end{align*} Но $A_2(\eps) = \varnothing$  (для непрерывной $g$ уже доказано), значит $\mu(A_2(\eps)) = 0$ .

 По неравенству Чебышева:
  \begin{align*}
  \mu(A_1(\eps)) \leqslant \frac{\left\| f-g \right\|_{L^{1}(X,\mu)}}{\frac{\eps}{3}}. \end{align*} 

  Также, по теореме \ref{theorem:hardy_littlewood} Харди-Литтлвуда \begin{align*}
   \mu(A_3(\eps)) \leqslant C \frac{\left\| f - g \right\|_{L^{1}(X, \mu)}}{\frac{\eps}{3}}
  ,\end{align*} где $C$ --- константа, зависящая лишь от меры.

  Тогда \begin{align*}
   \mu(A(\eps)) \leqslant \left\| f - g \right\|_{L^{1}(X,\mu)} \cdot \left( \frac{1}{\frac{\eps}{3}} + \frac{C}{\frac{\eps}{3}} \right)
  ,\end{align*} и это выполнено для любой $g \in C(X)$ ! Константа не зависит от $g$. Теперь так как $C(X)$ плотно в  $L^{1}(X,\mu)$ , то $\left\| f-g \right\|_{L^{1}(X,\mu)}$  можно сделать сколь угодно малым. Отсюда получаем $\mu(A(\eps)) = 0$ для любого  $\eps > 0$. Значит,  \begin{align*}
   \mu \left( \bigcap_{n=1}^{\infty} A\left(\frac{1}{n}\right) \right) = 0
  .\end{align*} Следовательно, при $\mu$ -почти всех $x \in X$  имеем \begin{align*}
   \frac{1}{\mu(B_r)} \int\limits_{B_r} \left| f(x) - f(y) \right| \, d\mu  (y) \to 0
  .\end{align*} 

  Теперь докажем оставшееся утверждение \eqref{equation:theorem_almost_all_points_are_lebesgue_points_2}: оно эквивалентно \begin{align*}
   \frac{1}{\mu(B_r)} \int\limits_{B_R} f(y) \, d\mu  (y) - f(x) \to 0 \iff \\
   \iff \frac{1}{\mu(B_r)} \int\limits_{B_r} \left( f(y) - f(x) \right) \, d\mu (y) \to 0
  ,\end{align*} а это уж точно следует из \eqref{equation:theorem_almost_all_points_are_lebesgue_points_1}.
\end{proof}
\begin{crly}
 Пусть $E \subset \R$ --- замкнутое подмножество. Тогда для $\lambda_1$-почти всех $x \in E$ существует предел \begin{align*}
  \lim_{\delta \to 0} \frac{\left| E \cap (x - \delta, x + \delta) \right|}{2\delta} = 1
 .\end{align*} Это ответ на мотивационную задачу \ref{problem:motivatinal_problem_good_points_in_closed_subset} в начале параграфа.
\end{crly}
\begin{proof}
 Можно считать $E \subset [-R, R]$, потому что всё локально. Мера $\lao \rvert_{\B([-R, R])}$  --- регулярная мера, удовлетворяющая условию удвоения на сепарабельном метрическом пространстве $[-R,R]$. По предыдущей теоремы для функции $\chi_E$ для  $\lao$-почти всех $x \in [-R, R]$  имеем \begin{align*}
  \frac{1}{\lao(B_{\delta})} \int\limits_{B_{\delta}} \chi_E \, d\mu  \to \chi_E(x)
 \end{align*} при $\delta \to 0$ . Если $x \in E$ , то правая часть равна $1$, а левая часть равна \begin{align*}
  \frac{\left| E \cap B_{\delta} \right|}{\left| B_{\delta} \right|}
 .\end{align*}

 Здесь остался один недоказанный момент: \textbf{регулярность меры Лебега}. Пока это отложим.
\end{proof}

\section{Произведение мер. Теоремы Тонелли и Фубини.}

Есть геометрическое наблюдение, называемое \textit{принципом Кавальери}: объём тела равен интегралу функции площади сечения по выбранной оси. Мы будем изучать аналог этого утверждения в теории меры. Интересно, что дискретная переформулировка этого утверждения --- теорема Тонелли (задача 2 из первого листочка), утверждающая, что если есть ряд по двум параметрам, то не важно, в каком порядке переменных брать сумму. Более общее, идея в том, что не важно, по какой переменной сначала нужно интегрировать.

Но прежде этого, мы вернемся на элементарный уровень, спустимся на землю с вершины теории меры.

\begin{lm}
 \label{lemma:product_of_semirings_is_semiring}
 Пусть  $\p$, $\mathcal{Q}$ --- полукольца. Тогда множество \begin{align*}
  \p \times \mathcal{Q} = \left\{ P \times Q \Mid P \in \p, Q \in \mathcal{Q} \right\}
 \end{align*}  --- полукольцо.
\end{lm}
\begin{proof}
 Проверим все три аксиомы полукольца.
 \begin{enumerate}
  \item $\varnothing \in \p \times \mathcal{Q}$, так как $\varnothing = \varnothing \times \varnothing$, и $\varnothing \in \p$, $\varnothing \in \mathcal{Q}$.
  \item Пусть $P_1 \times Q_1 \in \mathcal{P} \times \mathcal{Q}$ и  $P_2 \times Q_2 \in \p \times \mathcal{Q}$. Тогда \begin{align*}
    (P_1 \times Q_1) \cap (P_2 \times Q_2) = (P_1 \cap P_2) \times (Q_1 \cap Q_2) \in \p \times \mathcal{Q}
   ,\end{align*} так как $P_1 \cap P_2 \in \p$ и $Q_1 \cap Q_2 \in \mathcal{Q}$.
  \item Пусть $P_1 \times Q_1 \in \p \times \mathcal{Q}$ и $P_2 \times Q_2 \in \p \times \mathcal{Q}$. Удобно рассмотреть эти множества на плоскости в качестве сетки $3 \times 3$ (см. рис \ref{fig:semiring-product-set-difference}).
   \begin{figure}[ht]
    \centering
    \incfig{semiring-product-set-difference}
    \caption{Общее положение двух множеств из произведения полуколец.}
    \label{fig:semiring-product-set-difference}
   \end{figure}
   Рассмотрим множества $A = P_1 \setminus P_2, B = P_1 \cap P_2, C = P_2 \setminus P_1$ и $E = Q_1 \setminus Q_2, F = Q_1 \cap Q_2, G = Q_2 \setminus Q_1$. Каждое из множеств $A, B, C, E, F, G$ является дизъюнктным конечным объединением ячеек (в своём полукольце). Значит, каждая из девяти клеток является конечным объединением ячеек. А разность можно выразить явно через три ячейки сетки: \begin{align*}
    P_1 \times Q_1 \setminus P_2 \times Q_2 = (A \times E) \sqcup (B \times E) \sqcup (A \times F)
   .\end{align*}
 \end{enumerate}
\end{proof}
\begin{lm}
 Если $\p$ --- полукольцо, то \begin{align*}
  \A = \left\{ \bigsqcup_{k=1}^{N} P_k \Mid P_k \in \p,\; 1 \leqslant N < \infty \right\} \cup \left\{ X \right\}
 \end{align*}  --- алгебра.
\end{lm}
\begin{proof}\
 \begin{enumerate}
  \item $\varnothing, X \in \A$.
  \item Для пересечения: \begin{align*}
   \bigsqcup_{k=1}^{N} P_k \cap \bigsqcup_{j=1}^{M} Q_j = \bigsqcup_{k=1}^{N} \bigsqcup_{j=1}^{M} (P_k \cap Q_j)
  .\end{align*} 
 \item Равенство \begin{align*}
  X \setminus \bigsqcup_{k=1}^{\infty} P_k = \bigsqcup_{j=1}^{M} Q_j
 \end{align*} мы уже доказывали в начале первого семестра.
 \end{enumerate}
\end{proof}
\begin{df}
 Пусть $\A \subset X$, $\B \subset Y$ --- $\sigma$-алгебры. Тогда \begin{align*}
  \A \times \B = \left\{ A \times B \Mid A \in \A, B \in \B \right\}
 ,\end{align*} а $\A \otimes \B$  --- наименьшая $\sigma$-алгебра, содержащая $\A \times \B$.

 Если $E \in X \times Y$, то 
\begin{align*}
 E_x &= \left\{ y \in Y \Mid (x, y) \in E \right\} \subset Y \\
 E^{y} &= \left\{ x \in X \Mid (x, y) \in E \right\} \subset X
.\end{align*}

\begin{figure}[ht]
    \centering
    \incfig{projections}
    \caption{Сечение $E_x$.}
    \label{fig:projections}
\end{figure}

\end{df}

\begin{lm}
 \label{lemma:cross_sections_of_sigma_algebra_product_is_measurable}
 Если $E \in \A \otimes \B$, то для любых $x \in X$ и $y \in Y$ множество $E_x$ измеримо в $\B$, а множество $E^y$ измеримо в $\A$.
\end{lm}
\begin{proof}[\normalfont\textsc{Доказательство}]
 Заведём $\sigma$-алгебру 
\begin{align*}
\mathcal{C} = \left\{ E \subset \A \otimes \B \Mid E_x \in \B, E^{y} \in \A \right\}
.\end{align*} Достаточно понять, что $\mathcal{C}$ ---  $\sigma$ -алгебра:
\begin{enumerate}
 \item $\varnothing, X \times Y \in \mathcal{C}$.
 \item Если $E_k \in \mathcal{C}$, то \begin{align*}
   \left( \bigcup_{k=1}^{\infty} E_k \right)_x = \bigcup_{k=1}^{\infty} \underbrace{(E_k)_x}_{\in \mathcal{B}} \implies \bigcup_{k=1}^{\infty} (E_k)_x \in \mathcal{B}
 .\end{align*} Для сечения по $y$ проверяется аналогично.
\item Аналогично.
\end{enumerate}

Значит, $\mathcal{C}$ --- $\sigma$-алгебра. Кроме того, $\mathcal{C} \supset \A \times \B$, так как если $A \in \A$ и $B \in \B$, то \begin{align*}
 (A \times B)_x = \begin{cases}
  B, \text{ если } x \in A \\
  \varnothing, \text{ иначе.}
 \end{cases} \in \B
\end{align*} и \begin{align*}
(A \times B)^{y} = \begin{cases}
 A, \text{ если } y \in B  \\
 \varnothing, \text{ иначе.}
\end{cases} \in \A
\end{align*} Раз $\A \otimes \B$ наименьшая, то $\mathcal{C} = \A \otimes \B$, что завершает доказательство леммы.
\end{proof}
\begin{lm}
 \label{lemma:continuation_of_measure_on_sigma_algebra_product}

 Пусть $\A$ и $\B$ --- $\sigma$-алгебры в $X$, $Y$ соответственно. Пусть $\mu, \nu$ --- меры на $\A$, $\B$ соответственно. Заведём \begin{align*}
  \varphi_0(A \times B) = \mu(A) \cdot \nu(B), \quad A \in \A, B \in \B
 .\end{align*} Тогда $\varphi_0$ --- счётно-аддитивная функция на полукольце $\A \times \B$.
\end{lm}
\begin{proof}
 Пусть $A \times B = \bigsqcup_{i=1}^{\infty} A_i \times B_i$. Тогда \begin{align*}
  \chi_A(x) \cdot \chi_B(y) = \chi_{A \times B}(x, y) = \sum_{i=1}^{\infty} \chi_{A_i \times B_i}(x, y)  = \sum_{i=1}^{\infty} \chi_{A_i}(x) \cdot \chi_{B_i}(y)
 .\end{align*} Проинтегрируем всё по мере $\mu$, по переменной $x$:  \begin{align*}
 &\chi_B(y) \int\limits_{X} \chi_A(x) \, d\mu(x)   = \int\limits_{X} \left( \lim_{n \to \infty} \underbrace{\sum_{i=1}^{n} \chi_{A_i}(x) \cdot \chi_{B_i}(y)}_{f_n \geqslant 0,\; f_n \text{ возрастает }}  \right) \, d\mu(x) \implies \\
 \implies & \chi_B(y) \mu(A) = \lim_{n \to \infty} \sum_{i=1}^{n} \int\limits_{X} \chi_{A_i}(x) \chi_{B_i}(y) \, d\mu(x)
\end{align*} по теореме \ref{theorem:levi} Леви. Продолжим: \begin{align*}
\chi_B(y) \mu(A) = \sum_{i=1}^{\infty} \mu(A_i) \chi_{B_i}(y)
.\end{align*} Проинтегрируем ещё раз по $y$ по мере $\nu$ получаем то, что нужно: \begin{align*}
 \nu(B) \mu(A) = \sum_{i=1}^{\infty} \mu(A_i) \nu(B_i)
.\end{align*} 
\end{proof}

Далее у нас будет \textbf{соглашение}: в этом параграфе все меры $\sigma$-конечны.

\begin{df}
 Если $\mu$, $\nu$ --- меры на $\sigma$-алгебрах $\A$, $\B$ соответственно, то $\varphi = \mu \times \nu$ --- это стандартное продолжение функции множеств $\varphi_0$. Оно существует и единственно по теореме Каратеодори, потому что меры $\sigma$-конечны.
\end{df}

\begin{thm}[%
Принцип Кавальери]
\label{theorem:principle_cavalieri} 

Пусть $\mu$, $\nu$ --- конечные меры на  $\sigma$ -алгебрах $\A$, $\B$  в $X$, $Y$ соответственно. Тогда для любого множества  $E \in \A \otimes \B$  выполнено
\begin{enumerate}
 \item Отображение $x \mapsto \nu(E_x)$  измеримо относительно $\A$.
 \item Отображение $y \mapsto \mu(E^{y})$  измеримо относительно $\B$.
 \item \begin{align*}
   (\mu \times \nu)(E) = \int\limits_{X} \nu(E_x) \, d\mu = \int\limits_{Y} \mu(E^{y}) \, d\nu
 .\end{align*} 
\end{enumerate}
\end{thm}

\begin{df}
 Пусть $X$  --- множество, набор подмножеств $C \subset 2^{X}$ называется \textit{ монотонным классом }, если для любых $E_k \in C$  таких, что \begin{align*}
  E_1 \subset E_2 \subset \ldots \subset E_k \subset \ldots
 \end{align*} верно $\bigcup_{k=1}^{\infty} E_k \in C$ и для любых $\{E_{k}\}_{k=1}^{\infty} \subset C $  таких, что 


 Исправить!!!
\end{df}
\begin{lm}[%
о монотонном классе]
\label{lemma:about_monotonous_class}
 
Пусть $\A$ --- алгебра, $\B$ --- наименьшая $\sigma$-алгебра, содержащая $\A$, и $C$ --- наименьший монотонный класс, содержащий $\A$. Тогда $\B = C$.
\end{lm}
\begin{proof}
 Так как $\B$  --- монотонный класс, то $C \subset \B$ . Надо показать, что $C$  --- $\sigma$ -алгебра, и тогда получим $\B \subset C$ .

 Для $E \in C$ определим 
\begin{align*}
C(E) = \left\{ F \in C \Mid F \setminus E, F \cap E, E \setminus F \in C \right\}
.\end{align*} Заметим, что 
 \begin{enumerate}
  \item $C(E)$ --- монотонный класс по определению, и $C(E) \subset C$ по построению.
  \item Если $E, F \in C$, то $E \in C(F) \iff F \in C(E)$, так как построение симметрично.
  \item Если $E \in \A$, то $C \subset C(E)$ (и тогда $C = C(E)$), так как если $F \in \A$, то $F \in C(E)$, ведь $E,F \in \A$, и отсюда $F \setminus E, E \cap F, E \setminus F \in \A \subset C$. Значит, $\A \subset C(E)$, и следовательно $C \subset C(E)$.
  \item Если $E \in C$, то $C \subset C(E)$ (и тогда $C = C(E)$), так как если $F \in \A$, то $F \in C(E) \iff E \in C(F)$. По предыдущему пункту $E \in C(F)$ выполнено для любого $E$. Это означает, что $\A \subset C(E) \implies C \subset C(E)$.
 \end{enumerate}

 Мы поняли, что $C = C(E)$ для любого  $E \in C$. Но тогда, $C$  --- это алгебра, так как $C(E)$ --- это алгебра. Осталось понять, что $C$ ---  $\sigma$ -алгебра.

 Это верно, так как, если $\{A_{k}\}_{k=1}^{\infty} \subset C $, то \begin{align*}
  \bigcup_{k=1}^{\infty} A_k = \bigcup_{k=1}^{\infty} \left( \bigcup_{n=1}^{k} A_k \right) = \bigcup_{k=1}^{\infty} B_k \in C
 ,\end{align*} $B_k \in C$, так как $C$ --- алгебра и $B_k$ вложено возрастают. Значит, $C$ --- $\sigma$-алгебра, и лемма доказана.
\end{proof}

\begin{proof}[\normalfont\textsc{Доказательство принципа Кавальери \ref{theorem:principle_cavalieri}}]
 Пусть $C$ --- семейство всех множеств $E \in \A \otimes \B$, удовлетворяющих всем трём  свойствам принципа Кавальери.

 Докажем, что $C$  --- монотонный класс, содержащий алгебру, порожденную полукольцом $\A \times \B$. Тогда по лемме \ref{lemma:about_monotonous_class} будет $C = \A \otimes \B$, что завершит доказательство.

 Шаг 1. Докажем, что $\bigsqcup_{i=1}^{N} A_i \times B_i \in C $  для любого $N < \infty$, $A_i \in \A$, $B_i \in \B$. Пусть сначала $N = 1$, $E = A \times B$. Тогда \begin{align*}
  \nu(E_x) = \begin{cases}
   \nu(B), \text{ если } x \in A  \\
   0, \text{ иначе }
  \end{cases} = \nu(B) \cdot \chi_A(x)
 \end{align*} --- измеримая функция относительно $\A$, так как $A$  измеримое множество. Аналогично, $y \mapsto \mu(E^{y})$ измеримо относительно $\B$. Проверим третье свойство:  \begin{align*}
 (\mu \times \nu)(A \times B) = \mu(A) \nu(B) 
 \end{align*} по определению. Проинтегрируем: \begin{align*}
  \int\limits_{X} \nu(E_x) \, d\mu   = \nu(B) \int\limits_{X} \chi_A(x) \, d\mu   = \mu(A) \nu(B)
 .\end{align*} Аналогично для $Y$.

 Пусть теперь $N$ произвольное, $E = \bigsqcup_{i=1}^{\infty} A_i \times B_i $. Тогда \begin{align*}
  \nu(E_x) = \sum_{i=1}^{n} \nu(E_{i,x})
 ,\end{align*} где $E_i = A_i \times B_i$ . Каждая функция измерима, значит и сумма измерима. Первые два пункта проверили. Третий пункт верный по линейности.

 Шаг 2. Докажем, что $C$  --- монотонный класс. Проверим по определению. Возьмём $E_1 \subset E_2 \subset \ldots$  или $F_1 \supset F_2 \supset \ldots$  Пусть $E = \bigcup_{k=1}^{\infty} E_k$ и $F = \bigcap_{i=1}^{\infty} F_i $. Проверим (1) для $E$: \begin{align*}
  \nu (E_x) = \nu \left( \left( \bigcup_{k=1}^{\infty} E_k \right)_x \right) = \lim_{k \to \infty} \underbrace{\nu(E_{kx})}_{\text{ изм. }} 
 \end{align*} Следовательно, весь предел измерим, и всё ок. Проверим (1) для $F$: всё точно так же, но надо воспользоваться конечностью меры. \begin{align*}
 \nu(F_x) = \nu \left( \bigcap_{i=1}^{\infty} F_i \right) = \lim_{i \to \infty} \nu(F_{ix}) 
 .\end{align*} (2) проверяется симметрично.

 Осталось проверить само равенство (3). Пусть $E, E_k, F, F_i$ --- как раньше. Тогда \begin{align*}
  (\mu \times \nu)(E) &= \lim_{k \to \infty} (\mu \times \nu)(E_k) = \lim_{k \to \infty} \int\limits_{X} \nu(E_{kx}) \, d\mu   (x) =  \int\limits_{X} \nu(E_x)  \, d\mu  (x)
 \end{align*} так как функции $f_k$ возрастают и ограничены числом $\nu(Y)$. По теореме \ref{theorem:lebesgue-majoring-convergence} Лебега о мажорирующей сходимости. Для $F$ аналогично:

 \begin{align*}
  (\mu \times \nu)(F)  = \lim_{k \to \infty} (\mu \times \nu)(F_i) = \lim_{k \to \infty} \int\limits_{X} \nu(F_{ix}) \, d\mu = \int\limits_{X} \nu(F_x) \, d\mu(x)
 .\end{align*} Всё, мы показали, что $C$ --- монотонный класс. Тогда $C = \A \otimes \B$, и все множества хорошие. Принцип Кавальери доказан.
\end{proof}
\begin{df}
 $\lambda_n = \lambda_1 \times \ldots \times \lao$ --- \textit{мера Лебега} в $\R^{n}$.
\end{df}

