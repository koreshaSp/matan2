% 2022.12.02 Lecture 14

\section{Дифференциальные формы}

Тизер: сегодня мы приблизимся к понимаю, что же \textit{на самом деле} значит запись $\int_{a}^{b} f(x) \, dx $.

Для начала вспомним определение полилинейных отображений с первого курса. Они были как на анализе, так и на алгебре.

\begin{df}
 Функция $L \colon\, \underbrace{\R^{n} \times \ldots \times \R^{n}}_{q} \to \R  $ называется \textit{полилинейной формой порядка} $q$ в $\R^{n}$, если она линейна по каждому аргументу. То есть, для любого $i \in [q]$ и для любых векторов $v_1, \ldots, v_q \in \R^{n}$, $\tilde v_i \in \R^{n}$ верно
 \begin{align*}
  L(v_1, \ldots, v_i + \tilde v_i, \ldots, v_q) = L(v_1, \ldots, v_i, \ldots, v_q) + L(v_1, \ldots, \tilde v_i, \ldots, v_q)
 ,\end{align*} а также для любого $\alpha \in \R$ верно
 \begin{align*}
  L(v_1, \ldots, \alpha v_i, \ldots, v_q) = \alpha L(v_1, \ldots, v_i, \ldots, v_q)
 .\end{align*} 
\end{df}
\begin{prop*}
 Множество полилинейных форм порядка $q$ в $\R^{n}$ образует векторное пространство, обозначаемое $\mathcal{L}(\underbrace{\R^{n}, \ldots, \R^{n}}_{q}, \R)$.
\end{prop*}
Более того, пространство полилинейных форм нормированное и полное (то есть банахово) --- это было доказано на первом курсе.

\begin{claim*}
 Полилинейная форма полностью определяется набором значений на базисных векторах.
\end{claim*}
\begin{proof}[\normalfont\textsc{Доказательство}]
 Действительно, пусть есть полилинейная форма  $L$ порядка  $q$  в $\R^{n}$ . Пусть $e_1, \ldots, e_n$  --- стандартный базис в $\R^{n}$ . Пусть есть произвольный набор векторов $v_1, \ldots, v_q \in \R^{n}$ . Разложим каждый из них по базису:
 \begin{align*}
  v_i = \sum_{j=1}^{n} x_{i,j} e_j, \quad i \in [q]
 .\end{align*} Запишем значения формы на этом наборе векторов и воспользуемся полилинейностью:
 \begin{align}
  \label{equation:formula_polylinear_form}
  L(v_1, \ldots, v_q) = \sum_{j \in [n]^{q}} x_{1,j_1} \cdot \ldots \cdot x_{q, j_q} L(e_{j_1}, \ldots, e_{j_q})
 .\end{align} Таким образом, $L$ полностью определяется числами $L(e_{j_1}, \ldots, e_{j_q})$ по всем наборам $j \in [n]^{q}$.  Наоборот, легко проверить, что если по произвольным числам $L(e_{j_1}, \ldots, e_{j_q})$ задать функцию по формуле \eqref{equation:formula_polylinear_form}, то функция окажется полилинейной формой порядка $q$ в $\R^{n}$.
\end{proof}
\begin{crly*}
 Пространство полилинейных форм порядка $q$ в $\R^{n}$ имеет размерность $n^{q}$. Кроме того, набор форм вида
 \begin{align*}
  L(v_1,\ldots,v_q) = x_{1,j_1} \cdot \ldots \cdot x_{q,j_q}
 \end{align*} по всем $j \in [n]^{q}$ задаёт базис в этом пространстве.
\end{crly*}

\begin{exmpl*}
 Линейная форма порядка $0$ --- это просто число из $\R$. Линейная форма порядка $1$ --- это линейное отображение $\R^{n} \to \R$, также называемое \textit{линейной формой} ю
\end{exmpl*}
\begin{df}
 Полилинейная форма $L$ порядка $q$ в $\R^{n}$ называется \textit{кососимметрической}, если
 \begin{align*}
  L(v_1, \ldots, v_i, \ldots, v_j, \ldots, v_q) = -L(v_1, \ldots, v_j, \ldots, v_i, \ldots, v_q)
 \end{align*} для любых векторов $v_1, \ldots, v_q \in \R^{n}$ и любых $1 \leqslant i < j \leqslant q$.
\end{df}
\begin{prop*}
 Для полилинейных кососимметрических форм верно:
 \begin{itemize}
  \item $ L(v_1, \ldots, v_q) = \sgn \sigma \cdot L(v_{\sigma(1)}, \ldots, v_{\sigma(q)}) $ для любой перестановки $\sigma \in S_{q}$.
\item $L(v_1, \ldots, v_i, \ldots, v_i, \ldots, v_q) = 0$.
 \end{itemize}
\end{prop*}
\begin{exmpl}\
\begin{itemize}
 \item $dx_i$ --- полилинейная кососимметрическая форма порядка $1$ в $\R^{n}$,
  \begin{align*}
   ( dx_i )(h) = h_i
  ,\end{align*} где $h = (h_1, \ldots, h_n)$.
 \item $dx_i \otimes dx_j$ --- форма порядка $2$ в $\R^{n}$:
  \begin{align*}
   (dx_i \otimes dx_j)(h, g) = (dx_i)(h) \cdot (dx_j)(g) = h_i \cdot g_j
  .\end{align*} Форма полилинейная, но не кососимметрическая!
 \item Если $\omega_1$, $\omega_2$ --- полилинейные формы порядка $k_1$ и $k_2$, то
  \begin{align*}
   (\omega_1 \otimes \omega_2) (h_1, \ldots, h_{k_1}, g_1, \ldots, g_{k_2}) = \omega_1(h_1, \ldots, h_{k_1}) \cdot \omega_2(g_1, \ldots, g_{k_2})
  \end{align*}  --- полилинейная форма порядка $k_1 + k_2$.
\end{itemize} 
\end{exmpl}
\begin{exmpl}
 Важный пример полилинейной кососимметрической формы --- это \textit{определитель}:

 
\begin{align*}
 \left( dx_{i_1} \land \ldots \land dx_{i_k} \right)(h_1, \ldots, h_k) = \det \begin{pmatrix}
  h_{1,i_1} & \ldots & h_{1,i_k} \\
  \vdots & \ddots & \vdots \\
  h_{k,i_{1}} & \ldots & h_{k,i_k}
 \end{pmatrix},
\end{align*} где $h_{s,i_j}$ --- координата с номером $i_j$ вектора $h_s \in \R^{n}$. Это полилинейная кососимметрическая форма порядка $k$.

Кососиммтричность обосновывается тем, что при транспозиции двух соседних векторов знак определителя (и, следовательно, знак формы) меняется.

\end{exmpl}
\begin{thm}[%
из алгебры]
\label{theoream:determinant_is_only_polylinear_antisymmetric_form}
 Любая полилинейная кососимметрическая форма порядка $k$ в $\R^{n}$ имеет вид
 \begin{align*}
  \omega = \sum_{i \in [n]^{k}} a_{i_1,\ldots,i_k} \cdot dx_{i_1} \land \ldots \land dx_{i_k}
 ,\end{align*} где $c_{i_1,\ldots,i_k} \in \R$.
\end{thm}
\begin{proof}[\normalfont\textsc{Доказательство для $k=1$}]
 Действительно, в этому случае $\omega$ --- линейная форма в  $\R^{n}$ , значит у $\omega$  есть матрица
 \begin{align*}
  (\omega)(h) = \begin{pmatrix}
   a_1 & \ldots & a_n
  \end{pmatrix} \begin{pmatrix}
   h_1 \\
   \vdots \\
   h_n
  \end{pmatrix} = \sum_{i=1}^{n} a_i h_i = \sum_{i=1}^{n} a_i dx_i(h)
 .\end{align*}

 В общем случае $k \geqslant 2$ идея такая:
 \begin{align*}
  \omega = \sum_{\alpha \in [n]^{k}} c_{\alpha} dx_{i_1} \otimes \ldots \otimes dx_{i_k}
 ,\end{align*}  где $\alpha = (i_1, \ldots, i_k)$. Затем
 \begin{align*}
  \frac{1}{k!} \sum_{\omega} \omega(\sigma(h)) \,\mathrm{sgn}\;\omega = \omega = \sum_{\alpha} c_{\alpha} \underbrace{\frac{1}{k!} \sum_{\sigma} (dx_{\sigma(i_1) \otimes \ldots \otimes dx_{\sigma(i_k)}})\eps_{\alpha}}_{dx_{i_1} \land \ldots \land dx_{i_k}}
 \end{align*} 
\end{proof}
\begin{remrk}
 Если $i_1, \ldots, i_k$ --- набор индексов такой, что существуют $i_s, i_m$ такие, что $i_s = i_m$, но $s \neq m$, то $dx_{i_1} \land \ldots \land dx_{i_k} = 0$.
\end{remrk}
\begin{proof}
 Можно считать, что $s$ и $m$ соседние. Тогда, меняя их местами, с одной стороны меняется знак, а с другой --- ничего не происходит.

 Ещё можно сказать, что в матрице есть два одинаковых столбца, следовательно, определитель равен нулю.
\end{proof}
\begin{crly}
 В частности, единственная полилинейная кососимметрическая форма порядка $k > n$ на $\R^{n}$ --- это $0$.

 А единственная полилинейная кососимметрическая форма порядка $n$ на $\R^{n}$ --- это $C \det$ --- определитель, умноженный на константу $C \in \R$.
\end{crly}
\begin{df}[%
внешнее умножение дифференциальных форм]
Пусть есть две полилинейные кососимметрические формы $dx_{i_1} \land \ldots \land dx_{i_k}$  и $dx_{j_1} \land \ldots \land dx_{j_m}$. Тогда их \textit{внешним произведением} называется
 \begin{align*}
  (dx_{i_1} \land \ldots \land dx_{i_k}) \land (dx_{j_1} \land \ldots \land dx_{j_m}) = dx_{i_1} \land \ldots \land dx_{i_k} \land dx_{j_1} \land \ldots \land dx_{j_m}
 .\end{align*} Для общих форм определение продолжается по линейности.

 Например,
 \begin{align*}
  (dx \land dy + 2dy \land dx) \land dx &= dx \land dy \land dx + 2 dy \land dz \land dx = \\
  &= 2 dx \land dy \land dz
 .\end{align*} 
\end{df}

\begin{df}
 \textit{Внешняя форма} в области $\Omega \subset \R^{n}$ порядка $k$ -- это отображение из $\Omega$ в пространство полилинейных кососимметрических форм порядка $k$.
\end{df}
\begin{exmpl}
 $x dx \land dy + (x + z) dy \land dz$ --- пример внешней формы порядка $2$ на $\Omega = \R^{3}$. В точке $(1,2,3)$ эта внешняя форма равна $ dx \land dy + 4 dy \land dz $.
\end{exmpl}

В этой науке принято \textbf{соглашение}: функции --- это формы порядка $0$.

\begin{exmpl}
 Пусть $f \colon\, \Omega \to \R  $  --- гладкая функция, где $\Omega \subset \R^{n}$ --- область. Тогда дифференциал функции $f$
 \begin{align*}
  df = \frac{\partial f}{\partial x_1} dx_1 + \ldots + \frac{\partial f}{\partial x_n} dx_n
 \end{align*} --- это внешняя форма порядка порядка $1$.
\end{exmpl}
\begin{df}
 Общая внешняя форма 
\begin{align*}
 \omega = \sum_{i \in [n]^{k}} a_{i_1,\ldots,i_k}(x) \cdot dx_{i_1} \land \ldots \land dx_{i_k}
\end{align*} называется \textit{$C^{m}(\Omega)$-гладкой}, если все $a_{i_1, \ldots, i_k} \in C^{m}(\Omega, \R)$.

Это определение эквивалентно тому, что внешняя форма как отображение гладкая (по Фреше, то есть по определению из второго семестра).
\end{df}
\begin{df}
 \textit{Дифференциал} гладкой внешней формы $\omega$ равен
 \begin{align*}
   d \omega = \sum_{i \in [n]^{k}} (d a_{i_1,\ldots,i_k}) \land dx_{i_1} \land \ldots \land dx_{i_k}
 .\end{align*} 
\end{df}
\begin{exmpl*}
 \begin{align*}
  d((x^{2} + y^{2})dx + z^{2}dz) &= (2x dx + 2y dy) \land dx + 2z dz \land dz = \\
  &= -2y dz \land dy
 .\end{align*} 
\end{exmpl*}

\begin{df}
 Пусть $(\Omega, \Phi, S)$ --- поверхность порядка $k$, где $\Omega \subset \R^{k}$ --- область, $S \subset \R^{n}$, $\Phi \colon\; \Omega \to S$ --- гладкая биекция, $\mathrm{rk}\;\Phi(x) = k$ для любого $x \in \Omega$. Пусть
 \begin{align*}
  \Phi(x_1, \ldots, x_k) = \begin{pmatrix}
   \Phi_1(x_1, \ldots, x_k) \\
   \vdots \\
   \Phi_n(x_1, \ldots, x_k) \\
  \end{pmatrix}
 .\end{align*} 

 Пусть на $\R^{n}$ (или в окрестности $S$) задана внешняя форма
 \begin{align*}
  \omega = \sum_{i \in [n]^{k}} a_{i_1,\ldots,i_k}(x) \cdot dx_{i_1} \land \ldots \land dx_{i_k}
 .\end{align*} Тогда \textit{перенос} этой формы в $\Omega$ задаётся формулой
 \begin{align*}
  \omega^{\ast}(y) = \sum_{i \in [n]^{k}} a_{i_1,\ldots,i_k} (\Phi(y)) d\Phi_{i_1} \land \ldots \land d \Phi_{i_k}
 \end{align*}  --- это внешняя форма порядка $k$ в $\Omega$.
\end{df}
\begin{figure}[ht]
    \centering
    \incfig{surface_outer_form}
    \caption{surface_outer_form}
    \label{fig:surface_outer_form}
\end{figure}
\begin{exmpl}
 Найдём перенос формы
 \begin{align*}
  \omega = x dy \land dz
 \end{align*} со сферы в прямоугольник $(0, 2\pi) \times (-\pi / 2, \pi / 2) = \Omega$. Здесь $R$ --- фиксированный радиус сферы,
 \begin{align*}
  \Phi(\varphi, \psi) = \begin{pmatrix}
   R \cos \varphi \cos \psi \\
   R \sin \varphi \cos \psi \\
   R \sin \psi
  \end{pmatrix}
 .\end{align*} Правда, образ $\Phi$ --- это не вся сфера, а сфера минус множество меры нуль, но мы на это забьём.

 Перенесём:
 \begin{align*}
  \omega^{\ast} &= R \cos \varphi \cos \psi (d (R \sin \varphi \cos \psi)) \land d(R \sin \psi) = \\
  &= R^{3} \cos\varphi \cos\psi (\cos\varphi \cos \psi d\varphi - \sin\varphi\sin\psi d\psi) \land (\cos\psi d \psi) = \\
  &= R^{3} \cos^{2}\varphi\cos^{3}\psi d\varphi d\psi
 .\end{align*} 
\end{exmpl}

\begin{df}
 Пусть $\omega = f(x) \cdot dx_1 \land \ldots \land dx_k$  --- внешняя форма порядка $k$ на области $\Omega \subset \R^{k}$, где $f \in C(\Omega)$ . Тогда \textit{интегралом} внешней формы $\omega$ по измеримому подмножеству $E \subset \Omega$ называется
 \begin{align*}
  \int\limits_{E} \omega = \int\limits_{E} f(x) \, dx_1 dx_2 \ldots dx_k
 .\end{align*} 

 Для форм $\omega$  порядка $k$  на поверхности $(\Omega, \Phi, S)$  интеграл $\omega$
  \begin{align*}
  \int\limits_{\Phi(E)} \omega = \int\limits_{E} \omega^{\ast}
 ,\end{align*} где $E \subset \Omega \subset \R^{k}$. Правая часть уже определена, так как $\omega^{\ast}$ --- форма порядка $k$.
\end{df}

\begin{claim}
 Пусть $\omega = f_1(x) dx_1 + f_2(x) dx_2 + \ldots + f_{n}(x) dx_n$  --- внешняя форма порядка $1$  на гладкой кривой $\gamma = (\gamma_1, \ldots, \gamma_n )$ , $\gamma = \gamma(t)$ , $t \in (a,b)$ , $\left\| \gamma'(t) \right\|_{\R^{n}} \neq 0$ .

 Тогда
 \begin{align*}
  \int\limits_{\gamma} \omega = \int\limits_{a}^{b} \left( f_1(\gamma(t)) \gamma_1'(t) + \ldots f_n(\gamma(t)) \gamma_n'(t) \right) \, dt
 \end{align*} 
\end{claim}
\begin{proof}
 Найдём перенос внешней формы в интервал $(a, b)$:
  \begin{align*}
  \omega^{\ast} = \sum_{i=1}^{n} f_i(\gamma(t)) d \gamma_i(t) = \sum_{i=1}^{n} f_i(\gamma(t)) \gamma_i'(t) dt
 .\end{align*} По определению
 \begin{align*}
  \int\limits_{(a,b)} \omega^{\ast} = \int\limits_{1}^{b} \sum_{i=1}^{n} f_i(\gamma(t))\gamma_i'(t)  \, dt
 .\end{align*} Проверено.
\end{proof}

\textbf{Физический смысл}: если $F$ --- постоянное силовое поле в $\R^{n}$, то работа $F$ вдоль вектора $v \in \R^{n}$ --- это скалярное произведение $(F, v)$.

Если
\begin{align*}
 F = \begin{pmatrix}
  f_1(x) \\
  \vdots \\
  f_n(x)
 \end{pmatrix}
\end{align*}  --- переменное силовое поле в $\R^{n}$ , а $\gamma$  --- это кривая в $\R^{n}$ , $\gamma = \gamma(t)$ , $t \in (a, b)$ . Тогда работа $F$  вдоль $\gamma$ --- это интеграл
\begin{align*}
 A_{F,\gamma} = \int\limits_{\gamma} (F,v) \, dS
,\end{align*} где $v(t)$ --- единичный касательный вектор к кривой $\gamma$  в точке $\gamma(t)$.

\begin{figure}[ht]
    \centering
    \incfig{work_under_a_curve}
    \caption{work_under_a_curve}
    \label{fig:work_under_a_curve}
\end{figure}

\begin{claim}
 \begin{align*}
  A_{F,\gamma} = \int\limits_{\gamma} \omega
 ,\end{align*} где
 \begin{align*}
  \omega = f_1 dx_1 + \ldots f_n dx_n
 \end{align*} --- форма работы поля $F$.
\end{claim}
\begin{proof}
 Вычислим единичный касательный вектор к кривой $\gamma$ в точке $\gamma(t)$:
 \begin{align*}
  v(t) = \frac{\gamma'(t)}{\left\| \gamma'(t) \right\|}
 ,\end{align*} где $\gamma'(t) = (\gamma_1'(t), \ldots, \gamma_n'(t))^{\top}$ . Посмотрим на интеграл
 \begin{align*}
  \int\limits_{\gamma} (F,v) \, dS &= \int\limits_{a}^{b} (F(\gamma(t)), v(t)) \left\| \gamma' (t) \right\| \, dt = \\ 
  &= \int\limits_{a}^{b} \frac{\sum_{i=1}^{n} f_i(\gamma(t))\gamma_i'(t)}{\left\| \gamma'(t) \right\|} \left\| \gamma'(t) \right\| \, dt = \\
  &= \int\limits_{a}^{b} \sum_{i=1}^{n} f_i(\gamma(t))\gamma'_i(t) \, dt 
 .\end{align*} Правая часть равна
 \begin{align*}
  \int\limits_{\gamma} \omega = \int\limits_{(a,b)} \omega^{\ast} = \int\limits_{a}^{b} \sum_{i=1}^{n} f_i(\gamma(t))\gamma'_i(t) \, dt
 .\end{align*} 
\end{proof}
\begin{df}
 Пусть $\omega$ --- внешняя форма порядка $k$ с непрерывными коэффициентами в области $\Omega$. $\omega$ называется \textit{точной} формой, если существует внешняя форма $\kappa$ порядка $k - 1$ такая, что $d \kappa = \omega$.

 Для $1$-форм это означает, что существует функция $F \colon\; \Omega \to \R$ ($0$-форма) такая, что
 \begin{align*}
  dF = f_1 dx_1 + \ldots f_n dx_n = \omega
 .\end{align*} 
\end{df}
\begin{claim}
 \label{claim:integral_of_potential_curve_is_difference_on_ends}
 Пусть $\omega$  --- точная $1$-форма в $\Omega$. Пусть $\gamma$  --- гладкая кривая в $\Omega$. Тогда
 \begin{align*}
  \int\limits_{\gamma} \omega = F(\gamma(b)) - F(\gamma(a))
 ,\end{align*} где $F$ --- функция, такая, что $dF = \omega$.

 В частности, интеграл $\int_{\gamma} \omega  $  зависит лишь от начала и конца $\gamma$.
\end{claim}

\textbf{Физический смысл}: для потенциального силового поля работа вдоль кривой зависит только от начала и конца и кривой.

\begin{proof}[\normalfont\textsc{Доказательство утверждения \ref{claim:integral_of_potential_curve_is_difference_on_ends}}]
 Посчитаем
 \begin{align*}
  \int\limits_{\gamma} \omega &= \int\limits_{a}^{b} \left( f_1(\gamma(t)) \gamma'_1(t) + \ldots + f_n(\gamma(t))\gamma_n'(t) \right) \, dt = \\
  &= \int\limits_{a}^{b} \left( F(\gamma(t)) \right)'_t \, dt
 \end{align*}  в силу того, что
 \begin{align*}
  F'_1 = f_1, \ldots, F'_n = f_n
 .\end{align*} Тогда по формуле Ньютона-Лейбница
 \begin{align*}
  \int\limits_{\gamma} \omega = F(\gamma(b)) - F(\gamma(a))
 .\end{align*} 
\end{proof}

\begin{claim}
 \begin{align*}
  \int\limits_{\gamma} \omega = - \int\limits_{\tilde \gamma} \omega
 ,\end{align*} где $\tilde \gamma$ --- это кривая $\gamma$, проходимая в обратном направлении.
\end{claim}
\begin{proof}
  Пусть $\gamma \colon\, (a,b) \to \R^{n}$ . Тогда $\tilde \gamma \colon\, (a, b) \to \R^{n}$  задаётся формулой
 \begin{align*}
  \tilde \gamma (t) = \gamma(b - t + a)
 .\end{align*} Тогда
 \begin{align*}
  \int\limits_{\tilde \gamma} \omega &= \int\limits_{a}^{b} \left( f_1(\tilde \gamma(t))\tilde \gamma'_1(t) + \ldots + f_n(\tilde \gamma(t))\tilde \gamma'_n(t) \right) \, dt = \\
  &= -\int\limits_{a}^{b} \sum_{i=1}^{n} f_i(\gamma(b-t-a))\gamma'_i(s) \rvert_{s=b-t+a} \, dt = \\
  &= \begin{bmatrix}
   u = b - t + a & du = -dt
  \end{bmatrix} = \\
  &= - \int\limits_{a}^{b} \sum_{i=1}^{n} f_i(\gamma(u))\gamma'_i(u) \, du = \\
  &= -\int\limits_{\gamma} \omega
 .\end{align*} 
\end{proof}

Когда мы писали $\int_{a}^{b} f(x) \, dx $ --- это интеграл дифференциальной формы $f(x) dx$  по кривой --- отрезку $[a,b]$ в направлении от $a$ к $b$.

\begin{df}
 $\gamma_1, \gamma_2 \colon\, (a, b) \to \R^{n}$ --- \textit{эквивалентные кривые}, если существует гладкая биекция $s \colon\, (a,b) \to (a,b) $  такая, что $\gamma_2(t) = \gamma_1(s(t))$.
\end{df}

\begin{remrk*}
 Геометрически $\gamma_1$ и $\gamma_2$ --- одна и та же кривая, но по-разному параметризованная.
\end{remrk*}
\begin{claim}
 Для $1$-форм с непрерывными коэффициентами интеграл по гладкой кривой не зависит от выбора параметризации.  

 То есть
 \begin{align*}
  \int\limits_{\gamma_1} \omega = \int\limits_{\gamma_2} \omega
 ,\end{align*} если $\gamma_1$ эквивалентна $\gamma_2$.
\end{claim}
\begin{proof}
 \begin{align*}
  \int\limits_{\gamma_1} \omega &= \int\limits_{a}^{b} \sum_{i=1}^{n} f(\gamma_1(t))\gamma_{1,i}'(t) \, dt. \\
  \int\limits_{\gamma_2} \omega &= \int\limits_{a}^{b} \sum_{i=1}^{n} f(\gamma_2(t))\gamma_{1,i}'(t) \, dt =  \\
  &= \int\limits_{a}^{b} \sum_{i=1}^{n} f(\gamma_1(s(t))) \gamma_{1,i}'(y) \rvert_{y = s(t)} \cdot s'(t) \, dt = \\
  &= \begin{bmatrix}
   u = s(t) & du = s'(t) dt
  \end{bmatrix} = \\
  &= \int\limits_{a}^{b} \sum_{i=1}^{n} f(\gamma(u)) \gamma_{1,i}'(u) \, du = \\
  &= \int\limits_{\gamma_1} \omega
 .\end{align*}
\end{proof}
\begin{thm}[%
Стокса]
\label{theorem:stox}
 Пусть $M$  --- гладкое компактное ориентируемое многообразие с краем $\partial M$,  $\omega$  --- $C^{1}$ -гладкая дифференциальная форма на $M$; ориентации $M$  и $\partial M$  согласованы; порядок $\omega$ равен размерности $M$  минус $1$ и не меньше $0$.

 Тогда
 \begin{align}
  \label{equation:formula_stox}
  \int\limits_{\partial M} \omega = \int\limits_{M} d \omega
 .\end{align} 
\end{thm}

Формула \eqref{equation:formula_stox} называется \textit{формулой Стокса}.

Дадим несколько комментариев к теореме \ref{theorem:stox}. $\partial M$ --- это многообразие размерности  $\dim M - 1$.  Соответственно, $d \omega$  имеет порядок, равный $\dim M$.

Многообразия размерности $1$  ориентируемы и ориентация соответствует выбору начала и конца кривой.

\begin{df}
 Многообразие размерности $2$ в $\R^{3}$ называется \textit{оринтеируемым}, если на нём есть непрерывное поле единичных нормалей.
\end{df}
\begin{exmpl*}
 Сфера $S^{2}$ ориентируема, а лента Мёбиуса неориентируема.
\end{exmpl*}
\begin{exmpl}
 Пример многообразия с краем: полусфера
\begin{figure}[ht]
    \centering
    \incfig{halfsphere}
    \caption{halfsphere}
    \label{fig:halfsphere}
\end{figure}
\end{exmpl}

Частный случай формулы Стокса: $M = [a,b]$,  $\partial M = \left\{ a,b \right\}$, форма $\omega = F$. Тогда
 \begin{align*}
  \int\limits_{M} d\omega &= \int\limits_{M} d F \\
 \int\limits_{\partial M} F &= F(b) - F(a)
.\end{align*} Мы получили формулу Ньютона-Лейбница.
