% 2022.10.10 lecture 6

\begin{thm}[лемма Фату]
 \label{theorem:lemma_fatoo}
 Пусть $(X, \A, \mu)$ --- пространство с мерой, есть функции $f_n \geqslant 0$ --- измеримые и неотрицательные. Тогда \begin{align*}
  \int\limits_{X} \varliminf_{n\to \infty} f_n \, d\mu   \leqslant \varliminf_{n\to \infty} \int\limits_{X} f_n \, d\mu  
 .\end{align*} 
\end{thm}
\begin{proof}
 Пусть $g_n = \inf_{k \geqslant n} f_k$ --- измеримые неотрицательные функции такие, что $g_{n+1} \geqslant g_n$ для любого $n$. Значит, по теореме Леви \ref{theorem:levi} выполнено \begin{align*}
  \int\limits_{X} \lim_{n \to \infty} g_n \, d\mu  = \lim_{n \to \infty} \int\limits_{X} g_n \, d\mu   
 .\end{align*} Левая часть по определению равна \begin{align*}
 \mathrm{LHS} = \int\limits_{X} \varliminf_{n\to \infty} f_n \, d\mu  
 .\end{align*} Правую часть можно оценить так: \begin{align*}
 \mathrm{RHS} \leqslant \lim_{n \to \infty} \inf_{k\geqslant n} \int\limits_{X} f_n \, d\mu  = \varliminf_{n \to \infty} \int\limits_{X} f_n \, d\mu  
.\end{align*} Здесь мы воспользовались неравенством $\int_{X} \inf_{k\geqslant n} f_k \, d\mu  \leqslant \inf_{k \geqslant n} \int_{X} f_k \, d\mu  $, так как $\inf_{k \geqslant n} f_k \leqslant f_j$ для любого $j \leqslant n$.
\end{proof}
\begin{remrk}
 Неравенство в лемме Фату \ref{theorem:lemma_fatoo} бывает строгим: $\mu = \lao$, $f_n = \chi_{[n, n + 1]}$, $X = \R$. Тогда \begin{align*}
  \mathrm{LHS} &= \int\limits_{X} \varliminf_{n \to \infty} f_n \, d\mu  = \int\limits_{X} 0 \, d\mu  = 0, \\
  \mathrm{RHS} &= \varliminf_{n \to \infty} \int\limits_{X} f_n \, d\mu  = \varliminf_{n \to \infty} 1 = 1
 .\end{align*} 
\end{remrk}
\begin{exmpl}
 Занумеруем рациональные числа $\{x_{k}\}_{k=1}^{\infty} = \Q $. Посмотрим на последовательность $\{a_{k}\}_{k=1}^{\infty} $ такую, что \begin{align*}
  \sum_{k=1}^{\infty} \left| a_k \right| < \infty
 .\end{align*} Утверждается, что ряд \begin{align*}
 \sum_{k=1}^{\infty} \frac{a_k}{\sqrt{\left| x - x_k \right|}}
 \end{align*} сходится при почти всех $x \in \R$.
\end{exmpl}
\begin{proof}
 Пусть $c > 0$. Рассмотрим ряд \begin{align*}
  \int\limits_{-c}^{c} \sum_{k=1}^{\infty} \frac{\left| a_k \right|}{\sqrt{\left| x - x_k \right|}} \, dx &= \int\limits_{-c}^{c} \varliminf_{N \to \infty} \sum_{k=1}^{N} \frac{\left| a_k \right|}{\sqrt{\left| x - x_k \right|}}  \, dx  \leqslant \\ &\leqslant \varliminf_{N \to \infty} \sum_{k=1}^{N}  \int\limits_{-c}^{c} \frac{\left| a_k \right|}{\sqrt{\left| x - x_k \right|}} \, dx \leqslant \\ &\leqslant \sum_{k=1}^{\infty} \left| a_k \right| \cdot \delta(c)
 ,\end{align*} где \begin{align*}
 \delta(c) = \sup_{a \in \R} \int\limits_{-c}^{c} \frac{1}{\sqrt{\left| x - a \right|}}  \, dx  < \infty
 .\end{align*} Значит, $f$ конечна почти всюду на $(-c, c)$ для любого  $c > 0$. Следовательно, $f$ конечна почти всюду. Следовательно, ряд абсолютно сходится при почти всех $x \in \R$.
\end{proof}
\section{Интегральные неравенства}
\begin{thm}[неравенство Чебышева]
 Пусть $(X, \A, \mu)$ --- пространство с мерой, функция $f$ измерима, а число $\eps > 0$. Тогда \begin{align*}
  \mu \left( \left\{ x \mid \left| f(x) \right| \geqslant \eps \right\} \right) \leqslant \frac{1}{\eps} \int\limits_{X} \left| f \right| \, d\mu  
 \end{align*} 
\end{thm}
\begin{proof}
 Пусть $E_{\eps}$ --- то множество. Тогда \begin{align*}
  \mu(E_{\eps}) = \int\limits_{X} \chi_{E_{\eps}} \, d\mu  \leqslant \int\limits_{X} \frac{1}{\eps} \left| f \right| \chi_{E_{\eps}} \, d\mu  \leqslant \frac{1}{\eps} \int\limits_{X} \left| f \right| \, d\mu  
 .\end{align*} 
\end{proof}
\begin{thm}[неравенство Юнга]
 Пусть $a, b \in \R$, $p,q \geqslant 1$ такие, что $\frac{1}{p} + \frac{1}{q} = 1$. Тогда \begin{align*}
  \left| a \right| \cdot \left| b \right| \leqslant \frac{\left| a \right|^{p}}{p} + \frac{\left| b \right|^{q}}{q}
 .\end{align*} 
\end{thm}
\begin{proof}
 Функция $\varphi = \log x$ вогнута на $(0, +\infty)$, так как $\varphi'' = -\frac{1}{x^{2}} < 0$ на $(0, +\infty)$. Тогда \begin{align*}
  \log \left( \frac{1}{p}\left| \tilde a \right| + \frac{1}{q} \left| b \right| \right) \geqslant \frac{1}{p}\log \left| \tilde a \right| + \frac{1}{q}\left| q \right| = \log \left| \tilde a \right|^{\frac{1}{p}} \cdot \left| b \right|^{\frac{1}{q}}
 .\end{align*} Тогда \begin{align*}
  \left| \tilde a \right|^{\frac{1}{p}} \cdot \left| \tilde b \right|^{\frac{1}{q}} \leqslant \frac{1}{p}\left| \tilde a \right| + \frac{1}{q} \left| \tilde b \right|
 .\end{align*} Осталось взять $\tilde a = \left| a \right|^{p}$ и $\tilde b = \left| b \right|^{q}$.
\end{proof}
\begin{thm}[неравенство Гёльдера]
 \begin{align*}
  \int\limits_{X} \left| fg \right| \, d\mu  \leqslant \left( \int\limits_{X} \left| f \right|^{p} \, d\mu   \right)^{\frac{1}{p}} \cdot \left( \int\limits_{X} \left| g \right|^{q} \, d\mu   \right)^{\frac{1}{q}}
 ,\end{align*}  где $1 \leqslant p, q \leqslant \infty$, $\frac{1}{p} + \frac{1}{q} = 1$.
\end{thm}
\begin{proof}
 Обозначим \begin{align*}
  A &= \left( \int\limits_{X} \left| f \right|^{p} \, d\mu   \right)^{\frac{1}{p}} \\
  B &= \left( \int\limits_{X} \left| g \right|^{q} \, d\mu   \right)^{\frac{1}{q}}
 \end{align*} 
 Пусть $1 < p, q < \infty$ и $A, B < \infty$. Тогда неравенство равносильно тому, что \begin{align*}
  \int\limits_{X} \frac{\left| f \right|}{A} \cdot \frac{\left| g \right|}{B} \, d\mu   \leqslant 1
 .\end{align*} Применим неравенство Юнга под интегралом: \begin{align*}
 \int\limits_{X} \frac{\left| f \right|}{A} \cdot \frac{\left| g \right|}{B} \, d\mu \leqslant \frac{1}{p}\int\limits_{X} \frac{\left| f \right|^{p}}{A^{p}} \, d\mu   + \frac{1}{q} \int\limits_{X} \frac{\left| g \right|^{p}}{B^{q}} \, d\mu   = \frac{1}{p} + \frac{1}{q} = 1
 .\end{align*} При $p = 1$ и $q = \infty$ неравенство Гёльдера интерпретируем как \begin{align*}
 \int\limits_{X} \left| f g\right| \, d\mu  \leqslant \left( \int\limits_{X} \left| f \right| \, d\mu   \right) \cdot \sup_{x \in X} \left| g(x) \right|
.\end{align*} Это очевидно, так как $\left| fg \right| \leqslant \left| f \right| \cdot \sup_{x \in X} \left| g \right|$. Интегрируя неравенство получаем то, что нужно.

Пусть $A = 0$. Это означает, что  $f$ почти всюду равна нулю. Тогда и в левой части, и в правой части будет ноль. (для $B = 0$ то же самое)
\end{proof}
\begin{thm}[неравенство Минковского]
 \begin{align*}
  \left( \int\limits_{X} \left| f+g \right|^{p} \, d\mu \right)^{\frac{1}{p}} \leqslant \left( \int\limits_{X} \left| f \right|^{p} \, d\mu   \right)^{\frac{1}{p}} + \left( \int\limits_{X} \left| g \right|^{p} \, d\mu   \right)^{\frac{1}{p}}
 ,\end{align*} где $1 \leqslant p \leqslant \infty$.
\end{thm}
Это неравенство очень похоже на неравенство треугольника. В каком только пространстве?
\begin{proof}
 Пусть $p > 1$. Тогда
 \begin{align*}
  \int\limits_{X} \left| f+g \right|^{p} \, d\mu  &\leqslant \int\limits_{X} \left| f \right| \cdot \left| f + g \right|^{p - 1} \, d\mu  + \int\limits_{X} \left| g \right| \cdot \left| f+g \right|^{p-1} \, d\mu  \leqslant \\
  &\leqslant \left( \int\limits_{X} \left| f \right|^{p} \, d\mu   \right)^{\frac{1}{p}} \cdot \underbrace{\left( \int\limits_{X} \left| f+g \right|^{(p-1)q} \, d\mu   \right)^{\frac{1}{q}}}_{C} + \left( \int\limits_{X} \left| g \right|^{p} \, d\mu \right)^{\frac{1}{p}}\underbrace{\left( \int\limits_{X} \left| f+g \right|^{(p-1)q} \, d\mu   \right)^{\frac{1}{q}}}_{C} \leqslant \\
  &\leqslant \left( \left( \int\limits_{X} \left| f \right|^{p} \, d\mu   \right)^{\frac{1}{p}} + \left( \int\limits_{X} \left| g \right|^{p} \, d\mu   \right)^{\frac{1}{p}} \right)C
 .\end{align*} Но $(p-1)q = p$ (легко проверить). Поделим неравенство на  $C$: \begin{align*}
  \left( \int\limits_{X} \left| f+g \right|^{p} \, d\mu   \right)^{1 - \frac{1}{q}} \leqslant \left( \int\limits_{X} \left| f \right|^{p} \, d\mu   \right)^{\frac{1}{p}} + \left( \int\limits_{X} \left| g \right|^{p} \, d\mu   \right)^{\frac{1}{p}}
 ,\end{align*} и $1 - \frac{1}{q} = \frac{1}{p}$.

 При $p = 1$ все тривиально  \begin{align*}
  \int\limits_{X} \left| f + g \right| \, d\mu   \leqslant \int\limits_{X} \left| f \right| \, d\mu  + \int\limits_{X} \left| g \right| \, d\mu  
 .\end{align*}  При $p=\infty$ \begin{align*}
  \sup \left| f+g \right| \leqslant \sup \left| f \right| + \sup \left| g \right|
 ,\end{align*} что тоже верно.
\end{proof}
\begin{exmpl}
 Пусть $a_k, b_k \in \CC$, $k \geqslant 1$. Докажем неравенство \begin{align*}
  \sum_{k=1}^{\infty} \left| a_{k} b_k \right| \leqslant \left( \sum_{k=1}^{\infty} \left| a_k \right|^{p} \right)^{\frac{1}{p}} \cdot \left( \sum_{k=1}^{\infty} \left| b_k \right|^{q} \right)^{\frac{1}{q}}
 \end{align*} 
\end{exmpl}
\begin{proof}[\normalfont\textsc{Доказательство средствами теории меры}]
 Возьмём дискретную меру \begin{align*}
  \mu = \sum_{k=1}^{\infty} \delta_{\left\{ k \right\}}
 \end{align*} --- считающая мера на $\N = X$. Тогда для любой функции  $f$ верно \begin{align*}
  \int\limits_{X} \left| f \right| \, d\mu = \sum_{k=1}^{\infty} \int\limits_{\left\{ k \right\}} \left| f \right| \, d\mu  = \sum_{k=1}^{\infty} \left| f(k) \right|
 .\end{align*} Вернёмся к неравенству. Рассмотрим функции $f \colon\, k \mapsto a_k $ и $g_k \colon\, k \mapsto b_k$. Тогда по неравенству Гёльдера \begin{align*}
 &\int\limits_{\N} \left| fg \right| \, d\mu \leqslant \left( \int\limits_{\N} \left| f \right|^{p} \, d\mu   \right) ^{\frac{1}{p}} \cdot \left( \int\limits_{\N} \left| g \right|^{q} \, d\mu   \right)^{\frac{1}{q}} \iff \\ 
 \iff & \sum_{k=1}^{\infty} \left| a_k b_k \right| \leqslant \left( \sum_{k=1}^{\infty} \left| a_k \right|^{p} \right)^{\frac{1}{p}} \cdot \left( \sum_{k=1}^{\infty} \left| b_k \right|^{q} \right)^{\frac{1}{q}}
 .\end{align*} 
\end{proof}

Ряды --- это интегралы по считающим мерам. Вся теория рядов --- это очень частный случай теории меры.

\begin{thm}[неравенство Йенсена]
 Пусть $\varphi$ --- выпуклая функция на  $\R$, $f$ --- измерима и суммируема на $X$, $\mu$ --- мера на $\A$. Пусть  $\mu(X) = 1$. Тогда  \begin{align*}
  \varphi \left( \int\limits_{X} f \, d\mu   \right) \leqslant \int\limits_{X} \varphi(f(x)) \, d\mu  
 .\end{align*} 
\end{thm}
Упражнение: попробовать подставить считающую меру и получить обычное неравенство Йенсена.
\begin{proof}
 Для любой выпуклой функции выполнено неравенство \begin{align*}
   \frac{\varphi(x_1) - \varphi(x_2)}{x_1 - x_2} \leqslant \frac{\varphi(x_2) - \varphi(x_3)}{x_2 - x_3}, \quad \forall x_1 < x_2 < x_3
 .\end{align*} Зафиксируем $x_2$. Тогда \begin{align*}
  \frac{\varphi(x_1) - \varphi(x_2)}{x_1 - x_2} \leqslant a \leqslant \frac{\varphi(x_{2}) - \varphi(x_3)}{x_2 - x_3}
 .\end{align*} В качестве $a$ можно взять  \begin{align*}
 a = \inf_{x_3 > x_2} \frac{\varphi(x_2) - \varphi(x_3)}{x_2 - x_3}
 .\end{align*} Тогда \begin{align*}
 &\varphi(x_1) - \varphi(x_2) \leqslant a (x_2 - x_1) \\
  \implies &\varphi(x_1) \geqslant \varphi(x_2) + a(x_1 - x_2) \\
  &\varphi(x) \geqslant \varphi(x_2) + a(x - x_2) \quad \forall x < x_2
 .\end{align*} С другой стороны, \begin{align*}
 &\varphi(x_3) - \varphi(x_2) \geqslant a(x_3 - x_2) \\
 \implies &\varphi(x_3) \geqslant \varphi(x_2) + a(x_3 - x_2) \\
  \implies &\varphi(x) \geqslant \varphi(x_2) + a(x - x_2) \quad \forall x > x_2
 .\end{align*} Вывод: $\varphi(x) \geqslant \varphi(y) + a(x - y)$ для любого $x \in \R$, и некоторого $a$, которое зависит лишь от $y$ ($a$ это оценка на производную в точке $y$). Это называется что-то в духе <<полупроизводная>>.

 Докажем теперь неравенство Йенсена. Возьмём \begin{align*}
  y = \int\limits_{X} f \, d\mu  
 .\end{align*} Тогда существует $a \in \R$ такое, что \begin{align*}
  \varphi(f(x)) \geqslant \varphi \left( \int\limits_{X} f \, d\mu   \right) + a \left( f(x) - \int\limits_{X} f \, d\mu   \right)
 \end{align*}. Проинтегрируем обе части: \begin{align*}
  \int\limits_{X} \varphi(f(x)) \, d\mu   \geqslant \int\limits_{X} \varphi \left( \int\limits_{X} f \, d\mu   \right) \, d\mu  + a \left( \int\limits_{X} f \, d\mu - \int\limits_{X} \left( \int\limits_{X} f \, d\mu   \right) \, d\mu    \right)
 .\end{align*} Используя тот факт, что \begin{align*}
  \int\limits_{X} c \, d\mu  = c \mu(X) = c
 ,\end{align*} получаем требование.
\end{proof}
Сформулируем дискретный случай неравенства Йенсена. $X = \N$, $w_k \geqslant 0$ такие, что $\sum_{k=1}^{\infty} w_k = 1$. $\mu = \sum_{k=1}^{\infty} w_k \delta_{\left\{ w_k \right\}}$, $f(k) = a_k$ и  $\sum_{k=1}^{\infty} \left| a_k \right| \cdot w_k < \infty$. Пусть $\varphi$ выпуклая, тогда левая часть равна \begin{align*}
 \varphi \left( \int\limits_{X} f \, d\mu   \right) &= \varphi \left( \int\limits_{\bigcup_{k=1}^{\infty} \left\{ k \right\}} f \, d\mu   \right) = \varphi \left( \sum_{k=1}^{\infty} \int\limits_{\left\{ k \right\}} f \, d\mu   \right) = \varphi \left( \sum_{k=1}^{\infty} a_k w_k \right)
.\end{align*} А правая часть равна \begin{align*}
 \int\limits_{\N} \varphi(f(x)) \, d\mu  = \sum_{k=1}^{\infty} \int\limits_{\left\{ k \right\}} \varphi(f(x)) \, d\mu   = \sum_{k=1}^{\infty} \varphi(f(k)) w_k = \sum_{k=1}^{\infty} \varphi(a_k)w_k
.\end{align*} Получаем неравенство \begin{align*}
 \varphi \left( \sum_{k=1}^{\infty} a_k w_k \right) \leqslant \sum_{k=1}^{\infty} w_k \varphi(a_k)
.\end{align*} 

Мотивация: когда мы хотим доказать что-то интегральное, первое, что надо сделать --- проверить соответствующее утверждение в дискретном случае. Там получится теория рядов, и там, обычно, всё понятно.

\section{Пространства Лебега}
Пусть $(X, \A, \mu)$ --- пространство с мерой и $1 \leqslant p \leqslant \infty$ --- число. Зафиксируем их на весь параграф.
\begin{df}
 Пусть $f,g$ --- измеримы. Будем писать $f \sim g$, если $f = g$  $\mu$-почти всюду.
\end{df}
\begin{remrk}
 $f \sim g$ --- отношение эквивалентности.
\end{remrk}
\begin{df}
 Пространство Лебега 
\begin{align*}
 L^{p}(X, \mu) = \left\{ [f] \mid f \colon\, X \to [-\infty, +\infty],  \left( \int\limits_{X} \left| f \right|^{p} \, d\mu \right)^{\frac{1}{p}} < \infty  \right\}
\end{align*} для $p < \infty$, где $[f]$ --- класс эквивалентности $f$. Далее, \begin{align*}
L^{\infty}(X, \mu) = \left\{ [f] \mid \mathrm{ess} \sup \left| f \right|< \infty \right\}
,\end{align*} где существенный (essential) супремум определяется следующим образом: \begin{align*}
\mathrm{ess} \sup g &= \inf \left\{ c \in \R \mid g(x) \leqslant c \text{ $\mu$-почти всюду} \right\} \\
\mathrm{ess} \sup g &= +\infty, \text{ если таких $c$ нет. }
\end{align*} 
\end{df}
\begin{remrk}
 Запись $f \in L^{p}(X, \mu)$ означает, что выбран некоторый представитель его класса эквивалентности.
\end{remrk}
\begin{claim}
 $L^{p}(X,\mu)$ --- это линейное пространство относительно нормы \begin{align*}
  \|f\|_p = \begin{cases}
   \left( \int\limits_{X} \left| f \right|^{p} \, d\mu   \right)^{\frac{1}{p}}, \text{ если } p \geqslant 1  \\
   \mathrm{ess} \sup \left| f \right|, \text{ если } p=\infty
  \end{cases} 
 .\end{align*} 
\end{claim}
\begin{proof}
 Если $\|f\|_p = 0$, то  $\left| f \right| = 0$ $\mu$-почти всюду. Тогда $f = 0$ как элемент $L^{p}(X, \mu)$.

 \begin{align*}
  \| \alpha f \|_p = \left| \alpha \right| \cdot \| f \|_p \quad \forall \alpha \in \CC (\alpha \in \R)
 \end{align*} Если $f,g \in L^{p}(X,\mu)$, то $f + g \in L^{p}(X, \mu)$ и выполнено неравенство треугольника: \begin{align*}
  \| f + g \|_p \leqslant \|f \|_p + \|g\|_p
 \end{align*}  --- это неравенство Минковского. При $p=\infty$ нужно показать \begin{align*}
 \mathrm{ess} \sup \left| f+g \right| \leqslant \mathrm{ess} \sup \left| f \right| + \mathrm{ess} \sup \left| g \right|
.\end{align*} Заметим, что $\inf$ в определении  $\mathrm{ess} \sup$ достигается: если $\mathrm{ess} \sup h = C$, то существует множество $E$ такое, что $\mu(X \setminus E) = 0$ и $h \leqslant C$ на $E$. Докажем это: существуют множества $E_n$ полной меры, такие, что $h \leqslant C + \frac{1}{n}$ на $E_n$ (по определению нижней грани). Тогда  $E = \bigcap_{n=1}^{\infty} E_n $ подходит,  оно полной меры.
\end{proof}
\begin{df}
 $X$ --- \textit{банахово пространство}, если $X$ --- линейное нормированное пространство, и $X$ полно относительно своей нормы.
\end{df}
\begin{thm}
 \label{theorem:lebesgue_space_is_banach}
 $L^{p}(X, \mu)$ --- банахово пространство (оно полное).
\end{thm}
\begin{lm}
 Пусть $L$ --- линейное нормированное пространство. $L$ полно тогда и только тогда, когда для любых $x_n \in L$ таких, что \begin{align*}
  \sum_{k=1}^{\infty} \|x_n\| < \infty
 \end{align*}  существует предел \begin{align*}
  \lim_{N \to \infty} \sum_{n=1}^{\infty} x_n
 \end{align*} в $X$. То есть любой абсолютно сходящийся ряд сходится в $X$.
\end{lm}
\begin{proof}
 Пусть $L$ полно и $\sum_{n=1}^{\infty} \|x_n\| < \infty$. Положим \begin{align*}
  S_N = \sum_{n=1}^{\infty} x_n
 .\end{align*} Тогда \begin{align*}
  \| S_N - S_M \| \leqslant \sum_{n=N}^{M} \|x_n\|
 .\end{align*} По критерию Коши для числовых рядов \begin{align*}
  \lim_{N \to \infty} \sum_{n=N}^{M} \| x_n \| = 0 
 .\end{align*} Значит, $\{S_{N}\}_{N=1}^{\infty} $ --- фундаментальная последовательность. Так как пространство $L$ полное, то она сходится, то есть существует предел \begin{align*}
  \sum_{n=1}^{\infty} x_n := \lim_{N \to \infty} S_N 
 .\end{align*} 

 В обратную сторону: пусть любой абсолютно сходящийся ряд сходится и пусть дана последовательность $\{x_{k}\}_{k=1}^{\infty} $ --- фундаментальная в $L$. Тогда существует подпоследовательность $\{x_{n_k}\}_{k=1}^{\infty} $ такая, что \begin{align*}
  \|x_s - x_{n_k}\| \leqslant \frac{1}{2^{k}} \quad \forall s \geqslant n_k
 \end{align*} Построим последовательность $\{y_{k}\}_{k=1}^{\infty} $:  \begin{align*}
 y_1 &= x_{n_1} \\
 y_1 + y_2 &= x_{n_2} \\
 y_1 + y_2 +y_3 &= x_{n_3} \\
 &\vdots
\end{align*} Положим ещё $x_{n_0} = 0$. Тогда будет верна формула \begin{align*}
\sum_{k=1}^{\infty} \| y_k \| = \sum_{k=1}^{\infty} \| x_{n_k} - x_{n_{k-1}} \| \leqslant \sum_{k=1}^{\infty} \frac{1}{2^{k-1}} < \infty
.\end{align*} Следовательно, ряд $\sum_{k=1}^{\infty} y_k$ сходится, то есть существует предел \begin{align*}
\lim_{N \to \infty} \sum_{k=1}^{N} y_k = \lim_{N \to \infty} x_{n_N}  
.\end{align*} Значит, фундаментальная последовательность $\{x_{n}\}_{n=1}^{\infty} $ содержит сходящуюся подпоследовательность. Значит она сама сходится.
\end{proof}
\begin{proof}[\normalfont\textsc{Доказательство теоремы \ref{theorem:lebesgue_space_is_banach}}]
 Пусть $f_n \in L^{p}(X, \mu)$ такие, что $\sum_{n=1}^{\infty} \|f_n\|_p < \infty$. Нужно проверить, что ряд 
\begin{align*}
 F(x) = \sum_{n=1}^{\infty} f_n(x)
\end{align*} сходится при $\mu$-почти всех $x \in X$ и \begin{align*}
 \lim_{N \to \infty} \left\| \sum_{n=1}^{N} f_n - F \right\| = 0
.\end{align*} Докажем. \begin{align*}
\left( \int\limits_{X} \left( \sum_{n=1}^{\infty} \left| f_n(x) \right| \right)^{p} \, d\mu \right) ^{\frac{1}{p}} &= \left( \int\limits_{X} \varliminf_{N \to \infty} \left( \sum_{n=1}^{N} \left| f_n(x) \right| \right)^{p} \, d\mu \right)  ^{\frac{1}{p}} \leqslant \\
&\underbrace{\leqslant}_{\text{ лемма Фату }} \varliminf_{N\to \infty} \left( \int\limits_{X} \left( \sum_{n=1}^{N} \left| f_n(x) \right| \right)^{p} \, d\mu   \right)^{\frac{1}{p}} \leqslant \\
&\leqslant \varliminf_{N\to \infty} \sum_{n=1}^{N} \| f_n \|_p = \sum_{n=1}^{\infty} \|f_n\|_p < \infty \\
\implies &\sum_{n=1}^{\infty} \left| f_n(x) \right| < \infty \text{ $\mu$-почти всюду } \\
\implies &F(x) = \sum_{n=1}^{\infty} f_n(x) \text{ --- сходится $\mu$-почти всюду.}
\end{align*} Осталось оценить \begin{align*}
 \left\| F - \sum_{n=1}^{N} f_n(x) \right\|_p = \left\| \sum_{N+1}^{\infty} f_n \right\| \leqslant \sum_{N+1}^{\infty} \| f_n \|_p \to 0
\end{align*} 
\end{proof}
\begin{remrk}
 Пусть $\mu$ --- считающая мера на  $\N$, тогда  
\begin{align*}
 L^{p}(\N, \mu) = \left\{ \{x_{k}\}_{k=1}^{\infty} \mid \left( \sum_{k=1}^{\infty} \left| x_k \right|^{p} \right)^{\frac{1}{p}} < \infty  \right\} = \ell^{p}
.\end{align*}

\end{remrk}
