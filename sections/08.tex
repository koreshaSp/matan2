% 2022.10.21 lecture 8

\section{Точки Лебега множеств и функций.}

Рассмотрим мотивационную задачу.

\begin{problem}
 \label{problem:motivatinal_problem_good_points_in_closed_subset}
 Пусть $E \subset \R$ --- замкнутое подмножество. Будем говорить, что $x \in E$ --- \textit{хорошая точка}, если \begin{align*}
  \lim_{\eps \to 0} \frac{\left| E \cap (x - \eps, x + \eps) \right| }{\left| (x - \eps, x + \eps) \right|} = 1
 ,\end{align*} где $\left| S \right| = \lambda_1(S)$. Как понять, есть ли в $E$ хорошие точки, и много ли их?

 Любая внутренняя точка $E$ --- хорошая. С другой стороны, для отрезка $E = [l,r]$ есть ровно две плохие точки --- концы.

 \textbf{Ответ:} $\lao$-почти все точки  $E$ --- хорошие.
\end{problem}

В этом параграфе в основном будет про метрические пространства.

\begin{df}
 Пусть $(X, \tau)$ --- топологическое пространство. $\B(X)$ --- борелевская  $\sigma$-алгебра (наименьшая $\sigma$-алгебра, содержащая $\tau$). Пусть $\mu$ --- мера на $\B(X)$. Мера $\mu$ называется \textit{регулярной}, если для любого $E \in \B(X)$ \begin{align*}
  \mu(E) &= \inf \left\{ \mu(U) \mid U \text{ открытое и } E \subset U \right\} \\
  \mu(E) &= \sup \left\{ \mu(K) \mid K \text{ компактно и } K \subset E \right\}
 \end{align*} 
\end{df}
\begin{lm}
 Пусть $X$ --- метрический компакт и  $\mu$ --- конечная борелевская мера на $X$. Тогда $\mu$ регулярна.
\end{lm}
\begin{proof}
 Заведём $\sigma$-алгебру $\A$ следующим образом: $A \in \B(x)$ принадлежит $\A$, если для любого $\eps > 0$ существуют $C$ замкнутое и  $U$ открытое такие, что $C \subset A \subset U$, $\mu(U \setminus A) < \eps$ и $\mu(A \setminus C) < \eps$. Достаточно показать, что $\A = \B(X)$, так как в $X$ замкнутость и компактность --- одно и то же.

 Сначала поймём что замкнутые множества принадлежат $\A$. Действительно, если $E$ замкнутое, то можно взять $C = E$. Для $U$ возьмём любой $\eps > 0$ и построим открытое множество $U_{\eps}$, которое годится. Так как $E$ замкнуто, то $E = \bigcap_{n=1}^{\infty} V_n$, где $V_n = \left\{ x \mid \mathrm{dist}(x, E) < \frac{1}{n} \right\}$. Но в силу того, что $V_n \supset V_{n+1}$, то 
\begin{align*}
\mu(E) = \lim_{n \to \infty} \mu(V_n) 
\end{align*} по непрерывности конечной меры $\mu$. Поэтому существует $n$ такое, что $\mu(V_n) \leqslant \mu(E) + \eps$. Так как $V_n \supset E$, то $\mu(V_n \setminus E) < \eps$. Но $V_n$ открытое, поэтому можно взять $U_{\eps} = V_n$. $V_n$ открытое: $V_n = \bigcup_{x \in E} B(x, \frac{1}{n})$.


 В таком случае $\A = \B(X) \iff \A$ --- $\sigma$-алгебра. Проверим это.

 Если $A \in \A$, то и $A^{c} \in \A$. Действительно, если $C \subset A \subset U$ и $\mu(U \setminus A) < \eps$, $\mu(A \setminus C) < \eps$, то $U^{c} \subset A^{c} \subset C^{c}$, при этом $U^{c}$ замкнутое, а $C^{c}$ открытое. При этом 
\begin{align*}
 \mu(C^{c} \setminus A^{c}) &= \mu(C^{c} \cap A) = \mu((X \setminus C) \cap A) = \mu(A \setminus C) < \eps \\
 \mu(A^{c} \setminus U^{c}) &< \eps \text{ аналогично }
.\end{align*} Значит, симметричность есть.

$\varnothing \in \A$, так как $\varnothing$ и замкнутое, и открытое (можно взять $C=U=\varnothing$).

Теперь пусть $A_1, A_2, \ldots \in \A$. Найдём $C_1,C_2,\ldots$ и $U_1,U_2,\ldots$ такие, что $\mu(U_k \setminus A) < \frac{\eps}{2^{k}}$ и $\mu(A \setminus C_k) < \frac{\eps}{2^{k}}$. Тогда возьмём $U = U_1 \cup U_2 \cup \ldots$ и $C = C_1 \cup C_2 \cup \ldots$ и $A = A_1 \cup A_2 \cup \ldots$ Тогда \begin{align*}
 \mu(U \setminus A) &\leqslant \sum_{k=1}^{\infty} \mu(U_k \setminus A_k) < \eps \\
 \mu(A \setminus C) &\leqslant \sum_{k=1}^{\infty} \mu(A_k \setminus C_k) < \eps
\end{align*} Единственный момент, что $C$ может быть не замкнутой. Так как \begin{align*}
 \lim_{N \to \infty} \mu \left( A \setminus \bigcup_{k=1}^{N} C_k \right) = \mu \left( A \setminus \bigcup_{k=1}^{\infty} C_k \right) \leqslant \eps \\
 \implies 
\end{align*}  {\color{red} дописать!!!!}

\end{proof}
\begin{lm}[Урысона]
 \label{lemma:urison_metric_space}
 Пусть $X, d$ --- метрическое пространство и  $F_1, F_2$ --- замкнутые подмножества в $X$ и $F_1 \cap F_2 = \varnothing$. Тогда существует функция $f \colon\, X \to [0,1] $ такая, что $f \in C(X)$ и \begin{align*}
  f \rvert_{F_1} = 0 \\
  f \rvert_{F_2} = 1
 \end{align*}  (непрерывный спуск с единицы).
\end{lm}

Эта лемма очень полезная: она позволяет <<локализовывать>> многие задачи.

\begin{proof}
 Просто предъявим функцию: \begin{align*}
  f(x) = \frac{d(x, F_1)}{d(x, F_1) + d(x, F_2)}
 ,\end{align*}  где $d(x, F) = \inf_{y \in F} \left\| d(x, y) \right\|$ для $F \subset X$. Если $x \in F_1$, то $f(x) = 0$, так как  $d(x, F_1) = 0$. Если  $x \in F_2$, то $f(x) = \frac{d(x, F_1)}{d(x, F_1)} = 1$, причём $d(x, F_1) > 0$, так как  $F_1$ замкнутое и $x \notin F_1$ (если $F$ замкнуто, то $d(x, F) = 0 \iff x \in F$). Поэтому функция задано корректно на всём пространстве $X$. $0 \leqslant f \leqslant 1$ по построению.

 Осталось проверить, что $f$ непрерывна. Это верно, потому что функция $x \mapsto d(x, F)$ непрерывна при замкнутом $F$ --- проверим это: возьмём $x_n \to y$, \begin{align*}
  d(x_n, F) \to d(y, F) ? \\
  d(y, F) - \underbrace{d(x_n, y)}_{\to 0} \leqslant d(x_n, F) \leqslant \underbrace{d(x_n, y)}_{\to 0} + d(y, F) 
 .\end{align*} По теореме о двух милиционерах $d(x_n, F) \to d(y, F)$.
\end{proof}
\begin{thm}[%
]
  Пусть $K$ --- метрический компакт, $\mu$ --- борелевская мера на $K$, есть число $1 \leqslant p < \infty$. Тогда $C(K)$ --- плотное подмножество $L^{p}(K, \mu)$ (плотное, это значит замыкание совпадает со всем пространством).
 
\end{thm}
\begin{proof}
 Возьмём функцию $f \in L^{p}(K,\mu)$. Наша цель: найти $f_n \in C(K)$ такие, что \begin{align*}
  \left\| f - f_n \right\|_{L^{p}(K,\mu)} \to 0
 .\end{align*} Замечание: можно считать, что $f$ ограничена. Введём множества  \begin{align*}
 E_N &= \left\{ x \in K \mid \left| f(x) \right| \geqslant N \right\} \\
 \implies & \int\limits_{E_N} \left| f \right|^{p} \, d\mu   \to \int\limits_{\bigcap_{N=1}^{\infty} E_N} \left| f \right|^{p} \, d\mu \text{ при } N \to \infty
 \end{align*} по непрерывности меры $\nu(E) = \int_{E} \left| f \right|^{p} \, d\mu $. Далее,
 \begin{align*}
  \int\limits_{\bigcap_{N=1}^{\infty} E_N} \left| f \right|^{p} \, d\mu = \int\limits_{\left\{ x \mid \left| f(x) \right| = +\infty \right\}} \left| f \right|^{p} \, d\mu    = 0
 .\end{align*} Значит, если мы приблизим функцию $f$ на любом $E_N$ сколь угодно хорошо функциями, не превосходящими константы $C$, то мы приблизим $f$ на всём пространстве $K$.

 Итог: можно считать, что $\left| f \right|$ ограничена. По теореме \ref{theorem:approximation} об аппроксимации простыми функциями существуют $g_{\eps}$ --- простая, такая что $\left\| g_{\eps} - f \right\|_{L^{\infty}(E^{c}_N, \mu)} \leqslant \eps$. При этом \begin{align*}
  g_{\eps} \rvert_{E_N} = 0
 .\end{align*} Следовательно, можно считать, что $f$ простая. То есть \begin{align*}
 f = \sum_{k=1}^{M}  c_k \chi_{A_k}
.\end{align*} Если для любого $k$ и для любого $\eps > 0$ существует $h_{k,\eps}$ такая, что  \begin{align*}
\left\| \chi_{A_k} - h_{k,\eps} \right\|_{L^{p}(K,\mu)} < \frac{\eps}{2^{k}\left| c_k \right|}
,\end{align*} то \begin{align*}
\left\| f - \sum_{k=1}^{M} c_k h_{k,\eps} \right\| \leqslant \sum_{k=1}^{M}  \frac{\eps \left| c_k \right|}{\left| c_k \right| 2^{k}} \leqslant 2\eps
.\end{align*}

Таким образом, можно считать, что $f = \chi_A$, где $A \in \B(K)$. Так как $K$ --- метрический компакт то $\mu$ регулярна. Значит, для любого $\eps > 0$ существует $C_{\eps}, U_{\eps}$ такие, что $C_{\eps} \subset A \subset U_{\eps}$, $C_{\eps}$ замкнуто, $U_{\eps}$ открыто и 
\begin{align*}
\mu(U_{\eps} \setminus C_{\eps}) \leqslant \mu(U_{\eps} \setminus A) + \mu(A \setminus C_{\eps}) \leqslant 2\eps
.\end{align*} Теперь построим функцию $h_{\eps} \in C(K)$ такую, что $0 \leqslant h_{\eps} \leqslant 1$ всюду на $K$, $h_{\eps} \rvert_{C_{\eps}} = 1$ и $h_{\eps} \rvert_{U^{c}_{\eps}} = 0$ --- такая $h_{\eps}$ существует по лемме \ref{lemma:urison_metric_space} Урысона.

Наконец,
\begin{align*}
 \left\| f - h_{\eps} \right\|_{L^{p}(K,\mu)}^{p} &= \int\limits_{K} \left| \chi_A - h_\eps \right|^{p} \, d\mu  = \\ &= \underbrace{\int\limits_{U_{\eps}^{c}} \left| \chi_A - h_{\eps} \right|^{p} \, d\mu}_{=0}   + \int\limits_{U_{\eps} \setminus C_{\eps}} \left| \chi_A - h_{\eps} \right|^{p} \, d\mu  + \underbrace{\int\limits_{C_{\eps}} \left| \chi_A - h_{\eps} \right|^{p} \, d\mu}_{= 0}  = \\
 &= \int\limits_{U_{\eps} \setminus C_{\eps}} \left| \chi_A - h_{\eps} \right|^{p} \, d\mu  \leqslant \\
  & \leqslant \mu(U_{\eps} \setminus C_{\eps}) \cdot \max \left| \chi_A - h_{\eps} \right|^{p} \leqslant \\
  &\leqslant 2\eps
.\end{align*} {\color{red} Тут есть картинка}. Всё получилось! Мы приблизили характеристическую функцию. Значит, можно и любую измеримую приблизить.
\end{proof}
\begin{remrk*}
 $C(K)$ не плотно в $L^{\infty}(K, \mu)$.
\end{remrk*}
\begin{lm}[%
Витали]
\label{lemma:vitali}
Пусть $X$ --- сепарабельное метрическое пространство (то есть в $X$ существует всюду плотное счётное подмножество). Пусть $G$ --- семейство шаров в $X$, $0 < \rho(B) \leqslant R$ для любого $B \in G$, где $\rho(B)$ --- радиус шара $B$.

Тогда существует $\hat G \subset G$ такое, что $\hat G$ счётно, $B_1 \cap B_2 = \varnothing$ для любых $B_1, B_2 \in \hat G$, и \begin{align*}
 \bigcup_{B \in \hat G}  5B \supset \bigcup_{B \in G} B
,\end{align*} где $5B$ --- это шар с тем же центром, что и у  $B$, но $\rho(5B) = 5 \cdot \rho(B)$.
\end{lm}
В $\R^{n}$ вообще можно можно не увеличивать шары (теорема Безиковича).
\begin{proof}
 Считаем, что $X = \bigcup_{B \in G} B $. Положим 
\begin{align*}
G_j = \left\{ B \in G \mid \rho(B) \in \left(2^{-j-1}R, 2^{-j}R\right] \right\}
,\end{align*} где $R = \sup_{B \in G} \rho(B)$, $j \geqslant 0$.

Возьмём $G_0'$ --- произвольное дизъюнктное максимальное по включению подмножество шаров $G_0$: если $B \in G_0$, но $B \notin G_0'$, то существует $\tilde B \in G_0'$ такой, что $B \cap \tilde B \neq \varnothing$. В произвольном пространстве такое можно такое построить с помощью леммы Цорна. В сепарабельном пространстве можно без неё: возьмём $\{x_{k}\}_{k=0}^{\infty} $ -- счётное всюду плотное подмножество $\bigcup_{B \in G_0} B$. Возьмём шар $B_0 \ni x_0$. Если рассмотрены точки $x_0, \ldots, x_n$ и построены шары $B_0, \ldots, B_{m_n}$, то при рассмотрении точки $x_{n+1}$ поступим следующим образом:
 \begin{itemize}
  \item Если любой шар $B \in G_0$ такой, что $x_{n+1} \in B$, и $B$ пересекает $B_0 \cup \ldots \cup B_{m_n}$, то переходим к $x_{n+2}$.
  \item Если существует $B \in G_0$ такой, что $x_{n+1} \in B$ и $B$ не  пересекается с $B_0 \cup \ldots \cup B_{m_n}$, то берём $m_{n+1} = m_n + 1$ и  $B_{m_{n+1}} = B$.
\end{itemize} Легко видеть, что эта конструкция подходит.

Теперь положим $G_j'$ --- максимальное по включению дизъюнктное семейство шаров из $G_j$, не пересекающих $G_0' \cup \ldots \cup G_{j-1}'$. Теперь возьмём 
\begin{align*}
\hat G = \bigcup_{j=0}^{\infty} G_j'
.\end{align*} По построению  $\hat G \subset G$, $\hat G$ счётное, $\hat G$ дизъюнктное. Рассмотрим $B \in G$  и покажем, что $B \subset 5\tilde B$ для некоторого $\tilde B \in \hat G$. Если $B \in \hat G$, то всё очевидно (можно взять $\tilde B = B$). Если $B \in G \setminus \hat G$, то существует $j$ такое, что $B \in G_j \setminus G_j'$. Значит, $\rho(B) \in \left(2^{-j-1}R, 2^{-j}R\right]$. Кроме того, существует $\tilde B \in G_k'$, $0 \leqslant k \leqslant j$ такой, что $\tilde B \cap B \neq \varnothing$.

Покажем, что $B \subset 5\tilde B$ {\color{red} тут снова рисунок}. Достаточно показать \begin{align*}
 5\rho(\tilde B) \geqslant 2 \rho(B) + \rho(\tilde B) \iff \rho(B) \leqslant 2 \rho(\tilde B)
.\end{align*} Последнее верно, потому что в худшем случае $B$ и $\tilde B$ лежат в $G_j$. Если  $\tilde B \in G_k$, $k < j$, то всё даже лучше (радиус $\tilde B$ только больше).
\end{proof}
\begin{remrk*}
 В доказательстве леммы \ref{lemma:vitali} Витали счётность мы получили из конструкции семейства шаров. Но можно было бы взять неявную конструкцию по лемме Цорна и воспользоваться сепарабельностью пространства: если есть семейство дизъюнктных шаров в сепарабельном пространстве, то их обязательно не более, чем счётное число.
\end{remrk*}

\begin{df}
 Пусть $X$ --- метрическое пространство, $\B(X)$ --- борелевская $\sigma$-алгебра, $\mu$ --- борелевская мера на $X$, $f \in L^{1}(X, \mu)$ и \begin{align*}
  \mu(B) < \infty
 \end{align*} для любого шара $B \subset X$.

 Рассмотрим 
\begin{align*}
 (M^{\ast}f)(x) = \sup_{B \ni x \text{ --- шар }} \frac{1}{\mu(B)} \int\limits_{B} \left| f \right| \, d\mu  
\end{align*} --- \textit{максимальная функция Харди-Литтлвуда}.
\end{df}
\begin{remrk}
 $M^{\ast}f$ измерима, так как \begin{align*}
  E_t = \left\{ (M^{\ast} f)(x) > t \right\} 
 \end{align*} открыто для любого $t \in \R$ в $X$, потому что если $(M^{\ast}f)(x) > t$, то существует шар $B$ такой, что \begin{align*}
  \frac{1}{\mu(B)} \int\limits_{B} \left| f \right| \, d\mu   >t \implies B \subset E_t
 .\end{align*} 
\end{remrk}
\begin{thm}[%
Харди-Литтлвуда]
\label{theorem:hardy_littlewood}
 Пусть $X$ --- метрическое сепарабельное ограниченное пространство, $\rho(x, y) \leqslant R$ для любых $x,y \in X$. Пусть $\mu$ --- борелевская мера на $X$ такая, что $\mu(2B) \leqslant c \mu(B)$ для любого шара $B$ в $X$ и некоторого $c > 0$.

 Тогда существует константа $C > 0$ такая, что для любой $f \in L^{1}(X, \mu)$ имеет место оценка \begin{align*}
  \mu \left\{ x \mid (M^{\ast}f)(x) > \lambda \right\} \leqslant \frac{C \left\| f \right\|_{L'(X,\mu)}}{\lambda}
 \end{align*} для любого $\lambda > 0$.
\end{thm}
\begin{proof}
 Обозначим $E_{\lambda} = \left\{ x \mid (M^{\ast}f)(x) > \lambda \right\}$. Оказывается, \begin{align*}
  E_{\lambda} \subset \bigcup_{B \in G}  B
 ,\end{align*} где $G$ --- семейство шаров $B$ таких, что \begin{align*}
  \frac{1}{\mu(B)} \int\limits_{B} \left| f \right| \, d\mu   > \lambda
 .\end{align*} Воспользуемся леммой \ref{lemma:vitali} Витали и найдём $\hat G$. Тогда \begin{align*}
 \mu \left( \bigcup_{B \in G} B \right) \leqslant \mu \left( \bigcup_{B \in \hat G} 5B  \right) \leqslant \sum_{B \in \hat G} \mu(5B) \leqslant [2^{3} > 5] \leqslant c^{3} \sum_{B \in \hat G}  \mu(B).
 \end{align*} Для каждого $B$: \begin{align*}
 \mu(B) \leqslant \frac{\int\limits_{B} \left| f \right| \, d\mu}{\lambda}
 \end{align*} по неравенству. Поэтому, \begin{align*}
 \sum_{B \in \hat G}  \mu(B) \leqslant \frac{1}{\lambda} \sum_{B \in \hat G}  \int\limits_{B} \left| f \right| \, d\mu   \leqslant \frac{\int\limits_{X} \left| f \right| \, d\mu  }{\lambda} = \frac{\left\| f \right\|_{L^{1}(X,\mu)}}{\lambda}
 ,\end{align*} так как шары из $\hat G$ дизъюнктны. Теорема доказана.
\end{proof}
\begin{df}
 $x$ --- \textit{точка Лебега} функции  $f \in L^{1}(X,\mu)$, если \begin{align*}
  \lim_{\eps \to 0} \frac{1}{\mu(B_{\eps})} \int\limits_{B_{\eps}} \left| f(y) - f(x) \right| \, d\mu = 0
 ,\end{align*} где $\left\{ B_{\eps} \right\}$ --- произвольный набор шаров такой, что $\rho(B_{\eps}) = \eps$ и $B_{\eps} \ni x$.
\end{df}
\begin{df}
 Если $E$ --- борелевское множество. $x$ называется \textit{точкой Лебега} множества $E$, если $x$ --- точка Лебега функции $\chi_E$, то есть \begin{align*}
  \frac{\mu(E \cap B_{\eps}(X))}{\mu(B_{\eps}(x))} \to 1 \text{ при } \eps \to 0
 .\end{align*} 
\end{df}
