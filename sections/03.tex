% 2022.09.16 lecture 3

\begin{enumerate}
 \setcounter{enumi}{2}
 \item Пусть $A \in \A^{\ast} \cap \A_{\nu}$ такое, что $\mu(A) < \infty$. Тогда надо показать $\nu(A) = \mu(A)$.
 \begin{align*}
  \mu(A) = \mua(A) = \inf \left\{ \sum_{k=1}^{\infty} \mu(P_k) \mid P_k \in \p \colon\; A \subset \bigcup_{k=1}^{\infty} P_k \right\}
 .\end{align*} Возьмём любой набор $\left\{ P_k \right\}$ такой, что $\bigcup_{k=1}^{\infty} P_k \supset A$. Тогда:
 \begin{align*}
  \nu(A) \underbrace{\leqslant}_{\text{ так как оба множества } \in \A_{\nu}} \nu \left( \bigcup_{k=1}^{\infty} P_k \right) = \nu( \bigsqcup_{k=1}^{\infty}  \bigsqcup_{j=1}^{N_k} Q_{kj}) = \sum_{k=1}^{\infty} \underbrace{\sum_{j=1}^{N_k} \nu(Q_{kj})}_{\leqslant \nu(P_k)} \Rightarrow \\
  \Rightarrow \nu(A) \leqslant \sum_{k=1}^{\infty}  \nu(P_k) \underbrace{=}_{\text{по условию} P_k \in \p} \sum_{k=1}^{\infty}  \mu(P_k)
 .\end{align*} Возьмём инфимум и получим:
 \begin{align*}
  \nu(A) \leqslant \mu(A)
 .\end{align*} Это верно для любого $A \in \A^{\ast} \cap \A_{\nu}$.

 Теперь возьмём $P \in \p$ и $A \in \A^{\ast} \cap \A_{\nu}$. Посмотрим на равенство
 \begin{align*}
  \mu(P) = \mu(P \cap A^{c}) + \mu(P \cap A) \geqslant \nu(P \cap A^{c}) + \nu(P \cap A) = \nu(P) = \mu(P)
 .\end{align*} Значит, если $\mu(P)$ конечно, то все слагаемые конечные, и $\mu(P \cap A) = \nu(P \cap A)$ и $\mu(P \cap A^{c}) = \nu(P \cap A^{c})$: два строгих неравенства выглядят странно, но если бы одно из неравенств $\mu(P \cap A) \geqslant \nu(P \cap A)$, $\mu(P \cap A^{c}) \geqslant \nu(P \cap A^{c})$ было бы строгим, то мы пришли бы к противоречию.

 Так как $\mu(A) < \infty$, то $\forall \eps > 0 \colon\; \exists  \{P_{k}\}_{k=1}^{\infty} \subset \p,\; A \subset \bigcup_{k=1}^{\infty} P_k \colon\; $
 \begin{align*}
	 \mu(A) + \eps \geqslant \sum_{k=1}^{\infty} \mu(P_k), \text{в частности: } \mu(P_k) < \infty, \quad \forall k \in \N
 .\end{align*}
 Тогда
 \begin{align*}
  A = \bigcup_{k=1}^{\infty} (P_k \cap A)
 .\end{align*} Более того, используя лемму  о подчинённом разбиении, можно считать, что
 \begin{align*}
  A = \bigsqcup_{k=1}^{\infty} \bigsqcup_{j=1}^{N_k} (Q_{kj} \cap A)
 \end{align*}
 Теперь
 \begin{align*}
  \mu(A) = \sum_{k,j} \mu(Q_{kj} \cap A) = \sum_{kj} \nu(Q_{kj} \cap A) = \nu(A)
 .\end{align*} Доказательство окончено.
\end{enumerate}
\begin{df}
 $\mu_0 \colon\; \p \to [0, \infty]$ --- \textit{$\sigma$-конечна} на полукольце $\p \subset 2^{X}$, если существуют $\{P_k\}_{k=1}^{\infty} \subset \p $ такое, что
 \begin{align*}
  X = \bigsqcup_{k=1}^{\infty} P_k
 \end{align*} и $\mu_0(P_k) < \infty$ для любого $k$.
\end{df}
\begin{df}
 Пусть $\mu$ --- мера на $\sigma$-алгебре $\A \subset 2^{X}, \mu$ называется \textit{$\sigma$-конечной}, если существует счётный набор $\{A_k\}_{k=1}^{\infty} \subset \A $ такой, что $X = \bigsqcup_{k=1}^{\infty} A_k $, $\mu(A_k) < \infty$.
\end{df}
\begin{crly}
 Если в формулировке теоремы Каратеодори $\mu_0$ --- $\sigma$-конечна, то $\mu(A) = \nu(A)$ для любого $A \in \A^{\ast} \cap \A_{\nu}$ (даже если $\mu(A) = \infty$).
\end{crly}
\begin{proof}
 \begin{align*}
  A = \bigsqcup_{k=1}^{\infty} (A \cap P_k)
 ,\end{align*} где $P_k$ --- из определения $\sigma$-конечности. По условию  $\mu(P_k) < \infty$. Тогда
 \begin{align*}
  \mu(A) = \sum_{k=1}^{\infty} \mu(A \cap P_k) = \sum_{k=1}^{\infty} \nu(A \cap P_k) = \nu(A)
 .\end{align*}
\end{proof}
\begin{df}
 Пусть $\mu_0 \colon\; \p_1 \to [0, \infty)$ такова, что
 \begin{align*}
  \mu_0 \colon\; [a, b) \to b - a
 .\end{align*} Пусть $\lambda_1$ --- это мера  $\mu$ из теоремы Каратеодори для $\mu_0$, $\A_{\lambda_1}$ --- это $\sigma$-алгебра $\A^{\ast}$ из теоремы Каратеодори. Тогда $\lambda_1$ --- это \textit{мера Лебега}, $\A_{\lambda_1}$ --- \textit{$\sigma$-алгебра Лебега}. А множества из $\A_{\lambda_1}$ называются \textit{множествами, измеримыми по Лебегу}.
\end{df}
\begin{df}
 Мера $\mu$ в теореме Каратеодори называется \textit{стандартным продолжением}.
\end{df}
\begin{remrk}[формула для вычисления $\mu(A)$]
 \begin{align*}
  \mu(A) = \inf \left\{ \sum_{k=1}^{\infty} \mu_0(P_k) \mid P_k \in \p \colon\; A \subset \bigcup_{k=1}^{\infty} P_k \right\}
 \end{align*}
 для стандартного продолжения $\mu_{0}$ на $\A^{\ast}$.
\end{remrk}
\newcommand{\B}{\ensuremath \mathcal{B}}
\begin{df}
 \textit{Борелевская оболочка} множества $E$ --- наименьшая $\sigma$-алгебра, содержащая $E$. Обозначение: $\B(E)$.
\end{df}
\begin{df}
 Если $(X, \tau)$ --- топологическое пространство, то $\B(X)$ --- \textit{борелевская оболочка топологии}. 

 Для борелевской оболочки $\R$ есть стандартное обозначение $\B_1 = \B(R)$ --- наименьшая $\sigma$-алгебра, содержащая все открытые множества (и замкнутые! по аксиоме симметричности).
\end{df}
\begin{lm}
 \begin{align*}
	 \B_1 = \B( \p_1 ) \subset \A_{\lambda_1} \underbrace{\neq}_{\text{задачка}} 2^{\R}
 .\end{align*}
\end{lm}
\begin{proof}
 Включение $\B(\p_1) \subset \A_{\lambda_1}$ верно, так как $\p_1 \subset \A_{\lambda_1}$ (по условию теоремы Каратеодори) и $\A_{\lambda_1}$ --- это $\sigma$-алгебра.

 Равенство. Сначала докажем, что $\B_1 \subset \B(\p_1)$. Это верно, так как любое открытое $O \subset \R$ представимо в виде
 \begin{align*}
  O = \bigcup_{k=1}^{\infty} I_k, \quad I_k = (a_k, b_k),\; a_k \leqslant b_k
 ,\end{align*} так как $I_k$ --- класс эквивалентности по отношению $x \sim y \iff \exists I \subset O $ --- интервал, такой, что $x, y \in I$.

 Теперь напишем, что
 \begin{align*}
  (a_k, b_k) = \bigcup_{n=1}^{\infty} \left[a_k + \frac{1}{n}, b_k\right) \in \B(P_1) \implies O \in \B(\p_1)
 .\end{align*} Значит, $\B_1 \subset \B(\p_1)$, так как $\B_1$ --- наименьшая $\sigma$-алгебра.

 В обратную сторону: $\B(\p_1) \subset \B_1$, так как для любой ячейки $[a, b)$ имеет место равенство:
 \begin{align*}
  [a, b) = (a - 1, b) \setminus (-\infty, a) \in \B_1
 .\end{align*} Следовательно, $\B(\p_1) \subset \B_1$.
\end{proof}
\begin{exmpl}[про меру Лебега]\
 \begin{enumerate}
  \item $x \in \R \implies \left\{ x \right\} \in \B_1 \subset \A_{\lambda_1}$, так как точка --- замкнутая множества. Померим длину точки:
   \begin{align*}
    0 \leqslant \lambda_1(\left\{ x \right\}) \leqslant \lambda_1([x, x + \varepsilon)) = \varepsilon \quad \forall \eps > 0
   .\end{align*} Значит, $\lambda_1(\left\{ x \right\}) = 0$.
  \item Отрезок $[0, 1] \in \B_1$ (замкнутый). Его длина равна:
   \begin{align*}
    \lambda_1([0,1]) = \lambda_1([0, 1) \sqcup \left\{ 1 \right\}) = 1 + 0 = 1
   .\end{align*}
  \item $A$ открытое и непустое $ \implies A \in \B_1$ и $\lambda_1(A) > 0$.
   \begin{proof}
    Возьмём $x \in A$. Тогда существует $\eps > 0$ такой, что $(x - \eps, x + \eps) \subset A \implies [x, x + \eps) \subset A$. Тогда
    \begin{align*}
     \eps = \lambda_1( [x, x + \eps) ) \leqslant \lambda_1(A) \implies \lambda_1(A) > 0
    .\end{align*}
   \end{proof}
  \item $\lambda_1(\Q) = 0$.
   \begin{proof}
    \begin{align*}
     \lambda_1(\Q) = \sum_{x \in \Q} \lambda_1(\left\{ x \right\}) = 0
    .\end{align*}
   \end{proof}
  \item \begin{align*}
		  \lambda_1( \left\{ x \in [0,1] \mid x \notin \Q \right\} ) = \underbrace{\lambda_1(\Q \cap [0, 1])}_{\leq \lambda_1(\Q)} + \lambda_1(E) = \\
	= 0 + \lambda_1(E) = \underbrace{\lambda_1([0, 1])}_{=1} \Rightarrow \lambda_1(E) = 1
   .\end{align*}
  \item Если есть $\{E_{k}\}_{k=1}^{\infty} $ такие, что $\lambda_1(E_k) = 0$, то
   \begin{align*}
    \lambda_1 \left( \bigcup_{k=1}^{\infty} E_k \right) = 0
   .\end{align*}
   \begin{proof}
    \begin{align*}
     \lambda_1 \left( \bigcup_{k=1}^{\infty} E_k \right) \leqslant \sum_{k=1}^{\infty} \lambda_1(E_k) = 0
    .\end{align*}
   \end{proof}
  \item Обобщённое Канторово множество.
\begin{figure}[ht]
    \centering
	\incfig{пояснение-обобщенного-канторова-множества}
	\caption{Пояснение обобщенного Канторова множества}
	\label{fig:пояснение-обобщенного-канторова-множества}
\end{figure}

	  Шаг 0: из интервала $(0, 1)$ выкидываем интервал длины $a_0$. \\
	  Шаг 1: из оставшегося множества выкидываем $2^{1}$ открытых интервала длины $a_1$. \\
	  Шаг 2: выкидываем $2^{2}$ интервалов длины $a_2$. \\
	  Здесь $a_0, a_1, a_2, \ldots \geqslant 0$ достаточно маленькие чтобы можно было выкинуть:
	  
\begin{align*}
    \sum_{n=0}^{\infty} 2^{n} a_n \leqslant 1
   .\end{align*} Для стандартного Канторова множества $a_n = \frac{1}{3^{n+1}}$, проверим:
   \begin{align*}
    \sum_{n=0}^{\infty} \frac{2^{n}}{3^{n+1}} = \frac{1}{3} \sum_{n=0}^{\infty} \left( \frac{2}{3} \right)^{n} = \frac{1}{3} \cdot \frac{1}{1 - \frac{2}{3}} = 1
   .\end{align*} Если взять $a_n = \frac{1}{4^{n+1}}$, то
   \begin{align*}
    \sum_{n=0}^{\infty} \frac{2^{n}}{4^{n+1}} = \frac{1}{4} \cdot \frac{1}{1 - \frac{2}{4}} = \frac{1}{2}
   .\end{align*} Пусть $C_{\left\{ a_n \right\}}$ --- обобщённое Канторово множество. Утверждается, что оно \textit{измеримо по Борелю}: $C_{\left\{ a_n \right\}} \in \B_1$, так как оно замкнуто (или, например, мы счётное число раз выкидываем интервалы). С другой стороны,
   \begin{align*}
    \lambda_1 \left( C_{\left\{ a_n \right\}} \right)
    & = \lambda_{1} \left( [0, 1] \setminus \bigsqcup_{n=0}^{\infty} \bigsqcup_{k=1}^{2^{n}} I_{n,k} \right) = \\
    & =  1 - \sum  \lambda_1(I_{n,k}) = 1 - \sum_{n=0}^{\infty} 2^{n} a_n
   .\end{align*} Вывод: стандартное Канторово множество несчётное, но его длина равна нулю! При этом:
   \begin{align*}
    \lambda_1 \left( C_{\left\{ \frac{1}{4^{n+1}} \right\}} \right) = \frac{1}{2} > 0
   .\end{align*} Такие Канторовы множества с ненулевой длиной называются \textit{толстыми Канторовыми множествами}. Интересный факт: замыкание толстого Канторово множества не содержит интервалов, но имеет положительную длину, в отличие от $\Q$!
 \end{enumerate}
\end{exmpl}
\begin{df}
 Пусть  $\mu$ --- мера на $\sigma$-алгебре $\A$. $\mu$ называется \textit{полной}, если для любого множества $A \in \A$ такого, что $\mu(A) = 0$ и для любого  $B \subset A$ имеет место $B \in \A$.
\end{df}
\begin{remrk}
 Мера $\mu$ в теореме Каратеодори полна.
\end{remrk}
\begin{proof}
 Пусть $A \in \A^{\ast}$ такое, что $\mu(A) = 0$. Пусть  $B \subset A$. Рассмотрим произвольное $E \subset X$. Тогда:
 \begin{align*}
  \mua(E) & \leqslant \mua(E \cap B) + \mua(E \cap B^{c}) \leqslant \\
  & \leqslant \mua(E \cap A) + \mua(E) \leqslant \mua(A) + \mua(E) \leqslant 0 + \mua(E)
 .\end{align*} Значит, $\mua(E) = \mua(E \cap B) + \mua(E \cap B^{c})$ для любого $E \subset X$, и следовательно, $B \in \A^{\ast}$.
\end{proof}
\section{Измеримые функции. Теорема об аппроксимации.}
\begin{df}
 $(X, \A)$ --- \textit{измеримое пространство}, если  $\A$ ---  $\sigma$-алгебра подмножеств $X$.
\end{df}
В этом параграфе все функции заданы на некотором измеримом пространстве.
\begin{df}
 Функция $f \colon\, X \to [-\infty, \infty] $ \textit{измерима} (относительно $\A$), если для любого $a \in \R$ верно
 \begin{align*}
  f^{-1}((a, +\infty]) \in \A
 .\end{align*}
\end{df}
\begin{remrk}
 В определении можно заменить $(a, +\infty]$, например, на  $[a, +\infty]$:
 \begin{align*}
  f^{-1}([a, +\infty]) = \bigcap_{n=1}^{\infty} f^{-1}\left(\left(a - \frac{1}{n}, +\infty \right)\right) \\
  f^{-1}((a +\infty]) = \bigcup_{n=1}^{\infty}  f^{-1} \left( [a + \frac{1}{n}, +\infty] \right)
 .\end{align*}
\end{remrk}
\begin{crly}
 Прообраз любой точки измерим, если $f$ измеримо:
 \begin{align*}
  f^{-1}(\left\{ x \right\}) \in \A
 \end{align*} для любой точки $x \in [-\infty, +\infty]$.
\end{crly}
\begin{proof} Для $x \in \R$:
 \begin{align*}
  f^{-1}( \left\{ x \right\} ) = f^{-1}( [x, \infty] ) \setminus f^{-1} \left( (x, +\infty] \right)
 .\end{align*}  Для $x = +\infty$:
 \begin{align*}
  f^{-1} \left( \left\{ +\infty \right\} \right) = \bigcap_{n=1}^{\infty} f^{-1} \left( [n, +\infty] \right)
 .\end{align*} Сначала покажем для $[-\infty, +\infty]$:
 \begin{align*}
  \bigcup_{n=1}^{\infty} f^{-1}([-n, +\infty]) = f^{-1}( (-\infty, +\infty] ) \\
  f^{-1}(\left\{ -\infty \right\}) \sqcup f^{-1} \left( (-\infty, +\infty] \right) = X
 .\end{align*}
\end{proof}
\begin{crly}
 Для любого $[a, b)$
 \begin{align*}
  f^{-1}([a, b)) \in \A
 ,\end{align*} если $f$ измерима.
\end{crly}
\begin{proof}
 \begin{align*}
  f^{-1}( [a, b) ) = f^{-1} \left( [a, +\infty] \right) \setminus f^{-1} \left( [b, +\infty] \right)
 .\end{align*}
\end{proof}
\begin{thm}
 Пусть есть функция $f \colon\, X \to [-\infty, +\infty] $ измерима. Тогда для любого $E \in \B_1$ верно
 \begin{align*}
  f^{-1}(E) \in \A
 .\end{align*}
\end{thm}
\begin{proof}
 Заведём специально обученную $\sigma$-алгебру. Рассмотрим
 \begin{align*}
  \B = \left\{ A \subset \R \mid f^{-1}(A) \in \A \right\}
 .\end{align*} Это $\sigma$-алгебра (легко проверить все аксиомы):
 \begin{itemize}
  \item $\R \in \B$
  \item $\varnothing \in \B$
  \item $A_k \in \B$, тогда
   \begin{align*}
    f^{-1}\left(\bigcap_{k=1}^{\infty} A_k \right) = \bigcap_{k=1}^{\infty} f^{-1}(A_k) \in \A \implies \bigcap_{k=1}^{\infty} A_k \in \B
   \end{align*}
   Но $\p_1 \subset \B$ (см. следствия). Тогда $\B_1 = \B(\p_1) \subset \B$. Значит, для любого  $A \in \B_1$ $f^{-1}(A) \in \A$.
 \end{itemize}
\end{proof}
Иными словами, измеримая функция --- это та, у которой прообраз измеримого множества измерим(по Борелю).
\begin{df}
 Функция $f \colon\; X \to \R$ называется \textit{простой}, если она измерима и принимает конечное число значений.
\end{df}
\begin{remrk}
 Каждая простая функция $f$ имеет вид
 \begin{align*}
  f = \sum_{k=1}^{N} c_k \chi_{A_k}
 \end{align*} для некоторых $A_k \in \A$. Здесь
 \begin{align*}
  \chi_A(x) = \begin{cases}
   1, \text{ если } x \in A \\
   0, \text{ иначе }
  \end{cases}
 .\end{align*}
\end{remrk}
\begin{proof}
 \begin{align*}
  f(X) = \{c_{k}\}_{k=1}^{N}
 .\end{align*} Тогда
 \begin{align*}
  A_k = f^{-1} \left( \left\{ c_k \right\} \right) \in \A
 .\end{align*}
\end{proof}
Идея Лебега: нужно любую измеримую функцию приближать простыми.
\begin{lm}
 Пусть функции $f_n \colon\; X \to [-\infty, +\infty]$ измеримы. Тогда $\sup f_n$, $\inf f_n$,  $\limsup f_n$, $\liminf f_n$ измеримы. Также, если существует предел  $\lim_{n \to \infty} f_n(x)$ для любой точки $x \in X$, то $\lim f_n$ измерим.
\end{lm}
\begin{proof}
 $f(x) = \sup f_n(x)$:
 \begin{align*}
  f^{-1}((a, +\infty]) = \bigcup_{n=1}^{\infty} f^{-1}_n( (a, +\infty] )
 .\end{align*} $f(x) = \inf f_n(x)$:
 \begin{align*}
  f^{-1} \left( [-\infty, a) \right) = \bigcup_{n=1}^{\infty} f^{-1}_n \left( [-\infty, a) \right)
 .\end{align*}  $f(x) = \limsup f_n(x)$:
 \begin{align*}
  f(x) = \limsup f_n(x) = \displaystyle\lim_{N \to \infty} \underbrace{\sup_{n \geqslant N} f_n(x)}_{\text{ убывает } \forall x \in X} = \inf_{N \geqslant 1} \left( \sup_{n \geqslant N} f_n(x) \right)
 .\end{align*} Следовательно, $\limsup f_n$ измерим. $\liminf$ аналогично.

 Пусть теперь $f(x) = \lim f_n(x)$ существует. Тогда
 \begin{align*}
  f = \limsup f_n = \liminf f_n
 ,\end{align*} а они измеримы.
\end{proof}

Следующая теорема будет о том, что если взять какую попало измеримую функцию (но неотрицательную), то она приближается простыми.

