\documentclass[a4paper,14pt]{extarticle}
\usepackage[margin=0.5in]{geometry}
\usepackage[T1,T2A]{fontenc}
\usepackage[utf8]{inputenc}
\usepackage[russian]{babel}
\usepackage{amssymb}
\usepackage{amsthm}
\usepackage{amsmath}
\usepackage{textcomp}
\usepackage{gensymb}
\usepackage{systeme,mathtools}
\usepackage{indentfirst}
\usepackage{microtype}
\usepackage{hyperref}
\usepackage{enumerate}
\usepackage{enumitem}
\usepackage{array}
\usepackage{breqn}
\usepackage{verbatim}
\usepackage{listings}
\usepackage{color}
\usepackage{xcolor}
\usepackage{pgfplots}
\usepackage{transparent}
\usepackage{import}
\usepackage{xifthen}
\usepackage{pdfpages}
\pgfplotsset{compat=1.16}

\usepackage{tikz}
\usetikzlibrary{calc}
\usetikzlibrary{matrix}
\usetikzlibrary{patterns}
\usetikzlibrary{babel}
\usetikzlibrary{cd}

\newcommand{\incfig}[1]{%
 \def\svgwidth{\columnwidth}
 \import{./figures/}{#1.pdf_tex}
}

% counters and theorems
\newcounter{theoremCnt}

% numbered theorems
\theoremstyle{definition}
\newtheorem{df}[theoremCnt]{Определение}
\theoremstyle{plain}
\newtheorem{thm}[theoremCnt]{Теорема}
\theoremstyle{plain}
\newtheorem{lm}[theoremCnt]{Лемма}
\theoremstyle{plain}
\newtheorem{crly}[theoremCnt]{Следствие}
\theoremstyle{plain}
\newtheorem{prop}[theoremCnt]{Предложение}
\theoremstyle{definition}
\newtheorem{exmpl}[theoremCnt]{Пример}
\theoremstyle{definition}
\newtheorem{remrk}[theoremCnt]{Замечание}
\theoremstyle{definition}
\newtheorem{claim}[theoremCnt]{Утверждение}
\theoremstyle{definition}
\newtheorem{algorithm}[theoremCnt]{Алгоритм}
\theoremstyle{definition}
\newtheorem{problem}[theoremCnt]{Задача}

% non-numbered theorems
\theoremstyle{definition}
\newtheorem*{df*}{Определение}
\theoremstyle{plain}
\newtheorem*{thm*}{Теорема}
\theoremstyle{plain}
\newtheorem*{lm*}{Лемма}
\theoremstyle{plain}
\newtheorem*{crly*}{Следствие}
\theoremstyle{plain}
\newtheorem*{prop*}{Предложение}
\theoremstyle{definition}
\newtheorem*{exmpl*}{Пример}
\theoremstyle{definition}
\newtheorem*{remrk*}{Замечание}
\theoremstyle{definition}
\newtheorem*{claim*}{Утверждение}
\theoremstyle{definition}
\newtheorem*{algorithm*}{Алгоритм}
\theoremstyle{definition}
\newtheorem*{problem*}{Задача}

% small caps "proof" label
\addto\captionsrussian{\renewcommand\proofname{\normalfont\textsc{Доказательство}}}

% some aliases
\newcommand{\R}{\ensuremath \mathbb{R}}
\newcommand{\Q}{\ensuremath \mathbb{Q}}
\newcommand{\CC}{\ensuremath \mathbb{C}}
\newcommand{\Z}{\ensuremath \mathbb{Z}}
\newcommand{\N}{\ensuremath \mathbb{N}}
\newcommand{\eps}{\ensuremath \varepsilon}
\newcommand{\then}{\ensuremath \implies}

% project-local aliases
\newcommand{\A}{\ensuremath \mathfrak{A}}
\newcommand{\p}{\ensuremath \mathcal{P}}
\newcommand{\mua}{\ensuremath \mu^\ast}

\title{Матан СП 2 курс. Конспект лекций}
\author{Усачев Данила}

\begin{document}
\maketitle
\newpage
\section*{Организационное}
Теория: коллоквиума не будет, но можно его сделать по инициативе студентов в формате самопроверки.

О курсе: в этом семестре (третьем) в основном будет \textit{теория меры}. Программистам она нужна для того, чтобы понимать \textit{вероятность}. В этом семестре будет несколько \textit{глубоких} теорем (красивых фактов), каких не было в предыдущих семестрах. Глубокое осознание курса должно поднять слушателя на новый математический уровень.

\section{Суммирование неотрицательных семейств}

\begin{df}
 Пусть $ I $ --- это множество, для каждого $ i \in I $ сопоставлено число $ a_i \geqslant 0 $. Определим
 \begin{align*}
  \sum_{i \in I} a_i = \sup \left\{ \sum_{j \in F} a_j \mid F \subset I, F \text{ --- конечноe }\right\}
  .\end{align*}
\end{df}

\begin{thm}\
 \begin{itemize}
  \item Если $ \sum_{i \in I} a_i < \infty $, то существует не более чем счетное множество $ E \subset I $ такое, что $ a_i = 0 $ для всех $ i \in I \setminus E $.
  \item Eсли $ I $ не более, чем счётно, и $ I = \{i_k\}_{k \in \N} $, то
  \begin{align*}
   \sum_{i \in I} a_i = \sum_{k=1}^\infty a_{i_k}
   .\end{align*}
 \end{itemize}
\end{thm}
\begin{proof}\
 \begin{itemize}
  \item Пусть $ \sum_I a_i < \infty $. Тогда для любого $ N \in \N $ существует лишь конечное число $ i \in I $ таких, что $ a_i \geqslant 1/N $(иначе из них можно составить множество с неограниченной суммой). Значит,
  \begin{align*}
   \left\{ i \in I \mid a_i > 0 \right\} = \bigcup_{N \in \N} \left\{ i \in I \mid a_i \geqslant 1/N \right\} \subset E
   .\end{align*} Получается, существует множество $ E $ не более, чем счётное (как счётное объединение не более, чем счётных множеств).
  \item Есть тривиальное неравенство:
  \begin{align*}
   \sum_{k \in N} a_{i_k} \leqslant \sum_{i \in I} a_i
   .\end{align*} Потому что можно рассмотреть наборы вида $ F = \{i_1, i_2, \ldots, i_N\} $. Левая часть --- это супремум по таким $ F $ (супремум по частичным суммам) , а правая часть --- супремум по всем $ F $. Теперь докажем в обратную сторону. Пусть $ F \subset I $ конечноe. Тогда
  \begin{align*}
   \sum_{i \in F} a_i \leqslant \sum_{k=1}^N a_{i_k} \leqslant \sum_{k \in \N} a_{i_k}
   ,\end{align*} для некоторого $ N \in \N $. Тогда можно взять супремум:
  \begin{align*}
   \sup_F \sum_{i \in F} a_i \leqslant \sum_{k \in \N} a_{i_k}
   .\end{align*}
 \end{itemize}
\end{proof}
\begin{crly}
 Пусть $ I $ счётное множество, $ \{i_k\}_{k \in N} = \{i\} = I$ --- его нумерация двумя разными способами ($ \varphi \colon\; \N \to \N $ --- биекция). Тогда
 \begin{align*}
  \sum_{k \in N} a_{i_k} = \sum_{k \in \N} a_{i_{\varphi(k)}}
  .\end{align*}
\end{crly}
\begin{proof}
 Применить второй пункт предыдущей теоремы.
\end{proof}

\section{Наивная длина}

\begin{df}\
 \begin{itemize}
  \item Введём обозначения. Если $ X $ --- некоторое множество, то $ 2^X $ --- множество всех подмножеств множества $ X $.
  \item Если $ A \subset X $, то $ A^c = X \setminus A $.
  \item Если $ A, B \subset X $, то будем писать $ A \sqcup B $ вместо $ A \cup B $, если дополнительно известно, что $ A \cap B = \varnothing $.
  \item $ \R_+ = [0, +\infty) $.
 \end{itemize}
\end{df}
\begin{prop}
 Полезные формулы
 \begin{align*}
  \left( \bigcup_{i \in I} A_i \right)^c & = \bigcap_{i \in I} A_i^c \\
  \left( \bigcap_{i \in I} A_i \right)^c & = \bigcup_{i \in I} A_i^c
 \end{align*} для любого семейства $ i \in I $, $ A_i \subset X $.
\end{prop}
\begin{df}
 \textit{Наивная длина} --- это отображение $ \mu \colon\, 2^\R \to [0,\infty] $ такое, что
 \begin{itemize}
  \item $ \mu ([0, 1)) = 1 $
  \item $ \displaystyle \mu\left( \bigsqcup_{k=1}^\infty A_k \right) = \sum_{k=1}^\infty \mu( A_k ) \ \ \forall \{A_{k}\}_{k=1}^{\infty} \subset \R $
  \item $ \mu(A + x) = \mu(A) $ для любого $ A \subset \R $ и для любого $ x \in \R $ (где $ A + x = \{ a + x \mid a \in A \} $).
 \end{itemize}
\end{df}
\begin{thm}
 Наивной длины не существует.
\end{thm}
\begin{proof}
 Введём такое отношение эквивалентности: $ x \sim y $ для $ x,y \in [0, 1) $, если $ x-y \in \Q $. Тогда единичный полуинтервал бьётся на классы эквивалентности:
 \begin{align*}
  [0, 1) = \bigsqcup_{J \text{ --- класс экв. }} J
  .\end{align*} По аксиоме выбора из каждого класса $ J $ можно выбрать представителя $ x_J $. Обозначим
 \begin{align*}
  E = \{ x_J \mid J \text { --- класс экв. } \}
  .\end{align*} Наблюдение:
 \begin{align*}
  [0, 1) \subset \bigcup_{q \in [-1, 1] \cap \Q} (q + E)
  .\end{align*} Действительно, если $ x \in [0, 1) $, то существует класс $ J $ такой, что $ x - x_J = q \in Q $. Тем самым, $ x \in q + E $.

 Второе наблюдение:
 \begin{align*}
  \bigcup_{q \in \Q \cap [-1, 1]} (q + E) = \bigsqcup_{q \in \Q \cap [-1, 1]} (q + E)
  .\end{align*} Действительно, пусть $ (q_1 + E) \cap (q_2 + E) \ni x $. Тогда $ q_1 + x_{J_1} = x = q_2 + x_{J_2} $. Тогда
 \begin{align*}
  x_{J_1} - x_{J_2} = q_2 - q_1 \in \Q
  .\end{align*} Следовательно, $ x_{J_1} = x_{J_2} \;\Longrightarrow q_1 = q_2$. Значит, если они пересеклись, то они совпадают.

 Мы доказали два полезных включения:
 \begin{align*}
  [0, 1) \subset \bigsqcup_{q \in [-1, 1] \cap \Q} (q + E) \subset [-1, 2)
  .\end{align*} По аксиомам:
 \begin{align*}
  1 \leqslant & \sum_{q \in \Q \cap [-1, 1]} \mu(q + E) \leqslant 3 \\
  1 \leqslant & \sum_{q \in \Q \cap [-1, 1]} \mu(E) \leqslant 3
  .\end{align*} Но то, что по середине --- это либо $ 0 $, либо $ +\infty $. Противоречие!

 Для честности выведем монотонность по включению. Пусть $ A \subset B $. Тогда
 \begin{align*}
  \mu(B) = \mu(A \sqcup (B \setminus A)) = \mu(A) + \mu(B \setminus A) \geqslant \mu(A)
  .\end{align*} Показать $ \mu([-1, 2)) = 3 $ можно так:
 \begin{align*}
  \mu([-1, 2)) = \mu([-1, 0) \sqcup [0, 1) \sqcup [1, 2)) =\mu([-1, 0)) + \mu([0, 1)) + \mu([1, 2)) = 3 \cdot \mu([0, 1)) = 3
  .\end{align*}
\end{proof}

\section{Системы множеств и функции на них}

Итак, мы выяснили, что наивной длины нет. Классический способ обойти это (который мы будем изучать) --- определить длину не на всех множествах, а на \textit{хороших} множествах.

\begin{df}
 Пусть $ X $ --- множество. Набор подмножеств $ \mathcal{P} \subset 2^X $ --- \textit{полукольцо}, если
 \begin{enumerate}
  \item $ \varnothing \in P $
  \item $ P_1, P_2 \in \mathcal{P} \;\Longrightarrow\; P_1 \cap P_2 \in \mathcal{P} $
  \item $ P_1, P_2 \in \mathcal{P} \;\Longrightarrow\; P_1 \setminus P_2 = \bigsqcup_{j=1}^N Q_j, Q_j \in \mathcal{P}, N \in \N $
 \end{enumerate}
\end{df}
\begin{df}
 Пусть $ X $ --- множество. Набор подмножеств $ \A \in 2^X $ --- \textit{алгебра} множеств, если
 \begin{enumerate}
  \item $ \varnothing \in \A $
  \item $ A_1, A_2 \in \A \;\Longrightarrow\; A_1 \cap A_2 \in \A $
  \item $ A \in \A \;\Longrightarrow\; A^c \in \A $ (аксиома симметричности)
 \end{enumerate}
\end{df}
\begin{df}
 Набор подмножеств $ \A \subset 2^X $ --- \textit{$ \sigma $-алгебра}, если
 \begin{enumerate}
  \item $ \varnothing \in \A $
  \item $ \left\{ A_i \right\}_{i \in I} $, $ I $ --- счётное множество, $ A_i \in \A $. Тогда
  \begin{align*}
   \bigcap_{i \in I} A_i \in \A
   .\end{align*}
  \item $ A \in \A \;\Longrightarrow\; A^c \in \A $ (аксиома симметричности)
 \end{enumerate}
\end{df}
\begin{exmpl}\
 \begin{itemize}
  \item $ \sigma $-алгебра $ \;\Longrightarrow\; $ алгебра $ \;\Longrightarrow\; $ полукольцо. Последняя импликация: $ A_1 \setminus A_2 = A_1 \cap (A_2^c) $.
  \item $ 2^X $ --- $ \sigma $-алгербра
  \item $ \left\{ \varnothing, X \right\} $ --- $ \sigma $-алгебра.
  \item $ X = \R^2 $, $ \A = \left\{ E \subset X \mid E \text{ --- ограничено или } E^c \text { --- ограничено } \right\}  $. Тогда $ \A $ --- алгебра:
  \begin{enumerate}
   \item $ \varnothing \in \A $
   \item Пусть $ A, B \in \A $ и одно из них ограничено. Тогда $ A \cap B $ ограничено и $ A \cap B \in \A $. А если $ A^c, B^c $ ограничены, тогда $ (A \cap B)^c = A^c \cup B^c $ ограничено. Тогда $ A \cap B \in \A $.
   \item $ A \in \A \;\Longleftrightarrow\; A^c \in \A $
  \end{enumerate}
  При этом $ \A $ не является $ \sigma $-алгеброй:
  \begin{align*}
   \A \not\ni \R^2 \setminus \left\{ (0, x) \mid x \geqslant 0 \right\}  = \bigcap_{k=1}^\infty A_k, \quad A_k \in \A
   .\end{align*} Нужно взять
  \begin{align*}
   A_k = \R^2 \setminus \left\{ (0, x) \mid 0 \leqslant x \leqslant k \right\}
   .\end{align*}
  \item Если $ \A $ --- алгебра, то для любых $ A_1, \ldots, A_N \in \A $ верно $ \bigcup_1^N A_k \in \A $, так как
  \begin{align*}
   \bigcup_1^N A_k = \left( \bigcap_{k=1}^N A_k^c \right)^c
   .\end{align*}
  \item Аналогично, если $ \A $ --- $ \sigma $-алгебра, то для любого счетного $ I $ ($ A_i \in \A $ для любого $ i \in I $), то
  \begin{align*}
   \bigcup_{i\in I} A_i \in \A
   .\end{align*}
 \end{itemize}
\end{exmpl}

\begin{df}
 $ \mathcal{P}_1 = \left\{ [a, b) \mid a \leqslant b, a, b \in \R \right\} $ --- \textit{полукольцо ячеек}.
\end{df}
\begin{lm}
 $ \mathcal{P}_1 $ --- полукольцо.
\end{lm}
\begin{proof}\
 \begin{enumerate}
  \item $ \varnothing \in \mathcal{P_1} $, так как $ \varnothing = [0, 0) $.
  \item $$
   [a_1, b_1) \cap [a_2, b_2) =
   \begin{cases}
    \varnothing \text{, если } \max(a_1, a_2) \geqslant \min(b_1, b_2) \\
    [\max(a_1, a_2), \min(b_1, b_2) )
   \end{cases}
  $$
  \item Надо разобрать случаи. Но в каждом из них получается ячейка (иногда две ячейки).
 \end{enumerate}
\end{proof}
\begin{remrk}
 Полукольцо ячеек $ \mathcal{P}_1 $ не является алгеброй. Например, $ \R \setminus [0, 1) \neq [a, b)$ --- не ячейка.
\end{remrk}
\begin{lm}
 Если $ \A_i $ --- $ \sigma $-алгебра, $ i \in I $, $ I $ --- произвольное множество. Тогда $ \A = \bigcap_{i \in I} \A_i $ --- $ \sigma $-алгебра.
\end{lm}
\begin{proof}\
 \begin{enumerate}
  \item $ \varnothing \in \A $, так как $ \varnothing \in \A_i $ для всех $ i \in I $.
  \item $ A \in \A \;\Longrightarrow\; A \in \A_i  \forall i \in I \;\Longrightarrow\; A^c \in \A_i \forall i \in I \;\Longrightarrow\; A^c \in \bigcup_{i \in I} \A_i = \A$
  \item $ A_j \in \A \;\Longrightarrow\; A_j \in \A_i, \forall i, j  \;\Longrightarrow\; \bigcap_j A_j \in
   \A_i \;\Longrightarrow\; \bigcap_j A_j \in \A$
 \end{enumerate}
\end{proof}
\begin{lm}
 Если $ E \subset 2^X $, то существует единственная $ \A_E $ --- наименьшая по включению $ \sigma $-алгебра, содержащая $ E $.
\end{lm}
\begin{proof}
 Есть просто формула:
 \begin{align*}
  \A_E := \bigcap_{E \subset \A \text{---} \sigma \text{-алгебра}} \A
  .\end{align*} Пересечение не по пустому множеству, так как $ E \subset 2^X $ --- $ \sigma $-алгебра.
\end{proof}
\begin{df}
 Пусть $ \mathcal{P} $ --- полукольцо в $ X $ и $ \mu \colon\; \mathcal{P} \to [0, +\infty] $. $ \mu $ называется конечно аддитивной, если для любых $ P_1, P_2, \ldots, P_N $ таких, что
 \begin{align*}
  P = \bigsqcup_{k=1}^N P_k \in \mathcal{P}
 \end{align*} выполняется равенство
 \begin{align*}
  \mu\left( \bigsqcup_{k=1}^N P_k \right) = \sum_{k=1}^{N} \mu(P_k)
  .\end{align*} $ \mu $ счётно аддитивна, если для любого набора $ \left\{ P_i \right\}_{i \in I} $, $ I $ счётно,
 \begin{align*}
  P = \bigsqcup_{i \in I} P_i \subset \mathcal{P}
 \end{align*} выполняется равенство:
 \begin{align*}
  \mu\left( \bigsqcup_{i \in I} P_i \right) = \sum_{i \in I} \mu(P_i)
  .\end{align*}
\end{df}
\begin{df}
 Пусть $ \A $ --- $ \sigma $-алгебра в $ X $, $ \mu \colon\, \A \to [0, +\infty] $ --- \textit{мера}, если
 \begin{align*}
  \mu\left( \bigsqcup_{i \in I} A_i \right) = \sum_{i\in I}^{  } \mu(A_i)
 \end{align*} для любого дизъюнктного счётного набора множеств $ A_i \in \A$.
\end{df}
\begin{exmpl}\
 \begin{itemize}
  \item $ \mu \equiv 0 $, $ \A = 2^X $ (или $ \A $ любая)
  \item $ \mu \equiv +\infty $, $ \A = 2^X $ (или $ \A $ любая)
  \item Пусть $ X $ --- множество, $ E \subset X $. Определим \textit{считающую меру}:
  \begin{align*}
   \mu_E(A) =
   \begin{cases}
    \#(A \cap E), \text{ если  } (A \cap E) \text { конечно }, \\
    +\infty \text { иначе }
   \end{cases}
   .\end{align*} $ \A = 2^X $
  \item $ X = \R $, $$ \delta_0(A) =
   \begin{cases}
    1, \text{ если } 0 \in A \\
    0, \text { иначе }
   \end{cases}$$ --- \textit{дельта мера Дирака}.
  \item $ X = \R $, $  \left\{ a_k \right\}_{k=1}^\infty \subset \R_+ $, $ \delta_{x_k} $ --- дельта меры Дирака в точках $ x_k $. Тогда
  \begin{align*}
   \mu = \sum_{k=1}^\infty a_k \delta_{x_k}
  \end{align*} --- тоже мера. Вообще, в целом можно брать линейную комбинацию мер.
 \end{itemize}
\end{exmpl}
\begin{exmpl}
 $ \mu_0([a, b)) = b - a $ на $ \mathcal{P}_1 $. Тогда $ \mu_0 $ конечно аддитивна на $ \mathcal{P}_1 $: если $ [a, b)  = \bigsqcup_1^N [a_k, b_k) $, то
 \begin{align*}
  b - a = \sum_1^N (b_k - a_k) = \sum_1^N \mu_0([a_k, b_k))
  .\end{align*}
\end{exmpl}
\begin{lm}[о подчинённом разбиении]
 Пусть $ X $ --- множество, $ \mathcal{P} $ --- полукольцо, $ P_1, \ldots, P_N \in \mathcal{P} $, $ P \in \mathcal{P} $. Тогда:
 \begin{enumerate}
  \item $ \displaystyle P \setminus \bigcup_{k=1}^N P_k = \bigsqcup_{j=1}^M Q_j $, где $ Q_j \in \mathcal{P} $
  \item $ \displaystyle \bigcup_{k=1}^N P_k = \bigsqcup_{k=1}^N \bigsqcup_{j=1}^{M_k} Q_{k,j} $, где $ Q_{k,j} \in \mathcal{P} $ и $ Q_{k,j} \subset P_k $
 \end{enumerate}
\end{lm}
\begin{proof}\
 \begin{enumerate}
  \item По индукции. $ N = 1$ --- аксиома 3 полукольца. Пусть
  \begin{align*}
   P \setminus \bigcup_{k=1}^{N+1} P_k = \left( P \setminus \bigcup_{k=1}^N P_k \right) \setminus P_{N+1} =  \bigsqcup_{j=1}^{M_N} \underbrace{\left( Q_j \setminus P_{N+1} \right)}_{\text{уже знаем по базе индукции}}
   .\end{align*}
  \item
  \begin{align*}
    & P_1 = P_1 = Q_{1,1} \\
    & P_1 \cup P_2 = P_1 \sqcup (P_2 \setminus P_1) = Q_{1,1} \sqcup (Q_{2,1} \sqcup \ldots \sqcup Q_{2, M_2}) \\
    & P_1 \cup P_2 \cup P_3 = \underbrace{(P_1 \cup P_2)}_{\text{ знаем }} \sqcup (P_3 \setminus (P_1 \cup P_2)) \text{ <- знаем из пункта 1 } \\
    & P_1 \cup P_2 \cup \dots \cup P_N = \underbrace{(\bigcup_{k = 1}^{N - 1} P_k)}_{\text{ знаем }} \sqcup (P_N \setminus (\bigcup_{k = 1}^{N - 1} P_k)) \text{ <- знаем из пункта 1 }
   .\end{align*}
 \end{enumerate}
\end{proof}
\begin{crly}
 Пусть $ \mu \colon\, \mathcal{P} \rightarrow [0, +\infty]$ конечно-аддитивна. Пусть $ \bigsqcup_1^N P_k \subset P $, $ P_k, P \in \mathcal{P} $. Тогда
 \begin{align*}
  \sum_{k=1}^{N}  \mu(P_k) \leqslant \mu(P)
  .\end{align*}
\end{crly}
\begin{proof}
 $ P = \bigsqcup_1^N P_k \sqcup \bigsqcup_{j=1}^M Q_j $.
 \begin{align*}
  P = \bigsqcup_1^N P_k \sqcup \bigsqcup_{j=1}^M Q_j \\
  \mu(P) = \sum \mu(P_k) + \sum \mu(Q_j) \geqslant \sum \mu(P_k)
 \end{align*}
\end{proof}
\begin{crly}
 Если $ \mathcal{P} $, $ \mu $ как в предыдущем следствии и $ P \subset \bigcup_{k=1}^N P_k $, где $ P, P_k \in \mathcal{P} $. Тогда
 \begin{align*}
  \mu(P) \leqslant \sum_{k=1}^N \mu(P_k)
  .\end{align*}
\end{crly}
\begin{proof}
 \begin{align*}
  P = \bigcup_1^N (P_k \cap P) = \bigsqcup_{k=1}^N \bigsqcup_{j=1}^{M_k} Q_{k,j} \\
  \mu(P) = \sum_{k,j}^{} \mu(Q_{k,j}) = \sum_{k=1}^{N} \underbrace{\sum_j \mu(Q_{k,j})}_{\leqslant \mu(P_k) \text{ по пред. следствию }} \leqslant \sum_{k=1}^{N} \mu(P_k)
  .\end{align*}
\end{proof}
\begin{thm}[счётная аддитивность длины на полукольце]
 Функция множества $ \mu_0 \colon\, [a, b) \mapsto b - a $ на полукольце ячеек $ \mathcal{P}_1 $ является счётно-аддитивной.
\end{thm}
\begin{proof}
 Пусть $ [a, b) = \bigsqcup_{i \in I} [a_i, b_i) $, где $ I $ --- счётное множество. Возьмём $ F \subset I $ --- конечное подмножество. Тогда
 \begin{align*}
  \mu\left( \bigsqcup_{i \in F} [a_i, b_i) \right) = \sum_{i \in F}^{} \mu( [a_i, b_i) ) \\
  \mu\left( \bigsqcup_{i \in F} [a_i, b_i) \right) \leqslant \mu([a, b)) \text { <- следствие 22 }
  .\end{align*} Теперь надо доказать неравенство в обратную сторону. Для любого $ \varepsilon > 0 $ положим $ \tilde a_{i_k} = a_{i_k} - \varepsilon/2^k $ и $ \tilde b_{i_k} = b_{i_k} $, где $ \left\{ [a_{i_k}, b_{i_k}) \right\}  $ --- как-то занумерованное семейство $ \left\{ [a_i, b_i) \right\}_{i \in I}  $. Тогда для любого $ t $: $ a_i < t < b_i $:
 \begin{align*}
  [a_i, t] \subset \bigcup_{k \in \N} (\tilde a_{i_k}, \tilde b_{i_k})
  .\end{align*} Но слева \textbf{компактное} множество. Тогда, существует $ F $ конечное, такое что
 \begin{align*}
  [a_i, t) \subset \bigcup_{i_k \in F} [\tilde a_{i_k}, \tilde b_{i_k})
 \end{align*} (поменяли скобки). По предыдущему следствию:
 \begin{align*}
  t - a_i = & \mu([a_i, t)) \leqslant \sum_{i_k \in F}^{} \mu([\tilde a_{i_k}, \tilde b_{i_k})) \leqslant \sum_{i_k \in I}^{} (b_{i_k} - a_{i_k}) + \sum_{k \in \N} \frac{\varepsilon}{2^k} \leqslant \\
  \leqslant & \sum_{i \in I} \mu([a_i, b_i)) + \varepsilon
 \end{align*} для любого $ t < b_i $ и $ \varepsilon > 0 $. Устремим $ t \to b_i $ и $ \varepsilon \to 0 $.
\end{proof}

% 2022.09.09 lecture 2

\section{Внешние меры, $ \sigma $-алгебра Каратеодори}

Наша задача: продолжить функцию длины с ячеек на более широкий класс множеств. В самом начале этого пути есть метафизический вопрос: разные вещи являются счётно-аддитивными: масса, длина, ... Возможно окажется, что нам много раз придётся делать такое продолжение. Оказывается, что для всех этих разных явлений есть одна общая математическая конструкция.

\begin{df}
 Внешняя мера --- это $ \mu^\ast \colon\; 2^X \to [0, \infty] $ такое, что
 \begin{enumerate}
  \item $ \mu^\ast(\varnothing) = 0 $
  \item Если $ E \subset \displaystyle \bigcup_{k=1}^\infty A_k $, то $ \displaystyle \mu^\ast\left( E \right) \leqslant \sum_{k=1}^\infty \mu^\ast(A_k) $ --- полуаддитивность
 \end{enumerate}
\end{df}
\begin{exmpl}[внешняя мера, порождённая функцией множества]
 Пусть $ S \subset 2^X $ --- любое семейство подмножеств, $ \varnothing \in S $ и есть $ \mu_0 \colon\; S \to [0, \infty] $ --- любое отображение, $ \mu_0(\varnothing) = 0 $. Тогда:
 \begin{align*}
  \mu^\ast(A) = \inf \left\{ \sum_{k=1}^\infty \mu_0(A_k) \mid A \subset \bigcup_{k=1}^\infty A_k  \right\}
  .\end{align*} При этом, предполагается, что $ \inf \varnothing = +\infty $, то есть, если не существуют такие $ A_k \in S $ такое, что $ A \subset \bigcup A_k $, то $ \mu^\ast (A) = +\infty $.
\end{exmpl}
\begin{proof}\
 \begin{enumerate}
  \item $ \mu^\ast(\varnothing) = 0 $ --- ясно
  \item Пусть есть множества $ \left\{ E_k \right\}_{k=1}^\infty \subset 2^X$. Тогда нужно проверить, что
  \begin{align*}
   \mu^\ast\left( \bigcup E_k \right) \leqslant \sum_{k=1}^{\infty} \mu^\ast(E_k)
   .\end{align*} Если существует $ k_0 $ такое, что $ \mu^\ast(E_{k_0}) = \infty $, то доказывать нечего. Если же $ \mu^\ast(E_k) < \infty $ для любого $ k \in \N $, то тогда мы можем выбрать $ A_{kj} \in S $ такие, что $ \bigcup_j A_{kj} \supset E_k $ и $$ \mu^\ast(E_k) + \frac{\varepsilon}{2^k} \geqslant \sum_{k=1}^{\infty} \mu_0(A_{kj}) $$ Теперь возьмём и просуммируем:
  \begin{align*}
   \mu^\ast\left( \bigcup E_k \right) \leqslant \sum_{k,j} \mu_0(A_{k,j}) \leqslant \sum_k \sum_j \mu_0(A_{kj}) \leqslant \sum_{k} \mu^\ast(E_k) + \frac{\varepsilon}{2^k} \leqslant \varepsilon + \sum_k \mu^\ast(E_k)
   .\end{align*} Это по определению просто потому, что $ \bigcup E_k \subset \bigcup_{k,j} A_{kj} $. Теперь можно устремить $ \varepsilon $ к нулю (так как это верно для любого $ \varepsilon $) и получить нужное равенство.
 \end{enumerate}
\end{proof}
Получается, внешних мер очень много! Можно взять любую функцию и сделать из неё меру.

\begin{df}[$ \sigma $-алгебра, порождённая внешней мерой]
 Пусть есть внешняя мера $ \mu^\ast \colon\; 2^X \to [0,\infty] $. Тогда пусть \begin{align*}
  \A^\ast = \left\{ A \subset X \mid \forall E \subset X \colon\; \mu^\ast (E \cap A) + \mu^\ast(E \cap A^c) \right\}
  .\end{align*}
 Это и называется \textit{$ \sigma $-алгебра Каратеодори}.
\end{df}

\begin{lm}
 $ \A^\ast $ --- это алгебра.
\end{lm}
\begin{proof}\
 \begin{enumerate}
  \item $ \varnothing \in \A^\ast $ так как $ \mu^\ast(E) = \mu^\ast(E \cap \varnothing) + \mu^\ast(E \cap X) = 0 + \mu^\ast(E) $.
  \setcounter{enumi}{2}
  \item $ A \in \A^\ast \iff A^c \in \A $
  \setcounter{enumi}{1}
  \item $ \mu^\ast(E) \leqslant \mu^\ast(E \cap (A \cap B)) + \mu^\ast(E \cap (A \cap B)^c) $ --- полуаддитивность, т.к. $ \mu^\ast(E) = \mu^\ast(E_1 \cup E_2 \cup \ldots) \leqslant \sum_1^\infty \mu^\ast(E_k)$. Можно взять $ E_1 = E \cap (A \cap B) $, $ E_2 = E \cap (A \cap B)^c $ и $ E_3 = E_4 = \ldots = \varnothing $.

  Докажем в обратную сторону:
  \begin{align*}
   \mu^\ast(E) \leqslant \mu^\ast(E \cap (A \cap B)) + \mu^\ast(E \cap (A \cap B)^c) \leqslant \\
   \leqslant \mu^\ast(E \cap A \cap B) + \mu^\ast(E \setminus A) + \mu^\ast((E \setminus A) \cap B^c)
   .\end{align*} Так как
  \begin{align*}
   E \cap (A \cap B)^c = E \cap (A^c \cup B^c) = E \cap A^c \cup E \cap B^c = (E \setminus A) \cup (E \cap A \cap B^c \sqcup (E \ A) \cap B^c) = \\
   = (E \setminus A) \cup (E \cap A) \cap B^c
   .\end{align*} Далее,
  \begin{align*}
   \mu^\ast(E \cap A \cap B) + \mu^\ast(E \setminus A) + \mu^\ast((E \setminus A) \cap B^c) = \\
   = \mu^\ast(E \cap A) + \mu^\ast(E \setminus A)
   .\end{align*} Сгруппировали первый и третий член. Это равенство верно, так как $ B \in \A^\ast $. Потом
  \begin{align*}
   \mu^\ast(E \cap A) + \mu^\ast(E \setminus A) = \mu^\ast(E)
   .\end{align*} Так как $ A \in \A^\ast $. В итоге получили, что все неравенства --- равенства. В частности первое:
  \begin{align*}
   \mu^\ast(E) = \mu^\ast(E \cap (A \cap B)) + \mu^\ast(E \cap (A \cap B)^c)
   .\end{align*} Следовательно, $ A \cap B \in \A^\ast $
 \end{enumerate}
 Значит, $ \A^\ast $ --- $ \sigma $-алгебра.
\end{proof}
\begin{lm}
 $ \mu^\ast $ --- конечно-аддитивна на $ \A^\ast $:
 \begin{align*}
  \mu^\ast(A \sqcup B) = \mu^\ast(A) + \mu^\ast(B)
 \end{align*} для любых $ A, B \in \A^\ast,\; A \cap B = \varnothing $. Более того, для любого $ E \subset X $ верно
 \begin{align*}
  \mu^\ast(E \cap (A \sqcup B)) = \mu^\ast(E \cap A) + \mu^\ast(E \cap B)
  .\end{align*}
\end{lm}
\begin{proof}
 Так как $ A \in \A^\ast $, то
 \begin{align*}
  \mu^\ast(A \sqcup B) = \mu^\ast((A \sqcup B) \cap A) + \mu^\ast((A \sqcup B) \cap A^c) = \\
  = \mu^\ast(A) + \mu^\ast(B)
  .\end{align*} Теперь к <<более того>>:
 \begin{align*}
  \mu^\ast(E \cap (A \sqcup B)) = \mu^\ast(E \cap (A \sqcup B) \cap A) + \mu^\ast(E \cap (A \sqcup B) \cap A^c) = \\
  = \mu^\ast(E \cap A) + \mu^\ast(E \cap B)
  .\end{align*}
\end{proof}
\begin{lm}
 $ \A^\ast $ --- это $ \sigma $-алгебра.
\end{lm}
\begin{proof}
 Нужно доказать только свойство с счетным пересечением. Возьмём любой набор $ \left\{ A_k \right\}_{k=1}^\infty \subset \A^\ast $. Достаточно доказать, что
 \begin{align*}
  \bigcup A_k \in \A^\ast
  .\end{align*} (для этого нужно воспользоваться формулами де Моргана). Рассмотрим такие множества $ \tilde A_1 = A_1 $, $ \tilde A_2 = A_2 \setminus A_1 $, $ \tilde A_3 = A_3 \setminus (A_1 \cup A_2)$. Можно сказать, что
 \begin{align*}
  \bigcup_{k=1}^\infty A_k = \bigsqcup_{k=1}^\infty \tilde A_k
  .\end{align*} При этом все $ \tilde A_k \in \A^\ast $, так как $ \A^\ast $ это алгебра. Вывод: можно считать, $ A_k $ --- дизъюнктны и проверять, что $ \displaystyle \bigsqcup_{k=1}^\infty A_k \in \A^\ast $. Давайте проверим. Возьмём произвольное $ E \subset X $.
 \begin{align*}
  \mu^\ast(E) \leqslant \mu^\ast\left( E \cap \left( \bigsqcup_{k=1}^\infty A_k \right) \right) + \mu^\ast\left( E \cap \left( \bigsqcup_{k=1}^\infty A_k \right)^c \right) \leqslant \\
  \leqslant \sum_{k=1}^{\infty} \mu^\ast(E \cap A_k) + \mu^\ast\left( E \cap \left( \bigsqcup_1^N A_k \right)^c \right)
  .\end{align*} Неравенство по определению внешней меры. Неравенство для второго слагаемого: если $ C_1 \subset C_2 $, то $ \mu^\ast(C_1) \leqslant \mu^\ast(C_2) $ (пока не доказано).
 Далее,
 \begin{align*}
  \mu^\ast(E) \leqslant \mu^\ast\left( E \cap \left( \bigsqcup_{k=1}^N A_k \right) \right) + \mu^\ast\left( E \cap \left( \bigsqcup_{k=1}^N A_k \right)^c\right) + \sum_{k=N+1}^{\infty} \mu^\ast(E \cap A_k) = \\
  = \mu^\ast(E) + \sum_{k=N+1}^{\infty} \mu^\ast(E \cap A_k)
 \end{align*} по усиленной конечной аддитивности. Осталось проверить, что
 \begin{align*}
  \sum_{k=N+1}^{\infty} \mu^\ast(E \cap A_k) \to 0 \text{ при } n \to \infty
  .\end{align*} Это эквивалентно тому, что ряд
 \begin{align*}
  \sum_{k=1}^{\infty}  \mu^\ast (E \cap A_k) < \infty \iff \sup_{N \geqslant 1} \sum_1^N \mu^\ast(E \cap A_k) < \infty
  .\end{align*} Но
 \begin{align*}
  \sum_{1}^{N} \mu^\ast(E \cap A_k) = \mu^\ast\left(E \cap \left(\bigsqcup_{k=1}^N A_k\right)\right) \leqslant \mu^\ast(E)
 \end{align*} по монотонности внешней меры. Если $ \mu^\ast(E) $ конечно, то всё ОК, а иначе доказывать и нечего ($ \mu^\ast(E) = +\infty \then \displaystyle \mu^\ast\left(E \cap \left( \bigsqcup A_k \right)\right)  + \mu^\ast\left(E \cap \left( \bigsqcup A_k \right)^c\right) \leqslant \mu^\ast(E)$). Лемма доказана по модулю монотонности внешней меры.
\end{proof}
Самое сложное техническое место мы прошли.
\begin{lm}
 $ \mu^\ast \rvert_{\A^\ast} $ --- это мера.
\end{lm}
\begin{proof}
 Нужно доказать счётную аддитивность. Пусть $ A_k \in \A^\ast $, $ A = \bigsqcup_{k=1}^\infty A_k $. Напишем:
 \begin{align*}
  \sum_{k=1}^{N} \mu^\ast(A_k) = \mu^\ast\left( \bigsqcup_{k=1}^N A_k \right) \leqslant \mu^\ast(A) \leqslant \sum_{k=1}^{\infty}  \mu^\ast(A_k)
  .\end{align*} Переходя в левой части к пределу по $ N \to \infty $, получаем равенство.
\end{proof}
Остался один долг: нужно доказать монотонность внешней меры.
\begin{lm}
 Пусть $ \mu^\ast \colon\; 2^x \to [0, \infty] $ --- внешняя мера и $ C_1 \subset C_2 \subset X $. Тогда
 \begin{align*}
  \mu^\ast(C_1) \leqslant \mu^\ast(C_2)
  .\end{align*}
\end{lm}
\begin{proof}
 Очевидно по аксиоме 2.
\end{proof}
\section{Теорема Каратеодори о стандартном продолжении}
\begin{thm}[Каратеодори о стандартном продолжении]\
 Пусть $ \p $ --- полукольцо подмножеств $ X $, $ \mu_0 \colon\; \p \to [0, \infty] $ --- счётно-аддитивная функция на $ \p $, а $ \mu^\ast $ --- порождённая ей внешняя мера:
 \begin{align*}
  \mu^\ast(E) = \inf \left\{ \sum \mu_0(P_k) \mid E \subset \bigcup P_k,\; P_k \in \p \right\}, \quad E \subset X
  .\end{align*} Пусть $ \A^\ast $ --- $ \sigma $-алгебра Каратеодори, порождённая внешней мерой $ \mu^\ast $. Обозначим за $ \mu $ меру $ \mu^\ast \rvert_{\A^\ast} $ на $ \sigma $-алгебре $ \A^\ast $. Тогда:
 \begin{enumerate}
  \item $ \p \subset \A^\ast $
  \item $ \mu(P) = \mu_0(P) $ для каждого $ P \in \p $.
  \item Если $ \nu $ --- мера на $ \sigma $-алгебре $ \A_\nu \supset \p $, такая что $ \mu(P) = \mu_0(P) $ для любого $ P \in \p $, то для любого $ A \in \A^\ast \cap \A_\nu $ такого, что $ \mu(A) < +\infty $ имеет место равенство
  \begin{align*}
   \mu(A) = \nu(A)
   .\end{align*}
 \end{enumerate}
\end{thm}
Про что говорит нам теорема Каратеодори? $ \mu_0 $ может быть продолжена до меры единственным образом.
\begin{proof}\
 \begin{enumerate}
  \item Пусть $ P \in \p $. Чтобы показать, что $ P \in \A^\ast $, необходимо и достаточно проверить, что для каждого $ E \subset X $:
  \begin{align*}
   \mu^\ast(E) = \mu^\ast(E \cap P) + \mu^\ast(E \cap P^c)
   .\end{align*} У нас $ \leqslant $ есть. Если $ \mu^\ast(E) = +\infty $ то и $ \geqslant $ есть. Пусть теперь $ \mu^\ast(E) < +\infty $. Тогда для любого $ \varepsilon > 0 $ существуют $ P_k \in \p $ такие, что
  \begin{align*}
   E \subset \bigcup_{k=1}^\infty P_k
  \end{align*} и верно
  \begin{align*}
   \mu^\ast(E) + \varepsilon \geqslant \sum_{k=1}^{\infty} \mu_0(P_k)
   .\end{align*} Напишем
  \begin{align*}
   \mu^\ast(E) \leqslant \mua(E \cap P) + \mua(E \cap P^c) \leqslant \\
   \leqslant^{\text{***}} \sum_{k=1}^{\infty} \mu_0(P_k \cap P) + \sum_{k=1}^{\infty} \sum_{j=1}^{N_k}   \mu_0 (Q_{kj} ) \leqslant^{\text{****}} \sum_{k=1}^{\infty} \mu_0(P_k) \leqslant \mua(E) + \varepsilon
   .\end{align*} (***): $$ E \cap P^c \subset \bigcup_k (P_k \cap P^c) = \bigcup_{k=1}^\infty \bigsqcup_{j=1}^{N_k} Q_{kj} \cap P^c $$ --- третья аксиома полукольца. Из-за этого
  \begin{align*}
   \mua(E \cap P^c) \leqslant \sum_{k=1}^{\infty} \sum_{j=1}^{N_k}  \mu_0(Q_{kj})
  \end{align*} по определению $ \mua $.

  Теперь (****):
  \begin{align*}
   \sum_{k=1}^{N_k}  \mu(Q_{kj}) + \mu_0(P_k \cap P) \leqslant \mu_0(P_k)
  \end{align*} потому что
  \begin{align*}
   \left( \bigsqcup Q_{kj} \right) \sqcup \left( P_k \cap P \right) \subset P_k
   .\end{align*}
  \item Докажем, что $ \mu(P) = \mu_0(P) $ для $ P \in \p $. Нужно проверить, верно ли, что
  \begin{align*}
   \mu_0(P) = \inf \left\{ \sum_{k=1}^{\infty} \mu_0(P_k) \mid P \subset \bigcup_{P_k \in \p} P_k \right\} = \mu(P)
   .\end{align*} Так как $ P \subset P \in \p $, то $ \mu(P) \leqslant \mu_0(P) $. Пусть $ P \subset \displaystyle\bigcup_{k=1}^\infty P_k $, тогда $$
   P = \bigcup_{k=1}^\infty (P_k \cap P) = \bigsqcup_{k=1}^\infty \bigsqcup_{j=1}^{N_k} Q_{kj}
  $$ по лемме о подчинённом разбиении. Теперь применим $ \mu_0 $ и воспользуемся её счётно-аддитивностью:
  \begin{align*}
   \mu_0(P) = \sum_{k=1}^{\infty} \sum_{j=1}^{N_k} \mu_0(Q_{kj}) \leqslant \sum_{k=1}^{\infty} \mu_0(P_k)
   .\end{align*} Перейдём к инфимуму и получим, что
  \begin{align*}
   \mu_0(P) \leqslant \inf \left\{ \sum_{k=1}^{\infty} \mu_0(P_k) \right\}  = \mua(P) = \mu(P)
   .\end{align*}
  \item На следующей лекции.
 \end{enumerate}
\end{proof}

% 2022.09.16 lecture 3

Последняя часть: единственность меры в теореме Каратеодори.

\begin{proof}
 Пусть $A \in \A^{\ast} \cap \A_{\nu}$ такое, что $\mu(A) < \infty$. Тогда надо показать $\nu(A) = \mu(A)$. Сначала покажем, что
 \begin{align*}
  \mu(A) = \mua(A) = \inf \left\{ \sum_{k=1}^{\infty} \mu(P_k) \mid P_k \in \p \colon\; A \subset \bigcup_{k=1}^{\infty} P_k \right\}
  .\end{align*} С другой стороны, возьмём любой набор $\left\{ P_k \right\}$ такой, что $\bigcup_{k=1}^{\infty} P_k \supset A$. Тогда:
 \begin{align*}
  \nu(A) \underbrace{\leqslant}_{\text{ так как оба множества } \in \A_{\nu}} \nu \left( \bigcup_{k=1}^{\infty} P_k \right) = \nu( \bigsqcup_{k=1}^{\infty}  \bigsqcup_{j=1}^{N_k} Q_{kj}) = \sum \nu(Q_{kj}) = \sum_{k=1}^{\infty} \underbrace{\sum_{j=1}^{N_k} \nu(Q_{kj})}_{\leqslant \nu(P_k)} \leqslant \\
  \leqslant \nu(A) \leqslant \sum_{k=1}^{\infty}  \nu(P_k) = \sum_{k=1}^{\infty}  \mu(P_k)
  .\end{align*} Возьмём инфимум и получим:
 \begin{align*}
  \nu(A) \leqslant \mu(A)
  .\end{align*} Это верно для любого $A \in \A^{\ast} \cap \A_{\nu}$.

 Теперь возьмём $P \in \p$ и $A \in \A^{\ast} \cap \A_{\nu}$. Посмотрим на равенство
 \begin{align*}
  \mu(P) = \mu(P \cap A^{c}) + \mu(P \cap A) \geqslant \nu(P \cap A^{c}) + \nu(P \cap A) = \nu(P) = \mu(P)
  .\end{align*} Значит, если $\mu(P)$ конечно, то все слагаемые конечные, и $\mu(P \cap A) = \nu(P \cap A)$ и $\mu(P \cap A^{c}) = \nu(P \cap A^{c})$: если бы одно из неравенств $\mu(P \cap A) \geqslant \nu(P \cap A)$, $\mu(P \cap A^{c}) \geqslant \nu(P \cap A^{c})$ было бы строгим, мы пришли бы к противоречию.

 Так как $\mu(A) < \infty$, то $\forall \eps > 0 \colon\; \exists  \{P_{k}\}_{k=1}^{\infty} \subset \p,\; A \subset \bigcup_{k=1}^{\infty} P_k \colon\; $
 \begin{align*}
  \mu(A) + \eps \geqslant \sum_{k=1}^{\infty} \mu(P_k)
  .\end{align*} В частности,
 \begin{align*}
  \mu(P_k) < \infty, \quad \forall k \in \N
  .\end{align*}  Тогда
 \begin{align*}
  A = \bigcup_{k=1}^{\infty} (P_k \cap A)
  .\end{align*} Более того, используя лемму  о подчинённом разбиении, можно считать, что
 \begin{align*}
  A = \bigsqcup_{k=1}^{\infty} \bigsqcup_{j=1}^{N_k} (Q_{kj} \cap A)
 \end{align*}
 Теперь
 \begin{align*}
  \mu(A) = \sum_{k,j} \mu(Q_{kj} \cap A) = \sum_{kj} \nu(Q_{kj} \cap A) = \nu(A)
  .\end{align*} Доказательство окончено.
\end{proof}
\begin{df}
 $\mu_0 \colon\; \p \to [0, \infty]$ --- \textit{$\sigma$-конечна} на полукольце $\p \subset 2^{X}$, если существуют $\{P_k\}_{k=1}^{\infty} \subset \p $ такое, что
 \begin{align*}
  X = \bigsqcup_{k=1}^{\infty} P_k
 \end{align*} и $\mu_0(P_k) < \infty$ для любого $k$.
\end{df}
\begin{df}
 Пусть $\mu$ --- мера на $\sigma$-алгебре $\A \subset 2^{X}$. $\mu$ называется \textit{$\sigma$-конечной}, если существует счётный набор $\{A_k\}_{k=1}^{\infty} \subset \A $ такой, что $X = \bigsqcup_{k=1}^{\infty} A_k $, $\mu(A_k) < \infty$.
\end{df}
\begin{crly}
 Если в формулировке теоремы Каратеодори $\mu_0$ --- $\sigma$-конечна, то $\mu(A) = \nu(A)$ для любого $A \in \A^{\ast} \cap \A_{\nu}$ (даже если $\nu(A) = \infty$).
\end{crly}
\begin{proof}
 \begin{align*}
  A = \bigsqcup_{k=1}^{\infty} (A \cap P_k)
  ,\end{align*} где $P_k$ --- из определения $\sigma$-конечности. По условию  $\mu(P_k) < \infty$. Тогда
 \begin{align*}
  \mu(A) = \sum_{k=1}^{\infty} \mu(A \cap P_k) = \sum_{k=1}^{\infty} \nu(A \cap P_k) = \nu(A)
  .\end{align*}
\end{proof}
\begin{df}
 Пусть $\mu_0 \colon\; \p_1 \to [0, \infty)$ такова, что
 \begin{align*}
  \mu_0 \colon\; [a, b) \to b - a
  .\end{align*} Пусть $\lambda_1$ --- это мера  $\mu$ из теоремы Каратеодори для $\mu_0$, $\A_{\lambda_1}$ --- это $\sigma$-алгебра $\A^{\ast}$ из теоремы Каратеодори. Тогда $\lambda_1$ --- это \textit{мера Лебега}, $\A_{\lambda_1}$ --- \textit{$\sigma$-алгебра Лебега}. А множества из $\A_{\lambda_1}$ называются \textit{множествами, измеримыми по Лебегу}.
\end{df}
\begin{df}
 Мера $\mu$ в теореме Каратеодори называется \textit{стандартным продолжением}.
\end{df}
\begin{remrk}[формула для вычисления $\mu(A)$]
 \begin{align*}
  \mu(A) = \inf \left\{ \sum_{k=1}^{\infty} \mu_0(P_k) \mid P_k \in \p \colon\; A \subset \bigcup_{k=1}^{\infty} P_k \right\}
 \end{align*}
 для стандартного продолжения $\mu_{0}$ на $\A^{\ast}$.
\end{remrk}
\begin{df}
 \textit{Борелевская оболочка} $E$ --- наименьшая $\sigma$-алгебра, содержащая $E$. Обозначение: $\mathcal{B}(E)$.
\end{df}
\begin{df}
 Если $X, \tau$ --- топологическое пространство, то $\mathcal{B}(X)$ --- \textit{борелевская оболочка топологии}.

 Для борелевской оболочки $\R$ есть стандартное обозначение $\mathcal{B}_1 = \mathcal{B}(R)$ --- наименьшая $\sigma$-алгебра, содержащая все открытые множества (и замкнутые! по аксиоме симметричности).
\end{df}
\newcommand{\B}{\ensuremath \mathcal{B}}
\begin{lm}
 \begin{align*}
  \B_1 = \B( \p_1 ) \subset \A_{\lambda_1} = 2^{\R}
  .\end{align*}
\end{lm}
\begin{proof}
 Включение $\B(\p_1) \subset \A_{\lambda_1}$ верно, так как $\p_1 \subset \A_{\lambda_1}$ и $\A_{\lambda_1}$ --- это $\sigma$-алгебра.

 Равенство. Сначала докажем, что $\B_1 \subset \B(\p_1)$. Это верно, так как любое открытое $O \subset \R$ представимо в виде
 \begin{align*}
  O = \bigcup_{k=1}^{\infty} I_k, \quad I_k = (a_k, b_k),\; a_k \leqslant b_k
  ,\end{align*} так как $I_k$ --- класс эквивалентности по отношению $x \sim y \iff \exists I \subset O $ --- интервал, такой, что $x, y \in I$.

 Теперь напишем, что
 \begin{align*}
  (a_k, b_k) = \bigcup_{n=1}^{\infty} \left[a_k + \frac{1}{n}, b_k\right) \in \B(P_1) \implies O \in \B(\p_1)
  .\end{align*} Значит, $\B_1 \subset \B(\p_1)$, так как $\B_1$ --- наименьшая $\sigma$-алгебра.

 В обратную сторону: $\B(\p_1) \subset \B_1$, так как для любой ячейки $[a, b)$ имеет место равенство:
 \begin{align*}
  [a, b) = (a - 1, b) \setminus (-\infty, a) \in \B_1
  .\end{align*} Следовательно, $\B(\p_1) \subset \B_1$.
\end{proof}
\begin{exmpl}[про меру Лебега]\
 \begin{enumerate}
  \item $x \in \R \implies \left\{ x \right\} \in \B_1 \subset \A_{\lambda_1}$, так как точка --- замкнутая множества. Померим длину точки:
  \begin{align*}
   0 \leqslant \lambda_1(\left\{ x \right\}) = \lambda_1([x, x + \varepsilon)) = \varepsilon \quad \forall \eps > 0
   .\end{align*} Значит, $\lambda_1(\left\{ x \right\}) = 0$.
  \item Отрезок $[0, 1] \in \B_1$ (замкнутый). Его длина равна:
  \begin{align*}
   \lambda_1([0,1]) = \lambda_1([0, 1) \sqcup \left\{ 1 \right\}) = 1 + 0 = 1
   .\end{align*}
  \item $A$ открытое и непустое $ \implies A \in \B_1$ и $\lambda_1(A) > 0$.
  \begin{proof}
   Возьмём $x \in A$. Тогда существует $\eps > 0$ такой, что $(x - \eps, x + \eps) \subset A \implies [x, x + \eps] \subset A$. Тогда
   \begin{align*}
    \eps = \lambda_1( [x, x + \eps) ) \leqslant \lambda_1(A) \implies \lambda_1(A) > 0
    .\end{align*}
  \end{proof}
  \item $\lambda_1(\Q) = 0$.
  \begin{proof}
   \begin{align*}
    \lambda_1(\Q) = \sum_{x \in \Q} \lambda_1(\left\{ x \right\}) = 0
    .\end{align*}
  \end{proof}
  \item \begin{align*}
   \lambda_1( \left\{ x \in [0,1] \mid x \notin \Q \right\} ) \\
   = \lambda_1(\Q \cap [0, 1]) + \lambda_1(E) = 0 + \lambda_1(E) \lambda_1([0, 1]) = 1
   .\end{align*}
  \item Если есть $\{E_{k}\}_{k=1}^{\infty} $ такие, что $\lambda_1(E_k) = 0$, то
  \begin{align*}
   \lambda_1 \left( \bigcup_{k=1}^{\infty} E_k \right) = 0
   .\end{align*}
  \begin{proof}
   \begin{align*}
    \lambda_1 \left( \bigcup_{k=1}^{\infty} E_k \right) \leqslant \sum_{k=1}^{\infty} \lambda_1(E_k) = 0
    .\end{align*}
  \end{proof}
  \item Обобщённое Канторово множество. {\color{red} Тут нужен рисунок!!!}. Шаг 0:  выкидываем интервал длины $a_0$. Шаг 1: из оставшегося множества выкидываем $2^{1}$ открытых интервала длины $a_1$. Шаг 2: выкидываем $2^{2}$ интервалов длины $a_2$. Здесь $a_0, a_1, a_2, \ldots \geqslant 0$ достаточно маленькие чтобы можно было выкинуть:
  \begin{align*}
   \sum_{n=0}^{\infty} 2^{n} a_n \leqslant 1
   .\end{align*} Для стандартного Канторова множества $a_n = \frac{1}{3^{n+1}}$. Проверим:
  \begin{align*}
   \sum_{n=0}^{\infty} \frac{2^{n}}{3^{n+1}} = \frac{1}{3} \sum_{n=0}^{\infty} \left( \frac{2}{3} \right)^{n} = \frac{1}{3} \cdot \frac{1}{1 - \frac{2}{3}} = 1
   .\end{align*} Если взять $a_n = \frac{1}{4^{n+1}}$, то
  \begin{align*}
   \sum_{n=0}^{\infty} \frac{2^{n}}{4^{n+1}} = \frac{1}{4} \cdot \frac{1}{1 - \frac{2}{4}} = \frac{1}{2}
   .\end{align*} Пусть $C_{\left\{ a_n \right\}}$ --- обобщённое Канторово множество. Утверждается, что оно \textit{измеримо по Борелю}: $C_{\left\{ a_n \right\}} \in \B_1$, так как оно замкнуто (или, например, мы счётное число раз выкидываем интервалы). С другой стороны,
  \begin{align*}
   \lambda_1 \left( C_{\left\{ a_n \right\}} \right)
    & = \lambda_{1} \left( [0, 1] \setminus \bigsqcup_{n=0}^{\infty} \bigsqcup_{k=1}^{2^{n}} I_{n,k} \right) = \\
    & =  1 - \sum  \lambda_1(I_{n,k}) = 1 - \sum_{n=0}^{\infty} 2^{n} a_n
   .\end{align*} Вывод: стандартное Канторово множество несчётное, но его длина равна нулю! При этом:
  \begin{align*}
   \lambda_1 \left( C_{\left\{ \frac{1}{4^{n+1}} \right\}} \right) = \frac{1}{2} > 0
   .\end{align*} Такие Канторовы множества с ненулевой длиной называются \textit{толстыми Канторовыми множествами}. Интересный факт: замыкание толстого Канторово множества не содержит интервалов, но имеет положительную длину, в отличие от $\Q$!
 \end{enumerate}
\end{exmpl}
\begin{df}
 Пусть  $\mu$ --- мера на $\sigma$-алгебре $\A$. $\mu$ называется \textit{полной}, если для любого множества $A \in \A$ такого, что $\mu(A) = 0$ и для любого  $B \subset A$ имеет место $B \in \A$.
\end{df}
\begin{remrk}
 Мера $\mu$ в теореме Каратеодори полна.
\end{remrk}
\begin{proof}
 Пусть $A \in \A^{\ast}$ такое, что $\mu(A) = 0$. Пусть  $B \subset A$. Рассмотрим произвольное $E \subset X$. Тогда:
 \begin{align*}
  \mua(E) & \leqslant \mua(E \cap B) + \mua(E \cap B^{c}) \leqslant \\
          & \leqslant \mua(E \cap A) + \mua(E) \leqslant \mua(A) + \mua(E) \leqslant 0 + \mua(E)
  .\end{align*} Значит, $\mua(E) = \mua(E \cap B) + \mua(E \cap B^{c})$ для любого $E \subset X$, и следовательно, $B \in \A^{\ast}$.
\end{proof}
\section{Измеримые функции. Теорема об аппроксимации.}
\begin{df}
 $(X, \A)$ --- \textit{измеримое пространство}, если  $\A$ ---  $\sigma$-алгебра подмножеств $X$.
\end{df}
В этом параграфе все функции заданы на некотором измеримом пространстве.
\begin{df}
 Функция $f \colon\, X \to [-\infty, \infty] $ \textit{измерима} (относительно $\A$), если для любого $a \in \R$ верно
 \begin{align*}
  f^{-1}((a, +\infty]) \in \A
  .\end{align*}
\end{df}
\begin{remrk}
 В определении можно заменить $(a, +\infty]$, например, на  $[a, +\infty]$:
 \begin{align*}
  f^{-1}([a, +\infty]) = \bigcap_{n=1}^{\infty} f^{-1}\left(\left(a - \frac{1}{n}, +\infty \right)\right) \\
  f^{-1}((a +\infty]) = \bigcup_{n=1}^{\infty}  f^{-1} \left( [a + \frac{1}{n}, +\infty] \right)
  .\end{align*}
\end{remrk}
\begin{crly}
 Прообраз любой точки измерим, если $f$ измеримо:
 \begin{align*}
  f^{-1}(\left\{ x \right\}) \in \A
 \end{align*} для любой точки $x \in [-\infty, +\infty]$.
\end{crly}
\begin{proof} Для $x \in \R$:
 \begin{align*}
  f^{-1}( \left\{ x \right\} ) = f^{-1}( [x, \infty] ) \setminus f^{-1} \left( (x, +\infty] \right)
  .\end{align*}  Для $x = +\infty$:
 \begin{align*}
  f^{-1} \left( \left\{ +\infty \right\} \right) = \bigcap_{n=1}^{\infty} f^{-1} \left( [n, +\infty] \right)
  .\end{align*} Сначала покажем для $[-\infty, +\infty]$:
 \begin{align*}
  \bigcup_{n=1}^{\infty} f^{-1}([-n, +\infty]) = f^{-1}( (-\infty, +\infty] ) \\
  f^{-1}(\left\{ -\infty \right\}) \sqcup f^{-1} \left( (-\infty, +\infty] \right) = X
  .\end{align*}
\end{proof}
\begin{crly}
 Для любого $[a, b)$
 \begin{align*}
  f^{-1}([a, b)) \in \A
  ,\end{align*} если $f$ измерима.
\end{crly}
\begin{proof}
 \begin{align*}
  f^{-1}( [a, b) ) = f^{-1} \left( [a, +\infty] \right) \setminus f^{-1} \left( [b, +\infty] \right)
  .\end{align*}
\end{proof}
\begin{thm}
 Пусть есть функция $f \colon\, X \to [-\infty, +\infty] $ измерима. Тогда для любого $E \in \B_1$ верно
 \begin{align*}
  f^{-1}(E) \in \A
  .\end{align*}
\end{thm}
\begin{proof}
 Заведём специально обученную $\sigma$-алегбру. Рассмотрим
 \begin{align*}
  \B = \left\{ A \subset \R \mid f^{-1}(A) \in \A \right\}
  .\end{align*} Это $\sigma$-алгебра (легко проверить все аксиомы):
 \begin{itemize}
  \item $\R \in \B$
  \item $\varnothing \in \B$
  \item $A_k \in \B$, тогда
  \begin{align*}
   f^{-1}\left(\bigcap_{k=1}^{\infty} A_k \right) = \bigcap_{k=1}^{\infty} f^{-1}(A_k) \in \A \implies \bigcap_{k=1}^{\infty} A_k \in \B
  \end{align*}
  Но $\p_1 \subset \B$ (см. следствия). Тогда $\B_1 = \B(\p_1) \subset \B$. Значит, для любого  $A \in \B_1$ $f^{-1}(A) \in \A$.
 \end{itemize}
\end{proof}
Иными словами, измеримая функция --- это та, у которой прообраз измеримого множества измерим.
\begin{df}
 Функция $f \colon\; X \to \R$ называется \textit{простой}, если она измерима и принимает конечное число значений.
\end{df}
\begin{remrk}
 Каждая простая функция $f$ имеет вид
 \begin{align*}
  f = \sum_{k=1}^{N} c_k \chi_{A_k}
 \end{align*} для некоторых $A_k \in \A$. Здесь
 \begin{align*}
  \chi_A(x) = \begin{cases}
               1, \text{ если } x \in A \\
               0, \text{ иначе }
              \end{cases}
  .\end{align*}
\end{remrk}
\begin{proof}
 \begin{align*}
  f(X) = \{c_{k}\}_{k=1}^{N}
  .\end{align*} Тогда
 \begin{align*}
  A_k = f^{-1} \left( \left\{ c_k \right\} \right) \in \A
  .\end{align*}
\end{proof}
Идея Лебега: нужно любую измеримую функцию приближать простыми.
\begin{lm}
 Пусть функции $f_n \colon\; X \to [-\infty, +\infty]$ измеримы. Тогда $\sup f_n$, $\inf f_n$,  $\limsup f_n$, $\liminf f_n$ измеримы. Также, если существует предел  $\lim_{n \to \infty} f_n(x)$ для любой точки $x \in X$, то $\lim f_n$ измерим.
\end{lm}
\begin{proof}
 $f(x) = \sup f_n(x)$:
 \begin{align*}
  f^{-1}((a, +\infty]) = \bigcup_{n=1}^{\infty} f^{-1}_n( (a, +\infty] )
  .\end{align*} $f(x) = \inf f_n(x)$:
 \begin{align*}
  f^{-1} \left( [-\infty, a) \right) = \bigcup_{n=1}^{\infty} f^{-1}_n \left( [-\infty, a) \right)
  .\end{align*}  $f(x) = \limsup f_n(x)$:
 \begin{align*}
  f(x) = \limsup f_n(x) = \displaystyle\lim_{N \to \infty} \underbrace{\sup_{n \geqslant N} f_n(x)}_{\text{ убывает } \forall x \in X} = \inf_{N \geqslant 1} \left( \sup_{n \geqslant N} f_n(x) \right)
  .\end{align*} Следовательно, $\limsup f_n$ измерим. $\liminf$ аналогично.

 Пусть теперь $f(x) = \lim f_n(x)$ существует. Тогда
 \begin{align*}
  f = \limsup f_n = \liminf f_n
  ,\end{align*} а они измеримы.
\end{proof}

Следующая теорема будет о том, что если взять какую попало измеримую функцию (но неотрицательную), то она приближается простыми.

\end{document}
