% 2023.03.30 lecture 06
\documentclass[../complex-analysis.tex]{subfiles}
\begin{document}

\newpage
\section{Лемма Шварца. Нули ограниченных аналитических функций.}

\begin{df*}
 \textit{Единичным диском} называется $ \mathbb D = \left\{ z \in \CC \Mid \left| z \right| < 1 \right\} $.
\end{df*}

\begin{lm}[Шварца]
 Пусть $ f \colon\, \mathbb D \to \mathbb D $ --- аналитическая функция, причём $ f(0) = 0 $. Тогда $ \left| f(z) \right| \leqslant \left| z \right| $ в $ \mathbb D $, и равенство достигается в какой-то точке $ z \in \mathbb D $ тогда и только тогда, когда $f = \alpha z$, $ \left| \alpha  \right| = 1 $.
\end{lm}
\begin{proof}[\normalfont\textsc{Доказательство}]
Применим принцип максимума \eqref{theorem:maximum_principle}. Рассмотрим функцию $ g=\frac{f}{z} $. Функция аналитическая в $ \mathbb D $ (так как ряд не имеет свободного члена ($ f(0) = 0 $), и его можно поделить на $ z $). Найдём
 \begin{align*}
  \sup_{\left| z \right| < 1} \left| g(z) \right| &= \lim_{r \to 1} \sup_{\left| z \right| \leqslant r} \left| g(z) \right| = [\text{принцип максимума}] = \lim_{r\to 1} \sup_{\left| z \right| = r} \left| g(z) \right| = \\
  &= \lim_{r \to 1} r^{-1} \sup_{\left| z \right| = r} \underbrace{\left| f(z) \right|}_{\leq 1} \leqslant \lim_{r \to 1} r^{-1} = 1.
 \end{align*} Значит, $ \left| g \right| \leqslant 1 $ в $ \mathbb D $, то есть $ \left| f \right| \leqslant \left| z \right| $ в $ \mathbb D $.

 Пусть $ \left|f(z_0) \right| = \left| z_0 \right| $ в точке $ z_0 \in \mathbb D $. Тогда $ g(z_0) = 1 $, и по принципу максимума функция $ g $ --- константа в $ \mathbb D $, то есть $ f = \alpha \cdot z $, где $ \alpha = g(z_0) $, $ \left| \alpha \right|=1 $.
\end{proof}

\begin{df}
 Пусть $ f \colon\, \Omega \to \CC $ аналитична, и $ f(z_0) = 0 $ в некоторой точке $ z_0 \in \CC $. \textit{Кратностью нуля $ z_0 $} --- это число $ n \geqslant 1 $ такое, что
 \begin{align*}
  f(z) = (z-z_n)^{n} \cdot g(z)
 \end{align*} где $ g $ аналитична в $ \Omega $ и $ g(z_0) \neq 0 $.
\end{df}

По лемме \eqref{lemma:zero_multiplicity} о кратности нуля у каждой функции (кроме тождественного нуля) есть кратность во всех нулях.

\begin{thm}
 Пусть аналитическая функция $ f\colon\,\mathbb D \to \CC $  ограничена. Пусть $\{\lambda_{k}\}_{k=1}^{N} $,  где $ N \in \N \cup \left\{ \infty \right\} $ последовательность всех её нулей с учётом кратности (каждый нуль встречается в этой последовательности столько раз, какова его кратность).

 Тогда сходится ряд:
 \begin{align}
  \label{equation:zero_sum_conv}
  \sum_{k=1}^{N} \left( 1 - \left| \lambda_k \right| \right) < \infty.
 \end{align} Более того, обратное тоже верно: по нулям с условием \eqref{equation:zero_sum_conv} можно построить ограниченную аналитическую функцию с нулями в этих точках указанной кратности.
\end{thm}
Сходимость ряда \eqref{equation:zero_sum_conv} называется \textit{условием Бляшке}.
\begin{proof}[\normalfont\textsc{Доказательство}]
 Доказываем только в прямую сторону (в обратную сторону будет в листочке). Интересен только случай $ N = \infty $. Можно считать, что $ \left| f(z) \right| \leqslant 1$ в $ \mathbb D $.

 Рассмотрим функцию $ \mathcal B_n = \prod_{k=1}^{n} \frac{\lambda_k - z}{1 - \overline{\lambda_k}z} $, которая называется \textit{произведением Бляшке}. Эта функция аналитична в $ \mathbb D  $ и непрерывна в $ \overline {\mathbb D} $ (ведь $ \left| \lambda_k \right| < 1  $ и $ 1 - \overline{\lambda_k}z \neq 0 $). Кроме того,
 \begin{align*}
  \left| \mathcal B_n(\xi) \right| = 1
 \end{align*} для любого $ \xi \in \CC $, $ \left| \xi \right| = 1 $, потому что
 \begin{align*}
  \left| \frac{\lambda_k - \xi}{1 - \overline{\lambda_k}\xi} \right| = \left| \frac{\overline \xi \lambda_k - 1}{1 - \overline{\lambda_k} \xi} \right| = 1.
 \end{align*} По принципу максимума $ \left| \B_n(z) \right| Б= 1$ в $ \mathbb D $.

 Рассмотрим функцию $ g_n = \frac{f}{\B_n} $, эта функция также аналитична в $ \mathbb D $ по построению (нужно сравнить кратности нулей у $ f $ и у $ \mathcal B_n $ и воспользоваться леммой о кратности нуля).
 \begin{align*}
	 \sup_{\left| z \right| < 1} \left| g_n(z) \right| &= \lim_{r \to 1} \sup_{\left| z \right| = r} \frac{\left|f(z) \right|}{\left| B_n(z) \right|} = \lim_{r \to 1} \frac{\left| f(z) \right|}{1 + o(1)} \leqslant 1.
 \end{align*} Значит,
 \begin{align*}
  \left| f \right| = \left| \mathcal B_n \cdot g_n \right| \leqslant \left| \B_n \right|
 \end{align*} всюду в $ \mathbb D $. Подставим нуль:
 \begin{align*}
  \left| f(0) \right| \leqslant \left| \B_n(0) \right| = \prod_{k=1}^{n}\left| \lambda_k \right|.
 \end{align*} Пусть сначала $ f(0) \neq 0 $. Тогда
 \begin{align*}
  -\infty < \log \left| f(0) \right| = \sum_{k=1}^{n} \log \left| \lambda_k \right| = \sum_{k=1}^{n} \log (1 + (\left| \lambda_k \right| - 1)) \leqslant \\
  \leqslant [\log(1+x) \leqslant x \; \forall  x \in (-1, +\infty)  ] \leqslant \\
  \leqslant \sum_{k=1}^{n}\left( \left| \lambda_k \right| - 1 \right).
 \end{align*} Умножая на $ -1 $, получаем
 \begin{align*}
  \sum_{k=1}^{n} \left( 1 - \left| \lambda_k \right| \right) \leqslant \log \frac{1}{\left| f(0) \right|}
 \end{align*} Раз неравенство выполнено для любого $ n $, то ряд  \eqref{equation:zero_sum_conv} сходится.

 Теперь пусть $ f(0) = 0 $. Тогда достаточно рассмотреть  $ \tilde f(z) = \frac{f(z)}{z^{k}} $, где $ k $ --- кратность корня $ \lambda = 0 $. Дальше всё получается.

 В обратную сторону построение --- это бесконечное произведение Бляшке с нормировочными множителями:
  \begin{align*}
   f = \prod_{k=1}^{N} \frac{\left| \lambda_k \right|}{\lambda_k} \cdot \frac{\lambda_k -z}{1 - \overline{\lambda_k} z}
 \end{align*}
\end{proof}

\begin{exmpl*}
 Пусть $ c_k $ такие, что $ \sum \left| c_k \right| < \infty $. Тогда
 \begin{align*}
  f = \sum_{k=0}^{\infty} c_k z^{k}
 \end{align*} если $ f\left(1 - \frac{1}{n}\right) = 0 $, то $ c_k = 0 $ для любого $ k $.
\end{exmpl*}

\newpage
\section{Ряды Лорана.}

\begin{thm}
 Пусть $ f $ --- аналитическая функция в кольце
 \begin{align*}
  \Omega_{R,r} = \left\{ z \in \CC \mid r < \left| z - a \right| < R \right\}
 \end{align*} где $ a \in \CC $ и $ 0 \leqslant r < R \leqslant +\infty $. Тогда
 \begin{align}
  \label{equation:loran_series}
  f(w) = \sum_{z \in \Z} c_k(w - a)^{k}
\end{align} где ряды $ \sum_{k \geqslant 0} $ и $ \sum_{k < 0} $ сходятся равномерно на компактах в $ \Omega_{R,r} $. См. рисунок \eqref{fig:laurent-series-compacts}. Более того, коэффициенты $ c_k $ определены единственным образом:
 \begin{align*}
  c_k = \frac{1}{2\pi i} \varointctrclockwise\limits_{C_\rho}   \frac{f(z)}{(z-a)^{k+1}}\,dz,
 \end{align*} где $ r < \rho < R $, $ C_\rho = \left\{ z \in \CC \mid \left| z-a \right|=\rho \right\} $.
\end{thm}

\begin{figure}[h]

\begin{subfigure}{0.5\textwidth}
	\incfig[0.7]{laurent-series-compacts-limited}
    \caption{Ограниченный случай.}
    \label{fig:laurent-series-compacts-limited}
\end{subfigure}
\begin{subfigure}{0.5\textwidth}
	\incfig[0.8]{laurent-series-compacts-not-limited}
	\caption{Неограниченный случай.}
    \label{fig:laurent-series-compacts-not-limited}
\end{subfigure}

\caption{Пример компактов внутри кольца.}
\label{fig:laurent-series-compacts}
\end{figure}

\begin{df}
 Ряд \eqref{equation:loran_series} называется \textit{рядом Лорана} функции $ f $ в кольце $ \Omega_{R,r} $.
\end{df}
\begin{proof}[\normalfont\textsc{Доказательство}]
 Для простоты считаем $ a = 0 $. Проверим единственность: пусть ряд
 \begin{align*}
  f(w) = \sum_{k \in \Z}c_k w^{k}
 \end{align*} сходится равномерно на компактах в $ \Omega_{R,r} $. Проинтегрируем функцию по окружности:
 \begin{align*}
  \int\limits_{C_\rho} \frac{f(w)}{w^{j}} \,dw = \sum_{k \in \Z} c_k \int\limits_{C_\rho} w^{k-j}\,dw,
 \end{align*} потому что $ C_\rho $ --- компакт внутри $ \Omega_{R,r} $, и ряд сходится равномерно. Тогда
 \begin{align*}
  \int\limits_{C_\rho} w^{n}  \, dw = \int\limits_{0}^{2\pi}  \rho^{n} e^{i \cdot n \cdot t} \cdot \rho \cdot i e^{it}\,dt = \rho^{n+1} \cdot i \cdot \int\limits_{0}^{2\pi} e^{i(n+1)t}\,dt = \begin{cases}
   2\pi i, \text{ если } n+1=0 \\
   0, \text{ иначе }
  \end{cases} 
 \end{align*} Продолжим равенство:
 \begin{align*}
  = 2\pi i \cdot c_{j-1}.
 \end{align*} Тогда
 \begin{align*}
  c_j = \frac{1}{2\pi i} \int\limits_{C_\rho}   \frac{f(w)}{w^{j+1}}\,dw.
 \end{align*}

 Единственность проверена.

 Докажем существование.

 Сделаем хирургию: расширим кольцо, $ r < \tilde r < \tilde R < R $, нарисуем контур. $ w \in \Omega_{\tilde R, \tilde r} $.

\begin{figure}[ht]
    \centering
	\incfig[0.5]{laurent-series-point-surgery}
    \caption{Вырезание точки}
    \label{fig:laurent-series-point-surgery}
\end{figure}

 \begin{align*}
  \Gamma = C_{\tilde R} \cup C_{\tilde r} \cup \gamma_\eps^{+} \cup I_1 \cup I_2 \cup I_3 \cup I_4 \cup \gamma_\eps^{-}.
 \end{align*} Проинтегрируем по нему функцию
 \begin{align*}
  \frac{1}{2\pi i} \int\limits_{\Gamma} \frac{f(z)}{z - w}  \, dz = 0,
 \end{align*} так как контур стягиваем в области $ \Omega_{R,r} \setminus \left\{ w \right\} $. С другой стороны,
 \begin{align*}
  \frac{1}{2\pi i} \int\limits_{\Gamma} = \frac{1}{2\pi i} \varointctrclockwise \limits_{C_{\tilde R}}   - \frac{1}{2\pi i} \varointctrclockwise\limits_{C_{\tilde r}}  - \frac{1}{2\pi i} \varointctrclockwise\limits_{\gamma_\eps}   = 0.
 \end{align*} Тогда
 \begin{align*}
  \frac{1}{2\pi i} \int\limits_{\gamma_\eps} \frac{f(z)}{z-w}\,dz = \frac{1}{2\pi i} \int\limits_{C_{\tilde R}}   - \frac{1}{2\pi i}\int\limits_{C_{\tilde r}}.
\end{align*} Интеграл мы считали, когда доказывали теорему Коши-Гурса-Морера \eqref{theorem:cauchy-gursa-morer}.
 \begin{align*}
  \frac{1}{2\pi i} \int\limits_{\gamma_\eps}   \frac{f(z)}{z-w}\,dz = f(w)
 \end{align*}
 \begin{align*}
  f(w) = \frac{1}{2\pi i} \int\limits_{C_{\tilde R}}   - \frac{1}{2\pi i} \int\limits_{C_{\tilde r}}  = \frac{1}{2\pi i} \int\limits_{C_{\tilde R}}   \frac{f(z)}{z(1 - \frac{w}{z})}\,dz - \frac{1}{2\pi i} \int\limits_{C_{\tilde r}}   \frac{f(z)}{w(\frac{z}{w}-1)}\,dz = \\
  = \frac{1}{2\pi i}  \int\limits_{C_{\tilde R}} f(z) \cdot \sum_{k=0}^{\infty} \frac{w^{k}}{z^{k+1}}   + \frac{1}{2\pi i} \int\limits_{C_{\tilde r}}  f(z) \sum_{j=0}^{\infty}\frac{z^{j}}{w^{j+1}} = \\
  = \sum_{k \geqslant 0} w^{k} \cdot \left(\frac{1}{2\pi i} \int\limits_{C_{\tilde R}}   \frac{f(z)}{z^{k+1}}\,dz \right) + \sum_{j \geqslant 0} w^{-(j+1)} \left( \frac{1}{2\pi i} \int\limits_{C_{\tilde r}} \frac{f(z)}{z^{-j}}\,dz   \right) = \\
  = \sum_{k \geqslant 0} w^{k} \cdot \left(\frac{1}{2\pi i} \int\limits_{C_{\tilde R}}   \frac{f(z)}{z^{k+1}}\,dz \right) + \sum_{k \leqslant -1} w^{k} \left( \frac{1}{2\pi i} \int\limits_{C_{\tilde r}} \frac{f(z)}{z^{k + 1}}\,dz   \right)
 \end{align*} Получили что и требовалось.

 Ряды сходятся равномерно на $ \Omega_{\tilde{\tilde R}, \tilde{\tilde r}} $, где $ \tilde r < \tilde{\tilde r} < \tilde {\tilde R} < \tilde R $, так как если $ w \in \Omega_{\tilde{\tilde R}, \tilde{\tilde r}} $, то
 \begin{align*}
  \left| \frac{w}{z} \right| < [z \in C_{\tilde R}] < \frac{\tilde{\tilde R}}{\tilde R} < 1
 \end{align*} аналогично
 \begin{align*}
  \left| \frac{z}{w} \right| < \frac{\tilde r}{\tilde{\tilde r}} < 1
 \end{align*} Все хвосты оцениваются хвостами геометрических прогрессий.
\end{proof}

\newpage
\section{Изолированные особые точки аналитических функций.}
\begin{df}
 Точка $ a \in \CC $ называется \textit{изолированной особой точкой} функции $ f $, если $ f $ задана и аналитична в некоторой проколотой окрестности $ 0 < \left| z-a \right| < \eps $, но $ f $ не задана в точке $ a $.
\end{df}
\begin{exmpl}
 $ f(z) = \frac{1}{\sin z} $, $ z = 0 $ --- изолированная особая точка.
\end{exmpl}

\begin{df}
 Точка $ a \in \CC $ --- \textit{устранимая особая точка функции $ f $}, если $ a $--- изолированная особая точка и $ f $ ограничена в проколотой окрестности $ \Omega_{\eps,0} $ для некоторого $ \eps > 0 $.
\end{df}
\begin{lm}
 $ a $ --- устранимая особая точка тогда и только тогда, когда $ \exists  g  \colon\;  $ аналитическая функция в $ \left| z-a \right| < \eps $ такая, что $ f=g $ там, где эти функции определены (при $ z \neq a $).
\end{lm}
\begin{proof}[\normalfont\textsc{Доказательство}]
 Форма $ \omega = (z-a)f\,dz $ (доопределим $ 0 $ в точке $ a $) замкнута в $ \Omega_{\eps,0} $, имеет ограниченные коэффициенты в $ \Omega_{\eps,0} $ и непрерывно продолжается в $ \left| z-a \right| < \eps $.

 Значит форма $ \omega $ замкнутая в $ \left| z-a \right| < \eps $ (по лемме об устранении особенности \eqref{lemma:ob_ustranenii_osobennosti}). Значит,

 \begin{align*}
	 h(z) = \begin{cases}
		 (z - a) f(z), &z \neq a \\
		 0, &z = a
	 \end{cases}
 \end{align*} аналитична по теореме Коши-Гурса-Морера \eqref{theorem:cauchy-gursa-morer}. Кроме того, $ h(a) = 0 $. Значит, по лемме о кратности нуля $ h = (z-a)^{k}\tilde g $, где $ k \geqslant 1 $. Значит, $ (z-a)f = (z-a)^{k}\tilde g \implies f = (z-a)^{k-1}\tilde g $. Тогда $ g = (z-a)^{k-1}\tilde g $ нам подойдет.

 В обратную сторону очевидно, т.к. аналитическая функция в $C_\varepsilon$ ограничена. 
\end{proof}

\end{document}
