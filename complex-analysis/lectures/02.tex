\documentclass[../complex-analysis.tex]{subfiles}
\begin{document}

Заметим, что при выводе пункта~\ref{enum1:theorem:closed_1_form} из пункта~\ref{enum3:theorem:closed_1_form} радиус~$ r $ шара~$ B_0 $ с центром в точке~$ p $ мы выбирали абсолютно произвольный, лишь бы только шар помещался в область~$ \Omega $. Более того, мы могли бы выбирать не шар с центром в точке~$ p $, а открытый прямоугольник, или любую другую <<достаточно хорошую>> фигуру. Таким образом, с помощью понятия гомотопии мы из локальной точности формы получили почти-что глобальную точность. Более формально, мы получили следующий результат.

\begin{crly}
 \label{corollary:closed_form_is_exact_in_simply_connected}
 $ 1 $-форма~$ \omega $, замкнутая в односвязной области~$ \Omega $, является точной в $ \Omega $.
\end{crly}
\begin{proof}[\normalfont\textsc{Доказательство}]
 По теореме~\ref{theorem:exact_1_form} о точных формах нам достаточно доказать равенство~\eqref{eq:exact_1_form:integral_equals} для любых кусочно-гладких путей~$ \gamma_0, \gamma_1 $ в $ \Omega $ с совпадающими началом и концом. Так как область~$ \Omega $ односвязна, то любые такие пути гомотопны в $ \Omega $. Но тогда по теореме~\ref{theorem:closed_1_form} о замкнутых формах равенство~\eqref{eq:exact_1_form:integral_equals} действительно выполняется, что и требовалось доказать.
\end{proof}

Если дифференциальная форма~$ \omega $ не только непрерывная, но ещё и гладкая, то для неё есть ещё один интересный критерий замкнутости.

\begin{thm}[замкнутость гладких форм]
 \label{theorem:smooth_form_closed}
 $ C^{1} $-гладкая $ 1 $-форма~$ \omega $ замкнута в области~$ \Omega $ тогда и только тогда, когда выполнено $ d \omega = 0 $ в $ \Omega $.
\end{thm}
\begin{proof}[\normalfont\textsc{Доказательство}]
 Пусть гладкая форма~$ \omega $ замкнута в $ \Omega $, $ z_0 \in \Omega $ --- произвольная точка, а $ r > 0 $ --- такое число, что $ \omega = d F $ в $ B(z_0,r) \subset \Omega $, где $ F \colon\,\Omega\to\CC $ --- гладкая функция. Тогда в $ B(z_0,r) $ выполнено
 \begin{align*}
  d\omega&=d(dF) = d \left( F_x'\,dx + F_y'\,dy \right) = \\
  &= \left( F_{x x}''\,dx + F_{x y}''\,dy \right)\land dx + \left( F_{y x}''\,dx + F_{y y}''\,dy \right)\land dy = \\
  &= F_{x x}''\,dx \land dx + F_{xy}''\,dy \land dx + F_{yx}''\,dx\land dy + F_{y y}''\,dy \land dy = \\
  &= -F_{xy}''\,dx\land dy + F_{yx}''\,dx \land dy = (F_{yx}'' - F_{xy}'')\,dx\land dy.
 \end{align*} Но $ F_x' = P $ и $ F_y' = Q $ --- это $ C^{1} $-гладкие функции, поэтому верно $ F_{xy}'' = F_{yx}'' $. Получаем $ d\omega=0 $ в шаре $ B(z_0,r) $ (а значит и всюду).

 Теперь пусть выполнено $d \omega = 0$ в области~$\Omega$. Проверим замкнутость формы с помощью <<теста на прямоугольнике>> --- условия~\ref{enum3:theorem:closed_1_form} из теоремы~\ref{theorem:closed_1_form}. Пусть $ \Pi \subset \Omega $ --- замкнутый прямоугольник со сторонами, параллельными осям координат, а $ \partial\Omega $ --- его граница. Тогда по формуле Стокса:
 \begin{align*}
  \int_{\partial\Pi} \omega = \int_{\Pi} d\omega = \int_{\Pi} 0 = 0. 
 \end{align*} Так как мы не доказывали в полной мере формулу Стокса, здесь лучше воспользоваться формулой Грина, которая как минимум была доказана для стандартных областей (открытый прямоугольник уж точно является стандартной областью).
\end{proof}

\subsection{Примеры дифференциальных форм.}

\begin{exmpl}
 \label{example:form_a_plus_b_times_z_dz}
 Для коэффициентов $ a,b\in\CC $ рассмотрим дифференциальную форму
 \begin{align*}
  \omega = (a + bz)\,dz.
 \end{align*} Оказывается, форма~$ \omega $ точна в области~$ \CC $. Так как форма гладкая, а область $ \CC $ односвязная, то по следствию~\ref{corollary:closed_form_is_exact_in_simply_connected} и теореме~\ref{theorem:smooth_form_closed} достаточно показать $ d\omega = 0 $:
 \begin{align*}
  d\omega &= d(a + bz)\land dz = b\,dz \land dz = 0,
 \end{align*} где
 \begin{align*}
  dz\land dz = (dx + i\,dy) \land (dx + i\,dy) = i\,dy\land dx + i\,dx\land dy = 0.
 \end{align*}
\end{exmpl}

Следующая дифференциальная форма~\eqref{eq:form:dz/z-a} неожиданно окажется крайне полезной в дальнейшем построении теории.

\begin{exmpl}
 \label{example:form:dz/z-a}
 Пусть дана точка $a \in \CC$. Рассмотрим дифференциальную форму
 \begin{align}
  \label{eq:form:dz/z-a}
  \omega = \frac{dz}{z - a}
 \end{align} в области $\Omega = \CC \setminus \left\{ a \right\}$. Проверим, что $\omega$ замкнута в $\Omega$. Снова по теореме~\ref{theorem:smooth_form_closed} достаточно показать $ d\omega=0 $:
 \begin{align*}
  d \omega &= d \left( \frac{1}{z-a} \right) \land dz = \left( -\frac{1}{(z-a)^{2}}\,dx - \frac{i}{(z-a)^{2}}\,dy \right) \land dz = - \frac{dz\land dz}{(z-a)^{2}} = 0.
 \end{align*} Однако, так как область~$ \CC \setminus \left\{ a \right\} $ не односвязная, нельзя сказать, что форма~$ \omega $ точна в $ \CC \setminus \left\{ a \right\} $ (и мы уже сразу увидим, что она действительно не точна). Тем не менее, для формы~\eqref{eq:form:dz/z-a} есть крайне полезная формула интегрирования.
\end{exmpl}
\begin{lm}
 \label{lemma:form:dz/z-a}
 Пусть $ C $ --- окружность в $ \CC $, ориентированная против часовой стрелки, и не содержащая точку~$ a \in \CC $. Тогда
 \begin{align}
  \label{eq:int_form:dz/z-a}
  \int_{C} \frac{dz}{z-a} = \begin{cases}
   2\pi i, &\text{если $ C $ содержит точку $ a $ внутри;}\\
   0,&\text{иначе.}
  \end{cases} 
 \end{align}
\end{lm}
\begin{proof}[\normalfont\textsc{Доказательство}]
 Если точка~$a$ не лежит внутри $C$, то $ \int_{C} \frac{dz}{z-a} = 0 $, так как окружность~$ C $ стягиваема в точку в области~$ \CC\setminus \left\{ a \right\} $.

 Пусть теперь точка~$a$  лежит внутри $C$. Как часто бывает в комплексном анализе, доказательство будет <<на картинке>>: мы изменим некоторым хирургическим образом контур интегрирования так, чтобы интегралы на отдельных часах контура хорошо посчитались. В данном случае мы рассмотрим окружность~$ C_r $, направленную против часовой стрелки, с центром в точке~$ a $ и радиусом~$ r > 0 $, достаточно маленьким для того, чтобы окружность~$ C_r $ целиком помещалась во внутренность $ C $. Также соединим окружность $ C $ с окружностью $ C_r $ направленным отрезком $ I $, и рассмотрим контур $ \gamma = C + I + (-C_r) + (-I)$ (рисунок \ref{fig:gamma_with_tilde}).

 \begin{figure}[ht]
  \centering
  \incfig[0.8]{gamma_with_tilde}
  \caption{Контур интегрирования $ \gamma $.}
  \label{fig:gamma_with_tilde}
 \end{figure}

 Несмотря на то, что на рисунке~\ref{fig:gamma_with_tilde} отрезки $ I $ и $ -I $ нарисованы как разные отрезки (на некотором расстоянии друг от друга), они на самом деле являются одним и тем же отрезком, но направленным в разные стороны.

 Так как контур $ \gamma $ можно стянуть в точку в области $ \CC\setminus \left\{ a \right\} $, то
 \begin{align*}
  \int_{\gamma}  \omega = 0.
 \end{align*}

 С другой стороны, по аддитивности интеграла
 \begin{align*}
  \int_{\gamma} \omega = \int_{C} \omega + \int_{I} \omega + \int_{-C_r} \omega + \int_{-I} \omega = \int_{C} \omega - \int_{C_r} \omega.   
 \end{align*} Поэтому,
 \begin{align*}
  \int_{C} \omega &= \int_{C_r} \omega = \int_{C_r} \frac{dz}{z-a} = \begin{bmatrix}
   z = re^{it}+a, & dz = ire^{it}\,dt
  \end{bmatrix}=\\
  &= \int_{0}^{2\pi} \frac{ir e^{it}\,dt}{r e^{it} + a - a} = 2\pi i.
 \end{align*}
\end{proof}

Из формулы~\eqref{eq:int_form:dz/z-a} мгновенно следует, что форма~$ \frac{dz}{z-a} $ не точна в $ \CC \setminus \left\{ a \right\} $: ведь интеграл по окружности, содержащей внутри точку~$ a $, не равен нулю.

\subsection{Лемма об устранении особенности, лемма о первообразной.}

В последнем разделе параграфа докажем две технические леммы.

\begin{lm}[%
 об устранении особенности]
 \label{lemma:ob_ustranenii_osobennosti}
 Пусть $\omega$ --- непрерывная $1$-форма в области $\Omega$, замкнутая в области~$ \Omega \setminus \left\{ a \right\} $. Тогда $\omega$ также замкнута в $\Omega$.
\end{lm}
\begin{proof}[\normalfont\textsc{Доказательство}]
 Снова воспользуемся <<тестом на прямоугольнике>> --- условием~\ref{enum3:theorem:closed_1_form} теоремы~\ref{theorem:closed_1_form}. Возьмём любой замкнутый прямоугольник $\Pi \subset \Omega$, и проверим
 \begin{align*}
  \int_{\partial\Pi} \omega=0.
 \end{align*} Если $a \notin \Pi$, то доказывать нечего. Если же $ a $ лежит строго внутри $ \Pi $, то снова изменим контур интегрирования. Нарисуем вокруг точки~$ a $ окружность~$ C_\eps $ маленького радиуса~$ \eps > 0 $, разобьём её на верхнюю и нижнюю полуокружности ($ C_\eps^{+} $ и $ C_\eps^{-} $ соответственно), соединим её горизонтальными отрезками $ I_1 $ и $ I_2 $ с границей прямоугольника $ \partial\Pi $, и получим в сумме контур $ \gamma $ (рисунок \ref{fig:special_point_in_rectangle}).

 \begin{figure}[ht]
  \centering
  \incfig[0.8]{special_point_in_rectangle}
  \caption{Устранение особенности в прямоугольнике.}
  \label{fig:special_point_in_rectangle}
 \end{figure}

 Тогда
 \begin{align*}
  \int_{\gamma} \omega &= \int_{\partial\Pi} \omega + \int_{I_1} \omega + \int_{-I_1} \omega + \int_{I_2} \omega + \int_{-I_2} \omega + \int_{-C_\eps} \omega = \\
  &= \int_{\partial\Pi} \omega - \int_{C_\eps}  \omega.
 \end{align*} Но $ \gamma = \gamma^{+} + \gamma^{-} $, где $ \gamma^{+} $ --- верхняя часть кривой, а $ \gamma^{-} $ --- нижняя. Обе части можно стянуть в точку в области $ \Omega $, поэтому
 \begin{align*}
  \int_{\gamma} \omega = \int_{\gamma^{+}} \omega + \int_{\gamma^{-}}  \omega = 0,
 \end{align*} и, следовательно,
 \begin{align*}
  \int_{\partial\Pi} \omega = \int_{C_\eps}   \omega.
 \end{align*}

 Запишем $ \omega = P\,dx + Q\,dy $. Тогда по оценке модуля интеграла \eqref{eq:bound_on_absolute_value_of_int}:
 \begin{align*}
  \left| \int_{C_\eps} \omega  \right| \leqslant 2\pi\eps \cdot \max_{z \in C_\eps} \sqrt{ \left| P(z) \right|^{2} + \left| Q(z) \right|^{2} } \leqslant 2\pi B \eps,
 \end{align*} где $ B $ --- константа, не зависящая от $ \eps $ (непрерывная функция ограничена на компакте~$ \Pi $). Так как это верно для всякого $ \eps > 0 $, то
 \begin{align*}
  \int_{\partial\Pi}   \omega = 0.
 \end{align*} 

 В случае если точка $a$ лежит на границе прямоугольника (или является углом прямоугольника), то нужно по аналогии строить путь который ее огибает, и устремлять к нулю изменение.

\end{proof}

\begin{lm}[%
 о первообразной]
 \label{lemma:o_pervoobraznoi}
 Пусть $ f \colon\,\Omega\to\CC $ --- непрерывная функция на области~$ \Omega \subset \CC $ такая, что дифференциальная форма $ f\,dz $ точна в $ \Omega $. Тогда у функции $ f $ есть \emph{первообразная} в $ \Omega $: существует функция $ g \colon\,\Omega\to\CC $ такая, что выполнено
 \begin{align*}
  g'(z) := \lim_{h \to 0} \frac{g(z + h) - g(z)}{h} = f(z)
 \end{align*} при всех $ z \in \Omega $.
\end{lm}
\begin{proof}[\normalfont\textsc{Доказательство}]
 Раз форма $f\, dz$ точна в $ \Omega $, то существует функция~$F \colon\, \Omega \to \CC $ такая, что $f\,dz = dF$ всюду в $ \Omega $. Как можно было догадаться, в качестве первообразной нужно взять именно эту функцию:
 \begin{align*}
  g(z) = F(x, y),
 \end{align*} где $ z = x+iy \in \Omega $. Проверим, что она подходит: возьмём малое приращение $ h = h_1 + ih_2 \to 0 $ и запишем
 \begin{align*}
  g(z + h) &= F(x + h_1, y + h_2) = F(x,y) + F_x'(x,y) \cdot h_1 + F_y'(x,y) \cdot h_2 + o(h).
 \end{align*} Так как
 \begin{align*}
  dF = F_x'\,dx + F_y'\,dy = f\,dz = f\,dx + if\,dy,
 \end{align*} то
 \begin{align*}
  g(z+h) &= g(z) + f(z) \cdot h_1 + i f(z) \cdot h_2 + o(h) = \\
  &= g(z) + f(z) \cdot h + o(h),
 \end{align*} что и требовалось доказать.
\end{proof}

\end{document}
