% 2023.05.04 lecture 11
\documentclass[../complex-analysis.tex]{subfiles}
\begin{document}

\begin{lm}
 \label{lemma:Analytic Image of a Region}
 Образ $ f(\Omega) $ непостоянной функции~$ f $, аналитичной в области~$ \Omega \subset \CC $, тоже является областью.
\end{lm}
\begin{proof}[\normalfont\textsc{Доказательство}]
 $ f(\Omega) $ открыто по теореме~\ref{theorem:analytic_implies_open} и связно как непрерывный образ связного множества.
\end{proof}

\subsection{Теорема Римана об отображении.}

Как мы выяснили в предыдущем разделе, не все области являются конформно эквивалентными. В частности, односвязная область не может быть конформно эквивалентна неодносвязной (пример~\ref{example:Simple Connectivity is Invariant w respect to Conform Map}). 

Давайте упростим себе задачу и будем изучать конформную эквивалентность односвязных областей. Но снова получается, что не все односвязные области конформно эквивалентны: например, $ \CC $ не конформно эквивалентна $ \mathbb D $ (пример~\ref{example:C non Conform Equivalent with D}). Но оказывается, что если исключить этот случай, то все остальные односвязные области конформно эквивалентны друг другу, и об этом как раз и будет теорема Римана, к который мы подготавливались в течение нескольких параграфов.

\begin{thm}[Римана]
 \label{theorem:Riemann}
 Пусть $ \Omega \subset \CC $ --- односвязная область, $ \Omega \neq \CC $ и $ \Omega \neq \varnothing $. Тогда существует конформное отображение $ f \colon\,\Omega\to\mathbb D $.
\end{thm}
\begin{crly}
 Любые две односвязные области $ \Omega_1,\Omega_2 \subset \CC $, $ \Omega_i \neq \CC, \Omega_i \neq \varnothing $ конформно эквивалентны.
\end{crly}
\begin{proof}[\normalfont\textsc{Доказательство теоремы~\ref{theorem:Riemann} Римана}]
 Заметим, что по примеру~\ref{example:C non Conform Equivalent with D} любое конформное отображение сохраняет односвязность. В дальнейшем в доказательстве мы не будем это каждый раз упоминать.

 \begin{enumerate}
  \item \label{enum1:proof:Riemann Thm} Покажем, что не умаляя общности можно считать, что $ \Omega \subset \mathbb D $, и $ 0 \in \Omega $.

   Для этого нам достаточно показать существование конформного отображения из области~$ \Omega $ в ограниченную область~$ \tilde\Omega $: после этого можно применить конформное отображение вида $ z \mapsto a z + b $ для малого~$ a \in \CC $ и некоторого~$ b \in \CC $, и перевести $ \tilde\Omega $ в новую область, ограниченную диском~$ \mathbb D $, и содержащую точку~$ 0 $.

   Более того, заметим, что нам достаточно перевести область~$ \Omega $ в область~$ \tilde\Omega $, для которой существует точка~$ a \in \CC \setminus \tilde\Omega $, отделённая от $ \tilde\Omega $ (то есть $ B(a,\eps) \cap \tilde\Omega  = \varnothing $ для некоторого $ \eps > 0 $). В самом деле, тогда можно применить конформное отображение $ z \mapsto 1 / (z - a) $, которое переведёт область~$ \tilde\Omega $ в новую область, ограниченную диском $ B(0, 1 / \eps) $.

   Итак, зафиксируем произвольную точку~$ z_0 \in \CC \setminus \Omega $ (которая существует по условию). По теореме~\ref{theorem:exist_log_f} в области~$ \Omega $ существует аналитическая ветвь логарифма $ h(z) = \log (z - z_0) \colon\, \Omega \to \CC $ функции~$ z \mapsto z - z_0 $. Функция~$ h $ инъективна:
   \begin{align*}
    h(z_1) = h(z_2) \implies e^{h(z_1)} = e^{h(z_2)} \implies z_1 - z_0 = z_2 - z_0 \implies z_1 = z_2,
   \end{align*} 
   а значит, $h \colon\,\Omega \to \Omega_1$ --- конформное отображение, где $ \Omega_1 = h(\Omega) $ --- область значений функции~$ h $ (это действительно область по лемме~\ref{lemma:Analytic Image of a Region}).

   Покажем, что существует точка~$ a \in \CC \setminus \Omega_1 $, отделённая от $ \Omega_1 $. Заметим, что
   \begin{align}
    \label{eq:Omega1 not intersects with Omega1 + 2pi i:Riemann Thm}
    \Omega_1 \cap (\Omega_1 + 2\pi i) = \varnothing.
   \end{align}
   Действительно, если вдруг $ h(z_1) = h(z_2) + 2\pi i $, то
   \begin{align*}
    e^{h(z_1)} = e^{h(z_2) + 2\pi i} = e^{h(z_2)} \implies z_1 = z_2 \implies h(z_1) = h(z_1) + 2 \pi i,
   \end{align*} а это противоречие. Покажем, что в таком случае любая точка~$ a \in \Omega_1 + 2\pi i $ отделена от $ \Omega_1 $. Действительно, так как $ \Omega_1 + 2\pi i $ --- открытое множество, то точка~$ a $ входит в него вместе со своей окрестностью; эта же окрестность не пересекается с $ \Omega_1 $ по~\eqref{eq:Omega1 not intersects with Omega1 + 2pi i:Riemann Thm}.

   Далее мы будем полагать $ \Omega \subset \mathbb D $ и $ 0 \in \Omega $.

  \item \label{enum2:proof:Riemann Thm} Рассмотрим семейство~$ \mathcal R $ инъективных аналитичных функций~$ f\colon\,\Omega \to \mathbb D $, таких, что $ f(0) = 0 $, в которое также добавлен тождественный нуль $ f \equiv 0 $ на~$ \Omega $. Обратим внимание на то, что в семействе~$ \mathcal R $ по шагу~\ref{enum1:proof:Riemann Thm} есть функция~$ f $, отличная от тождественного нуля: например, можно взять тождественную функцию~$ \mathrm{id}\colon\,\Omega \to \Omega $, поскольку $ \Omega \subset \mathbb D $, и $ 0 \in \Omega $.

   Докажем, что семейство~$ \mathcal R $  <<компактно>>: из всякой последовательности функций~$ \{f_{n}\}_{n=1}^{\infty} \subset \mathcal R $ можно выделить подпоследовательность~$ \{f_{i_n}\}_{n=1}^{\infty}  $, сходящуюся к функции~$ f \in \mathcal R $ равномерно на компактах в $ \Omega $.

   Если в последовательности тождественный нуль встречается бесконечно много раз, то можно создать подпоследовательность из нулей, а иначе выкинем конечное число нулей в начале последовательности.

   Благодаря тому, что функции~$f_n$ аналитичны и равномерно ограничены, по теореме \ref{theorem:montel} Монтеля существует подпоследовательность $ \{f_{i_n}\}_{n=1}^{\infty}$, сходящаяся к функции~$f$ равномерно на компактах в $ \Omega $, причём по замечанию~\ref{remark:Montel Theorem Analytic Limit} функция~$ f $ аналитична в $ \Omega $, и по понятным причинам $ f(0) = 0 $. По теореме~\ref{theorem:analytic_injective} (следствие из теоремы Гурвица) функция~$ f $ либо постоянна, либо инъективна. Если $ f $ постоянна, то $ f \equiv 0 $ на $ \Omega $ (так как $ f(0) = 0 $). В противном случае функция~$ f $ инъективна. При этом функция~$ f $ действует в $ \mathbb D $: по предельному переходу в неравенстве она уж точно действует в $ \overline{\mathbb D} $, а так как по теореме~\ref{theorem:analytic_implies_open} образ $ f(\Omega) $ --- открытое подмножество~$ \CC $, то $ f(\Omega) \subset \mathop{\mathrm{Int}} \overline{\mathbb D} = \mathbb D $. Так или иначе, $ f \in \mathcal R $, и семейство~$ \mathcal R $ компактно.

  \item \label{enum3:proof:Riemann Thm} Пусть есть конформное отображение~$ f \in \mathcal R $, такое, что $ f(\Omega) \neq \mathbb D $ (если получилось равенство, то теорема уже доказана). Докажем, что тогда существует другое конформное отображение~$ g \in \mathcal R $, такое, что
   \begin{align}
    \label{eq:Larger Derivative at Zero:proof:Riemann Thm}
    \left| f'(0) \right| < \left| g'(0) \right|.
   \end{align}

   Возьмём произвольную точку~$ w \in \mathbb D \setminus f(\Omega) $. Рассмотрим уже встречавшуюся нам ранее функцию, называющуюся \emph{множителем Бляшке}:
   \begin{align*}
    T_1(z) = \frac{z-w}{1 - \overline w z}.
   \end{align*} Эта функция является конформным отображением $ T_1\colon\mathbb D \to \mathbb D $ (замечание~\ref{remark:T1_conformal}). Поскольку $ T_1(f(z)) \neq 0 $ при $ z \in \Omega $ (ведь $ w \notin f(\Omega) $), то у функции $ T_1 \circ f $ в односвязной области~$ \Omega $ по теореме~\ref{theorem:exist_log_f} существует аналитическая ветвь логарифма $\log T_1(f(z))$, которая, очевидно, также инъективна. Так как $ T_1 $ действует в $ \mathbb D $, то
   \begin{align*}
    \Real \log T_1(f(z)) = \log \left| T_1(f(z)) \right| < 0,
   \end{align*} то есть функция~$ z \mapsto \log T_1(f(z)) $ действует из $ \Omega $ в $ L := \left\{ z \in \CC : \Real z < 0 \right\} $.

   В таком случае рассмотрим функцию
   \begin{align*}
    T_2(z) = \frac{z-a}{z+\overline a},
   \end{align*} являющуюся конформным отображением~$ T_2 \colon\, L \to \mathbb D $ по примеру~\ref{example:Re z < 0 Conform Map to D}, где
   \begin{align*}
    a = \log T_1(f(0)).
   \end{align*}

   Соберём всё вместе: рассмотрим функцию
   \begin{align}
    \label{eq:g Definition:proof:Riemann Thm}
    g(z) = T_2(\log(T_1(f(z)))).
   \end{align} Функция~$ g \colon\,\Omega \to \mathbb D $ инъективна и аналитична в $ \Omega $ (как композиция инъективных и аналитичных функций), и при этом
   \begin{align*}
    g(0) = T_2(\log T_1(f(0))) = T_2(a) = 0.
   \end{align*} Таким образом, $ g \in \mathcal R $.

   Осталось показать неравенство~\eqref{eq:Larger Derivative at Zero:proof:Riemann Thm}. Перепишем \eqref{eq:g Definition:proof:Riemann Thm}, пользуясь обратимостью конформных отображений:
   \begin{align*}
    &T_2(\log T_1(f(z))) = g(z)\\
    \iff\;&\log T_1(f(z)) = T_2^{-1}(g(z))\\
    \iff\;&T_1(f(z)) = \exp(T_2^{-1}(g(z)))\\
    \iff\;&f(z) = T_1^{-1}(\exp(T_2^{-1}(g(z)))).
   \end{align*} То есть,
   \begin{align*}
    f(z) = \varphi(g(z)),
   \end{align*} где
   \begin{align*}
    \varphi(z) = T_1^{-1}(\exp(T_2^{-1}(z))).
   \end{align*} Функция~$ \varphi $ действует из~$ \mathbb D $ в $ \mathbb D $, так как $ T_2^{-1}\colon\,\mathbb D \to L $, $ \exp\colon\,L \to \mathbb D $, и $ T_1^{-1}\colon\,\mathbb D \to \mathbb D $. Функция~$ \varphi $ аналитична как композиция аналитичных, но она не инъективна, так как функция~$ \exp\colon\,L\to\mathbb D $ не инъективна.

   Докажем, что $ \left| \varphi'(0) \right| < 1 $. Так как $ \varphi\colon\,\mathbb D \to \mathbb D $, и $ \varphi(0) = 0 $, то по лемме~\ref{lemma:schwarz} Шварца либо $ \varphi(z) = \alpha z $ в $ \mathbb D $ для некоторого $ \left| \alpha \right| = 1 $, либо
   \begin{align*}
    \left| \varphi(z) \right| < \left| z \right|,
   \end{align*} всюду в $ \mathbb D $. Первого случая не может быть, так как функция $ z \mapsto \alpha z $, $ \left| \alpha \right|=1 $ инъективна, а~$ \varphi $ не инъективна. Значит, верно второе; тогда для любого $ z \in \mathbb D $ выполнено
   \begin{align*}
    \left| \frac{\varphi(z)}{z} \right| < 1,
   \end{align*} и так как функция $ z \mapsto \left| \varphi(z) / z \right| $ непрерывна на компакте~$ \overline B(0, 1 / 2) $, то
   \begin{align*}
    \left| \frac{\varphi(z)}{z} \right| \leqslant \delta < 1, \quad z \in \overline B(0, 1 / 2).
   \end{align*} Следовательно,
   \begin{align*}
    \left|\varphi'(0)\right| = \left| \lim_{z \to 0} \frac{\varphi(z)}{z} \right| \leqslant \lim_{z \to 0}  \left| \frac{\varphi(z)}{z} \right| \leqslant \delta < 1.
   \end{align*} Итого,
   \begin{align*}
    \left| f'(0) \right| = \left| \varphi'(g(0)) \cdot g'(0) \right| = \left| \varphi'(0) \right| \cdot \left| g'(0) \right| < \left| g'(0) \right|.
   \end{align*}

  \item Рассмотрим величину
   \begin{align*}
    A = \sup_{\psi \in \mathcal R} \left| \psi'(0) \right|.
   \end{align*} Заметим, что $ A > 0 $, поскольку по шагу~\ref{enum1:proof:Riemann Thm} существует хотя бы одна функция~$ f \in \mathcal R $, не равная тождественному нулю, и тогда по теореме~\ref{theorem:Derivative of Injective Analytic Fun is not 0} $ \left| f'(0) \right| > 0 $. Тогда по определению супремума существует такая последовательность функций $ \{f_{n}\}_{n=1}^{\infty} \subset \mathcal R $, что
   \begin{align*}
    \lim_{n \to \infty} \left| f_n'(0) \right| = A.
   \end{align*} По шагу~\ref{enum2:proof:Riemann Thm} из этой последовательности можно выделить подпоследовательность~$ \{f_{i_n}\}_{n=1}^{\infty}  \subset\mathcal R$, сходящуюся к функции~$ f\in\mathcal R $ равномерно на компактах в области~$ \Omega $. Следовательно (например, по теореме~\ref{theorem:Derivative Convergence Example}), функции~$ f_{i_n}' $ также сходятся к функции~$ f' $ равномерно на компактах в $ \Omega $, и, в частности
   \begin{align*}
    A = \lim_{n \to \infty} \left| f_{i_n}'(0) \right| = \left| f'(0) \right|.
   \end{align*} Таким образом, функция $ f \in \mathcal R $ такова, что для любой другой функции~$ g \in \mathcal R $
   \begin{align}
    \label{eq:f has largest Derivate at Zero:proof:Riemann Thm}
    \left| f'(0) \right| \geqslant \left| g'(0) \right|.
   \end{align}

   Осталось понять, что~$ f $ и есть искомое конформное отображение~$ f \colon\,\Omega\to\mathbb D $. Действительно, $ f $ не равна тождественному нулю в $ \Omega $ по~\eqref{eq:f has largest Derivate at Zero:proof:Riemann Thm}, и поэтому $ f $ --- конформное отображение области~$ \Omega $ на свой образ. Тогда, если предположить $ f(\Omega) \neq \mathbb D $, то по шагу~\ref{enum3:proof:Riemann Thm} существует другое конформное отображение~$ g \in \mathcal R $, такое, что выполнено~\eqref{eq:Larger Derivative at Zero:proof:Riemann Thm}, а это противоречит \eqref{eq:f has largest Derivate at Zero:proof:Riemann Thm}!
 \end{enumerate}
\end{proof}

\begin{remrk}
 \label{remark:T1_conformal}
 Множитель Бляшке
 \begin{align*}
  T_1(z) = \frac{z - w}{1 - \overline w z}, \quad w \in \mathbb D
 \end{align*}
 является конформным отображением $ T_1 \colon\,\mathbb D \to \mathbb D $.
\end{remrk}
\begin{proof}
 Заметим, что если $\left| z \right| = 1$, то
 \begin{align*}
  \left| T_1(z) \right| = \left|\frac{z - w}{1 - \overline w z}\right| = \left|\frac{1 - w \overline z}{1 - \overline w z}\right| = 1.
 \end{align*} Тогда по принципу максимума (теорема~\ref{theorem:maximum_principle}) имеем $ \left| T_1(z) \right| < 1 $ всюду в $ \mathbb D $.

 Покажем теперь, что функция~$ T_1 $ обратима. Выберем произвольную точку $ \zeta \in \mathbb D $, и найдём её прообраз:
 \begin{align*}
  &T_1(z) = \zeta \iff \frac{z - w}{1 - \overline w z} = \zeta \iff z - w = \zeta - \overline w \zeta z \iff \\
  \iff\;& z(1 + \overline w \zeta) = \zeta + w \iff z = \frac{\zeta + w}{1 + \overline w \zeta}.
 \end{align*} Функция $ \zeta \mapsto (\zeta + w) / (1 + \overline w \zeta) $ также является множителем Бляшке с параметром~$ -w $, и по уже доказанному она действует в $ \mathbb D $. Таким образом, множитель Бляшке является конформным отображением $ T_1 \colon\,\mathbb D \to \mathbb D $.
\end{proof}

\end{document}
