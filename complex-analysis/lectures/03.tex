% 2023.03.09 lecture 03
\documentclass[../complex-analysis.tex]{subfiles}
\begin{document}

\newpage
\section{Степенные ряды.}

К сожалению, на первом курсе матанализа мы во многом упустили теорию рядов, в то время как некоторый её сюжет нам пригодится для построения теории аналитических функций. В этом параграфе мы коснёмся \emph{степенных рядов} --- своего рода <<бесконечных многочленов>>.

\begin{df}
 \textit{Степенным рядом} с центром в точке $ z_0 \in \CC $ и коэффициентами $\{c_{n}\}_{n=0}^{\infty} \subset \CC $  называется формальное выражение
 \begin{align*}
  F(z) = \sum_{n=0}^{\infty}c_n(z-z_0)^{n}.
 \end{align*} 
\end{df}

\subsection{Теорема Адамара.}

Так как слагаемых в степенном ряду бесконечно много, то встаёт вопрос о сходимости степенного ряда. В одной точке ряд уж точно сходится (в точке $z_0$ имеем $ F(z_0) = c_0 $), а в других --- априори не ясно.

\newcommand{\Rconv}{\ensuremath R_{\text{сх}}}

\begin{df}
 \textit{Радиусом сходимости} степенного ряда $ F(z) = \sum_{n=0}^{\infty} c_n(z-z_0)^{n}$ называется число
 \begin{align*}
  \Rconv = \sup \left\{ \rho \geqslant 0 \mid \text{ряд $ F(z) $ сходится в $ B(z_0, \rho) $} \right\}.
 \end{align*}
\end{df}

Радиус сходимости может быть как нулевой, так и равный $ +\infty $ (когда степенной ряд сходится всюду в $ \CC $).

Следующий результат является ключевым в теории степенных рядов: теорема Адамара предоставляет общую формулу, которая по абы какому степенному ряду выдаёт его радиус сходимости.

\begin{thm}[Адамара]
 \label{theorem:adamar}
 Радиус сходимости степенного ряда
 \begin{align*}
  F(z) = \sum_{n=0}^{\infty} c_n(z - z_0)^{n}
 \end{align*} равен
 \begin{align*}
  \Rconv = \frac{1}{L},
 \end{align*} где
 \begin{align*}
  L = \limsup_{n \to \infty} \left| c_n \right|^{1 / n}.
 \end{align*}
\end{thm}

Теорема~\ref{theorem:adamar} верна и при критических случаях: если $ L = 0 $, то $ \Rconv = +\infty $ (то есть ряд сходится всюду в $ \CC $), а если $ L = +\infty $, то $ \Rconv = 0 $ (то бишь есть сколь угодно близкая к нулю точка, в которой ряд не сходится).

\begin{proof}[\normalfont{\textsc{Доказательство теоремы~\ref{theorem:adamar}}}]
 Докажем сначала неравенство в одну сторону:
 \begin{align*}
  \Rconv \geqslant \frac{1}{L}.
 \end{align*} Если $ L = +\infty $, то доказывать нечего (и так понятно, что $ \Rconv \geqslant 0 $), поэтому считаем $ 0 \leqslant L < +\infty $.

 Рассмотрим любой радиус $ \rho \in (0, 1/L) $ и покажем $ \Rconv \geqslant \rho $. Для $ \rho $ существует число~$\eps > 0 $ такое, что
 \begin{align*}
  \rho \leqslant \frac{1-\eps}{L}.
 \end{align*} Перепишем неравенство:
 \begin{align*}
  \limsup\limits_{n\to \infty} \rho \left| c_n \right|^{1 / n} \leqslant 1 - \eps.
 \end{align*} Тогда при всех $ n > N $ верно
 \begin{align*}
  \rho \left| c_n \right|^{1 / n} \leqslant 1 - \frac{\eps}{2},
 \end{align*} что эквивалентно
 \begin{align*}
  \left| c_n \right| \rho^{n} \leqslant \left( 1 - \frac{\eps}{2} \right)^{n}.
 \end{align*} Значит, для любой точки~$z \in B(z_0, \rho)$ верно
 \begin{align*}
  \left| \sum_{n=0}^{\infty} c_n(z-z_0)^{n} \right| \leqslant \OO(1) +  \sum_{n=N}^{\infty} \left| c_n \right| \rho^{n} \leqslant \OO(1) +  \sum_{n=N}^{\infty} \left(1 - \frac{\eps}{2}\right)^{n} < \infty.
 \end{align*} Таким образом, $R_{\text{сх}} \geqslant \rho$. Так как это верно для любого $\rho \in (0, 1 / L)$, то
 \begin{align*}
  R_{\text{сх}} \geqslant \frac{1}{L}.
 \end{align*}

 Осталось доказать в другую сторону. Предположим, ряд $ F(z) $ сходится в замкнутом шаре $ \overline B(z_0,\rho) $ для некоторого $\rho > \frac{1}{L}$. Тогда
 \begin{align*}
  \rho \geqslant \frac{1 + \eps}{L}
 \end{align*} для некоторого $\eps > 0$, из чего следует
 \begin{align*}
  \limsup\limits_{ n \to \infty } \rho \left| c_n \right|^{1/ n} \geqslant 1 + \eps.
 \end{align*} Тогда при бесконечно многих $ n $ выполняется
 \begin{align*}
  \rho \left| c_n \right|^{1 / n} \geqslant 1 + \frac{\eps}{2},
 \end{align*} или же
 \begin{align*}
  \left| c_n \right|\rho^{n} \geqslant \left( 1 + \frac{\eps}{2} \right)^{n}.
 \end{align*} Тогда в точке~$ w = z_0 + \rho $ не может быть сходимости, так как общий член ряда не стремится к нулю:
 \begin{align*}
  \left| c_n(w-z_0)^{n} \right| = \left| c_n \right| \cdot \rho^{n} \geqslant \left( 1 + \frac{\eps}{2} \right)^{n} \to +\infty
 \end{align*} для бесконечно многих $ n $.
\end{proof}

\subsection{Примеры степенных рядов.}

Рассмотрим несколько канонических примеров степенных рядов, и применим к ним теорему Адамара.

\begin{exmpl}[экспонента]
 \emph{Экспонентой} называется степенной ряд
 \begin{align*}
  e^{z} = \sum_{n=0}^{\infty}\frac{z^{n}}{n!}.
 \end{align*} Найдём радиус сходимости экспоненты, пользуясь теоремой~\ref{theorem:adamar} Адамара и формулой Стирлинга:
 \begin{gather*}
  n! = \sqrt{2\pi n} \left( \frac{n}{e} \right)^{n} \cdot (1 + o(1)),\\
  \left( n! \right)^{1 / n} = e^{\frac{1}{2n}\log(2\pi n)} \cdot \frac{n}{e} \cdot (1 + o(1))^{1 / n} = \frac{n}{e} \cdot e^{o(1)} \cdot e^{1 / n \cdot o(1)} = \frac{n}{e} \cdot (1 + o(1)),\\
  L = \limsup_{n \to \infty} \left( \frac{1}{n!} \right)^{1 / n} = \limsup_{n \to \infty} \frac{e}{n} = 0,\\
  \Rconv = \frac{1}{L} = +\infty.
 \end{gather*} Таким образом, ряд $ e^{z} $ сходится всюду в $ \CC $.
\end{exmpl}

\begin{exmpl}
 Ряды
 \begin{align*}
  \cos z = \frac{e^{iz}+e^{-iz}}{2}, && \sin z = \frac{e^{it}-e^{-iz}}{2i}
 \end{align*} также сходятся всюду в $ \CC $.
\end{exmpl}

\begin{exmpl}[сумма геометрической прогрессии]
 Рассмотрим ряд
 \begin{align*}
  \frac{1}{1-z}=\sum_{n=0}^{\infty} z^{n}.
 \end{align*} По теореме~\ref{theorem:adamar} Адамара:
 \begin{align*}
  \Rconv = \left[\limsup\limits_{n\to \infty} 1^{1/n}\right]^{-1} = 1.
 \end{align*} 
\end{exmpl}
\begin{exmpl}
 Рассмотрим многочлен $ \sum_{k=0}^{N} c_{k} z^{k}$. Тогда по теореме~\ref{theorem:adamar} Адамара:
 \begin{align*}
  \Rconv = \left[ \limsup_{n \to \infty} \left| c_n \right|^{1 / n} \right]^{-1} = \left[ \limsup_{n \to \infty} 0 \right]^{-1} = +\infty.
 \end{align*} 
\end{exmpl}

\subsection{Комплексная дифференцируемость степенных рядов.}

\begin{crly}
 \label{corollary:complex_differential_of_power_series}
 Пусть степенной ряд
 \begin{align*}
  f(z) = \sum_{n=0}^{\infty} c_n(z - z_0)^{n}
 \end{align*} имеет радиус сходимости~$ \Rconv $. Тогда в любой точке сходимости~$ z \in B(z_0,\Rconv) $ ряд~$ f(z) $ имеет комплексную производную, равную
 \begin{align*}
  f'(z) = \lim_{h \to 0} \frac{f(z + h) - f(z)}{h} = \sum_{n=1}^{\infty}nc_n(z-z_0)^{n-1},
 \end{align*} причём ряд в правой части имеет тот же радиус сходимости $\Rconv$.
\end{crly}

\begin{lm}
 \label{lemma:complex_n_w_inequality}
 Для любых различных чисел $z, w \in \CC$, $z \neq w$ и для любого $n \geqslant 1$ верно неравенство
 \begin{align*}
  \left|\frac{z^{n} - w^{n}}{z - w} \right| \leqslant n \left(\left| z \right|^{n-1} + \left| w \right|^{n-1}\right)
 \end{align*} 
\end{lm}
\begin{proof}[\normalfont\textsc{Доказательство}]
 В силу симметрии можно считать, что $\left| z \right| \geqslant \left| w \right|$. Выделим главный член асимптотики:
 \begin{align*}
  \left| \frac{z^{n}-w^{n}}{z-w} \right| &= \left| z \right|^{n-1} \cdot \left| \frac{1 - \left( \frac{w}{z} \right)^{n}}{1 - \frac{w}{z}} \right| = \left| z \right|^{n-1} \left| 1 + \frac{w}{z} + \ldots + \left( \frac{w}{z} \right)^{n-1} \right| \leqslant \\
  &\leqslant n \cdot \left| z \right|^{n-1} \leqslant n \left( \left| z \right|^{n-1} + \left| w \right|^{n-1} \right).
 \end{align*} 
\end{proof}

\begin{proof}[\normalfont\textsc{Доказательство следствия \ref{corollary:complex_differential_of_power_series}}]
 Для простоты будем считать $z_0 = 0$ (иначе сдвинем). Сначала докажем, что радиус сходимости ряда
 \begin{align*}
  \sum_{n=1}^{\infty}nc_nz^{n-1}
 \end{align*} также равен $ \Rconv $. В самом деле,
 \begin{align*}
  \limsup_{n \to \infty} \left| (n+1)c_{n+1} \right|^{1 / n} = \limsup_{n \to \infty} \left| c_{n} \right|^{1 / n} = \Rconv^{-1},
 \end{align*} поскольку
 \begin{align*}
  \lim_{n \to \infty} (n+1)^{1 / n} = 1.
 \end{align*} Поэтому, по теореме~\ref{theorem:adamar} Адамара радиусы действительно совпадают.

 Теперь возьмём любую точку $ z \in B(z_0, \Rconv) $ и найдём производную $ f $ в этой точке:
 \begin{align}
  \label{eq:derivate_of_series:swap} f'(z) &= \lim_{h \to 0} \frac{f(z+h)-f(z)}{h} =  \lim_{h \to 0} \sum_{n=0}^{\infty} c_n \cdot \frac{(z+h)^{n}-z^{n}}{h}=\\
  \notag &= \sum_{n=0}^{\infty} c_n \cdot \lim_{h \to 0} \frac{(z+h)^{n}-z^{n}}{h} = \sum_{n=0}^{\infty} c_n \cdot \lim_{h\to0} \frac{nz^{n-1}h + o(h)}{h} = \sum_{n=0}^{\infty}nc_nz^{n-1}.
 \end{align} Нужно лишь обосновать переход~\eqref{eq:derivate_of_series:swap}. Здесь мы выстрелим из пушки по воробьям: воспользуемся теоремой Лебега о мажорируемой сходимости из теории меры. Здесь пространство с мерой --- это считающая мера на  $ \N \cup \left\{ 0 \right\} $; от нас требуется найти неотрицательную функцию $ g $, то есть последовательность $ \{g_{n}\}_{n=0}^{\infty} \subset \R_+$ такую, что
 \begin{align*}
  \left| c_n \cdot \frac{(z+h)^{n}-z^{n}}{h} \right| \leqslant g_n,
 \end{align*} и
 \begin{align*}
  \sum_{n=0}^{\infty} g_n < +\infty.
 \end{align*}

 Пусть $ \rho $ --- такое число, что $ \left| z \right| < \rho < \Rconv $. При малых $ h $ также будет верно $ \left| z+h \right| < \rho $. Тогда по лемме~\ref{lemma:complex_n_w_inequality} можно написать оценку
 \begin{align*}
  \left| c_n \cdot \frac{(z+h)^{n}-z^{n}}{h} \right| \leqslant \left| c_n \right| \cdot n \cdot \left( \left| z+h \right|^{n-1} + \left| z \right|^{n-1} \right) \leqslant 2n \left| c_n \right| \rho^{n-1}.
 \end{align*} Раз так, то положим
 \begin{align*}
  g_n = 2 n \left| c_n \right|\rho^{n-1}.
 \end{align*}

 Осталось проверить суммируемость функции $ g $, то есть сходимость ряда
 \begin{align*}
  \sum_{n=1}^{\infty} n \left| c_n \right| \rho^{n - 1} < \infty.
 \end{align*} Но этот ряд сходится, поскольку $ \rho < \Rconv $, а $ \Rconv $, как мы уже установили, и есть радиус сходимости указанного ряда.
\end{proof}

\newpage
\section{Аналитические функции.}

\subsection{Теорема Коши-Гурса-Морера.}

Наконец, мы готовы дать определение \emph{аналитической} функции. Как часто бывает, здесь есть несколько эквивалентных определений, и их эквивалентность сама по себе является фундаментальным результатом, который в данном случае называется \emph{теоремой Коши-Гурса-Морера}.

\begin{thm}[%
 Коши-Гурса-Морера]
 \label{theorem:cauchy-gursa-morer}
 Пусть $\Omega \subset \CC$  --- область, $f\colon\,\Omega \to \CC$  --- функция. Тогда следующие условия равносильны.
 \begin{enumerate}
  \item \label{enum1:theorem:cauchy_gurs_morer} Функция $ f $ в любой точке $z \in \Omega$ имеет \emph{комплексную производную:}
   \begin{align*}
    f'(z) := \lim_{h \to 0} \frac{f(z+h)-f(z)}{h}.
   \end{align*} 
  \item \label{enum2:theorem:cauchy_gurs_morer} Дифференциальная форма $f\,dz$ непрерывна и замкнута в области~$\Omega$.
  \item \label{enum3:theorem:cauchy_gurs_morer} Для любой точки $ z_0 \in \Omega $ существует такой открытый диск $ B(z_0, r) \subset \Omega $, что для любой точки~$ z \in B(z_0,r) $ выполнено равенство
   \begin{align}
    \label{eq:series:cauchy_gurs_morer}
    f(z) = \sum_{n=0}^{\infty} c_n(z-z_0)^{n},
   \end{align} где ряд~\eqref{eq:series:cauchy_gurs_morer} сходится на $B(z_0, r)$ и называется \emph{рядом Тейлора} функции $ f $ в точке $ z_0 $.
 \end{enumerate}
 Более того,  если \ref{enum1:theorem:cauchy_gurs_morer}-\ref{enum3:theorem:cauchy_gurs_morer} выполнены, то в пункте \ref{enum3:theorem:cauchy_gurs_morer} можно взять максимально возможный радиус $r = \mathrm{dist}(z_0, \CC \setminus \Omega).$
\end{thm}
\begin{df}
 Функция~$ f $, удовлетворяющая условиям \ref{enum1:theorem:cauchy_gurs_morer}-\ref{enum3:theorem:cauchy_gurs_morer} в теореме~\ref{theorem:cauchy-gursa-morer} Коши-Гурса-Морера, называется \textit{аналитической} в области $ \Omega $.
\end{df}
\begin{proof}[\normalfont\textsc{Доказательство теоремы~\ref{theorem:cauchy-gursa-morer}}]\
 \begin{itemize}
  \item Из условия \ref{enum1:theorem:cauchy_gurs_morer} следует условие \ref{enum2:theorem:cauchy_gurs_morer}. Ясно, что из комплексной дифференцируемости функции~$ f $ на $ \Omega $ следует её непрерывность на $ \Omega $, поэтому дифференциальная форма $ f\,dz $ уж точно непрерывна.

   Замкнутость формы $ f\,dz $ мы, как обычно, проверим <<тестом на прямоугольнике>> (условие~\ref{enum3:theorem:closed_1_form} из теоремы~\ref{theorem:closed_1_form}): для любого замкнутого прямоугольника $ \Pi \subset\Omega $ со сторонами, параллельными осям координат, нужно проверить $ \int_{\partial\Pi} f\,dz=0  $. Проверим это от противного: предположим, что существует прямоугольник $ \Pi_0 \subset \Omega $ и число $ \eps > 0 $ такое, что
   \begin{align*}
    \left| \int_{\partial\Pi_0} f\,dz   \right| \geqslant \eps \delta_0^{2},
   \end{align*} где $ \delta_0 $ --- диаметр (длина диагонали) прямоугольника $ \Pi_0 $.

   Разобьём прямоугольник $ \Pi_0 $ на четыре равных прямоугольника $ Q_1, Q_2, Q_3, Q_4 $ (рисунок \ref{fig:theorem_cauchy_gurs_morer_rect}).

   \begin{figure}[ht]
    \centering
    \incfig[0.5]{theorem_cauchy_gurs_morer_rect}
    \caption{Разбиение прямоугольника на 4 меньших.}
    \label{fig:theorem_cauchy_gurs_morer_rect}
   \end{figure}

   По аддитивности интеграла верно
   \begin{align*}
    \int_{\partial\Pi_0} f\,dz = \int_{\partial Q_1} f\,dz + \int_{\partial Q_2}  f\,dz + \int_{\partial Q_3} f\,dz + \int_{\partial Q_4} f\,dz,  
   \end{align*} поэтому среди четырёх прямоугольников существует прямоугольник $ Q_i $ такой, что
   \begin{align*}
    \left| \int_{\partial Q_i} f\,dz \right| \geqslant \frac{\eps \delta_0^{2}}{4} = \eps \delta_1^{2},
   \end{align*} где $ \delta_1 = \delta_0 / 2 $ --- диаметр каждого из четырёх прямоугольников. Обозначим этот прямоугольник через $ \Pi_1 = Q_i $.

   По прямоугольнику~$\Pi_1$ таким же образом построим прямоугольник~$\Pi_2$, затем по $ \Pi_2 $ построим $\Pi_3$, и так далее. Мы получим последовательность вложенных прямоугольников
   \begin{align*}
    \Pi_0 \supset \Pi_1 \supset \Pi_2 \supset \ldots,
   \end{align*} обладающих свойством \begin{align}
    \label{eq:large_rect_int:cauchy_gurs_morer}
    \left| \int_{\partial\Pi_n} f\,dz  \right| \geqslant \eps\delta_n^{2}.
   \end{align} По теореме о вложенных компактах их пересечение не пустое:
   \begin{align*}
     a  \in \bigcap_{n=0}^{\infty}\Pi_n.
   \end{align*} Теперь, пользуясь комплексной дифференцируемостью функции~$ f $, выразим интеграл функции по контуру $ \partial\Pi_n $ при $ n \to \infty $ в терминах производной в точке $ z $:
   \begin{align*}
    \int_{\partial\Pi_n} f(z)\,dz &= \int_{\partial\Pi_n} \left( f(a) + f'(a) \cdot (z-a) + o(z-a) \right)dz = \\
    &= \int_{\partial\Pi_n} \left( p+qz \right)dz + \int_{\partial\Pi_n} o(z-a)\,dz
   \end{align*} для некоторых комплексных чисел~$ p,q\in\CC $. Но так как форма~$ (p+qz)\,dz $ замкнута в $ \CC $ (пример~\ref{example:form_a_plus_b_times_z_dz}), то левое слагаемое равно нулю, и тогда по оценке интеграла \eqref{eq:bound_on_absolute_value_of_int}:
   \begin{align*}
    \int_{\partial\Pi_n} f\,dz  &= \int_{\partial\Pi_n}  o(z-a)\,dz = \int_{\partial\Pi_n} o(\delta_n) =   o(\delta_n^{2}).
   \end{align*} Но это противоречит \eqref{eq:large_rect_int:cauchy_gurs_morer}! Таким образом, тест на прямоугольнике пройдён, и форма~$ f\,dz $ действительно замкнута в $ \Omega $.

  \item Из условия \ref{enum1:theorem:cauchy_gurs_morer} следует условие \ref{enum3:theorem:cauchy_gurs_morer}. Пусть $ D \subset \Omega$ --- открытый диск с центром в точке~$ a \in \Omega $ и любым (хоть максимально возможным) радиусом~$ r > 0 $. Зафиксируем точку~$ b \in D $, для неё рассмотрим функцию
   \begin{align*}
    g(z) = \frac{f(z)-f(b)}{z-b}
   \end{align*} и соответствующую ей дифференциальную форму $ g\,dz $. Так как функция~$ f $ по условию~\ref{enum1:theorem:cauchy_gurs_morer} комплексно дифференцируема в точке~$ b $, то функция~$ g $ определена и непрерывна на диске~$ D $. Более того, функция~$ g $ комплексно дифференцируема (то есть удовлетворяет условию~\ref{enum1:theorem:cauchy_gurs_morer}) в области~$ D \setminus \left\{ b \right\} $ как частное комплексно дифференцируемых функций (ведь знаменатель $ z-b $ не обращается в нуль в области~$ D \setminus \left\{ b \right\} $).

   \begin{figure}[ht]
    \centering
    \incfig{theorem_cauchy_gurs_morer_1_to_3}
    \caption{Импликация из \ref{enum1:theorem:cauchy_gurs_morer} в \ref{enum3:theorem:cauchy_gurs_morer}.}
    \label{fig:theorem_cauchy_gurs_morer_1_to_3}
   \end{figure}

   Тогда, по уже доказанному следованию из условия~\ref{enum1:theorem:cauchy_gurs_morer} в условие~\ref{enum2:theorem:cauchy_gurs_morer}, дифференциальная форма~$ g\,dz $ замкнута в области~$ D \setminus \left\{ b \right\} $. Но так как она также непрерывна в~$ D $, то по лемме~\ref{lemma:ob_ustranenii_osobennosti} об устранении особенности форма $ g\,dz $ замкнута в диске~$ D $.

   Пусть радиус $ \rho $ выбран так, что $ \left| b-a \right| < \rho < r $; обозначим через $ C_\rho $  окружность с центром в $ a $  и радиусом~$ \rho $. Проинтегрируем форму~$ g\,dz $ по этой окружности:
   \begin{align}
    \label{eq:int_g_eq_0:cauch_gurs_morer}
    \int_{C_\rho} g\,dz = 0,
   \end{align} поскольку форма~$ g\,dz $ замкнута в области~$ D $, а окружность~$ C_\rho $ можно стянуть в точку. С другой стороны,
   \begin{align}
    \label{eq:int_g_eq_diff:cauchy_gurs_morer}
    \int_{C_\rho} g\,dz = \int_{C_\rho} \frac{f(z)\,dz}{z-b} - f(b) \int_{C_\rho} \frac{dz}{z-b}.
   \end{align}

   По формуле~\eqref{eq:int_form:dz/z-a} (лемма~\ref{lemma:form:dz/z-a}) имеем
   \begin{align*}
    \int_{C_\rho} \frac{dz}{z-b} = 2\pi i.
   \end{align*} Подставляя обратно в \eqref{eq:int_g_eq_0:cauch_gurs_morer} и \eqref{eq:int_g_eq_diff:cauchy_gurs_morer}, получаем, так называемую, \emph{формулу Коши}:
   \begin{align}
    \label{eq:cauchy_formula:cauchy_gurs_morer}
    f(b) = \frac{1}{2\pi i} \int_{C_\rho} \frac{f(z)\,dz}{z-b}.
   \end{align}

   Далее, дробь $ 1/(z-b) $ можно разложить в сумму геометрической прогрессии:
   \begin{align}
    \label{eq:sum_geom_series:cauchy_gurs_morer}
    \frac{1}{z-b} = \frac{1}{(z-a)+(a-b)} = \frac{1}{z-a} \cdot \frac{1}{1 - \frac{b-a}{z-a}} = \sum_{n=0}^{\infty} \frac{(b-a)^{n}}{(z-a)^{n+1}}.
   \end{align} Ряд~\eqref{eq:sum_geom_series:cauchy_gurs_morer} сходится для всех $ z \in C_\rho $, причём равномерно по $ z $, ведь
   \begin{align*}
    \left| \frac{b-a}{z-a} \right| = \frac{\left| b-a \right|}{\rho} < 1,
   \end{align*} причём равномерность не нарушится после умножения на $ f(z) $. Подставим полученное в \eqref{eq:cauchy_formula:cauchy_gurs_morer}:
   \begin{align*}
    f(b) &= \frac{1}{2\pi i} \int_{C_\rho} f(z) \cdot \sum_{n=0}^{\infty}\frac{(b-a)^{n}}{(z-a)^{n+1}}.
   \end{align*} Так как ряд~\eqref{eq:sum_geom_series:cauchy_gurs_morer} сходится равномерно (всё суммируемо), то по теореме Фубини можно поменять местами сумму и интеграл. Получаем
   \begin{align}
    \label{eq:taylor_series:cauchy_gurs_morer}
    f(b) = \sum_{n=0}^{\infty} c_n (b-a)^{n},
   \end{align} где
   \begin{align*}
    c_n = \frac{1}{2\pi i} \int_{C_\rho} \frac{f(z)\,dz}{(z-a)^{n+1}}.
   \end{align*} Равенство~\eqref{eq:taylor_series:cauchy_gurs_morer} верно для любого $ b \in B(a,\rho) $.

   {\color{red} Почему коэффициенты $ c_n $ не зависят от $ \rho $?}

   Теперь можно устремить $ \rho \to r $ и получить равенство~\eqref{eq:taylor_series:cauchy_gurs_morer} на всём диске $ D $.

  \item Из условия~\ref{enum3:theorem:cauchy_gurs_morer} следует условие~\ref{enum1:theorem:cauchy_gurs_morer}. Пусть функция $ f $ раскладывается в ряд Тейлора
   \begin{align*}
    f(z) = \sum_{n=0}^{\infty} c_n(z-z_0)^{n}
   \end{align*} в открытом диске $ D $ с центром в точке $ z_0 \in \Omega $. Тогда по следствию~\ref{corollary:complex_differential_of_power_series} о комплексной дифференцируемости степенных рядов $ f $ имеет комплексную производную на диске~$ D $, и, в частности, в точке $ z_0 $.

  \item Наконец, из условия~\ref{enum2:theorem:cauchy_gurs_morer} следует условие~\ref{enum3:theorem:cauchy_gurs_morer}. Пусть форма~$ f\,dz $ замкнута в $ \Omega $, а $ D \subset \Omega $ --- открытый диск с центром в точке $ \Omega $. Так как область~$ D $ односвязная, то по следствию~\ref{corollary:closed_form_is_exact_in_simply_connected} форма~$ f\,dz $ точна в $ D $. По лемме~\ref{lemma:o_pervoobraznoi} о первообразной существует функция $ g\colon D\to\CC $ такая, что $ g'(z) = f(z) $ всюду в $ D $. Эта функция~$ g $ удовлетворяет условию~\ref{enum1:theorem:cauchy_gurs_morer} в области~$ D $, а значит по уже доказанному следованию она удовлетворяет условию~\ref{enum3:theorem:cauchy_gurs_morer}, то есть выполнено равенство
   \begin{align*}
    g(z) = \sum_{n=0}^{\infty} a_n(z-z_0)^{k},
   \end{align*} причём ряд сходится в $ D $. Но по следствию~\ref{corollary:complex_differential_of_power_series} о комплексной производной степенного ряда верно
   \begin{align*}
    f(z) = g'(z) = \sum_{n=1}^{\infty}na_n(z-z_0)^{n-1},
   \end{align*} причём ряд также сходится в $ D $. Следовательно, $ f $ тоже удовлетворяет условию~\ref{enum3:theorem:cauchy_gurs_morer}.
 \end{itemize}
\end{proof}

Заметим, что дифференциальные формы в теореме~\ref{theorem:cauchy-gursa-morer} Коши-Гурса-Морера играют ключевую роль: без них мы не смогли бы вывести из комплексной дифференцируемости (условия~\ref{enum1:theorem:cauchy_gurs_morer}) разложение в ряд Тейлора (условие~\ref{enum3:theorem:cauchy_gurs_morer}).

\end{document}
