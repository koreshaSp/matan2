% 2023.04.20 lecture 09
\documentclass[../complex-analysis.tex]{subfiles}
\begin{document}
 Более подробно о доказательстве теоремы Руше. Функции $ f $ и $ g $ аналитичны в $ \Omega $, $ \left| g \right| < \left| f \right| $ на $ \partial\Omega $, $ f, g \in C^{1}(\overline\Omega) $. Нужно было доказать для всех $ k $
 \begin{align*}
  \Delta_{\gamma_k} \arg \left( 1 + \frac{g}{f} \right) = 0.
 \end{align*} Мы заметили, что
 \begin{align*}
  \Real\left(1 + \frac{g}{f}\right) = 1 + \Real \frac{g}{f} > 1 - \left| \frac{g}{f} \right| > 0 
 \end{align*} на $ \partial\Omega $. Отсюда вывод
 \begin{align*}
  \cos(\psi(t)) > 0
 \end{align*} если $ \psi $ такая, что  $ 1 + \frac{g(\gamma_k(t))}{f(\gamma_k(t))} = \left| 1 + \frac{g(\gamma_k(t))}{f(\gamma_k(t))} \right|e^{i\psi(t)} $. Поэтому, $ \psi(t) \in (- \pi / 2, \pi / 2) + 2\pi k (t) $ для любого  $ t $ . Здесь $ k(t) \in \Z $. Поэтому
  \begin{align*}
  \psi([0,1]) \subset \bigcup_{k \in \Z} (-\pi / 2, \pi / 2) + 2\pi k.
 \end{align*} Слева стоит связное множество (как непрерывный образ связного). Второе множество можно нарисовать {\color{red} рисунок}. Значит, весь образ лежит только в одном интервале: существует $ k_0 \in \Z $: $ \psi([0,1]) \subset (-\pi / 2, \pi / 2) + 2\pi k_0 $. Тогда
 \begin{align*}
  \left|\psi(1) - \psi(0) \right| < \pi,
 \end{align*} а с другой стороны,
 \begin{align*}
  \left( 1+ \frac{g}{f} \right)(\gamma_k(0)) = \left(1 + \frac{g} f\right)(\gamma_k(1)) \\
  \implies e^{i\psi(0)} = e^{i\psi(1)}.
 \end{align*} См (*). Значит, $ \psi(1) - \psi(0) = 2\pi N $, $ N \in \Z $. Значит
 \begin{align*}
  \left|2\pi N \right| < \pi \implies N = 0.
 \end{align*} Значит $ \psi(1) -\psi(0)=0 $, поэтому
 \begin{align*}
  \Delta_{\gamma_k} \arg \left( 1 + \frac{g}{f} \right) = 0.
 \end{align*}

 \begin{exmpl}
  Найдём число корней многочлена $ 0.1+2z^{3} + z^{10} = 0 $ в кольце $ 1 < \left| z \right|  < 2 $. Рассмотрим область $ \Omega = B(0,1) $. Рассмотрим функцию
  \begin{align*}
   f = 2z^{3}, &&g = 0.1 + z^{10}.
  \end{align*} Тогда $ \left| g \right| \leqslant 1.1 $ на $ \left| z \right|=1 $, а $ \left| f \right| = 2 > \left| g \right| $ на $ \left| z \right|=1 $. Значит, $f + g$
  имеет столько же нулей, сколько и $ f $: то есть три штуки (с учётом кратности). Кроме того, на $ \left| z \right| = 1 $ нет нулей.

  А теперь посчитаем сколько нулей в $ \Omega=B(0,2) $. Но выберем там другие буквы $ f $ и $ g $:
  \begin{align*}
   f = z^{10}, && g = 0.1  + 2z^{3}.
  \end{align*} Тогда $ \left| f \right| = 1024 $ на $ \left| z \right|=2 $, а $ \left| g \right| < 0.1 + 16 $ на $ \left| z \right| = 2 $. По теореме Руше у $ f+g $ столько же нулей, сколько и у $ f $. То есть $ 10 $ нулей (с учётом кратности) в круге $ B(0, 2) $. 

  Итого, в кольце $ 1 < \left| z \right| < 2 $ будет $ 10-3 = 7 $ нулей.
 \end{exmpl}

 Теорема Руше часто используется в задачах теории возмущений. Другой пример: оператор Шрёдингера, про частицу в потенциальном поле.

 \begin{crly}[теорема Гурвица]
  Пусть $ f_k $ --- аналитические функции в области $ \Omega $, и пусть $ f_k $ сходятся на компактах в $ \Omega $ к функции $ f $. Пусть $ z_0 \in \Omega $: $ f(z_0)=0 $. Тогда либо $ f \equiv 0 $ в $ \Omega $, либо $ z_0 $ --- корень кратности $ k_0 \in \N $, и существует $ \eps > 0 $ и $ N \in \N $ такие, что
  \begin{itemize}
   \item $ B(z_0, \eps) \subset \Omega $.
   \item $ f_k $ имеет ровно $ k_0 $ нулей в $ B(z_0, \eps) $ с учётом кратности для любого $ k > N $.
  \end{itemize}
 \end{crly}
 \begin{proof}[\normalfont\textsc{Доказательство}]
  Поймём, что $ f $ аналитична в $ \Omega $ (очевидный тест на прямоугольниках). Считаем, что $ f \not\equiv 0 $ в $ \Omega $. Тогда по лемме о кратности нуля $ f = g\cdot(z - z_0)^{k_0} $, где $ g $ аналитична в $ \Omega $, $ g(z_0) \neq 0 $. Значит, существует $ \eps > 0 $ такое, что $ B(z, 2\eps) \subset \Omega $, и $ \left| g(z) \right| > \delta > 0 $ для любого $ z $: $ \left| z - z_0 \right| = \eps $ (по непрерывности $ \delta = \left| g(z_0) \right| / 2 $). Значит
  \begin{align*}
   \left|f(z) \right| > \delta \eps^{k_0}, \qquad \left| z-z_0 \right|=\eps.
  \end{align*} Напишем
  \begin{align*}
   f_k = f + (f_k - f) = f + g_k.
  \end{align*} Так как $ f_k \rightrightarrows f $ равномерно на $ \left| z \right| = \eps $, то существует число $ N $ такое, что
  \begin{align*}
   \left|g_k(z) \right| < \frac{\delta \cdot \eps^{k_0}}{2}
  \end{align*} для любого $ \left| z-z_0 \right|=\eps $ и $ k \geqslant N $. По теореме Руше $ f $ и $ f + g_k $ имеет одинаковое число нулей в $ B(z_0,\eps) $ при $ k \geqslant N $.
 \end{proof}

 Мы доказали чуть больше: $ \exists \eps_0 > 0 \colon\; \forall \eps \in (0, \eps_0) \colon\; B(z,\eps) \in \Omega \colon\; \exists N \colon\;\ldots  $ и так далее.

 \begin{thm}
  Пусть $ f_k $ --- аналитические функции в $ \Omega $, $ f_k $ сходятся равномерно на компактах к функции $ f $. Пусть $ f_k $ инъективны в $ \Omega $. Тогда либо $ f \equiv \mathrm{const} $, либо $ f $ инъективна.
 \end{thm}
 \begin{proof}[\normalfont\textsc{Доказательство}]
  \begin{align*}
   f = \lim f_k
  \end{align*} --- аналитическая функция в $ \Omega $ (тест на прямоугольниках). Пусть существуют точки $ z_0,w_0 \in \Omega $ такие, что $ f(z_0) - f(w_0) = 0 $. Рассмотрим функции $ F_k \colon\, z \mapsto f_k(z) - f_k(w_0) $, а
  \begin{align*}
   F := f(z) - f(w_0)
  \end{align*} Дано $F_k$ сходится к $ F $ равномерно на компактах (от вычитания константы ничего не поменялось). С другой стороны, $ F(z_0) = 0 $. Возьмём $ \eps $ такое, что $ \left| z_0 - w_0 \right| / 4 > \eps $ и достаточно маленькое для того, чтобы выполнялась теорема Гурвица для $ F_k $ и $ F $. Пусть $ N $ --- построили по $ \eps $ из т. Гурвица такое, что $ F_k $ имеет ноль в $ B(z_0, \eps) $. Значит $ \forall n > N(\eps) \colon\; z_n \in B(z_0,\eps): F_n(z_n) = 0 \implies f_n(z_n) - f_n(w_0) = 0 $. Но так как $ f_n $ инъективны, то $ z_n \in w_0 $. Но $ z_n \in B(z_0, \eps) $. То есть $ \left|z_n - w_0 \right| < \eps < \frac{\left| z_0-w_0 \right|}{4} $, противоречие. Если же $ F \equiv 0 $, то $ f \equiv \mathrm{const} $, что разрешено.
 \end{proof}
 
 \begin{thm}
  Пусть $ f $ --- инъективная аналитическая функция в $ \Omega $. Тогда $ f'(z) \neq 0 $ для всех $ z \in \Omega $.
 \end{thm}
 Можно доказать вещественными методами ($ f \colon \R^{2}\to\R^{2} $).
 \begin{proof}[\normalfont\textsc{Доказательство}]
  Рассмотрим функции
  \begin{align*}
   f_n = \left(f(z + \frac{1}{n}) - f(z) \right)n, \qquad n \in \N. 
\end{align*} заданы при больших $ n $ в шаре $ B(z_0, \eps) $ таком, что $ B(z_0, 2\eps) \subset \Omega $. Кроме того, $ f_n $ равномерно на всём шаре $ \overline {B(z_0,\eps)} $ сходится к $ f' $. Равномерность оценим
  \begin{align*}
   f_n(z) = \frac{f\left(z + \frac{1}{n}\right) - f(z)}{z + \frac{1}{n} - z} = g_z \left( \frac{1}{n} \right),
  \end{align*} где \begin{align*}
   g_z(w) = \frac{f(z+w)-f(z)}{w}
  \end{align*} $ g_z(0) = f'(z) $. Поэтому,
  \begin{align*}
   \left|f_n(z) - f'(z) \right| = \left|g_z(1 / n) - g_z(0) \right| \leqslant \max_{\left| \lambda \right| \leqslant \eps} \left| g'_z(\lambda) \right| \cdot \left| 1 / n - 0 \right|.
  \end{align*}
  \begin{align*}
   g'_z(w) = \frac{f'(z + w)w - f(z + w) + f(z)}{w^{2}}
  \end{align*}
  \begin{align*}
   \max_{\left| \lambda \right| \leqslant \eps} \left| g'_z(\lambda) \right| = \max_{\left| \lambda \right|=\eps} \left| g'_z(\lambda) \right| \leqslant \frac{1}{\eps^{2}} \max_{\zeta \in B(z_0,2\eps)} \left| f'(\zeta) \right| \eps + \left| f(\zeta) \right| \cdot 2 < C.
  \end{align*}

  Доказали равномерность. Если $ \exists z_0 \colon\; f'(z_0) = 0  $, то при больших $ n $
  \begin{align*}
   n(f(z + 1 / n) - f(z))
  \end{align*} имеет корень по теореме Гурвица. Тогда $ f $ не инъективна!
 \end{proof}


 На этом немного остановимся в теории аналитических функций. Наша цель --- самая красивая теорема Римана.

 Сегодня будем заниматься компактностью, и докажем классическую теорему из функционального анализа.

 \newpage
 \section{Теорема Арцела-Асколи.}

 \begin{df}
  Пусть $ (X,\rho) $ --- метрическое пространство, $ \left\{f_\alpha\right\}_{\alpha \in A}  $ --- семейство комплекснозначных, непрерывных на $ X $ функций. Говорят, что семейство $ \left\{f_\alpha\right\}_{\alpha\in A}  $ \textit{равностепенно непрерывно}, если для любого $ \eps > 0 $ существует $ \delta > 0 $ такое, что из $ \rho(x_1, x_2) < \delta $ следует
  \begin{align*}
   \left|f_\alpha(x_1) - f_\alpha(x_2) \right| < \eps
  \end{align*} для любого $ \alpha \in A $.
 \end{df}
 \begin{df}
  Семейство $ \left\{f_\alpha\right\}_{\alpha \in A}  $ \textit{равномерно ограничено}, если
  \begin{align*}
   \sup_{x \in X, \alpha \in A} \left| f_\alpha(x) \right| < \infty.
  \end{align*}
 \end{df}

 \begin{lm}
 	Пусть $(X, \rho)$ --- метрический компакт, то тогда $X$ --- сепарабельное.
 \end{lm}

 \begin{proof}[\normalfont\textsc{Доказательство}]
	 Возьмем покрытие $X$ шарами радиуса $\frac{1}{2}$, извлечем конечное подпокрытие.
	 После этого проделаем такую же процедуру для покрытия $X$ шарами радиуса $\frac{1}{3}$  и опять извлечем конечное подпокрытие.
	 Продолжаем по аналогии для всякого $n \in N$ .

	 Тогда центры этих шаров будут счетным всюду плотным множеством, т.к. для каждой точки и любого $\eps$ найдется шар радиуса $< \eps$ в котором эта точка лежит, следовательно найдется центр на расстоянии $< \eps$. 
 \end{proof}

 \begin{thm}[Арцела-Асколи]
  Пусть $ (X,\rho) $ --- метрический компакт.  $ \left\{f_\alpha\right\}_{\alpha \in A}  $ --- равномерно ограниченное равностепенно непрерывное семейство функций. Тогда $ \left\{f_\alpha\right\}_{\alpha \in A} $ предкомпактно, то есть существует $ \exists \left\{\alpha_k\right\} \colon\;  $ --- набор различных элементов $ A $ такой, что
  \begin{align*}
   f_{\alpha_k} \rightrightarrows_X f,
  \end{align*} где $ f $ непрерывна на $ X $.
 \end{thm}
 \begin{proof}[\normalfont\textsc{Доказательство}]
  Считаем $ A = \N $ . Так как $ X $  сепарабельный, то возьмём $ \left\{x_k\right\}  $ --- счётное всюду плотное множество в $X$

  Т.к. $\{ f_\alpha \}$ равномерно ограничены, значит можно выделить сходящуюся подпоследовательность $f_{n_1}, f_{n_2}, \ldots $ сходящуюся в точке $x_1$.
  Но это подпоследовательность тоже будет равномерно ограничена, поэтому из нее можно выделить подпоследовательность сходящуюся в точке $x_2$.
  Продолжая эту процедуру мы получаем семейство последовательностей где каждая вложена в предыдущую и сходится в конкретной точке.
  
\begin{figure}[ht]
    \centering
	\incfig[0.7]{cantor_diagonalization_theorem_arcel_askol}
    \caption{Диагональный метод Кантора.}
    \label{fig:cantor_diagonalization_theorem_arcel_askol}
\end{figure}

  Теперь мы можем взять диагональные элементы из каждой подпоследовательности и получить последовательность которая сходится во всех точках $x_i$.
  См. рис \eqref{fig:cantor_diagonalization_theorem_arcel_askol}. 
\end{proof}

\end{document}

