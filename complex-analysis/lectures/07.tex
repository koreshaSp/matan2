% 2023.04.06 lecture 07
\documentclass[../complex-analysis.tex]{subfiles}
\begin{document}

\subsection{Полюса.}

\begin{df}[полюс]
 Неустранимая изолированная особая точка~$ a\in\CC $ функции~$ f $ называется \emph{полюсом}, если для некоторого натурального числа $ n \geqslant 1 $ выполнено
 \begin{align*}
  f(z) = \OO \left( \frac{1}{(z-a)^{n}} \right)
 \end{align*} при $ z \to a $. \emph{Порядком} полюса~$ a $ называется минимальное такое число $ n \geqslant 1 $.
\end{df}

Устранимую особую точку можно считать полюсом порядка~$ 0 $, ведь в этом случае при $ z \to a $ выполнено $ f(z) = \OO (z-a)^{-0} = \OO(1) $, что и означает локальную ограниченность.

\begin{exmpl}\
 \begin{itemize}
  \item Точка $ z=0 $ является полюсом порядка~$ 2 $ функции $ f(z) =  1/z^{2} $.
  \item Точка $ z = 0 $ является полюсом порядка~$ 1 $ функции~$ f(z) = \sin z / z^{2} $, так как $ f(z) \sim 1 / z $ при $ z \to 0 $.
 \end{itemize}
\end{exmpl}

\begin{thm}
 \label{theorem:Pole}
 Изолированная особая точка~$ a\in\CC $ функции~$ f $ является полюсом порядка~$ n \geqslant 1 $, если и только если ряд Лорана функции~$ f $ в проколотой окрестности~$ \dot B(a,\eps) $ имеет вид
 \begin{align}
  \label{eq:Pole Laurent Series}
  f(z) = \sum_{k=-n}^{\infty} c_k(z-a)^{k},
 \end{align} где $ c_{-n} \neq 0 $.
\end{thm}
\begin{proof}[\normalfont\textsc{Доказательство}]
 Если ряд Лорана функции~$ f $ в проколотой окрестности точки~$ a $ имеет вид~\eqref{eq:Pole Laurent Series}, то совершенно очевидно выполняется
 \begin{align*}
  f(z) \sim \frac{c_{-n}}{(z-a)^{n}}
 \end{align*} при $ z \to a $, причём так как $ c_{-n} \neq 0 $, то полюс~$ a $ имеет порядок ровно $ n $.

 Наоборот, пусть $ a $ --- полюс порядка $ n \geqslant 1 $ функции~$ f $. Рассмотрим функцию
 \begin{align*}
  g(z) = (z-a)^{n}f(z).
 \end{align*} Функция~$ g $ аналитична в проколотой окрестности точки~$ a $, и $ g(z) = \OO(1) $ при $ z \to a $ (так как $ f(z) = \OO(z-a)^{-n} $). По лемме~\ref{lm:Removable Singularity} $ a $ --- устранимая особая точка функции~$ g $, поэтому в окрестности точки~$ a $ функция~$ g $ (с до-определённым значением в точке~$ a $) раскладывается в ряд Тейлора:
 \begin{align*}
  g(z) = \sum_{j=0}^{\infty} b_j(z-a)^{j}.
 \end{align*} Но тогда проколотой окрестности точки $ a $ имеем в точности разложение~\eqref{eq:Pole Laurent Series}, где $ c_k = b_{k+n} $. При этом $ c_{-n} = b_0 \neq 0 $, так как $ a $ --- полюс порядка ровно~$ n $ функции~$ f $.
\end{proof}

\subsection{Существенные особенности.}

\begin{df}
 Неустранимая изолированная особая точка~$ a\in\CC $ функции~$ f $, не являющаяся полюсом, называется \emph{существенной особой точкой}.
\end{df}

\begin{thm}
 Пусть $ a\in\CC $ --- изолированная особая точка функции~$ f $. Тогда следующие условия равносильны.
 \begin{enumerate}
  \item \label{enum1:Essential Singularity} $ a $ --- существенная особая точка.
  \item \label{enum2:Essential Singularity} Для любого $ n \geqslant 0 $ выполнено
   \begin{align*}
    \limsup_{z \to a} \left| (z-a)^{n} f(z) \right| = +\infty.
   \end{align*}
  \item \label{enum3:Essential Singularity} Ряд Лорана функции~$ f $ в проколотой окрестности точки~$ a $
   \begin{align*}
    f(z) = \sum_{k \in \Z} c_n(z-a)^{k}
   \end{align*} имеет бесконечное число ненулевых коэффициентов $ c_k \neq 0 $ при $ k < 0 $.
 \end{enumerate}
\end{thm}
\begin{proof}[\normalfont\textsc{Доказательство}]\
 \begin{itemize}
  \item Из условия~\ref{enum1:Essential Singularity} следует условие~\ref{enum2:Essential Singularity}. Действительно, если для некоторого $ n \geqslant 0 $ выполнено
   \begin{align*}
    \limsup_{z \to a} \left| (z-a)^{n} f(z) \right| < +\infty,
   \end{align*} то $ f(z) = \OO (z-a)^{-n} $ при $ z \to a $, и тогда $ f $ является полюсом (или устранимой особенностью).
  \item Из условия~\ref{enum2:Essential Singularity} следует условие~\ref{enum3:Essential Singularity}. Действительно, если бы у ряда Лорана было лишь конечно много ненулевых отрицательных коэффициентов, то мы бы получили
   \begin{align*}
    \lim_{z \to a} (z-a)^{n}f(z) = c_{-n},
   \end{align*} где $ c_{-n} \neq 0 $ --- наименьший ненулевой коэффициент.
  \item Из условия~\ref{enum3:Essential Singularity} следует условие~\ref{enum1:Essential Singularity}. Действительно, если $ a $ --- устранимая особенность, или полюс, то по следствию~\ref{corollary:Removable Singularity in Laurent Series} и теореме~\ref{theorem:Pole} ряд Лорана имеет лишь конечное число коэффициентов.
 \end{itemize}
\end{proof}

\begin{exmpl}
 Точка $ z=0 $ является существенной особой точкой функции~$ f(z)= e^{1 / z} $, так как для любого $ n \geqslant 0 $
 \begin{align*}
  \limsup_{z \to 0} \left| z \right|^{n} \cdot \left| e^{1 / z} \right| \geqslant \limsup_{x \to 0+} x^{n} e^{1 / x} = \limsup_{x \to +\infty} \frac{e^{x}}{x^{n}} = +\infty.
 \end{align*}
\end{exmpl}

Для общего развития скажем, что у функций без существенных особенностей есть отдельное название.

\begin{df}
 Пусть $ \Omega \subset \CC $ --- область, а $ E \subset \Omega $ --- конечное число точек. Функция~$ f $, аналитичная в области~$ \Omega \setminus E $ и не имеющая среди точек множества~$ E $ существенных особых точек, называется \emph{мероморфной} в области~$ \Omega $.
\end{df}

\subsection{Вычет аналитической функции.}

\begin{df}
 Пусть функция~$ f $ аналитична в проколотой окрестности точки~$ a \in \CC $. Тогда \emph{вычетом} функции~$ f $ в точке~$ a $ называется коэффициент $ c_{-1} $ её ряда Лорана в точке~$ a $
 \begin{align*}
  f(z) = \sum_{k \in \Z} c_k (z-a)^{k}.
 \end{align*} Обозначение: $ \residue_a f = c_{-1} $.
\end{df}

\begin{exmpl}
 Если $ a $ --- устранимая особая точка функции $ f $, то $ \residue_a f = 0 $.
\end{exmpl}

\begin{exmpl}
 \label{example:Residue of f / z-a}
 Пусть функция~$ f $ аналитична в области~$ \Omega \ni a $, и функция
 \begin{align*}
  h(z) = \frac{f(z)}{z - a}
 \end{align*} определена и аналитична в области~$ \Omega \setminus \left\{ a \right\} $. Тогда 
 \begin{align}
  \label{eq:Residue of f / z-a}
  \residue_a h = f(a).
 \end{align} Действительно, функция~$ f $ раскладывается в ряд Тейлора в окрестности точки~$ a $:
 \begin{align*}
  f(z) = \sum_{j=0}^{\infty} c_j (z-a)^{j},
 \end{align*} и тогда $ h(z) $ раскладывается в ряд Лорана в проколотой окрестности  $ a $:
 \begin{align*}
  h(z) = \sum_{k=-1}^{\infty} c_{k+1}(z-a)^{k},
 \end{align*} и коэффициент при $ (z-a)^{-1} $ равен $ c_{0} = f(a) $.
\end{exmpl}
\begin{exmpl}
 Пусть функция~$ f $ аналитична в области~$ \Omega \ni a $, и функция
 \begin{align*}
  h(z) = \frac{f(z)}{(z - a)^{k}},
 \end{align*} где $ k \geqslant 1 $, определена и аналитична в области~$ \Omega \setminus \left\{ a \right\} $. Тогда 
 \begin{align*}
  \residue_a h = \frac{f^{(k-1)}(a)}{(k-1)!}
 \end{align*} Действительно,
 \begin{align*}
  h(z) = \frac{1}{(z-a)^{k}}\sum_{j=0}^{\infty} \frac{f^{(j)}(a)}{j!}(z-a)^{j} = \sum_{n \geqslant -k}^{\infty} c_n(z-a)^{n},
 \end{align*} где
 \begin{align*}
  c_n = \frac{f^{(n+k)}(a)}{(n+k)!}.
 \end{align*} Тогда
 \begin{align*}
  \residue_a h = c_{-1} = \frac{f^{(k-1)}(a)}{(k-1)!}.
 \end{align*}
\end{exmpl}
\begin{exmpl}
 \label{example:logarithmic_derivative1}
 Пусть функция~$ f $ аналитична в области~$ \Omega $, и $ a \in \Omega $ --- корень кратности $ n \geqslant 1 $ функции $ f $. Тогда
 \begin{align*}
  \residue_a \frac{f'(z)}{f(z)} = n.
 \end{align*}
\end{exmpl}
Функция $ f'(z)/f(z) $  называется \textit{логарифмической производной} функции $ f $. Так как множество нулей аналитической функции дискретно (следствие~\ref{corollary:Zeroes of analytic fun is a Discrete Set}), то особая точка $ a $ является изолированной.
\begin{proof}[\normalfont\textsc{Доказательство}]
 По лемме~\ref{lemma:zero_multiplicity} о кратности нуля
 \begin{align*}
  f(z)=(z-a)^{N}g(z),
 \end{align*} где функция~$ g $ аналитична в $ \Omega $, и $ g(a) \neq 0 $. Тогда при $ z \to a $
 \begin{align*}
  \frac{f'(z)}{f(z)} &= \frac{n(z-a)^{n-1}g(z) + (z-a)^{n}g'(z)}{(z-a)^{n}g(z)} = \\
  &= \frac{n (z-a)^{n - 1} g(z)}{(z - a)^{n} g(z)} + \frac{(z-a)^{n} g'(z)}{(z-a)^{n} g(z)}= \frac{n}{z-a} + \OO(1).
 \end{align*} Значит, $ a $  --- полюс первого порядка функции $ f'(z)/f(z) $, и
 \begin{align*}
  \residue_a \frac{f'(z)}{f(z)} = n.
 \end{align*}
\end{proof}

\begin{exmpl}
 \label{example:logarithmic_derivative2}
 Пусть $ a\in\CC $ --- полюс порядка~$ n \geqslant 1 $ аналитической функции~$ f $. Тогда
 \begin{align*}
  \residue_a \frac{f'(z)}{f(z)} = -n.
 \end{align*}
\end{exmpl}
\begin{proof}[\normalfont\textsc{Доказательство}]
 По теореме~\ref{theorem:Pole} в окрестности точки~$ a $ выполнено
 \begin{align*}
  f(z) = \frac{g(z)}{(z-a)^{n}},
 \end{align*} где функция~$ g $ аналитична в окрестности~$ a $, и $ g(a) \neq 0 $. Поэтому
 \begin{align*}
  \frac{f'(z)}{f(z)} = \frac{ \frac{g'(z) \cdot (z-a)^{n} - n(z-a)^{n-1}g(z)}{(z-a)^{2n}} }{\frac{g(z)}{(z-a)^{n}}} = -\frac{n}{z-a} + \OO(1)
 \end{align*} при $ z \to a $. Значит, $ a $ --- полюс первого порядка функции $ f'(z) / f(z) $, и
 \begin{align*}
  \residue_a \frac{f'(z)}{f(z)} = -n.
 \end{align*}
\end{proof}

\newpage
\section{Теорема Коши о вычетах.}

В этом параграфе мы изучим \emph{теорему Коши о вычетах} --- мощном инструменте для вычисления интеграла от аналитической функции по замкнутому контуру. Этот результат применяется очень часто, в том числе и при вычислении вещественных интегралов.

В условии теоремы нам придётся наложить некоторые ограничения на область определения аналитической функции; введём для этих ограничений специальное название.

\begin{df}
 Замкнутый путь $ \gamma\colon\,[a,b] \to \CC $ называется \textit{простым}, если у него нет самопересечений: $\gamma |_{[a,b)}$ --- инъективное отображение.
\end{df}

\begin{df}
 Область~$ \Omega \subset \CC $ называется \emph{стандартной}, если она ограничена, и её граница $ \partial\Omega = \overline \Omega \setminus \Omega $ представляет собой объединение конечного числа непересекающихся простых кусочно-гладких замкнутых путей.
\end{df}

Пример стандартной области изображён на рисунке~\ref{fig:standart-region}.

\begin{figure}[ht]
 \centering
 \incfig[0.7]{standart-region}
 \caption{Стандартная область.}
 \label{fig:standart-region}
\end{figure}

\begin{conventn}
 \label{convention:Standard Region Canonical Parameterization}
 Пусть $ \Omega \subset\CC$ --- стандартная область. Обычно, например, когда мы будем писать выражение
 \begin{align*}
  \varointctrclockwise_{\partial \Omega} f \, dz,
 \end{align*} мы будем подразумевать, что граница $ \partial \Omega $ ориентирована так, что при обходе вдоль неё область~$ \Omega $ остаётся слева.

 Словосочетанию <<область остаётся слева>> можно придать математическую строгость: для любого пути $ \gamma_k $, являющегося составной частью границы~$ \partial\Omega $, и для любого~$ t $ при достаточно малом~$ \eps > 0 $ выполнено
 \begin{align*}
  \gamma_k(t) + i\eps\gamma_k'(t) \in \Omega.
 \end{align*} Для понимания см. рисунок~\ref{fig:region-traversal-direction}. Существование такой параметризации для произвольной стандартной области --- вопрос сложный, но и его можно строго доказать.
\end{conventn}

\begin{figure}[ht]
 \centering
 \incfig[0.5]{region-traversal-direction}
 \caption{Направление обхода контура.}
 \label{fig:region-traversal-direction}
\end{figure}

\begin{thm}[%
 Коши о вычетах]
 \label{theorem:cauchy_residue}
 Пусть $ \Omega \subset \CC$ --- стандартная область, и $ E \subset \Omega $ --- конечное число точек в ней. Пусть функция~$ f $ аналитична в области~$ \Omega \setminus E $, и непрерывно продолжается на границу области: $ f\in C(\overline\Omega \setminus E) $. Тогда
 \begin{align}
  \label{eq:Cauchy Residue Theorem}
  \varointctrclockwise_{\partial \Omega} f\,dz = 2\pi i \sum_{a \in E} \residue_{a} f.
 \end{align}
\end{thm}

Оказывается, теорема~\ref{theorem:cauchy_residue} очень просто выводится из теоремы Стокса. Однако теорема Стокса не была доказана нами в полной мере, поэтому этот вывод не будет являться полным доказательством.

\begin{proof}[\normalfont\textsc{Вывод теоремы~\ref{theorem:cauchy_residue} из формулы Стокса}]
 Рассмотрим область
 \begin{align*}
  \tilde \Omega = \Omega \setminus \bigcup_{a \in E} \overline B(a,\eps),
 \end{align*} где число $ \eps > 0 $ достаточно маленькое, чтобы все замкнутые диски $ \overline B(a,\eps) $, $ a \in E $ лежали в~$ \Omega $ и не пересекались (рисунок~\ref{fig:theorem_cauchy_omega_tilde}). Область~$ \tilde \Omega $ также является стандартной областью.

 \begin{figure}[ht]
  \centering
  \incfig[0.5]{theorem_cauchy_omega_tilde}
  \caption{Область $\tilde \Omega$.}
  \label{fig:theorem_cauchy_omega_tilde}
 \end{figure}

 Дифференциальная форма~$ f\,dz $ непрерывна на $ \overline{\tilde \Omega} $, а также $ C^{1} $-гладкая и замкнутая в области~$ \tilde \Omega $ по теореме~\ref{theorem:cauchy-gursa-morer} Коши-Гурса-Морера: по теореме~\ref{theorem:smooth_form_closed} имеем $ d(f\,dz) = 0 $ в области~$ \tilde\Omega $. Тогда по формуле Стокса
 \begin{align}
  \label{eq:Stokes Formula:Cauchy Residue Thm}
  \varointctrclockwise_{\partial \tilde \Omega}   f\,dz = \int_{\tilde \Omega} d(f\,dz)  = \int_{\tilde\Omega} 0\,dx\land dy = 0.
 \end{align} Но по аддитивности интеграла
 \begin{align}
  \label{eq:Additivity:Cauchy Residue Thm}
  \varointctrclockwise_{\partial \tilde \Omega} f\,dz = \varointctrclockwise_{\partial \Omega} f\,dz - \sum_{a \in E} \varointctrclockwise_{C(a,\eps)}f\,dz,
 \end{align} где $ C(a,\eps) $ --- окружность с центром в $ a \in E $ и радиусом~$ \eps $, проходимая против часовой стрелке. Подставляя \eqref{eq:Stokes Formula:Cauchy Residue Thm} в \eqref{eq:Additivity:Cauchy Residue Thm}, получаем
 \begin{align}
  \label{eq:Almost:Cauchy Residue Thm}
  \varointctrclockwise_{\partial\Omega} f\,dz = \sum_{a \in E}\varointctrclockwise_{C(a,\eps)} f\,dz.
 \end{align} По теореме~\ref{theorem:Laurent series of analytic function} Лорана в проколотой окрестности каждой точки~$ a \in E $ функция~$ f $ раскладывается в ряд Лорана
 \begin{align}
  \label{eq:Laurent Series:Cauchy Residue Thm}
  f(z)=\sum_{k\in\Z}c_k(z-a)^{k},
 \end{align} причём ряд~\eqref{eq:Laurent Series:Cauchy Residue Thm} сходится равномерно на компактах в этой проколотой окрестности. Тогда
 \begin{align*}
  \varointctrclockwise_{C(a,\eps)}f(z)\,dz &= \varointctrclockwise_{C(a,\eps)} \left[ \sum_{k \in \Z} c_k(z-a)^{k} \right]dz = \sum_{k\in\Z} c_k \cdot \left[ \varointctrclockwise_{C(a,\eps)} (z-a)^{k}\,dz \right] = \\
  &= \sum_{k\in\Z}c_k \cdot \left[ \varointctrclockwise_{C(0,\eps)} \zeta^{k}\,d\zeta \right] = 2\pi i \cdot c_{-1} = 2\pi i \cdot \residue_a f,
 \end{align*} где переставлять местами знак суммы и интеграла можно по равномерной сходимости ряда~\eqref{eq:Laurent Series:Cauchy Residue Thm}. Подставляя полученное в \eqref{eq:Almost:Cauchy Residue Thm}, получаем в точности \eqref{eq:Cauchy Residue Theorem}, что и требовалось доказать.
\end{proof}

В качестве примера применения теоремы~\ref{theorem:cauchy_residue} Коши о вычетах, обобщим уже известную нам \emph{интегральную формулу Коши} (следствие~\ref{corollary:cauchy_formula}).

\begin{crly}[интегральная формула Коши]
 \label{corollary:Cauchy Integral Formula}
 Пусть функция~$ f $ аналитична в стандартной области~$ \Omega \subset \CC $, и непрерывно продолжается на её границу: $ f \in C(\overline \Omega) $. Тогда для любой точки $ z_0 \in \Omega $ верна \emph{интегральная формула Коши:}
 \begin{align}
  \label{eq:Cauchy Integral Formula}
  f(z_0) = \frac{1}{2\pi i} \varointctrclockwise_{\partial\Omega} \frac{f(z)\,dz}{z-z_0}.
 \end{align}
\end{crly}
\begin{proof}[\normalfont\textsc{Доказательство}]
 Так как аналитическая функция~$ f(z) / (z-z_0) $ в области~$ \Omega $ имеет лишь одну особенность в точке $ z = z_0 $, то по теореме~\ref{theorem:cauchy_residue} Коши о вычетах:
 \begin{align*}
  \varointctrclockwise_{\partial \Omega} \frac{f(z)\,dz}{z-z_0} = 2\pi i \cdot \residue_{z_0} \frac{f(z)}{z-z_0} = 2\pi i  \cdot f(z_0),
 \end{align*} где вычет был вычислен по формуле~\eqref{eq:Residue of f / z-a} (пример~\ref{example:Residue of f / z-a}).
\end{proof}

\begin{thm}
 \label{theorem:Derivative Convergence Example}
 Пусть функции $ f, g_1, g_2, \ldots, g_n, \ldots $ аналитичны в стандартной области $ \Omega \subset \CC $ и непрерывны на $ \overline \Omega $. Пусть также выполнено
 \begin{align}
  \label{eq:Condition:Derivative Convergence Example}
  \lim_{n \to \infty} \max_{z \in \partial\Omega} \left| f(z)-g_n(z) \right| = 0.
 \end{align} Тогда производные~$ g_n' $ сходятся к $ f' $ равномерно на компактах в $ \Omega $.
\end{thm}

\begin{lm}
 \label{lemma:Derivative Cauchy Integral Formula}
 Пусть функция~$ f $ аналитична в стандартной области~$ \Omega \subset \CC $ и непрерывна на $ \overline\Omega $. Тогда для любой точки~$ z_0 \in \Omega $ выполнено
 \begin{align}
  \label{eq:Derivative Cauchy Integral Formula}
  f'(z_0) = \frac{1}{2\pi i} \varointctrclockwise_{\partial\Omega} \frac{f(z)\,dz}{(z-z_0)^{2}}.
 \end{align}
\end{lm}
\begin{proof}[\normalfont\textsc{Доказательство}]
 Пользуясь интегральной формулой Коши~\eqref{eq:Cauchy Integral Formula} (следствие~\ref{corollary:Cauchy Integral Formula}), раскроем производную функции~$ f $ в точке $ z_0 $:
 \begin{align*}
  f'(z_0) &= \lim_{h\to 0} \frac{f(z_0 + h) - f(z_0)}{h} = \lim_{h \to 0}  \frac{1}{2\pi i}\varointctrclockwise_{\partial\Omega} \frac{f(z)}{h} \left[ \frac{1}{z-(z_0+h)} - \frac{1}{z-z_0} \right]dz = \\
  &= \lim_{h \to 0} \frac{1}{2\pi i} \varointctrclockwise_{\partial\Omega} \frac{f(z)\,dz}{(z - z_0 - h)(z - z_0)} = \frac{1}{2\pi i} \varointctrclockwise_{\partial\Omega} \frac{f(z)\,dz}{(z-z_0)^{2}}.
 \end{align*} Осталось лишь обосновать предельный переход под знаком интеграла по теореме Лебега о мажорируемой сходимости. В самом деле, при малых~$ h $:
 \begin{align*}
  \left| \frac{f(z)}{(z - z_0 - h)(z-z_0)} \right|  \leqslant \frac{2 \left| f(z) \right|}{\left| z-z_0 \right|^{2}} = \OO(1),
 \end{align*} так как точка~$ z_0 $ отделена от границы~$ \partial\Omega $ ($ \left| z-z_0 \right| \geqslant \delta$ для некоторой константы $ \delta > 0$ и для всех $ z\in\partial\Omega $), а непрерывная функция~$ \left| f(z) \right| $ достигает максимума на компакте~$ \partial\Omega $.
\end{proof}

\begin{proof}[\normalfont\textsc{Доказательство теоремы~\ref{theorem:Derivative Convergence Example}}]
 Пользуясь формулой~\eqref{eq:Derivative Cauchy Integral Formula} (лемма~\ref{lemma:Derivative Cauchy Integral Formula}), запишем
 \begin{align*}
  \left| f'(w) - g_n'(w) \right| &= \left| \frac{1}{2\pi i} \varointctrclockwise_{\partial\Omega} \frac{(f(z)-g_n(z))\,dz}{(z-w)^{2}}  \right| \leqslant \frac{1}{2\pi} \varointctrclockwise_{\partial\Omega} \frac{\max_{\partial\Omega} \left| f-g_n \right|}{\left| z-w \right|^{2}}\,dz.
 \end{align*} Тогда по~\eqref{eq:Condition:Derivative Convergence Example} 
 \begin{align*}
  \left| f'(w) - g_n'(w) \right| \leqslant \frac{1}{2\pi} \varointctrclockwise_{\partial\Omega} \frac{o(1)\,dz}{\left| z-w \right|^{2}} \leqslant \frac{l(\partial\Omega) \cdot o(1)}{2\pi} \cdot \max_{z \in \partial\Omega} \frac{1}{\left| z-w \right|^{2}} \rightrightarrows 0
 \end{align*} при $ n \to \infty $, причём сходимость равномерная по $ w $ на компактах в $ \Omega $ (так как любой компакт $ K \subset \Omega $ отделён от границы).
\end{proof}

\end{document}
