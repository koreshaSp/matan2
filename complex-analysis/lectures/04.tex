% 2023.03.16 lecture 4
\documentclass[../complex-analysis.tex]{subfiles}
\begin{document}

Уже в самом доказательстве теоремы~\ref{theorem:cauchy-gursa-morer} нам пришлось выяснить некоторые нетривиальные факты про аналитические функции. Оформим их как следствия.

\begin{crly}[формула Коши]
 \label{corollary:cauchy_formula}
 Пусть функция~$ f  $ аналитична в области~$ \Omega \subset \CC $, и точка~$ z_0 $ лежит в $ \Omega $ вместе с замкнутым диском с центром в $ z_0 $ и радиусом $ \rho > 0 $. Пусть точка~$ b \in B(z_0,\rho) $. Тогда верна \emph{формула Коши:}
 \begin{align}
  \label{eq:Cauchy Formula}
  f(b) = \frac{1}{2\pi i} \int_{C_\rho} \frac{f(z)\,dz}{z - b},
 \end{align} где $ C_\rho $ --- окружность с центром в $ z_0 $ и радиусом $ \rho $, проходимая против часовой стрелки.
\end{crly}
\begin{crly}
 \label{corollary:coefficients_of_analytic_function}
 Пусть функция~$ f $ аналитична в области~$ \Omega \subset\CC$. Тогда ряд Тейлора функции~$ f $ в точке~$ z_0 \in \Omega $
 \begin{align*}
  f(z) = \sum_{n=0}^{\infty}c_n(z-z_0)^{n},
 \end{align*} сходящийся на открытом диске $ B(z_0,r) \subset \Omega $, имеет коэффициенты
 \begin{align}
  \label{eq:coefficient_of_analytiv_function_taylor_series}
  c_n = \frac{1}{2\pi i} \int_{C_\rho} \frac{f(z)\,dz}{(z-z_0)^{n+1}},
 \end{align} где $ C_\rho $ --- окружность с центром в точке~$ z_0 $ и радиусом~$ \rho \in (0, r) $, проходимая против часовой стрелки. В частности, интеграл~\eqref{eq:coefficient_of_analytiv_function_taylor_series} не зависит от выбора $ \rho $.
\end{crly}

Следствия \ref{corollary:cauchy_formula} и \ref{corollary:coefficients_of_analytic_function} были установлены в доказательстве теоремы~\ref{theorem:cauchy-gursa-morer} при выводе условия~\ref{enum3:theorem:cauchy_gurs_morer} из условия~\ref{enum1:theorem:cauchy_gurs_morer}. 

\begin{crly}
 Если функция $ f \colon\,\Omega\to\CC $ аналитична в области~$ \Omega \subset \CC$, то и её производная $ f' \colon\,\Omega\to\CC $ существует, и также аналитична в области~$ \Omega $.

 Следовательно, аналитическая функция бесконечно гладкая, и любая её $ k $-я производная аналитична.
\end{crly}
\begin{proof}[\normalfont\textsc{Доказательство}]
 Пусть $ f $ аналитична в области~$ \Omega $. Тогда в любой окрестности~$ B(z_0, r) \subset \Omega $ ряд Тейлора
 \begin{align*}
  f(z) = \sum_{n=0}^{\infty} c_n (z-z_0)^{n}
 \end{align*} по следствию~\ref{corollary:complex_differential_of_power_series} комплексно дифференцируем, и его производная равна
 \begin{align*}
  f'(z) = \sum_{n=1}^{\infty} nc_n (z-z_0)^{n-1},
 \end{align*} причём ряд сходится в том же диске $ B(z_0,r) $. Но тогда функция $ f' \colon\,\Omega\to\CC $ по условию~\ref{enum3:theorem:cauchy_gurs_morer} аналитична в $ \Omega $, что и требовалось доказать.
\end{proof}

\begin{crly}
 \label{corollary:taylor_series}
 Пусть функция~$ f $ аналитична в области~$ \Omega \subset \CC $, и $ B(z_0, r) \subset \Omega $. Тогда
 \begin{align*}
  f(z) = \sum_{n=0}^{\infty} \frac{f^{(n)}(z)}{n!}
 \end{align*} для всех $ z \in B(z_0,r) $.
\end{crly}

Следствие~\ref{corollary:taylor_series} также было установлено в доказательстве теоремы~\ref{theorem:cauchy-gursa-morer} при выводе условия~\ref{enum3:theorem:cauchy_gurs_morer} из условия~\ref{enum1:theorem:cauchy_gurs_morer}. Оно показывает, что ряд Тейлора в старом смысле совпадает с рядом Тейлора аналитической функции.

Наконец, многие <<разумные>> операции с аналитическими функциями на выходе выдают также аналитическую функцию.
\begin{crly}
 Сумма, разность, произведение, частное (при том условии, что знаменатель не обнуляется) аналитических в области~$ \Omega \subset \CC $ функций аналитичны в $ \Omega $. Если функции $ f \colon\,\Omega_1 \to \Omega_2 $ и $ g \colon\,\Omega_2 \to \CC $ аналитичны в областях~$ \Omega_1 \subset \CC$ и $ \Omega_2 \subset \CC $ соответственно, то их композиция $ g \circ f \colon\,\Omega_1 \to \CC $ аналитична в $ \Omega_1 $.
\end{crly}
\begin{proof}[\normalfont\textsc{Доказательство}]
 Тривиально выводится с помощью комплексной дифференцируемости --- условия~\ref{enum1:theorem:cauchy_gurs_morer} из теоремы~\ref{theorem:cauchy-gursa-morer}.
\end{proof}

\subsection{Теорема Лиувилля.}

Рассмотрим один из <<простых>> случаев аналитической функции, когда она аналитична всюду.

\begin{df}
 \emph{Целой функцией} называется функция~$ f\colon\,\CC\to\CC $, аналитическая в области~$ \CC $.
\end{df}

Теорема Лиувилля является основным результатом о целых функциях.

\begin{thm}[Лиувилля]
 \label{theorem:liuvill}
 Пусть целая функция~$ f\colon\,\CC\to\CC $ ограничена: существует такое $ M > 0 $, что $ \left| f(z) \right| \leqslant M $ всюду в $ \CC $. Тогда функция $ f $ постоянна на $ \CC $: $ f(z) = c_0 $.
\end{thm}
\begin{proof}[\normalfont\textsc{Доказательство}]
 С одной стороны, ряд Тейлора функции~$ f $ в точке~$ 0 $
 \begin{align*}
  f(z) = \sum_{n=0}^{\infty} c_n z^{n}
 \end{align*} сходится всюду в $ \CC $ по теореме~\ref{theorem:cauchy-gursa-morer} Коши-Гурса-Морера. С другой стороны, по формуле~\eqref{eq:coefficient_of_analytiv_function_taylor_series} (следствие~\ref{corollary:coefficients_of_analytic_function}):
 \begin{align*}
  c_n = \frac{1}{2\pi i} \int_{C_\rho} \frac{f(z)\,dz}{z^{n+1}},
 \end{align*} причём радиус~$ \rho > 0 $ мы вольны взять любой. Тогда при стремлении $ \rho \to \infty $ имеем оценку
 \begin{align*}
  \left| c_n \right| = \frac{1}{2\pi} \left| \int_{C_\rho} \frac{f(z)\,dz}{z^{n+1}}  \right|\leqslant \frac{2\pi\rho}{2\pi} \cdot \max_{\left| z \right|=\rho} \frac{\left| f(z) \right|}{\rho^{n+1}} \leqslant \frac{M}{\rho^{n}}.
 \end{align*} Так как при $ n \geqslant 1 $ правая часть стремится к нулю при $ \rho \to \infty $, то $ c_n = 0 $ при $ n \geqslant 1 $, то есть $ f(z) = c_0 $ всюду в $ \CC $.
\end{proof}

В качестве примера применения теоремы Лиувилля (и комплексного анализа в целом), докажем основную теорему алгебры.

\begin{thm}[основная теорема алгебры]
 У всякого непостоянного многочлена есть корень в $ \CC $.
\end{thm}
\begin{proof}[\normalfont\textsc{Доказательство}]
 Пусть у непостоянного многочлена
 \begin{align*}
  p(z) = c_0 + c_1 z + \ldots + c_n z^{n}, \qquad n \geqslant 1,\; c_n \neq 0
 \end{align*}
 нет корней. Рассмотрим функцию
 \begin{align*}
  f(z) = \frac{1}{p(z)}.
 \end{align*} Так как функция $ p(z) $ целая и не обнуляется в $ \CC $, то функция $ f(z) $ также целая.

 Заметим, что функция~$ f $ ограничена. Действительно, с одной стороны непрерывная функция~$ f $ ограничена на любом компакте~$ \overline B(0, R) $, $ R > 0 $, а с другой стороны
 \begin{align*}
  \lim_{z \to \infty} \left| f(z) \right| = \lim_{z \to \infty} \frac{1}{\left| p(z) \right|} = \lim_{z \to \infty} \frac{1}{\left| c_n \right| \cdot \left| z \right|^{n}} = 0.
 \end{align*} Тогда по теореме~\ref{theorem:liuvill} Лиувилля функция~$ f $ постоянна на $ \CC $ (и тогда $ p $ тоже), а это противоречие!
\end{proof}

\subsection{Теорема единственности.}

Изучим, как устроены нули аналитических функций.

\begin{lm}[о кратности нуля]
 \label{lemma:zero_multiplicity}
 Пусть функция $ f $  аналитична в области~$ \Omega \subset \CC $ и не равна тождественно нулю на $ \Omega $, а точка~$ z_0 \in \Omega $ --- нуль этой функции: $ f(z_0) = 0 $. Тогда существует единственное натуральное число $ n_0 \geqslant 1 $, называемое \emph{кратностью нуля} функции~$ f $ в точке~$ z_0 $, такое, что
 \begin{align*}
  f(z) = (z - z_0)^{n_0} \cdot g(z),
 \end{align*} где функция $ g $ аналитична в $ \Omega $, и $ g(z_0) \neq 0 $.
\end{lm}
Для удобства можно считать, что лемма~\ref{lemma:zero_multiplicity} верна и при $ f(z_0) \neq 0 $: тогда кратность нуля~$ n_0 $ попросту равна нулю. Также можно считать, что в случае тождественного нуля (когда $ f \equiv 0 $ в $ \Omega $) каждая точка $ z_0 \in \Omega $ является нулём $ f $ бесконечной кратности.
\begin{proof}[\normalfont\textsc{Доказательство леммы~\ref{lemma:zero_multiplicity} в предположении диска с не нулём}]\item
 Предположим сначала, что существует такой проколотый диск $ D \setminus \left\{ z_0 \right\} $ с центром в $ z_0 $, что $ f(z) \neq 0 $ для некоторого $ z\in D \setminus \left\{ z_0 \right\} $ (в дальнейшем мы докажем, что это всё равно следует из условия $ f \not\equiv 0 $ на $ \Omega $). Ряд Тейлора
 \begin{align}
  \label{eq:taylor_series:zero_multiplicity}
  f(z) = \sum_{n=0}^{\infty}c_n(z-z_0)^{n}
 \end{align} сходится на диске~$ D $, причём $ f(z_0) = c_0 = 0 $. Так как $ f(z) \neq 0 $ для некоторого $ z \in D $, то ряд~\eqref{eq:taylor_series:zero_multiplicity} обязан иметь хотя бы один ненулевой коэффициент $ c_n \neq 0 $. Пусть  $ n_0 \geqslant 1 $  --- номер наименьшего такого коэффициента $ c_{n_0} \neq 0 $.  Тогда в $ D $ выполнено
 \begin{align*}
  f(z) = (z-z_0)^{n_0} \sum_{n=0}^{\infty} c_{n + n_0}(z-z_0)^{n},
 \end{align*} то есть функция $ g(z) = f(z) / (z-z_0)^{n_0} $ аналитична в $ D $. Но так как она также аналитична в $ \Omega \setminus \left\{ z_0 \right\} $, то она аналитична в $ \CC $. При этом, $ g(z_0) = c_{n_0} \neq 0 $.
\end{proof}

От сделанного предположения нам поможет избавиться следующая, крайне существенная сама по себе, теорема о единственности аналитической функции.

\begin{thm}[теорема единственности]
 \label{theorem:uniqueness}
 Пусть функции $ f_1,f_2 $, аналитические в области~$ \Omega $, совпадают на некотором множестве $ E \subset \Omega $, которое имеет предельную точку в $ \Omega $. Тогда $ f_1(z) = f_2(z) $ всюду в $ \Omega $.
\end{thm}
\begin{proof}[\normalfont\textsc{Доказательство}]
 По существу нам нужно доказать, что если множество~$ E $ нулей аналитической в $ \Omega $ функции~$ g = f_1 - f_2 $ имеет предельную точку, то $ f \equiv 0 $ на $ \Omega $.

 Пусть $ z_\ast \in \Omega $ --- предельная точка множества~$ E $, то есть любая проколотая окрестность $ \dot U $ точки $ z_\ast $ пересекается с $ E $. Докажем сначала, что $ g $ тождественно равна нулю на любом диске $ B(z_\ast,r) $. Докажем от противного: предположим, что функция $ g $ не тождественный нуль на диске $ B(z_\ast, r) $. Тогда по лемме~\ref{lemma:zero_multiplicity} о кратности нуля (с учётом предположения $ g \not\equiv 0 $ на $ B(z_0,r) $) для некоторого $ n_0 \geqslant 0 $ функция
 \begin{align*}
  h(z) = \frac{g(z)}{(z-z_\ast)^{n_0}}
 \end{align*} аналитична в $ \Omega $, и $ h(z_\ast) \neq 0 $. Но все нули $ g $ также являются нулями $ h $, поэтому существует последовательность нулей $ z_1, z_2, \ldots $ функции~$ h $ такая, что $ z_n \to z_\ast $. По непрерывности~$ h $ получаем $ h(z_n) = 0 $, а это противоречие! Мы доказали, что $ g \equiv 0 $ на некотором открытом диске $ B(z_\ast, r) $, где $ z_\ast $ --- предельная точка множества нулей функции~$ g $.

 Теперь рассмотрим множество
 \begin{align*}
  F = \mathop{\mathrm{Int}} \left\{ z \in \Omega \mid g(z) = 0 \right\},
 \end{align*} где внутренность берётся в метрическом пространстве~$ \Omega $. Докажем, что $ F = \Omega $.

 Предположим $ F \neq \Omega $. С одной стороны, множество~$ F $ по построению открыто в пространстве $ \Omega $, и не пусто, как мы показали ранее. С другой стороны, покажем, что множество~$ F $ замкнуто в пространстве~$ \Omega $. Действительно, пусть $ t_\ast \in \Omega $ --- предельная точка множества $ F $. Тогда по уже доказанному получаем, что $ g \equiv 0 $ в некоторой окрестности точки $ t_\ast $, то есть $ t_\ast \in F $. В таком случае, мы нашли подмножество $ F \subset \Omega $, не пустое, не равное всему пространству~$ \Omega $, одновременно открытое и замкнутое, а такого не бывает в связном пространстве $ \Omega $.
\end{proof}

Теперь можно закончить доказательство леммы~\ref{lemma:zero_multiplicity} о кратности нуля.

\begin{proof}[\normalfont\textsc{Полное доказательство леммы~\ref{lemma:zero_multiplicity}}]
 Действительно, если предположение выше не выполняется, то есть существует открытый диск, состоящий из нулей функции~$ f $, то по теореме единственности~\ref{theorem:uniqueness} $ f $ тождественна равна нулю на $ \Omega $, чего не может быть по условию.
\end{proof}

\begin{exmpl}
 Пусть $ f\colon\,\CC\to\CC $ --- целая функция, такая, что
 \begin{align*}
  f \left( \frac{1}{n} \right) = \sin \frac{1}{n}
 \end{align*} для всех натуральных $ n \geqslant 1 $. Тогда $ f(z) = \sin z $ всюду в $ \CC $.
\end{exmpl}
\begin{proof}[\normalfont\textsc{Доказательство}]
 Функции $ f(z) $ и $ \sin z $ совпадают на множестве
 \begin{align*}
  E = \left\{ \frac{1}{n} \Mid n \in \N \right\},
 \end{align*} имеющем предельную точку $ 0 \in \CC $. По теореме единственности~\ref{theorem:uniqueness} совпадение функций есть и всюду в $ \CC $.
\end{proof}

\begin{df}
 Подмножество~$ E \subset \Omega $ области~$ \Omega $ называется \emph{дискретным множеством}, если любая предельная точка $ E $ не лежит в $ \Omega $.
\end{df}
\begin{prop}
 \label{proposition:Discrete Set is Countable}
 Дискретное множество не более, чем счётно.
\end{prop}
\begin{proof}[\normalfont\textsc{Доказательство}]
 Действительно, каждую точку~$ z \in E $ можно отделить от множества~$ E $ некоторым открытым диском (иначе точка~$ z $ была бы предельной точкой~$ E $), а в каждом диске можно выбрать рациональное комплексное число.
\end{proof}

\begin{crly}
 \label{corollary:Zeroes of analytic fun is a Discrete Set}
 Множество нулей аналитической в области~$ \Omega $ функции~$ f $, не равной тождественно нулю в $ \Omega $, является дискретным.
\end{crly}
\begin{proof}[\normalfont\textsc{Доказательство}]
 Непосредственно следует из теоремы единственности~\ref{theorem:uniqueness}.
\end{proof}

\subsection{Условия Коши-Римана.}

Есть ещё одно эквивалентное определение аналитической функции, рассматривающая функцию не как комплекснозначную функцию одного комплексного переменного, а как пару вещественнозначных функций двух вещественных переменных.

\begin{thm}[условия Коши-Римана]
 \label{theorem:cauchy_riman}
 Пусть $ \Omega \subset \CC $ --- область, $ f\colon\,\Omega \subset \CC $ --- функция. Обозначим через $ u, v \colon\, \Omega \to \R $ функции, такие, что
 \begin{align*}
  f(x + iy) = u(x, y) + iv(x,y)
 \end{align*} для всех $ x+iy \in \CC $. Тогда следующие условия равносильны.
 \begin{enumerate}
  \item \label{enum1:cauchy_riman} Функция~$ f $ аналитична в области~$ \Omega $.
  \item \label{enum2:cauchy_riman} Функции~$ u,v $ $ C^{1} $-гладкие, а также всюду в $ \Omega $ выполнены следующие равенства, называемые \emph{условиями Коши-Римана:}
   \begin{align*}
    \begin{cases}
     u'_x = v'_y, \\
     u'_y = -v'_x.
    \end{cases} 
   \end{align*}
 \end{enumerate}
\end{thm}
\begin{proof}[\normalfont\textsc{Доказательство}]
 По теореме~\ref{theorem:cauchy-gursa-morer} Коши-Гурса-Морера функция~$ f $ аналитична в $ \Omega $ тогда и только тогда, когда дифференциальная форма~$ f \, dz $ замкнута в $ \Omega $. А по теореме~\ref{theorem:smooth_form_closed} о замкнутости $ C^{1} $-гладкой формы, форма замкнута тогда и только тогда, когда $ d(f\,dz) = 0 $. Раскроем левую часть:
 \begin{align*}
  d(f\,dz) &= d((u+iv)\,dz) = (u_x'\,dx + u_y'\,dy + iv_x'\,dx + iv_y'\,dy)\land dz = \\
  &= ((u_x' + iv_x')\,dx + (u_y' + iv_y')\,dy) \land (dx + i\,dy) = \\
  &= (-v_x' + iu_x')\,dx \land dy + (u_y' + iv_y')\,dy \land dx = \\
  &= (-u_y' - v_x' + i(u_x' - v_y'))\,dx\land dy.
 \end{align*} Тогда $ d(f\,dz) = 0 $, если и только если
 \begin{align*}
  \begin{cases}
   -u_y' - v_x' = 0\\
   u_x' - v_y' = 0
  \end{cases} \iff
  \begin{cases}
   u_x' = v_y'\\
   u_y' = -v_x'.
  \end{cases} 
 \end{align*} Отметим, что гладкость формы при импликации из \ref{enum2:cauchy_riman} в \ref{enum1:cauchy_riman} есть по условию, а при импликации из \ref{enum1:cauchy_riman} в \ref{enum2:cauchy_riman} есть по гладкости функции~$ f $.
\end{proof}

\subsection{Неравенство Лагранжа.}

\begin{lm}
 \label{lemma:derivative_of_analytic_f_is_exact_form}
 Пусть функция~$ f $ аналитична в области~$ \Omega \subset \CC $. Тогда дифференциальная форма~$ f'(z)\,dz $ точна в $ \Omega $, и, более того, функция~$ f $ (рассматриваемая как функция двух вещественных переменных) является её первообразной: $ df = f'(z)\,dz $.
\end{lm}
\begin{proof}[\normalfont\textsc{Доказательство}]
 В самом деле,
 \begin{align*}
  d(f(z)) &= d(f(x + iy)) = f'(x+iy) \cdot \frac{\partial}{\partial x}(x+iy)\,dx + f'(x+iy) \cdot \frac{\partial}{\partial y}(x+iy)\,dy = \\
  &= f'(z) \, dx + f'(z) \cdot i\,dy = f'(z)\,dz,
 \end{align*} где $ l(\gamma) $ --- длина пути $ \gamma $.
\end{proof}

\begin{thm}[неравенство Лагранжа]
 \label{theorem:Lagrange_inequality}
 Пусть функция~$ f $ аналитична в области~$ \Omega \subset \CC$, а $ \gamma $ --- кусочно-гладкий путь в $ \Omega $, соединяющий точки $ z_1 $ и $ z_2 $. Тогда
 \begin{align}
  \label{eq:lagrange_inequality}
  \left| f(z_1) - f(z_2) \right| \leqslant l(\gamma) \cdot \max_{z \in \gamma} \left| f'(z) \right|.
 \end{align} 
\end{thm}
\begin{proof}[\normalfont\textsc{Доказательство}]
 По лемме~\ref{lemma:derivative_of_analytic_f_is_exact_form} функция~$ f $ является первообразной точной дифференциальной формы~$ f'(z)\,dz $. Тогда по формуле~\eqref{eq:integral_of_exact_form} (замечание~\ref{remark:integral_of_exact_form}):
 \begin{align*}
  f(z_2) - f(z_1) = \int_{\gamma} f'(z)\,dz. 
 \end{align*} Тогда по оценке модуля интеграла:
 \begin{align*}
  \left| f(z_1) - f(z_2) \right| = \left| \int_{\gamma} f'(z)\,dz  \right| \leqslant l(\gamma) \cdot \max_{z \in \gamma} \left| f'(z) \right|.
 \end{align*}
\end{proof}

Из неравенства~\eqref{eq:lagrange_inequality} Лагранжа непосредственно выводится следующее.

\begin{remrk}
 Если $ f $ аналитична в области~$ \Omega $, и отрезок~$ [z_1,z_2] \subset \Omega$, то
 \begin{align*}
  \left| f(z_1) - f(z_2) \right| \leqslant \left| z_1-z_2 \right| \cdot \max_{z \in [z_1, z_2]} \left| f'(z) \right|.
 \end{align*}
\end{remrk}

\begin{remrk}[аналитические функции с нулевой производной]
 \label{remark:zero_derivative_analytical}
 Если функция~$ f $ аналитична в области~$ \Omega $, и её производная~$ f'(z) $ тождественно равна нулю в $ \Omega $, то $ f(z) $ постоянна в $ \Omega $.
\end{remrk}
\begin{proof}[\normalfont\textsc{Доказательство}]
 Действительно, так как в линейно связном множестве~$ \Omega $ любые две точки $ z_1 $ и $ z_2 $ можно соединить путём $ \gamma $, то по неравенству~\eqref{eq:lagrange_inequality} Лагранжа
 \begin{align*}
  \left| f(z_1)-f(z_2) \right| \leqslant l(\gamma) \cdot 0 = 0.
 \end{align*}
\end{proof}

\end{document}
