В этом семестре будет \textit{комплексный анализ}. Эта наука в первую очередь изучает \textit{аналитические функции} --- то есть те функции, которые совпадают со своим рядом Тейлора в окрестности любой точки (где определена функция).

Рассмотрим две вещественные бесконечно гладкие функции $\frac{1}{1+x^{2}}$ и $e^{-x^{2}}$

\begin{figure}[ht]
 \centering
 \incfig{similar-plots}
 \caption{}
 \label{fig:similar-plots}
\end{figure}

Запишем ряды Тейлора этих функций в точке $0$:
\begin{align}
 \label{equation:introduction:taylor_series_of_1_over_1_plus_x_squared}
 \frac{1}{1+x^{2}} &= \sum_{k=0}^{\infty} (-x^{2})^{k}, \\
 \label{equation:introduction:taylor_series_of_e_to_minus_x_squared}
 e^{-x^{2}} &= \sum_{k=0}^{\infty}  \frac{(-x^{2})^{k}}{k!}.
\end{align} Ряд \eqref{equation:introduction:taylor_series_of_1_over_1_plus_x_squared} сходится при $\left| x \right| < 1$, а ряд \eqref{equation:introduction:taylor_series_of_e_to_minus_x_squared} --- при любых $x \in \R$.

Продолжим первую функцию в область $\CC$ комплексных чисел: получим функцию $\frac{1}{1 + z^{2}}$. Однако, есть две особые точки: $i$ и $-i$: в этих двух точках функция не определена. Тем не менее, ряд \begin{align*}
 \frac{1}{1+z^{2}} = \sum_{k=0}^{\infty} (-z^{2})^{k}
\end{align*}  сходится на комплексном единичном круге $\left| z \right| < 1$. {\color{red} ещё две картинки}

{\color{red} Не понятно, что показывает иллюстрация --- выпилить её или довести до ума.}

\section{Точные и замкнутые дифференциальные формы.}

\begin{conventn*}
 Начиная с этого параграфа, плоскость $\R^{2}$ мы часто будем отождествлять с областью комплексных чисел $\CC$ и будем писать $\R^{2} \cong \CC$. Так, вектор $ \begin{psmallmatrix}
  h_1 \\ h_2
 \end{psmallmatrix} \in \R^{2}$ отождествляется с комплексным числом $h_1 + i h_2 \in \CC$. В каких-то случаях нам будет удобнее говорить о $\R^{2}$, а в других случаях --- о $\CC$.
\end{conventn*}

Как это не было бы удивительно, чтобы эффективно и быстро построить теорию комплексного анализа, необходимо прибегнуть к дифференциальным формам порядка $1$. Поэтому, вспомним соответствующие определения из предыдущего семестра.

В определениях форм есть два важных отличия от предыдущего семестра. Во-первых, формы теперь могут принимать комплексные значения. Во-вторых, мы будем рассматривать формы, действующие только из $\R^{2}$ и имеющие порядок $1$. 

\begin{df}
 \label{definition:linear_form}
 \textit{Линейной формой} (порядка $1$) будем называть функцию $L \colon\, \R^{2} \to \CC$ со следующими свойствами:
 \begin{enumerate}
  \item $L(h_1 + h_2) = L(h_1) + L(h_2)$ для всех $h_1, h_2 \in \R^{2}$.
  \item \label{enum2:definition:linear_form} $L(\alpha h) = \alpha L(h)$ для всех $\alpha \in \R$ и $h \in \R^{2}$.

 \end{enumerate}
\end{df}
\begin{exmpl}
 Линейные формы $dx, dy \colon\, \R^{2} \to \CC$ определены формулами:
 \begin{align*}
  dx \begin{pmatrix}
   x \\ y
   \end{pmatrix} = x, & &dy \begin{pmatrix}
   x \\ y
  \end{pmatrix} = y.
 \end{align*} Легко проверить, что $dx$ и $dy$ удовлетворяют определению \ref{definition:linear_form}.
\end{exmpl}

\begin{remrk}
 Лучше считать, что линейные формы действуют из $\R^{2}$. Можно считать, что линейная форма действует из $\CC$, но в таком случае возникает следующий нюанс. Условие $\alpha \in \R$ в пункте \ref{enum2:definition:linear_form} становится критическим. Хоть линейная форма и может принимать комплексные значения, и должна быть линейной по $\R$, \textbf{она совершенно не обязана быть линейной по $\CC$}.

 Например, форма $dx$ не линейна по $\CC$: возьмём вектор $h = \begin{psmallmatrix}
  1 \\ 0
 \end{psmallmatrix}$, который отождествлён с комплексным числом $z = 1$ и комплексный коэффициент $\alpha = i$:
 \begin{align*}
  dx(\alpha z) = dx(i) = 0, & &\alpha \cdot dx(z) = i \cdot 1 = i.
 \end{align*} 
\end{remrk}
\begin{remrk}
 \label{remark:linear_forms_is_linear_space}
 Линейные формы образуют линейное пространство над полем $\CC$, которое мы будем обозначать $\LF$ (конечно, это даже банахово пространство). Формы $dx$ и $dy$ образуют базис в этом пространстве: всякая линейная форма имеет вид $z_1\,dx + z_2\,dy$ для некоторых $z_1, z_2 \in \CC$.
\end{remrk}
\begin{exmpl*}
 Важная линейная форма: $dz = dx + i\,dy$. По сути эта форма --- функция $\R^{2} \to \CC$, задающая отождествление $\R^{2} \cong \CC$.
\end{exmpl*}

\begin{df}
 \textit{Дифференциальной формой} порядка $1$ в $\CC$ (или \textit{$1$-формой}) называется отображение $\omega \colon\, \Omega \to \LF$, которое каждой точке $z$ из области $\Omega \subset \CC$ сопоставляет линейную форму $\omega(z) \in \LF$ (в смысле определения \ref{definition:linear_form}).
\end{df}
\begin{remrk*}
 В силу замечания \ref{remark:linear_forms_is_linear_space} всякая $1$-форма $\omega \colon\, \Omega \to \LF$ имеет вид
 \begin{align}
  \label{equation:1-form}
  \omega = P(x,y)\,dx + Q(x,y)\,dy
 ,\end{align} где $P, Q \colon\, \Omega \to \CC$ --- функции.
\end{remrk*}

\begin{exmpl*}
 $\omega = x \, dy$ --- $1$-форма. В точке $2 \in \CC$ на векторе $ \begin{psmallmatrix}
  1 \\ 3
 \end{psmallmatrix} \in \R^{2}$ эта форма принимает значение
 \begin{align*}
  \left(\omega (2)\right) \begin{psmallmatrix}
   1 \\ 3
   \end{psmallmatrix}= (2 \, dy) \begin{psmallmatrix}
   1 \\ 3
  \end{psmallmatrix} = 2 \cdot 3 = 6
 .\end{align*} 
\end{exmpl*}
\begin{exmpl*}
 \begin{align*}
  z\,dz &= (x + iy)(dx + i\,dy) = (x + iy)\,dx + (-y +ix)\,dy
 .\end{align*} Здесь $P(x,y) = x + iy$ и $Q(x,y) = -y + ix$.
\end{exmpl*}
\begin{df*}
 $1$-форма $\omega \colon\, \Omega \to \LF$, представленная в виде \eqref{equation:1-form}, называется \textit{непрерывной}, если функции $P,Q \colon\, \Omega \to \CC$ непрерывны. Форма $\omega$ называется $C^{k}$-гладкой, если $P$ и $Q$ --- $C^{k}$-гладкие (здесь $k \in \N \cup \left\{ \infty \right\}$).
\end{df*}

Мы вспомнили определение дифференциальных форм порядка $1$. Теперь вспомним определение объектов, по которым мы умеем интегрировать такие формы. Эти объекты --- поверхности размерности $1$ в $\CC$, то есть \textit{пути}.

\begin{df*}
 \textit{Путём} в $\CC$ называется непрерывная функция $\gamma \colon [a,b] \to \CC$. Путь $\gamma$ называется \textit{гладким}, если $\gamma \in C^{1}([a,b],\CC)$ --- гладкая функция. Путь $\gamma$ называется \textit{кусочно-гладким}, если $\gamma$ --- кусочно-гладкая функция, то есть $\gamma = \gamma_1 + \ldots + \gamma_n$, где $\gamma_i$ --- гладкие пути.
\end{df*}

\begin{df*}
 Если $\gamma_1 \colon [a,b] \to \CC$ и $\gamma_2 \colon [b,c] \to \CC$ --- пути, то \textit{суммой путей $\gamma_1$ и $\gamma_2$} называется путь $(\gamma_1 + \gamma_2)\colon\, [a,c] \to \CC$, определённый по формуле
 \begin{align*}
  (\gamma_1 + \gamma_2)(t) = \begin{cases}
   \gamma_1(t), \text{ если } t \in [a,b],  \\
   \gamma_2(t), \text{ если } t \in [b,c].
  \end{cases} 
 \end{align*} 
\end{df*}

\begin{notatn*}
 Если $\gamma \colon\, [a,b] \to \CC$ --- путь, то путь $-\gamma \colon [a,b] \to \CC$ определяется по формуле:
 \begin{align*}
  -\gamma(t) = \gamma(b - t + a)
 \end{align*} --- это тот же путь, но проходимый в обратном направлении.
\end{notatn*}
\begin{notatn*}
 Если $\gamma \colon\, [c, d] \to \CC$ --- путь, то $b(\gamma) = \gamma(c) \in \CC$ --- \textit{начало пути} (begin), а $e(\gamma) = \gamma(d) \in \CC$ --- конец пути (end).
\end{notatn*}

Теперь вспомним (и обобщим) понятие интеграла $1$-формы по пути в $\CC$.

\begin{df}[интеграл $1$-формы по пути]
 Пусть $\Omega \subset \CC$ --- область, $\gamma \colon [a,b] \to \Omega$ --- кусочно-гладкий путь, и $\omega \colon\, \Omega \to \LF$ --- непрерывная $1$-форма. Тогда \textit{интегралом} от $1$-формы $\omega$ по пути  $\gamma$ называется число
 \begin{align*}
  \int\limits_{\gamma} \omega = \int\limits_{a}^{b} \left( P(\gamma(t)) \gamma_1'(t) + Q(\gamma(t)) \gamma_2'(t) \right) dt,
 \end{align*} где $\gamma(t) = (\gamma_1(t),\gamma_2(t))$ (или $\gamma(t) = \gamma_1(t) + i \gamma_2(t)$).
\end{df}
\begin{lm}[основная оценка интеграла]
 \begin{align*}
  \left| \int\limits_{\gamma} \omega  \right| \leqslant \max_{t \in [a,b]} \sqrt{\left| P(\gamma(t)) \right|^{2} + \left| Q(\gamma(t)) \right|^{2}} \cdot l(\gamma),
 \end{align*} где $l(\gamma)$ --- длина пути $\gamma$.
\end{lm}
\begin{proof}
 Воспользуемся <<обычной>> основной оценкой интеграла (утверждение \ref{claim:basic_estimation_of_integral}), а также неравенством \eqref{equation:lemma:cauchy_bunyakovsky_schwarz_inequality} КБШ для пространства $\CC^{2}$:
 \begin{align*}
  \left| \int\limits_{\gamma} \omega  \right| &\leqslant \int\limits_{a}^{b} \left| P(\gamma(t))\gamma_1'(t) + Q(\gamma(t))\gamma_2'(t) \right|dt \leqslant \\
  &\leqslant \int\limits_{a}^{b} \sqrt{\left| P(\gamma(t)) \right|^{2} + \left| Q(\gamma(t)) \right|^{2}} \cdot \sqrt{\left| \gamma_1'(t) \right|^{2} + \left| \gamma_2'(t) \right|^{2}} dt \leqslant \\
  &\leqslant \max_{t \in [a,b]} \sqrt{\left| P(\gamma(t)) \right|^{2} + \left| Q(\gamma(t)) \right|^{2}} \cdot \int\limits_{a}^{b} \left| \gamma'(t) \right|  dt
  .\end{align*} Интеграл $\int_{a}^{b} \left| \gamma'(t) \right|dt$ и есть в точности длина пути $l(\gamma)$. Неравенство КБШ было применено к векторам $ \begin{psmallmatrix}
  P(\gamma(t)) \\ Q(\gamma(t))
  \end{psmallmatrix}, \begin{psmallmatrix}
  \gamma_1'(t) \\ \gamma_2'(t)
 \end{psmallmatrix} \in \CC^{2}$.
\end{proof}

Дадим новое важное топологическое определение.
\begin{df}[гомотопия]
 \textit{Гомотопией} в области $\Omega \subset \CC$ называется непрерывное отображение $H \colon [a,b] \times [0,1] \to \Omega$, $H = H(s,t)$. 
\end{df}

Гомотопии часто применяются, когда работают с путями.

\begin{df}[гомотопные пути]
 \label{definition:path-homotopy}
 Пусть $\Omega \subset \CC$ --- область, $\gamma_0, \gamma_1\colon [a,b] \to \Omega$ --- два пути в $\Omega$. Путь $\gamma_0$ \textit{гомотопен} пути $\gamma_1$ в области $\Omega$, если их начало и конец совпадают ($b(\gamma_0) = b(\gamma_1)$, $e(\gamma_0) = e(\gamma_1)$), а также существует гомотопия $H \colon [a,b] \times [0,1] \to \Omega$  такая, что для любого $s \in [a,b]$:
 \begin{align*}
  H(s,0) = \gamma_0(s), & &H(s,1) = \gamma_1(s).
 \end{align*} Иными словами, два пути гомотопны, если один можно переделать в другой непрерывным образом.
\end{df}

Пример двух гомотопных путей см. на рисунке \ref{fig:path-homptopy}.

\begin{figure}[ht]
 \centering
 \incfig[0.5]{path-homptopy}
 \caption{Пути и гомотопия между ними.}
 \label{fig:path-homptopy}
\end{figure}

Приведём пример, который показывает, что область $\Omega$ в определении \ref{definition:path-homotopy} гомотопии путей важна.

\begin{exmpl}
 Рассмотрим рисунок \ref{fig:paths-without-homptopy}. Путь $\gamma_0$ --- верхняя полуокружность, направленная по часовой стрелке, а путь $\gamma_1$ ---  нижняя полуокружность, направленная против часовой стрелки. Пути $\gamma_0$ и $\gamma_1$ гомотопны в области $\CC$, но не гомотопны в области $\CC \setminus \left\{ 0 \right\}$.

 Интуитивное объяснение такое: мы не можем непрерывно перевести $\gamma_0$ в $\gamma_1$, миновав выколотую точку $0$. Формальное доказательство давать не будем (если интересно доказательство, см. фундаментальную группу единичной окружности).
\end{exmpl}

\begin{figure}[ht]
 \centering
 \incfig[0.7]{paths-without-homptopy}
 \caption{Пути гомотопны в $\CC$ (слева), но не гомотопны в $\CC \setminus \left\{ 0 \right\}$ (справа).}
 \label{fig:paths-without-homptopy}
\end{figure}

Вспомним определение \textit{точных} дифференциальных форм, которое мы уже давали раньше.

\begin{df}[%
 точная $1$-форма]
 $1$-форма $\omega \colon\, \Omega \to \LF$ называется \textit{точной} в области $\Omega \subset \CC$, если существует гладкая функция $F \colon\, \Omega \to \CC$ ($0$-форма) такая, что $\omega = dF$.
\end{df}

Рассмотрим следующий нетривиальный критерий точности формы.

\begin{thm}
 \label{theorem:exact_1_form}
 Пусть $\omega$ --- непрерывная $1$-форма в области $\Omega \subset \CC$. Тогда следующие условия равносильны:
 \begin{enumerate}
  \item \label{enum1:theorem:exact_1_form} $\omega$ точна в $\Omega$.
  \item \label{enum2:theorem:exact_1_form} Для любых кусочно-гладких путей $\gamma_0, \gamma_1 \colon [0,1] \to \Omega$ с совпадающими началом и концом ($b(\gamma_0) = b(\gamma_1)$ и $e(\gamma_0) = e(\gamma_1)$) верно $ \int_{\gamma_0}  \omega = \int_{\gamma_1} \omega $. То есть, интеграл формы по пути зависит лишь от начала и от конца пути.
 \end{enumerate}
\end{thm}
\begin{proof}\
 \begin{itemize}
  \item Из пункта \ref{enum1:theorem:exact_1_form} в пункт \ref{enum2:theorem:exact_1_form}. Пусть существует гладкая функция $F \colon\,\Omega \to \CC$ такая, что
   \begin{align*}
    \omega = d F = F'_x \, dx + F'_y \, dy
   .\end{align*} Возьмём любой кусочно-гладкий путь $\gamma \colon [0,1] \to \Omega$ и проинтегрируем по нему $\omega$, используя формулу Ньютона-Лейбница:
   \begin{align*}
    \int\limits_{\gamma} \omega &= \int\limits_{\gamma} dF = \int\limits_{0}^{1} \left( F'_x(\gamma(t)) \gamma_1'(t)  + F'_y(\gamma(t)) \gamma_2'(t) \right) dt = \\
    &= \int\limits_{0}^{1} \left( F(\gamma(t)) \right)'_t dt = F(\gamma(1)) - F(\gamma(0)) = F(e(\gamma)) - F(b(\gamma))
   .\end{align*} Полученное выражение зависит только от начала и конца пути.
  \item Из пункта \ref{enum2:theorem:exact_1_form} в пункт \ref{enum1:theorem:exact_1_form}. Имея знание о пункте \ref{enum2:theorem:exact_1_form} мы красивым образом сконструируем функцию $F$. Выберем произвольную точку $p_0 \in \Omega$. Для любой точки $p \in \Omega$ определим
   \begin{align*}
    F(p) = \int\limits_{\gamma(p_0,p)}  \omega
   ,\end{align*} где $\gamma(p_0,p)$ --- произвольный кусочно-гладкий путь в  $\Omega$, соединяющий $p_0$ и $p$. Это определение корректно, так как по пункту \ref{enum2:theorem:exact_1_form} интеграл формы не зависит от выбора пути. Кроме того, путь между $p_0$ и $p$ существует, так как $\Omega$ --- линейно связное множество ({\color{red} однако не ясно, почему этот путь кусочно-гладкий}).

   Проверим, что построенная функция $F$ подходит. Возьмём произвольную точку $p \in \Omega$. Рассмотрим малое приращение $h \in \CC$, $h \to 0$:
   \begin{align*}
    &F(p + h) - F(p) = \int\limits_{\gamma(p_0,p + h)}  \omega - \int\limits_{\gamma(p_0,p)}  \omega = \int\limits_{\gamma(p_0,p) + [p,p+h]} \omega - \int\limits_{\gamma(p_0,p)}   \omega = \int\limits_{[p,p+h]} \omega.
   \end{align*} Здесь $[p,p+h] \subset \Omega$ --- отрезок, соединяющий точки $p$ и $p + h$, рассматриваемый как путь в $\Omega$. Так как $h$ очень мало, а $\Omega$ открыто, то отрезок лежит в $\Omega$. Этот отрезок параметризуется функцией $\gamma \colon [0,1] \to \Omega$, $\gamma(t) = p + th$. Производная этой функции равна $\gamma'(t) = (\gamma'_1(t), \gamma'_2(t)) = (h_1, h_2)$, где $h = h_i + i h_2$. Запишем $\omega = P(z)\,dx + Q(z)\,dy$, где $P,Q$ --- непрерывные функции. Подставим всё это:
   \begin{align*}
    F(p+h)-F(p) &= \int\limits_{0}^{1} \left( P(p+th) h_1 + Q(p+th) h_2 \right)dt = \\
    &=  \int\limits_{0}^{1} \left( (P(p) + o(1))h_1 + (Q(p) + o(1))h_2 \right)dt = \\
    &= \int\limits_{0}^{1} \left( P(p)h_1 + Q(p)h_2 + o(h) \right)dt = \\
    &= P(p)h_1 + Q(p)h_2 + o(h).
   \end{align*} Получается, функция $F$ по определению дифференцируема в точке $p$, и её дифференциал на приращении $h$ равен
   \begin{align*}
    d_p F \begin{pmatrix}
     h_1 \\ h_2
     \end{pmatrix} = P(p) h_1 + Q(p) h_2 = \left( \omega(p) \right) \begin{pmatrix}
     h_1 \\ h_2
    \end{pmatrix}
   ,\end{align*} что и требовалось доказать.
 \end{itemize}
\end{proof}

Теперь введём определение \textit{замкнутой} дифференциальной формы, которое имеет связь с точными формами.

\begin{df}[замкнутая $1$-форма]
 $1$-форма $\omega \colon\,\Omega\to\LF$ называется \textit{замкнутой} в области $\Omega$, если $\omega$ локально точна: для любой точки $p \in \Omega$ существует окрестность $U_p \ni p$, $U_p \subset \Omega$ такая, что $\omega$ точна в $U_p$.
\end{df}

Как и для точных форм, есть следующие нетривиальные критерии замкнутости формы.

\begin{thm}
 \label{theorem:closed_1_form}
 Пусть $\omega$ --- непрерывная $1$-форма на области $\Omega \subset \CC$. Тогда следующие условия эквивалентны:
 \begin{enumerate}
  \item \label{enum1:theorem:closed_1_form} $\omega$ замкнута в $\Omega$.
  \item \label{enum2:theorem:closed_1_form} Если пути $\gamma_0, \gamma_1 \colon [0,1] \to \Omega$ гомотопны в $\Omega$, то $\int_{\gamma_0} \omega = \int_{\gamma_1} \omega $.
  \item \label{enum3:theorem:closed_1_form} Если $\Pi$ --- замкнутый прямоугольник в $\Omega$, а $\partial \Pi$ --- его граница, то $\int_{\partial \Pi} \omega = 0$ (см. рисунок \ref{fig:closed-rectangle-and-its-border}).
   \begin{figure}[ht]
    \centering
    \incfig[0.5]{closed-rectangle-and-its-border}
    \caption{Замкнутый прямоугольник в области и его граница.}
    \label{fig:closed-rectangle-and-its-border}
   \end{figure}
 \end{enumerate}
\end{thm}
\begin{proof}\
 \begin{itemize}
  \item Из пункта \ref{enum1:theorem:closed_1_form} в пункт \ref{enum2:theorem:closed_1_form}. Пусть $\omega$ замкнута в $\Omega$. Пусть пути $\gamma_0$, $\gamma_1$ гомотопны в $\Omega$, $H \colon [0,1] \times [0,1] \to \Omega$ --- их гомотопия. Заметим, что образ
   \begin{align*}
    H([0,1] \times [0,1]) = K
   \end{align*}  является компактным (как непрерывный образ компакта).
   \begin{prop}
    \label{prop:eps_0:theorem:closed_1_form}
    Существует число $\eps_0 > 0$ такое, что для любой точки $p \in K$ форма $\omega$ точна в шаре $B(p,\eps_0) \subset \Omega$. 
   \end{prop}
   \begin{proof}[\normalfont\textsc{Доказательство}]
    Действительно, так как $\omega$ замкнута, то для любой точки $p \in K$ существует число $\eps(p) > 0$ такое, что $\omega$ точна в шаре $B(p,2\eps(p))\subset\Omega$. Семейство открытых шаров $\{B(p,\eps(p))\}_{p \in K}$ образует открытое покрытие компакта $K$:  выделим из него конечное подпокрытие $\left\{ B(p_k, \eps(p_k)) \right\}_{k=1}^{N}$. Возьмём $\eps_0 = \min \left\{ \eps(p_1), \ldots, \eps(p_N) \right\}$.

    Проверим, что $\eps_0$ подходит: пусть $p \in K$ --- любая точка. Тогда $p \in B(p_k,\eps(p_k))$ для некоторого $k$. В таком случае по неравенству треугольника $B(p, \eps_0)\subset B(p_k,2\eps(p_k))$. Тогда $\omega$ тем более точна в $B(p,\eps_0)$.
   \end{proof}

   Гомотопия $H$ равномерно непрерывна на компакте $[0,1]\times[0,1]$ (по теореме Кантора с первого курса): для любого $\eps > 0$ существует число $\delta(\eps) > 0$, такое, что для любых двух точек $w_1,w_2 \in [0,1] \times [0,1]$ условие $\left\| w_1 - w_2 \right\| < \delta(\eps)$ влечёт $\left| H(w_1)-H(w_2) \right| < \eps$.

   Возьмём натуральное число $N \geqslant 1$ такое, что $1 / N < \delta(\eps_0 / 2)$, где $\eps_0$ взято из предложения \ref{prop:eps_0:theorem:closed_1_form}.

   Для всякого $t \in [0,1]$ обозначим путь $\gamma_t \colon\, s \mapsto H(s,t)$ ($s \in [0,1]$). Покажем равенство следующих интегралов:
   \begin{align}
    \label{equation:small_gamma_step:theorem:closed_1_form}
    \int\limits_{\gamma_{\frac{k}{N}}}  \omega = \int\limits_{\gamma_{\frac{k+1}{N}}}  \omega,
   \end{align} где $k \in \left\{ 0 \ldots N - 1 \right\}$. Ясно, что из \eqref{equation:small_gamma_step:theorem:closed_1_form} всё последует: будет цепочка из $N$ равенств, в конце установим равенство интегралов по $\gamma_0$ и $\gamma_1$.

   Зафиксируем $k \in \left\{ 0, \ldots, N-1 \right\}$ и обозначим $\eta = \gamma_{\frac{k}{N}}$  и $\rho = \gamma_{\frac{k+1}{N}}$. Разобьём оба пути на маленькие кусочки:
   \begin{align*}
    \eta = \eta_0 + \eta_1 + \ldots + \eta_{N-1}, & &\rho = \rho_0 + \rho_1 + \ldots + \rho_{N-1},
   \end{align*} где
   \begin{align*}
    \eta_j = \eta \rvert_{\left[\frac{j}{N},\frac{j+1}{N}\right]  }, & &\rho_j = \rho \rvert_{\left[\frac{j}{N},\frac{j+1}{N}\right]  },
   \end{align*} для всех $j \in \left\{ 0, \ldots, N - 1 \right\}$. Теперь для каждого $j$ соединим концы и начала путей $\eta_j$, $\rho_j$ отрезками $I_j$, $J_j$ соответственно. Отрезок $I_j$ направим от $\eta_j$ к $\rho_j$, а отрезок $J_j$ направим от $\rho_j$ к $\eta_j$. Получим направленный контур $C_j = \eta_j + I_j + (-\rho_j) + J_j$, изображённый на рисунке \ref{fig:closed_1_form_small_contur}.

   \begin{figure}[ht]
    \centering
    \incfig{closed_1_form_small_contur}
    \caption{Малый контур $C_j = \eta_j + I_j + (-\rho_j) + J_j$, в котором форма $\omega$ точна.}
    \label{fig:closed_1_form_small_contur}
   \end{figure}

   За точку $P$ обозначим начало пути $\eta_j$. Заметим, что расстояние от любой точки контура $C_j$ до $P$ не превосходит $\eps_0$. Действительно, это следует из того, что
   \begin{align*}
    C_j = H(\partial \Pi),
   \end{align*} где $\partial \Pi$ --- граница прямоугольника
   \begin{align*}
    \Pi = \left[ \frac{k}{N}, \frac{k+1}{N} \right] \times \left[ \frac{j}{N}, \frac{j+1}{N} \right].
   \end{align*} Так как из любой точки $\partial \Pi$ мы можем за один или два шага длиной меньше $1 / N$ дойти до точки $(k / N, j / N)$, а $N$ выбрано так, что $1 / N < \delta(\eps_0 / 2)$, то из любой точки контура мы можем за один или два шага длиной менее $\eps_0 / 2$ дойти до точки $P$. Следовательно, расстояние от неё до $P$ меньше $\eps_0$.

   Иными словами, $C_j \subset B(P,\eps_0)$. Но так как форма $\omega$ точна в шаре $B(P,\eps_0)$, то
   \begin{align*}
    \int\limits_{C_j} \omega = 0.
   \end{align*} Просуммируем теперь это равенство по всем $j \in \left\{ 0, \ldots,N-1 \right\}$, сократив интегралы по парам отрезков вида $I_j$, $J_{j+1}$:
   \begin{align*}
    \sum_{j=0}^{N-1} \int\limits_{C_j} \omega = \int\limits_{\eta + (- \rho)}  \omega = 0.
   \end{align*} Следовательно,
   \begin{align*}
    \int\limits_{\eta} \omega = \int\limits_{\rho} \omega,  
   \end{align*} что и требовалось доказать.
  \item Из пункта \ref{enum2:theorem:closed_1_form} в пункт \ref{enum3:theorem:closed_1_form}. Если весь прямоугольник $\Pi \subset \Omega$, то его границу $\partial \Pi$ можно стянуть в точку в $\Omega$. То есть, $\partial\Pi$ гомотопна постоянному пути $\gamma_{z_0}(t) = z_0$. Значит,
   \begin{align*}
    \int\limits_{\partial\Pi} \omega = \int\limits_{\gamma_{z_0}}   \omega = 0.
   \end{align*} 
\begin{figure}[ht]
    \centering
    \incfig{rectangle_into_point}
    \caption{rectangle_into_point}
    \label{fig:rectangle_into_point}
\end{figure}

\item Из пункта \ref{enum3:theorem:closed_1_form} в пункт \ref{enum1:theorem:closed_1_form}. Возьмём любую точку $z_0 \in \Omega$. Нужно найти функция $F \colon\, \R^{2} \to \CC$ такую, что $dF = \omega$ в $B(z_0,\eps)$ для некоторого $\eps > 0$. Положим
 \begin{align*}
  F(x,y) = \int\limits_{\gamma(x,y)} \omega,
 \end{align*} где $\gamma(x,y)$ --- это путь из двух отрезков, параллельных осям $x$ и $y$ с началом в точке $z_0 = (x_0, y_0)$ и концом в точке $(x,y)$.

\begin{figure}[ht]
    \centering
    \incfig{parrallel_path_in_oblast}
    \caption{parrallel_path_in_oblast}
    \label{fig:parrallel_path_in_oblast}
\end{figure}

 Как и раньше, проверяется, что $dF = \omega$.

\begin{figure}[ht]
    \centering
    \incfig{differential_of_parallel_path_f}
    \caption{differential_of_parallel_path_F}
    \label{fig:differential_of_parallel_path_f}
\end{figure}

  \begin{align*}
   \int\limits_{\gamma(z+h)}  \omega - \int\limits_{\gamma(z)}  \omega = \int\limits_{\Gamma} \omega. 
  \end{align*} Далее так же.
 \end{itemize}
\end{proof}

