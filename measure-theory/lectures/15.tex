% 2022.12.09 Lecture 15
\documentclass[../measure-theory.tex]{subfiles}
\begin{document}

\subsection{Формула Грина.}

Рассмотрим некоторые приложения теоремы Стокса: её частные случаи в $\R^{2}$ и $\R^{3}$.

\begin{df*}
 Пусть есть два пути $\gamma_1 \colon [a,b] \to \R^{n}$ и $\gamma_2 \colon [b,c] \to \R^{n}$, причём $\gamma_1(b) = \gamma_2(b)$. Тогда \textit{cумма путей} $(\gamma_1 + \gamma_2) \colon [a,c] \to \R^{n}$ задаётся так:
 \begin{align*}
  (\gamma_1 + \gamma_2)(t) = \begin{cases}
   \gamma_1(t), \text{ если } t \in [a,b],  \\
   \gamma_2(t), \text{ если } t \in [b,c].
  \end{cases} 
 \end{align*} 
\end{df*}

\begin{thm}[%
 формула Грина]
 \label{theorem:green_formula}
 Пусть $\Omega \subset \R^{2}$ --- ограниченная область с кусочно-гладкой границей: 
\begin{align*}
 \partial \Omega = \bigcup_{k=1}^{N}  \gamma_{k,1} + \ldots + \gamma_{k,n_k}
,\end{align*} где $\gamma_{k,j}$ --- гладкие пути. Пусть $\partial \Omega$ параметризована так, что область $\Omega$ остаётся слева при обходе вдоль границы. Пусть $P,Q \in C^{1}(\overline \Omega, \R)$ --- гладкие функции. Тогда верна \textbf{формула Грина:}
 \begin{align}
  \label{equation:green_formula}
  \int\limits_{\delta \Omega} P(x,y) \, dx + Q(x,y)\, dy  = \int\limits_{\Omega} (Q'_x - P'_y)\,dx\,dy
 .\end{align}
\end{thm}

Пример такой области показан на рисунке \ref{fig:oblast_border_sum_of_smooth_paths}.

\begin{figure}[ht]
 \centering
 \incfig{oblast_border_sum_of_smooth_paths}
 \caption{Ограниченная область в $\R^{2}$ с кусочно-гладкой границей.}
 \label{fig:oblast_border_sum_of_smooth_paths}
\end{figure}

Оказывается, формула \eqref{equation:green_formula} Грина --- это частный случай формулы \eqref{equation:formula_stox} Стокса.

\begin{proof}[\normalfont\textsc{Вывод формулы Грина из формулы Стокса}]\
 Рассмотрим дифференциальную форму
 \begin{align*}
  \omega = P \, dx + Q \, dy
 .\end{align*} Рассмотрим замыкание области $\overline \Omega$ как многообразие $M$ размерности $2$ (с тождественной параметризацией). Тогда $\partial M = \partial \Omega$, где $\partial M$ --- край многообразия, а $\partial \Omega$ --- граница множества (в топологическом смысле). Ориентации многообразий согласованы за счёт условия <<область остаётся слева при обходе вдоль границы>> (хотя вопрос ориентации мы не рассматривали строго). Тогда по формуле \eqref{equation:formula_stox} Стокса:
 \begin{align*}
  \int\limits_{\partial \Omega} P\,dx + Q\,dy &= \int\limits_{\Omega} d(P\,dx+Q\,dy) = \\
  &= \int\limits_{\Omega} (P'_x \, dx + P'_y\, dy) \land dx + (Q'_x\, dx + Q'_y\, dy) \land dy = \\
  &= \int\limits_{\Omega} P'_y \, dy \land dx + Q'_x \, dx \land dy = \int\limits_{\Omega} (Q'_x - P'_y) \, dx \land dy = \\
  &= \int\limits_{\Omega} (Q'_x - P'_y) \,dx\,dy
 .\end{align*} Последний переход совершён с учётом того, что параметризация многообразия тождественная: переносить форму не нужно, можно просто стереть крышечки.
\end{proof}

Поскольку мы не доказали теорему \ref{theorem:stox} Стокса, и даже не до конца понимаем её формулировку, то мы приведём другое доказательство теоремы \ref{theorem:green_formula}.

\begin{proof}[\normalfont\textsc{Элементарное доказательство формулы Грина}]\
 Сначала будем считать, что $Q = 0$, а $\Omega$ --- область, стандартная относительно оси $x$, то есть \begin{align*}
  \Omega = \left\{ (x, y) \Mid x \in (a, b),\; y \in (\varphi_1(x), \varphi_2(x)) \right\}
 ,\end{align*} где $\varphi_1, \varphi_2 \in PC^{1}([a, b])$ и $\varphi_1 < \varphi_2$ на $(a, b)$. Здесь $PC^{1}([a,b])$ --- кусочно-гладкие функции на отрезке $[a,b]$. Пример см. на рисунке \ref{fig:oblast_elementary_on_ox}.

 \begin{figure}[ht]
  \centering
  \incfig[0.7]{oblast_elementary_on_ox}
  \caption{Область, стандартная относительно оси $x$.}
  \label{fig:oblast_elementary_on_ox}
 \end{figure}

 В таком случае

 \begin{align*}
  &\int\limits_{\partial \Omega} P(x,y) \, dx + Q(x,y) \, dy  = \int\limits_{\gamma_1 + \gamma_2 + \gamma_3 + \gamma_4} P(x,y) \, dx = \\
  = &\int\limits_{\gamma_1} P(x,y)\,dx + \int\limits_{\gamma_2} P(x,y)\, dx + \int\limits_{\gamma_3} P(x,y)\, dx + \int\limits_{\gamma_4} P(x,y)\, dx
 .\end{align*} Интегралы на вертикалях (пути $\gamma_1$ и $\gamma_3$) равны нулю, потому что там $dx = 0$.  Параметризуем путь $\gamma_2$ как $(x, \varphi_1(x))$ (при $x \in [a,b]$) и путь $-\gamma_4$ как $(x, \varphi_2(x))$ (при $x \in [a,b]$). Тогда
 \begin{align*}
  \int\limits_{\partial \Omega} P(x,y)\,dx &= \int\limits_{a}^{b} P(x, \varphi_1(x)) \, dx - \int\limits_{a}^{b} P(x, \varphi_2(x)) \, dx  = \\
  &= \int\limits_{a}^{b} (P(x, \varphi_1(x)) - P(x, \varphi_2(x))) \, dx 
 .\end{align*} Правая часть формулы \eqref{equation:green_formula} по теореме \ref{theorem:fubini} Фубини и формуле Ньютона-Лейбница равна:
 \begin{align*}
  \int\limits_{\Omega} -P'_y \, dx \, dy &= \int\limits_{a}^{b} \int\limits_{\varphi_1(x)}^{\varphi_2(x)} (-P'_y(x,y)) \, dy \, dx = \int\limits_{a}^{b} \left( -P(x, \varphi_2(x)) + P(x, \varphi_1(x)) \right) \, dx 
 .\end{align*} Итак, для этого частного случая доказали.

 Аналогично можно доказать формулу \eqref{equation:green_formula}, если область $\Omega$ стандартная относительно оси $y$, и $P = 0$.
 Значит, формула Грина верна для любых функций $P, Q$ и для области $\Omega$, стандартной относительно обеих осей $x$ и $y$ (пример см. на рисунке  \ref{fig:oblast_standard_on_both_axis}).

 \begin{figure}[ht]
  \centering
  \incfig[0.7]{oblast_standard_on_both_axis}
  \caption{Область, стандартная относительно обеих осей.}
  \label{fig:oblast_standard_on_both_axis}
 \end{figure}

 В общем случае, представим область $\Omega$ в виде объединения конечного числа непересекающихся областей $\Omega_i$ с кусочно-гладкой границей $\partial \Omega_i$, стандартных относительно обеих осей $x$ и $y$, и конечного числа путей, составляющих границы $\partial \Omega_i$. Пример см. на рисунке \ref{fig:oblast_union_of_standard_oblasts}.

 \begin{figure}[ht]
  \centering
  \incfig[0.7]{oblast_union_of_standard_oblasts}
  \caption{Разбиение области в объединение стандартных областей.}
  \label{fig:oblast_union_of_standard_oblasts}
 \end{figure}

 Для всех областей $\Omega_i$ формула  \eqref{equation:green_formula} уже доказана. Просуммируем по  $i$ обе части формулы: 
 \begin{align*}
  \sum_{i=1}^{N} \int\limits_{\partial \Omega_i} P\, dx + Q\,dy = \sum_{i=1}^{N} \int\limits_{\Omega_i} (Q'_x - P'_y) \, dx \, dy 
 .\end{align*} Правая часть равна $\int_{\Omega} (Q'_x - P'_y) \, dx \,dy  $  по аддитивности интеграла Лебега. А левая часть равна
 \begin{align*}
  \int\limits_{\partial \Omega} P\, dx + Q\, dy
 ,\end{align*}  так как интегралы по всем внутренним путям сокращаются. Например, на рисунке \ref{fig:oblast_union_of_standard_oblasts} отрезок $p_1 p_2$ внесёт вклад в левую часть два раза: сначала он будет проинтегрирован в направлении от $p_1$ к $p_2$ (как часть границы $\partial \Omega_i$), а потом --- в направлении от $p_2$ к $p_1$ (как часть границы $\partial \Omega_j$). Эти интегралы сократятся, и отрезок $p_1 p_2$ внесёт вклад $0$ в левую часть.
\end{proof}
\begin{remrk*}
 Разбиение области $\Omega$ на объединение стандартных областей $\Omega_i$ в простых случаях можно построить явно. Построение такого разбиения в общем случае оставим недоказанным.
\end{remrk*}

\begin{exmpl*}
 Посчитаем площадь лепестка лемнискаты $\left( x^{2} + y^{2} \right)^2 = x^{2} - y^{2}$.

 \begin{figure}[ht]
  \centering
  \incfig[0.7]{lepestok_lemniskati}
  \caption{Лепесток лемнискаты.}
  \label{fig:lepestok_lemniskati}
 \end{figure}

 Введём полярные координаты
 \begin{align*}
  x &= r \cos\varphi, \\
  y &= r \sin\varphi
  .\end{align*} Тогда уравнение превратится в \begin{align*}
  r^{4} = r^{2} \cos (2 \varphi) \iff r = \sqrt{ \cos (2 \varphi) }
 .\end{align*} $\varphi \in [-\pi / 4, \pi / 4]$. Пользуясь тем, что якобиан полярной замены равен $r$, получаем
 \begin{align*}
  \lambda_2(\Omega) &= \int\limits_{-\pi / 4}^{\pi / 4} \int\limits_{0}^{\sqrt{\cos(2\varphi)}} r  \, dr  \, d\varphi = \int\limits_{-\pi / 4}^{\pi / 4} \frac{\cos(2\varphi)}{2} \, d\varphi = \frac{\sin (2\varphi)}{4} \bigg\rvert_{-\pi / 4} ^{\pi / 4} = \frac{1}{2}
.\end{align*} 

Второй способ. По формуле \eqref{equation:green_formula} Грина ($P = 0$, $Q = x$) есть равенство
\begin{align*}
 \int\limits_{\partial \Omega} x \, dy = \int\limits_{\Omega} dx \, dy = \lambda_2(\Omega)
.\end{align*} Теперь нужно посчитать интеграл по кривой $\partial \Omega$. Возьмём ту же параметризацию. Тогда
\begin{align*}
 \int\limits_{\partial \Omega}  x \, dy &= \int\limits_{-\pi / 4}^{\pi / 4} r(\varphi) \cos\varphi \, d \left( r(\varphi) \sin\varphi \right) = \int\limits_{-\pi / 4}^{\pi / 4} \sqrt{\cos(2\varphi)} \cos \varphi \, d \left( \sqrt{\cos(2\varphi)} \sin \varphi \right) = \\
 &= \int\limits_{-\pi / 4}^{\pi / 4} \sqrt{\cos(2\varphi)}\cos\varphi \left( - \frac{2\sin(2\varphi) \cdot \sin\varphi}{2\sqrt{\cos(2\varphi)}} + \sqrt{\cos(2\varphi) } \cos \varphi \right)\,d\varphi = \\
 &= \int\limits_{-\pi / 4}^{\pi / 4} \left( -\cos\varphi\sin\varphi\sin(2\varphi) + \cos(2\varphi) \cos^{2}\varphi \right)\,d\varphi = \\
 &= \int\limits_{-\pi / 4}^{\pi / 4} \left( -\frac{\sin^{2}(2\varphi)}{2} + \cos(2\varphi) \frac{1 + \cos(2\varphi)}{2} \right)\,d\varphi = \\
 &= \frac{1}{2}\int\limits_{-\pi / 4}^{\pi / 4} \left( \cos^{2}(2\varphi) - \sin^{2}(2\varphi) + \cos(2\varphi) \right)\,d\varphi = \\
 &= \frac{1}{2}\int\limits_{-\pi / 4}^{\pi / 4} \cos(4\varphi)\,d\varphi + \frac{1}{2}\int\limits_{-\pi / 4}^{\pi / 4}  \cos(2\varphi)\,d\varphi = \\
 &= 0 + \frac{1}{2} \cdot \frac{1}{2} \cdot \sin(2\varphi) \rvert_{-\pi / 4}^{\pi / 4} = \frac{1}{2}.
\end{align*} 
\end{exmpl*}

\subsection{Поток поля через поверхность. Дивергенция.}

Рассмотрим физический смысл интегрирования $2$-форм по двумерным поверхностям в $\R^{3}$.

\begin{notatn*}
 Для сокращения записей вместо $\det A$ будем писать $\left| A \right|$.
\end{notatn*}

\begin{df*}
 \textit{Векторным произведением} векторов $u,v \in \R^{3}$ называется вектор
 \begin{align*}
  u \times v = 
  \begin{vmatrix}
   i & j & k \\
     & u & \\
     & v & \\
  \end{vmatrix} = \begin{vmatrix}
  i & j & k \\
  u_x & u_y & u_z \\
  v_x & v_y & v_z \\
  \end{vmatrix}.
 \end{align*} Здесь $i,j,k$ --- стандартный базис в $\R^{3}$, $u_x, v_x, u_y, v_y, u_z, v_z$ --- координаты векторов. Общеизвестный факт: $u \times v \perp u$, $u \times v \perp v$.
\end{df*}

\begin{df*}
\textit{Потоком} постоянного поля $F \in \R^{3}$ через параллелограмм, натянутый на вектора $u,v \in \R^{3}$ в направлении векторного произведения $u \times v$ называется число
 \begin{align*}
  \begin{vmatrix}
   F \\
   u \\
   v
  \end{vmatrix} = \begin{vmatrix}
  F_x & F_y & F_z \\
  u_x & u_y & u_z \\
  v_x & v_y & v_z \\
  \end{vmatrix}.
 \end{align*} Он равен объёму параллелепипеда, натянутого на вектора $u$, $v$ и $F$ (с точностью до знака). См. рисунок \ref{fig:flow_of_constant_field_through_parallelogramm}.

 \begin{figure}[ht]
  \centering
  \incfig{flow_of_constant_field_through_parallelogramm}
  \caption{Поток постоянного поля $F$ через параллелограмм, натянутый на вектора.}
  \label{fig:flow_of_constant_field_through_parallelogramm}
 \end{figure}
\end{df*}

\begin{df*}
 \textit{Потоком} переменного поля $F \colon\, \R^{3} \to \R^{3}$ через ориентированную двумерную поверхность $S$ в $\R^{3}$ называется число
 \begin{align*}
  \int\limits_{S} (F(x), n(x)) \, dS
 ,\end{align*}  где $n(x)$ --- единичная нормаль к поверхности $S$ в точке $x \in S$ (выбранная в соответствии с ориентацией поверхности). Интеграл берётся по поверхностной мере Лебега на $S$.
\end{df*}

Поток характеризует объем жидкости (заряда, \ldots), протекающей через поверхность за единицу времени. См. рисунок \ref{fig:flow_of_changing_field_through_surface}.

\begin{figure}[ht]
 \centering
 \incfig{flow_of_changing_field_through_surface}
 \caption{Поток переменного поля $F$ через поверхность $S$.}
 \label{fig:flow_of_changing_field_through_surface}
\end{figure}

\begin{remrk}
 \label{remark:flow_throw_surface}
 Пусть ориентированная поверхность $S$ параметризована гладкой биекцией $\Phi \colon\, \Omega \to S$, $\Phi(u,v) =(\Phi_1(u,v),\Phi_2(u,v),\Phi_3(u,v))$. Тогда единичная нормаль к поверхности $S$ в точке $\Phi(x)$ записывается так:
 \begin{align}
  \label{equation:normal_formula}
  n(\Phi(x)) &= \frac{\Phi'_u \times \Phi'_v}{\left| \Phi'_u \times \Phi'_v \right|} = \frac{ \begin{vmatrix}
    i & j & k \\
    \Phi'_{1,u} & \Phi'_{2,u} & \Phi'_{3,u} \\
    \Phi'_{1,v} & \Phi'_{2,v} & \Phi'_{3,v} \\
    \end{vmatrix} }{\sqrt{  \begin{vmatrix}
     \Phi'_{2,u} & \Phi'_{3,u} \\
     \Phi'_{2,v} & \Phi'_{3,v} \\
     \end{vmatrix}^{2} +  \begin{vmatrix}
     \Phi'_{1,u} & \Phi'_{3,u} \\
     \Phi'_{1,v} & \Phi'_{3,v} \\
     \end{vmatrix}^{2} +  \begin{vmatrix}
     \Phi'_{1,u} & \Phi'_{2,u} \\
     \Phi'_{1,v} & \Phi'_{2,v} \\
  \end{vmatrix}^2 }} = \\
  \nonumber
  &=  \frac{\Phi'_u \times \Phi'_v}{\sqrt{\det G_{\Phi}(x)}}
 ,\end{align} где последнее равенство верно по формуле Бине-Коши для $J_{\Phi}^{\top} J_{\Phi}$.

 Вычислим теперь подынтегральное выражение:
 \begin{align*}
  (F(\Phi(x)),n(\Phi(x))) = \frac{1}{\sqrt{\det G_{\Phi}(x)}} \begin{vmatrix*}
   F_1(\Phi(x)) & F_2(\Phi(x)) & F_3(\Phi(x)) \\
   \Phi'_{1,u}(x) & \Phi'_{2,u}(x) & \Phi'_{3,u}(x) \\
   \Phi'_{1,v}(x) & \Phi'_{2,v}(x) & \Phi'_{3,v}(x) \\
  \end{vmatrix*}
 .\end{align*}  Определитель
 \begin{align*}
  \begin{vmatrix*}
   F_1(\Phi(x)) & F_2(\Phi(x)) & F_3(\Phi(x)) \\
   \Phi'_{1,u}(x) & \Phi'_{2,u}(x) & \Phi'_{3,u}(x) \\
   \Phi'_{1,v}(x) & \Phi'_{2,v}(x) & \Phi'_{3,v}(x) \\
  \end{vmatrix*}
 \end{align*} --- это поток постоянного поля $F(\Phi(x)) \in \R^{3}$ через параллелограмм, натянутый на вектора $\Phi'_u(x), \Phi'_v(x) \in \R^{3}$.
\end{remrk}

Аналогично случаю работы по кривой, оказывается, что поток переменного поля можно выразить как интеграл некоторой $2$-формы, называемой \textit{формой потока}, по поверхности $S$.

\begin{df*}
 \textit{Формой потока} переменного поля $F \colon\, \R^{3} \to \R^{3}$ называется $2$-форма $\omega_{\Pi} \colon\, \R^{3} \to \bigwedge^{2}(\R^{3})^{\ast}$, заданная по формуле
 \begin{align}
  \label{equation:flow_form_of_vector_field}
  \omega_{\Pi} = F_1 \, dy \land dz + F_2 \, dz \land dx + F_3 \, dx \land dy
 .\end{align} Здесь $F = (F_1,F_2,F_3)$ --- координатные функции.
\end{df*}
\begin{claim}
 Поток переменного поля $F \colon\, \R^{3} \to \R^{3}$ через ориентированную двумерную поверхность $S$ равен
 \begin{align*}
  \int\limits_{S} \omega_{\Pi} = \int\limits_{S} (F, n)\, dS,
 \end{align*} где $\omega_{\Pi}$ --- форма потока поля $F$.
\end{claim}
\begin{proof}\

 Пусть поверхность $S$ параметризована гладкой биекцией $\Phi \colon\, \Omega \to S$, $\Omega \subset \R^{2}$. Вычислим левую часть по определению интеграла $2$-формы:
 \begin{align*}
  &\int\limits_{S} \omega_{\Pi}  = \int\limits_{S} F_1 \, dy \land dz + F_2 \, dz \land dx + F_3 \, dx \land dy = \\
  = &\int\limits_{\Omega} (F_1 \circ \Phi) \, d\Phi_2 \land d\Phi_3 + (F_2 \circ \Phi) \, d \Phi_3 \land d \Phi_1 + (F_3 \circ \Phi) \, d \Phi_1 \land d\Phi_2.
 \end{align*} Вычислим внешние произведения дифференциалов:
 \begin{align*}
  d \Phi_1 \land d\Phi_2 &= (\Phi'_{1,u}\,du + \Phi'_{1,v}\,dv) \land (\Phi'_{2,u}\,du + \Phi'_{2,v}\,dv) = \\
  &= \Phi'_{1,u} \Phi'_{2,v} \, du \land dv + \Phi'_{1,v} \Phi'_{2,u} \, dv \land du = \\
  &= (\Phi'_{1,u}\Phi'_{2,v} - \Phi'_{1,v}\Phi'_{2,u})\,du\land dv = \\
  &= \begin{vmatrix}
   \Phi'_{1,u} & \Phi'_{2,u} \\
   \Phi'_{1,v} & \Phi'_{2,v} \\
  \end{vmatrix} \, du \land dv, \\
  d \Phi_2 \land d \Phi_3 &= \begin{vmatrix}
   \Phi'_{2,u} & \Phi'_{3,u} \\
   \Phi'_{2,v} & \Phi'_{3,v} \\
  \end{vmatrix}\,du\land dv, \\
  d \Phi_3 \land d \Phi_1 &= \begin{vmatrix}
   \Phi'_{3,u} & \Phi'_{1,u} \\
   \Phi'_{3,v} & \Phi'_{1,v} \\
  \end{vmatrix}\,du\land dv.
 .\end{align*} Продолжим раскрывать левую часть:
 \begin{align*}
  = &\int\limits_{\Omega}  \left( (F_1 \circ \Phi) \begin{vmatrix}
    \Phi'_{2,u} & \Phi'_{3,u} \\
    \Phi'_{2,v} & \Phi'_{3,v} \\
    \end{vmatrix} + (F_2 \circ \Phi) \begin{vmatrix}
    \Phi'_{3,u} & \Phi'_{1,u} \\
    \Phi'_{3,v} & \Phi'_{1,v} \\
    \end{vmatrix} + (F_3 \circ \Phi) \begin{vmatrix}
    \Phi'_{1,u} & \Phi'_{2,u} \\
    \Phi'_{1,v} & \Phi'_{2,v} \\
  \end{vmatrix} \right)\,du\land dv = \\
  = &\int\limits_{\Omega} \begin{vmatrix}
   F_1(\Phi(u,v)) & F_2(\Phi(u,v)) & F_3(\Phi(u,v)) \\
   \Phi'_{1,u}(u,v) & \Phi'_{2,u}(u,v) & \Phi'_{3,u}(u,v) \\
   \Phi'_{1,v}(u,v) & \Phi'_{2,v}(u,v) & \Phi'_{3,v}(u,v) \\
  \end{vmatrix} \, du \, dv
 .\end{align*} 

 Теперь пользуясь замечанием \ref{remark:flow_throw_surface} раскроем правую часть:
 \begin{align*}
  \int\limits_{S} (F,n) \, dS &= \int\limits_{S} \frac{1}{\sqrt {\det G_{\Phi}(\Phi^{-1}(x))}} \begin{vmatrix}
   F_1 & F_2 & F_3 \\
   \Phi'_{1,u} & \Phi'_{2,u} & \Phi'_{3,u} \\
   \Phi'_{1,v} & \Phi'_{2,v} & \Phi'_{3,v} \\
  \end{vmatrix} \, dS = \\
  &= \int\limits_{\Omega} \frac{\sqrt{\det G_{\Phi}(u,v)}}{\sqrt{\det G_{\Phi}(u,v)}} \begin{vmatrix}
   F_1 & F_2 & F_3 \\
   \Phi'_{1,u} & \Phi'_{2,u} & \Phi'_{3,u} \\
   \Phi'_{1,v} & \Phi'_{2,v} & \Phi'_{3,v} \\
  \end{vmatrix} \, d\lambda_2 = \\
  &= \int\limits_{\Omega} \begin{vmatrix}
   F_1 & F_2 & F_3 \\
   \Phi'_{1,u} & \Phi'_{2,u} & \Phi'_{3,u} \\
   \Phi'_{1,v} & \Phi'_{2,v} & \Phi'_{3,v} \\
  \end{vmatrix} \, d\lambda_2
 .\end{align*} В числителе возник множитель $\sqrt{\det G_{\Phi}(u,v)}$ из-за перехода из поверхностной меры Лебега в карту. Равенство проверено.
\end{proof}

Сформулируем формулу Стокса в этой ситуации.

\begin{thm}[%
 формула Гаусса-Остроградского]
 \label{theorem:divergence_theorem}
 Пусть $\Omega \subset \R^{3}$  --- ограниченная область, её граница $\partial \Omega$  --- кусочно-гладкое многообразие (то есть объединение конечного числа гладких двумерных многообразий и кривых). Пусть параметризация многообразия $\partial \Omega$ выбрана так, чтобы нормаль к поверхности $\partial \Omega$, соответствующая этой параметризации (формула \eqref{equation:normal_formula}), торчала наружу области $\Omega$.

 Пусть $F \in C^{1}(\overline \Omega, \R^{3})$ --- гладкое векторное поле. Тогда поток поля $F$ через поверхность $\partial \Omega$ выражается \textbf{формулой Гаусса-Остроградского:}
 \begin{align*}
  \int\limits_{\partial \Omega} \omega_{\Pi} = \int\limits_{\Omega} \diverg F \, d\lambda_3
 ,\end{align*} где $\omega_{\Pi}$ --- форма потока поля $F$ (формула \eqref{equation:flow_form_of_vector_field}), а 
 \begin{align*}
  \diverg F = F'_{1,x} + F'_{2,y} + F'_{3,z} = \mathrm{trace}\,J_F
 \end{align*} --- \textbf{дивергенция} векторного поля $F$ --- функция $\diverg F \colon\, \overline \Omega \to \R$.
\end{thm}

\begin{proof}[\normalfont\textsc{Вывод формулы Гаусса-Остроградского из формулы Стокса}]
 \begin{align*}
  \int\limits_{\partial \Omega} \omega_{\Pi} &= \int\limits_{\Omega} d \omega_{\Pi} = \int\limits_{\Omega} dF_1 \land dy \land dz + dF_2 \land dz \land dx + dF_3 \land dx \land dy = \\
  &= \int\limits_{\Omega}  F'_{1,x} \, dx \land dy \land dz + F'_{2,y} \, dy \land dz \land dx + F'_{3,z}  \, dz \land dx \land dy = \\
  &= \int\limits_{\Omega}  \left( F'_{1,x} + F'_{2,y} + F'_{3,z} \right)\,dx \land dy \land dz = \int\limits_{\Omega}  \diverg F \, d\lambda_3.
 \end{align*} Действительно, в каждом множителе $dF_i$ из трёх слагаемых выживет только одно. Например,
 \begin{align*}
  dF_1 \land dy \land dz &= (F'_{1,x} \, dx + F'_{1,y} \, dy + F'_{1,z} \, dz) \land dy \land dz = F'_{1,x} \, dx \land dy  \land dz + 0 + 0
 .\end{align*} 
\end{proof}

Обсудим физический смысл дивергенции. 

\begin{thm}
 Пусть $\Omega \subset \R^{3}$ --- область, $F \in C^{1}(\overline \Omega, \R^{3})$ --- гладкое векторное поле, и $p \in \Omega$ --- точка. Тогда
 \begin{align*}
  \diverg F(p) = \lim_{\eps \to 0} \frac{1}{\left| B_{\eps}(p) \right|} \int\limits_{S_{\eps}(p)} (F,n)\, dS
 ,\end{align*} где $\left| B_{\eps}(p) \right|$ --- объём шара с радиусом $\eps$ в $\R^{3}$;
 \begin{align*}
  S_{\eps}(p) = \left\{ q \in \R^{3} \Mid \left| p-q \right| = \eps \right\}
 \end{align*} --- сфера с центром в точке $p$ и радиусом $\eps$; $n(q)$ --- единичная нормаль к сфере $S_{\eps}(p)$ в точке $q$.
\end{thm}
\begin{proof}
 Во-первых, можно считать, что $p = 0$. Если это не так, то сдвинем всё параллельным переносом, от этого ничего не поменяется (преобразование $q \mapsto q - p$ --- изометрия).

 Вычислим подынтегральное выражение в точке $q = (q_x,q_y,q_z) \in S_{\eps}(0)$, пользуясь дифференцируемостью отображения $F$ в точке $0$:
 \begin{align*}
  (F(q),n(q)) &= \frac{1}{\eps} \left( \begin{pmatrix}
    F_1(0) \\
    F_2(0) \\
    F_3(0) \\
    \end{pmatrix} + \begin{pmatrix}
    F'_{1,x}(0) q_x + F'_{1,y}(0) q_y + F'_{1,z}(0) q_z + o(\eps) \\
    F'_{2,x}(0) q_x + F'_{2,y}(0) q_y + F'_{2,z}(0) q_z + o(\eps) \\
    F'_{3,x}(0) q_x + F'_{3,y}(0) q_y + F'_{3,z}(0) q_z + o(\eps)
    \end{pmatrix}, \begin{pmatrix}
    q_x \\ q_y \\ q_z
  \end{pmatrix} \right)
 .\end{align*} Здесь мы также воспользовались понятным равенством $n(q) = q / \eps$: вектор $q$ является нормалью к сфере $S_{\eps}(0)$ в точке $q$, торчащей наружу и имеющей длину $\eps$.

 Проинтегрируем теперь это выражение по сфере $S_{\eps}(0)$. Заметим, что для любого фиксированного вектора $v \in \R^{3}$ верно
 \begin{align*}
  \int\limits_{S_{\eps}(0)} (v,n(q)) \, dS(q) = 0
 ,\end{align*} ведь эта функция нечётная: она меняет знак при переходе к противоположной точке на сфере. Поэтому, первое слагаемое $\frac{1}{\eps}(F(0), q)$ уничтожается.

 Далее, при различных индексах $i,j \in \left\{ x,y,z \right\}$, $i \neq j$ имеем
 \begin{align*}
  \int\limits_{S_{\eps}(0)} q_i \cdot q_j \, dS = 0
 ,\end{align*} также в силу нечётности (можно отразить сферу по оси $i$, от этого подынтегральное выражение поменяет знак).

 После убирания нулей остаётся
 \begin{align*}
  \int\limits_{S_{\eps}(0)} (F,n)\,dS &= \frac{1}{\eps}\int\limits_{S_{\eps}(0)} \left( F'_{1,x}(0)q_x^{2} + F'_{2,y}(0)q_y^{2} + F'_{3,z}(0)q_z^{2} + o(\eps^{2}) \right)dS
 .\end{align*} Простой заменой координат получаем следующее хитрое равенство
 \begin{align*}
  \int\limits_{S_{\eps}(0)} q_x^{2} \, dS &= \int\limits_{S_{\eps}(0)} q_y^{2} \, dS = \int\limits_{S_{\eps}(0)} q_z^{2} \, dS = \frac{1}{3} \int\limits_{S_{\eps}(0)} (q_x^{2} + q_y^{2} + q_z^{2})\,dS = \\
  &= \frac{1}{3} \int\limits_{S_{\eps}(0)} \eps^{2}\, dS = \frac{\eps^{2} \left| S_{\eps}(0) \right|}{3} = \frac{4 \pi \eps^{4}}{3} = \eps \left| B_{\eps}(0) \right|
 .\end{align*} Подставим и раскроем скобки:
 \begin{align*}
  \int\limits_{S_{\eps}(0)} (F,n)\,dS &= \frac{1}{\eps} (F'_{1,x}(0) + F'_{2,y}(0) + F'_{3,z}(0)) \cdot \eps \left| B_{\eps}(0) \right| + \frac{1}{\eps} \cdot o(\eps^{2}) \left| S_{\eps}(0) \right| = \\
  &= \diverg F(0) \cdot \left| B_{\eps}(0) \right| + o(\eps^{3})
 .\end{align*} После деления на $|B_{\eps}(0)|$ получаем в пределе $\diverg F(x)$, что и требовалось.  
\end{proof}

\begin{remrk*}
 Интеграл
 \begin{align*}
  \frac{1}{\left| B_{\eps}(p) \right|} \int\limits_{S_{\eps}(p)} (F,n)\,dS
 \end{align*} --- это нормированный поток поля $F$ через сферу $S_{\eps}(x)$. В пределе при $\eps \to 0$ получаем величину, которая характеризует \textit{мощность} источника (или стока, если величина отрицательная) в точке $p$. Это и есть физический смысл дивергенции поля $F$ в точке $p$.
\end{remrk*}

\end{document}
