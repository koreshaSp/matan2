\documentclass[a4paper,14pt]{extarticle}
\usepackage[margin=0.5in]{geometry}
\usepackage[T1,T2A]{fontenc}
\usepackage[utf8]{inputenc}
\usepackage[russian]{babel}
\usepackage{amsmath}
\usepackage{amssymb}
\usepackage{amsthm}
\usepackage{array}
\usepackage{breqn}
\usepackage{color}
\usepackage{enumerate}
\usepackage{enumitem}
\usepackage{gensymb}
\usepackage{hyperref}
\usepackage{import}
\usepackage{indentfirst}
\usepackage{listings}
\usepackage{microtype}
\usepackage{pdfpages}
\usepackage{pgfplots}
\usepackage{subfiles}
\usepackage{systeme,mathtools}
\usepackage{textcomp}
\usepackage{transparent}
\usepackage{verbatim}
\usepackage{xcolor}
\usepackage{xifthen}
\usepackage{esint}
\usepackage{subcaption}
\pgfplotsset{compat=1.16}

\hfuzz=6.41pt 

\newcommand{\incfig}[2][1]{%
 \def\svgwidth{#1\columnwidth}
 \import{./figures/}{#2.pdf_tex}
}
\graphicspath{{figures/}}

\pdfsuppresswarningpagegroup=1

% counters and theorems
\newcounter{theoremCnt}
\counterwithin{theoremCnt}{section}
\counterwithin{equation}{section}
\counterwithin{figure}{section}

\makeatletter
\@addtoreset{section}{part}
\makeatother

% numbered theorems
\theoremstyle{definition}
\newtheorem{df}[theoremCnt]{Определение}
\theoremstyle{plain}
\newtheorem{thm}[theoremCnt]{Теорема}
\theoremstyle{plain}
\newtheorem{lm}[theoremCnt]{Лемма}
\theoremstyle{plain}
\newtheorem{crly}[theoremCnt]{Следствие}
\theoremstyle{plain}
\newtheorem{prop}[theoremCnt]{Предложение}
\theoremstyle{definition}
\newtheorem{exmpl}[theoremCnt]{Пример}
\theoremstyle{definition}
\newtheorem{remrk}[theoremCnt]{Замечание}
\theoremstyle{definition}
\newtheorem{claim}[theoremCnt]{Утверждение}
\theoremstyle{definition}
\newtheorem{algorithm}[theoremCnt]{Алгоритм}
\theoremstyle{definition}
\newtheorem{problem}[theoremCnt]{Задача}
\theoremstyle{definition}
\newtheorem{exercise}[theoremCnt]{Упражнение}

% non-numbered theorems
\theoremstyle{definition}
\newtheorem*{df*}{Определение}
\theoremstyle{definition}
\newtheorem*{notatn*}{Обозначение}
\theoremstyle{definition}
\newtheorem*{conventn*}{Соглашение}
\theoremstyle{plain}
\newtheorem*{thm*}{Теорема}
\theoremstyle{plain}
\newtheorem*{lm*}{Лемма}
\theoremstyle{plain}
\newtheorem*{crly*}{Следствие}
\theoremstyle{plain}
\newtheorem*{prop*}{Предложение}
\theoremstyle{definition}
\newtheorem*{exmpl*}{Пример}
\theoremstyle{definition}
\newtheorem*{remrk*}{Замечание}
\theoremstyle{definition}
\newtheorem*{claim*}{Утверждение}
\theoremstyle{definition}
\newtheorem*{algorithm*}{Алгоритм}
\theoremstyle{definition}
\newtheorem*{problem*}{Задача}
\theoremstyle{definition}
\newtheorem*{exercise*}{Упражнение}

% small caps "proof" label
\addto\captionsrussian{\renewcommand\proofname{\normalfont\textsc{Доказательство}}}

% some aliases
\newcommand{\R}{\ensuremath \mathbb{R}}
\newcommand{\Q}{\ensuremath \mathbb{Q}}
\newcommand{\CC}{\ensuremath \mathbb{C}}
\newcommand{\Z}{\ensuremath \mathbb{Z}}
\newcommand{\N}{\ensuremath \mathbb{N}}
\newcommand{\eps}{\ensuremath \varepsilon}
\newcommand{\then}{\ensuremath \implies}
\newcommand{\Mid}{\ensuremath \;\middle|\;}

% project-local aliases
\newcommand{\A}{\ensuremath \mathfrak{A}}
\newcommand{\B}{\ensuremath \mathcal{B}}
\newcommand{\p}{\ensuremath \mathcal{P}}
\newcommand{\mua}{\ensuremath \mu^\ast}
\newcommand{\LL}{\ensuremath \mathcal{L}}
\newcommand{\lan}{\ensuremath \lambda_n}
\newcommand{\alan}{\ensuremath \A_{\lan}}
\newcommand{\LF}{\ensuremath \LL(\R^2, \CC)}
\newcommand{\lao}{\ensuremath \lambda_{1}}
\newcommand{\res}{\ensuremath \mathrm{res}}

\DeclareMathOperator{\sgn}{sgn}
\DeclareMathOperator{\Real}{Re}
\DeclareMathOperator{\Imaginary}{Im}
\DeclareMathOperator*{\esssup}{ess\,sup}
\DeclareMathOperator{\Ker}{Ker}
\DeclareMathOperator{\diam}{diam}
\DeclareMathOperator{\rk}{rk}
\DeclareMathOperator{\diverg}{div}

\title{Математический анализ III (СП). \\ Теория меры. \\ Конспект лекций, осень 2022 г.}
\author{Исходный набор в \TeX: Данила Усачев. \\ Правки, ревью: Николай Чухин, \\ Всеволод Васькин, Кирилл Митькин, \\ Андрей Петрухин. \\ CI: Максим Хабаров, Андрей Хорохорин.}
\date{}

\begin{document}
\maketitle
\maketitle
Конспект лекций \href{https://math-cs.spbu.ru/people/bessonov-r-v/}{\color{blue}Романа Викторовича Бессонова} по математическому анализу в третьем семестре, прочтённый осенью 2022 года второму курсу СПбГУ МКН <<Современное программирование>> (набор 2021 года).

В третьем семестре анализа изучается \emph{теория меры}. Для программистов теория меры нужна в первую очередь для того, чтобы осознать понятие \textit{вероятности}: этот курс является пререквизитом к курсу теории вероятностей (который изучается в четвёртом семестре). Данный курс содержит несколько \textit{глубоких} теорем --- красивых фактов, подобных которым не было в предыдущих двух семестрах анализа. Глубокое осознание курса должно поднять слушателя на новый математический уровень.

Начало курса мотивировано попытками математически строго построить \emph{длину} на вещественной прямой; для этой цели мы вводим абстрактное определение \emph{меры}, и доказываем \emph{теорему Каратеодори о стандартном продолжении}, являющуюся основным способом получить пространство с мерой. Далее изучается теория \emph{интеграла Лебега}: очень гибкое, в отличие от интеграла Римана, понятие, позволяющее интегрировать почти-что произвольную функцию в произвольном пространстве с мерой. Здесь доказываются классические результаты о предельном переходе под знаком интеграла: теорема Леви, теорема Лебега о мажорируемой сходимости, лемма Фату. Также затрагиваются пространства Лебега $ L^{p} $.

Особенность курса --- наличие \emph{теоремы Радона-Никодима}~--- очень красивого результата теории меры, описывающего взаимоотношение между произвольными двумя мерами на одном измеримом пространстве. Доказательство фон Неймана, приведённое здесь, использует методы функционального анализа, в связи с чем мы заранее изучаем гильбертовы пространства и теорему Рисса.

В оставшейся части семестра фокус смещается с абстрактной теории меры к более осязаемым пространствам: изучаются точки Лебега метрических пространств, рассматривается конструкция произведения мер, и, в частности, мера Лебега в $ \R^{n} $. Здесь доказывается \emph{принцип Кавальери}, а также классические теоремы \emph{Тонелли} и \emph{Фубини} о перестановке местами знаков интеграла. Также обсуждаются меры Лебега на поверхностях.

Заканчивается курс объёмным параграфом о \emph{дифференциальных формах} --- более изощрённой теории интегрирования, являющейся далёким и нетривиальным обобщением формулы Ньютона-Лейбница. Формулируется (но не доказывается) \emph{формула Стокса}, а также приводятся её интересные приложения, в частности, в математической физике.

\subsubsection*{Как читать}
Исходный код конспекта доступен в  \href{https://github.com/koreshaSp/matan2/releases}{\color{blue}репозитории GitHub} под лицензией \href{https://creativecommons.org/licenses/by/4.0/}{\color{blue}Creative Commons Attribution 4.0 International} (см. файл \texttt{LICENSE}).

Последнюю версию конспекта всегда можно найти в \href{https://github.com/koreshaSp/matan2/releases}{\color{blue}Releases}.

\subsubsection*{Как внести вклад}
Об опечатках и ошибках можно сообщать, создав новую \href{https://github.com/koreshaSp/matan2/issues}{\color{blue}Issue} на GitHub.
Добавить в/улучшить конспект можно через \href{https://github.com/koreshaSp/matan2/pulls}{\color{blue}Pull Request}: нужно создать ветку/форк, внести правки, а затем создать Pull Request в ветку main.
Можно ревьювить Pull Request'ы, отмечая ошибки.

\newpage
\tableofcontents
\newpage

\subfile{lectures/01.tex}
\subfile{lectures/02.tex}
\subfile{lectures/03.tex}
\subfile{lectures/04.tex}
\subfile{lectures/05.tex}
\subfile{lectures/06.tex}
\subfile{lectures/07.tex}
\subfile{lectures/08.tex}
\subfile{lectures/09.tex}
\subfile{lectures/10.tex}
\subfile{lectures/11.tex}
\subfile{lectures/12.tex}
\subfile{lectures/13.tex}
\subfile{lectures/14.tex}
\subfile{lectures/15.tex}

\end{document}

